\chapter{Geometrie auf der Kugeloberfläche\label{chapter:kugel}}
\lhead{Geometrie auf der Kugeloberfläche}
\begin{refsection}
\chapterauthor{Melina Staub und Fabian Schmid}

\section{Einleitung}
Seit jeher fasziniert den Menschen die Fahrt zur See. Nicht grundlos
ist die Seefahrt eine der wichtigsten und ältesten Tätigkeiten der
Menschheit. Der innere Drang neue Weltmeere und unbekannte Gebiete
zu entdecken, die Fahrt zur See zu erleichtern und erträglicher zu
machen, trieben die Menschen an, die Schiffe dieser Welt immer
weiter zu entwickeln.

Auch die Form der Erde spielte dabei eine wichtige Rolle. Denn die
Idee der Erde als Kugel ist älter als man zu denken vermag. Bereits
der Schüler Aristoteles des antiken griechischen Philosophen Platon
beschrieb in seiner Schrift {\em Über den Himmel} aus dem 4.
Jahrhundert v. Chr. etliche Gründe, welche für die Gestalt der Erde
als Kugel sprechen:
\begin{itemize}
      \item Sämtliche schweren Körper streben zum Mittelpunkt,
      nämlich der Erde. Da sie dies von allen Seiten her gleichmässig
      tun und die Erde im Mittelpunkt steht, muss sie eine kugelrunde
      Gestalt annehmen.
      \item Bei von der Küste wegfahrenden Schiffen wird der Rumpf vor
      den Segeln der Sicht verborgen.
      \item In südlichen Ländern erscheinen südliche Sternbilder
      höher über dem Horizont.
      \item Der Erdschatten bei einer Mondfinsternis ist stets rund.
\end{itemize}
Als Christoph Kolumbus im Jahr 1492 die Idee hatte, auf dem westlichen
Weg nach Indien zu segeln, stiess er beim portugiesischen König auf
taube Ohren.
Die Tatsache, dass die Erde eine Kugelgestalt hat, war damals nur
in den gebildeten Kreisen bekannt. Lediglich Kulturen welche noch
den Entwicklungsstand der Bronzezeit hatten, so auch die Bibel,
glaubten noch an die Gestalt der Erde als Scheibe.
Diese Sachlage war also nicht der Grund der fehlenden Unterstützung
Kolumbus durch die damalige Seefahrernation Portugal,  sondern die
grosse Entfernung von Europa nach Indien Indien über den westlichen
Seeweg.

Dadurch, dass die alten Griechen bereits sehr früh von der Erde als
Kugel wussten, konnte Eratosthenes etliche Jahre vor Christus
ziemlich genau den Erdumfang berechnen. Klar war, dass der östliche
Weg von Indien nach Asien etwa 10\,000\,km beträgt. Mit einem
Erdumfang von ca. 40\,000\,km, müsste also der Weg über den Westen
etwa 30\,000\,km weit sein.
Zu weit für die damalige Zeit und deren Schiffe. Kolumbus bildete
sich aber ein, in wenigen Wochen über den Westen nach Asien und
Indien zu gelangen. Dies aufgrund der Tatsache, dass er die arabischen
Längenmasse falsch in die damals üblichen Europäischen umgerechnet
hatte.
Durch diesen Fehler und die Überschätzung der Länge der Erdfläche
Asiens, war Kolumbus überzeugt, Westindien mit den ihm zur Verfügung
stehenden Flotten erreichen zu können.

Erst durch viel Überzeugungsarbeit am spanischen Hof, setzte Kolumbus
sich  durch und segelte westlich über den Atlantik und entdeckte
schlussendlich Amerika, nicht Indien.
Dabei hatte er Glück, dass sich zwischen dem europäischen und
asiatischen Kontinent noch der amerikanische befand, ansonsten wäre
er auf seiner Expedition schlichtweg verhungert.

Wäre er nicht zurückgekommen, hätte man nicht widerlegt, dass die
Erde eine Kugel ist. Es hätte nur belegt werden können, das man das
Meer über den Westen mit den damaligen Schiffen nicht durchqueren
konnte.
Der praktische und greifbare Beweis, dass die Erde eine Kugel ist,
lieferte rund 30 Jahre später der Portugiese Fernando Magellan. Mit
seiner Weltumsegelung und der Ankunft in den Philippinen bewies er
definitiv, dass die Erde eine Kugelgestalt hat. Die Kritiker
verstummten grösstenteils, denn noch heute gibt es Gesellschaften,
welche die Erde als Scheibe betrachten.

Nun wollen wir uns die Frage stellen, wie die alten Seefahrer ohne
GPS und jeglichen modernen Navigationssystemen auf hoher See wussten,
wo sie sich befanden und was die Sterne mit alledem zu tun haben.
Reisen Sie mit uns zurück in eine Zeit der Sextanten, Kompasse und
Sternkarten --- In die Zeit der Seefahrer und Entdecker.


\section{Gross- und Kleinkreise}
Kreise auf einer Kugeloberfläche lassen sich in zwei verschiedene
Arten einteilen: Gross- und Kleinkreise.

\subsection{Grosskreise}
%BILD GROSSKREISE
\begin{figure}
\centering
\includegraphics[width=0.4\textwidth]{kugel/Grosskreise.jpg}
\caption{Verschiedene Grosskreise auf einer Kugel, mit dem Zentrum
der Kugel als einheitlichen Mittelpunkt $M_{\text{Kugel}}$ und einem
einheitlichen Radius $r$ aller Kreise.}
\end{figure}

Es gibt unendlich viele Möglichkeiten eine Kugel in zwei gleich
grosse Stücke zu zerschneiden, daher gibt es auch unendlich viele
Grosskreise.

\begin{definition}
Ein Grosskreis ist ein grösstmöglicher Kreis auf einer Kugeloberfläche.
Sein Mittelpunkt fällt immer mit dem Mittelpunkt der Kugel zusammen.
Ein Schnitt durch einen Grosskreis, teilt die Kugel (auf jeden Fall)
in zwei (gleich grosse) Hälften.
\label{skript:kugel:satz:Grosskreis}
\index{Grosskreis}%
\end{definition}

Mithilfe der Längengrade die Grosskreise sind, wurde unsere Erde
in gleichmässige Felder ebendieser unterteilt. Die Längengrade
stehen orthogonal auf dem Äquator. Mit Schnittpunkten verschiedener
Grosskreise lassen sich sphärische Dreiecke bilden, auf welchen
sich die sphärische Trigonometrie anwenden lässt.


\subsection{Kleinkreise}
Die Kleinkreise eignen sich im Gegensatz zu den Grosskreisen {\em
nicht} für die sphärische Trigonometrie.  Dies liegt daran, dass
ihre Mittelpunkte nicht mit dem der Kugel zusammenfallen und die
Radien variieren können.
Sie werden lediglich für Messgrössen, Winkelabständen oder zur
Bestimmung des Höhenwinkels eines Gestirns verwendet.

%BILD KLEINKREISE
\begin{figure}[hbtp]
\centering
\includegraphics[width=0.4\textwidth]{kugel/Kleinkreise.jpg}
\caption{Verschiedene Kleinkreise auf einer Kugel mit verschiedenen
Kreismittelpunkten $M_{Kleinkreis}$ und Radien $r$.}
\end{figure}

\begin{definition}
Unter Kleinkreis versteht man jene Kreise auf einer Kugeloberfläche,
deren Ebenen nicht den Kugelmittelpunkt enthalten.
\label{skript:kugel:satz:Kleinkreis}
\index{Kleinkreis}%
\end{definition} 

Auf der Erdoberfläche sind die Breitengrade (-kreise) allesamt
Kleinkreise. Davon ausgenommen ist der Äquator. Er bildet zwar einen
Breitengrad, ist aber ein Grosskreis, da sein Mittelpunkt durch
denjenigen der Erde verläuft.
Die Breitengrade verlaufen parallel zum Äquator in Richtung Norden
und Süden. Sie bilden mit den Längengraden ein Raster auf der
Erdoberfläche. Mit dessen Hilfe lässt sich ein Ort eindeutig auf
der Erde definieren und man kann mit den Koordinaten seinen Standort
exakt bestimmen.
Für die Navigation, sind sowohl die für die sphärische Trigonometrie
hilfreichen Grosskreise (Längengrade) von Nöten, wie auch die
Kleinkreise (Breitengrade). Nur mit beiden Angaben in Kombination
lassen sich Standorte eindeutig bestimmen.



\section{Sphärische Dreiecke / Kugeldreieck}
\index{sphärisches Dreieck}%
\index{Dreieck, sphärisches}%
Der Begriff sphärisches Dreieck oder Kugeldreieck wird folgendermassen
definiert

\begin{definition}
Ein sphärisches Dreieck oder Kugeldreieck, ist eine durch drei
Grosskreise begrenzte Figur auf der Kugeloberfläche.
\end{definition} 

Dabei können wir sphärische Dreiecke in vier für uns wesentliche
Dreiecksarten unterteilen:

\begin{itemize}
\item Allgemeine Kugeldreiecke (Nicht Eulersche Dreiecke)
\item Kugelzweiecke
\item Eulersche Dreiecke
\end{itemize}


\subsection{Allgemeine Kugeldreiecke (Nicht Eulersche Dreiecke)}
Ähnlich dem Dreieck in der Ebene hat das Dreieck auf der Kugel
Seiten und Winkel. Allerdings werden die Dreiecksseiten nicht im
Längenmass angegeben, sondern im Bogenmass. Es handelt sich dabei
um Kreisbögen und keine Strecken.
Hinzu kommt, dass die Innenwinkelsumme eines Kugeldreiecks immer
grösser als $180^{\circ}$ ist. Bei allgemeinen Kugeldreiecken kann
diese Summe bis auf $900^{\circ}$ anwachsen.

%BILD KUGELDREIECKE ARTEN
\begin{figure}[htbp]
\centering
\includegraphics[width=0.9\textwidth]{kugel/Dreiecksarten.jpg}
\caption{Allgemeine Kugeldreiecke (grün) auf einer Kugel}
\end{figure}


\subsection{Kugelzweiecke} 
\index{Kugelzweieck}%
Zwei Grosskreise auf der Kugeloberfläche zerlegen diese in paarweise
gleich grosse Kugelzweiecke, welche die Kugeloberfläche gleichmässig
einnehmen. Würden wir für den Winkel
$\alpha = 90^{\circ}$ wählen, würden die Kugelzweiecke je einen
Viertel der Kugeloberfläche einnehmen. Die Längen der entstandenen
Zweiecksseiten, haben auf jedenfall die Länge
$180^{\circ}$, was auch $\pi$ entspricht.
Der Flächeninhalt wird dabei einzig durch den Winkel $\alpha$
zwischen den beiden Grosskreisen bestimmt, was im Ausdruck \eqref{V5}
aufgezeigt wird.

%BILD ZWEIECKE
\begin{figure}[htbp]
\centering
\includegraphics[width=0.4\textwidth]{kugel/Zweieck.jpg}
\caption{Bildung von Zweiecken durch Grosskreise}
\end{figure}

Den Flächeninhalt des Zweiecks berechnen wir wie folgt
\begin{equation}
A_\text{Zweieck}
=
\underbrace{4 \pi r^{2}}_{\displaystyle\text{$A_\text{Kugel}$}} \cdot \underbrace{\frac{ \alpha }{ 2 \pi }}_{\displaystyle\text{$A_\text{Kugelsegment}$}}
=
2 r^{2} \alpha .
\label {V5}
\end{equation}

Auf der Erdoberfläche finden wir Vierundzwanzig uns allbekannte
Zweiecke - Die Zeitzonen. Auf dieses Thema wird im
Abschnitt~\ref{Zeitzonen} \nameref{Zeitzonen} näher eingegangen.


\subsection{Eulersche Dreiecke} \label{Euler} 
\index{Dreieck, eulersches}%
\index{Eulersches Dreieck}%
Legt man drei Grosskreise auf eine Kugeloberfläche, bilden sich
dabei acht Dreiecke.
Jedes dieser Dreiecke heisst Eulersches Dreieck\footnote{%
Leonard Euler (1707-1783), berühmter Schweizer Mathematiker und Physiker. 
Ein nicht Eulersches Dreieck ist beispielsweise, das Äussere Dreieck
des Dreieckes
$\triangle{ABC}$.}.
Diese werden weder durch die Verlängerung ihrer Seiten durchschnitten, 
noch haben sie Dreiecksseiten, welche grösser sind als $180^{\circ}$.

%BILD EULERSCHE DREIECKE
\begin{figure}[htbp]
\centering
\includegraphics[width=0.4\textwidth]{kugel/EulerschesDreieck.jpg}
\caption{Drei Grosskreise bilden acht Eulersche Dreiecke}
\label{V10}
\end{figure}


In den nachstehenden Erklärungen und Herleitungen sprechen wir
ausschliesslich von Eulerschen Dreiecken, da die umgeformten
Winkelsätze der ebenen Trigonometrie nur auf diese Art von
Kugeldreiecken angewendet werden können.

Aus der Ebenen Trigonometrie folgt die Formel für die Innenwinkelsumme
aus dem Wechselwinkelsatz
\begin{align*}
\alpha + \beta + \gamma &= 180^{\circ}.
\end{align*}

Für die Innenwinkelsumme in der sphärischen Trigonometrie gilt dies
nicht. Obschon die Trigonometrie der Sphäre einige Gemeinsamkeiten
zur Trigonometrie der Ebene aufweist, kann nicht alles übernommen
werden.
So lässt sich auch die Innenwinkelsumme eines ebenen Dreiecks von
$180^{\circ}$ nicht in einem Eulerschen Dreieck übernehmen.
Denn in der Sphäre liegt diese zwischen
\[
\begin{aligned}
\text{$180^{\circ}$ bis $540^{\circ}$}
&
&\text{oder}
&
\text{$ \pi$  bis $ 3\pi$}.
\end{aligned}
\]
Daraus können wir schliessen, dass eine einzelne Seite durchaus die
Grösse $180^{\circ}$ oder $\pi$ annehmen kann, jedoch nicht mehr.
Ansonsten wäre es ein allgemeines Kugeldreieck und damit kein
Eulersches. Dies würde wiederum bedeuten, dass wir die sphärische
Trigonometrie nicht anwenden dürfen.


\section{Polardreieck}
\index{Polardreieck}%
Für jedes sphärische Dreieck auf der Kugeloberfläche gibt es das
zugehörige Polardreieck\footnote{%
Das Polardreieck verdient seinen Namen den beiden Polen der Erdkugel,
dem Nord- und Südpol, denn sie bilden die beiden Pole des Äquators
ab.}. Wenn man einen Grosskreis auf die Kugeloberfläche projiziert,
bilden die Schnittpunkte der Senkrechten durch den Mittelpunkt $M$
dabei die beiden Pole.

%ERKLAERUNG
\begin{figure}
\centering
\includegraphics[width=1\textwidth]{kugel/Polar.jpg}
\caption{Die Polarbögen zweier Grosskreise bilden einen Schnittwinkel
$\alpha$, welche durch einen Grosskreisbogen (Polarbogen) der
Seitenlänge $a’$ verbunden werden.}
\label{V8}
\end{figure}

Betrachten wir nun die Abbildung \eqref{V8}, wird ersichtlich, dass
die Beziehung
\begin{align*}
{\alpha + a’} = {\pi}
\end{align*}
gilt. 

%POLARDREIECK
\begin{figure}[htbp]
\centering
\includegraphics[width=0.4\textwidth]{kugel/Polardreieck.jpg}
\caption{Das sphärische Dreieck (blau) und sein duales Polardreieck
(hellblau) auf der Kugeloberfläche.}
\end{figure}

Somit können wir das Polardreieck eines sphärischen Dreieckes
bestimmen und stellen fest, dass die Längen des Polardreiecks und
die gegenüberliegenden Winkel des sphärischen Dreiecks sich immer
zu $\pi$ ergänzen lassen. Die folgenden Beziehungen
\begin{align*}
\alpha + a^{\prime} &= \pi & \alpha^{\prime} + a &= \pi \\
\beta + b^{\prime} &= \pi & \beta^{\prime} + b &= \pi \\
\gamma + c^{\prime} &= \pi & \gamma^{\prime} + c &= \pi
\end{align*}
entstehen.


\section{Dreiecksfläche und sphärischer Exzess} \label{Flaeche}
\index{Exzess, sphärischer}%
\index{sphärischer Exzess}%
Betrachten wir das hellgraue Dreieck in der Abbildung \eqref{V10}
ist $A_{\triangle{ABC}}$ dessen Flächeninhalt. Dieser lässt sich
folgendermassen berechnen
\begin{align*}
A_{\triangle{ABC}} &= r^{2}(\alpha + \beta + \gamma - \pi).
\end{align*}
Dass diese Formel zum gewünschten Flächeninhalt führt, lässt sich
folgendermassen beweisen:
Als erstes berechnen wir die einzelnen Flächeninhalte der Zweiecke
$A$, $B$ und $C$
\begin{align*}
\text{Zweieck A}
&=
\triangle{ABC} + \triangle{A'BC} = 2 \alpha r^{ 2 } = A_{ \alpha }\\
\text{Zweieck B}
&=
\triangle{ABC} + \triangle{AB'C} = 2 \beta r^{ 2 } = A_{ \beta }\\
\text{Zweieck C}
&=
\triangle{ABC} + \triangle{ABC’} = 2 \gamma r^{ 2 } = A_{ \gamma }
\end{align*}
Die drei Zweiecke und die beiden kleinen grauen Eulersche Dreiecke
bedecken zusammen die halbe Kugel, was in der Abbildung \eqref{V9}
ersichtlich wird.

%BILD DREIECKSFLÄCHE
\begin{figure}[htbp]
\centering
\includegraphics[width=0.4\textwidth]{kugel/HalbeKugel.jpg}
\caption{Durch Drehen des braunen sphärischen Dreieckes, wird
deutlich illustriert, dass die halbe Kugeloberfläche durch Drei-
und Zweiecke vollständig bedeckt wird.}
\label{V9}
\end{figure}

Nun setzen wir die Summe der Flächeninhalte der Zweiecke mit dieser
der halben Kugel gleich
\begin{align*}
2 \alpha r^2 + 2 \beta r^2 + 2 \gamma  r^2 = \pi r^2 + A_{ \triangle{ ABC }}.
\end{align*}
Für den Flächeninhalt erhalten wir
\begin{align*}
A_{ \triangle{ ABC }}
&=
r^{ 2 }\underbrace{(\alpha + \beta + \gamma - \pi)}_{\displaystyle\text{Sphärischer Exzess}}.
\end{align*}

\begin{definition}
Der Flächeninhalt eines Eulerschen Dreiecks ist die Summe seiner
Winkel multipliziert mit dem Radius im Quadrat.
\label{skript:kugel:satz:Flaecheninhalt}
\index{Flächeninhalt}%
\end{definition}


Die Winkelsumme abzüglich $\pi$ ist der sphärische Exzess $\epsilon$.
$\epsilon$ gibt die Abweichung zu $\pi$ an und somit der Innenwinkelsumme
von $180^{\circ}$
\begin{equation}
\epsilon = \alpha + \beta + \gamma - \pi.
\end{equation}
Folglich ist der Exzess direkt mit dem Flächeninhalt $\triangle A$
eines Eulerschen Dreiecks verbunden
\begin{align*}
\epsilon =\frac{A_{\triangle{ ABC }}}{r^2} = {\alpha + \beta + \gamma - \pi}.
\end{align*}

Mit einer maximalen Innenwinkelsumme bei Eulerschen Dreiecken von
$540^{\circ}$ ist der Exzess maximal $360^{\circ}$.
\[
0^{\circ} \le \epsilon \le 360^{\circ}
\qquad \text{oder} \qquad
\pi \le \epsilon \le 2\pi
\qquad \Rightarrow \qquad
0 \le \text{Fläche} \le 2 \pi r^2.
\]

Berechnen wir den maximalen sphärischen Exzess bei allgemeinen
Kugeldreiecken würde dieser maximal $720^{\circ}$ annehmen können.
\[
0^{\circ} \le \epsilon \le 720^{\circ}
\qquad \text{oder} \qquad
\pi \le \epsilon \le 4\pi
\qquad \Rightarrow \qquad
0 \le A_{\triangle{ ABC }} \le 4 \pi r^2.
\]


\subsection{Grenzfall --- Satz von Legendre}
Würden wir den sphärischen Exzess in der ebenen Trigonometrie
anwenden wäre dieser immer $\epsilon=0$. Betrachten wir nun sehr
kleine Kugeldreiecke oder solche mit grossen aber endlichen Radien,
würde die Innenwinkelsumme $\pi$ nur wenig übersteigen. Dies besagt,
dass wir sphärische Dreiecke mit geringer Grösse durch „Verebnung“
annähernd als solche der Ebenen Trigonometrie betrachten können.
Diese Erkenntnis beschreibt der Satz von Legendre.

\begin{satz}Ein kleines sphärisches Dreieck kann näherungsweise 
wie ein ebenes Dreieck mit denselben Seiten berechnet 
werden, wenn alle Winkel des ebenen Dreiecks die um 
je ein Drittel des sphärischen Exzesses verminderten 
Winkel des sphärischen Dreiecks nimmt. 
\\
\quad \quad - Adrien-Marie Legendre (1752-1833), Paris 1787
\label{skript:kugel:satz:Legendre}
\end{satz}
\index{Satz von Legendre}%

%BILD LEGENDRE
\begin{figure}[htbp]
\centering
\includegraphics[width=0.5\textwidth]{kugel/SphaerischerExzess.jpg}
\caption{Blick senkrecht von Oben auf ein sphärisches Dreieck (blau)
mit einem eingeschriebenen ebenen Dreieck (hellblau). Die 6 Winkel
übersteigen die Innenwinkelsumme von $180^{\circ}$ nur sehr gering.
Wir können das sphärische Dreieck annähernd als ein solches der
Ebene betrachten.}
\end{figure}

Durch den {\em Satz von Legendre} lässt sich ein Zusammenhang
zwischen der Trigonometrie der Ebene und jener auf der Kugel
herstellen.


\section{Sphärisch analoge Winkelfunktionen}
Euklid von Alexandria\footnote{%
Euklid war ein griechischer Mathematiker. Er lebte wahrscheinlich
3 Jahrhunderte vor Christus. In seinem berühmtesten Werk {\em
Elemente} fasst er die Arithmetik und Geometrie seiner Zeit zusammen.
Sein Werk {\em Elemente} war 2000 Jahre lang als Lehrbuch in Gebrauch
und war bis Mitte des 19. Jahrhunderts nach der Bibel das weit
verbreitetste Buch der Weltliteratur.}  beschrieb die Grundbegriffe
der Ebenen Geometrie mittels Punkt, Gerade, Ebene, Winkel und
Dreieck. Ebendiese Dreiecke lassen sich mithilfe der ebenen
Trigonometrie beschreiben. Dabei gelten die uns bekannten
trigonometrischen Winkelsätze.
Der Sinussatz ist nichts anderes als der Strahlensatz des ebenen Dreiecks:

\begin{satz}
Das Verhältnis der Längen zweier Seiten ist gleich dem Verhältnis
der Sinuswerte der gegenüberliegenden Winkel:
\begin{align*}
\frac{ a }{\sin(\alpha) } &= \frac{ b }{\sin(\beta)} = \frac{ c }{\sin(\gamma)}
\end{align*}
\label{skript:kugel:satz:SinussatzEbene}
\index{Sinussatz, eben}%
\end{satz}

Der Kosinussatz stellt Beziehungen zwischen den Seiten und den
Kosinuswerten der Winkel im Dreieck der Ebene her.

\begin{satz}
Die Beziehung zwischen den drei Seiten und dem Kosinus eines der Winkel.
\begin{align*}
c^{ 2 } &= a^{ 2 } + b^{ 2 } - 2ab\cdot \cos(\gamma)\\
b^{ 2 } &= a^{ 2 } + c^{ 2 } - 2ab\cdot \cos(\beta)\\
a^{ 2 } &= b^{ 2 } + c^{ 2 } - 2ab\cdot \cos(\alpha)
\end{align*}
\label{skript:kugel:satz:KosinussatzEbene}
\end{satz}
\index{Kosinussatz, eben}%

Um diese Winkelfunktionen auf der Kugeloberfläche anwenden zu können,
benötigen wir die sphärische Trigonometrie. Die oben beschriebenen
Sätze lassen sich auf der Kugel nicht anwenden, sie werden aber als
Grundlage und Gedankenstütze zur Herleitung der Sätze für das
Kugeldreieck benötigt.


\subsection{Sphärischer Sinussatz}
Wir betrachten das folgende sphärische Dreieck auf einem Teilstück
der Kugeloberfläche mit dem Radius $r= \overline{MA} = \overline{MB}
= \overline{MC}$. Wir fügen ein ebenes Dreieck $\triangle_{ADE}$
in das Kugelstück ein, welches den Eckpunkt $A$ beinhaltet und die
halbe Abbildung des sphärischen Dreieckes bildet.

%BILD SINUSSATZ
\begin{figure}[hbtp]
\centering
\includegraphics[width=0.6\textwidth]{kugel/Sinussatz.jpg}
\caption{Hilfsskizze um die Beziehung auf der Kugel rund um den
sphärischen Sinussatz zu illustrieren und aufzuzeigen.}
\end{figure}

Um den Sphärischen Sinussatz zu beweisen, möchten wir die Höhe
$h_{A}$ auf zwei verschiedene Arten schreiben
\begin{align}
h_{A} = \overline{AF} &= \overline{AD} \cdot \sin(\beta) = r \cdot \sin(c) \cdot \sin(\beta).  
\label {V1}
\end{align}
Aus einer anderen Sichtweise lässt sich die Strecke $h_{A}$ wie folgt schreiben
\begin{align}
h_{A} = r \cdot \sin(b) \cdot \sin(\gamma).  
\label {V2}
\end{align}
Durch Gleichsetzen der Ausdrücke \eqref{V1} und \eqref{V2} eliminieren
wir den Radius $r$ und es entsteht der Ausdruck
\begin{align*}
\sin(c) \cdot \sin(\beta) &= \sin(b) \cdot \sin(\gamma) \\
\Rightarrow \quad \quad
\frac{\sin (b)}{\sin (c)} &= \frac{\sin (\beta)}{\sin (\gamma)}.
\end{align*}
Analog dazu könnte man auch die Höhe $h_{B}$ nehmen und würde erhalten
\begin{align*}
\sin(c) \cdot \sin(\alpha) &= \sin(a) \cdot \sin(\gamma) \\
\Rightarrow \quad \quad
\frac{\sin (a)}{\sin (c)} &= \frac{\sin (\alpha)}{\sin (\gamma)}.
\end{align*}
Aus diesen Erkenntnissen lässt sich der Sinussatz zusammenfassen:

\begin{satz}
Die Sinuswerte der Seiten, verhalten sich wie die Sinuswerte der
gegenüberliegenden Winkel.
\begin{align*}
\sin a : \sin b : \sin c &= \sin \alpha : \sin \beta : \sin \gamma \\
\Rightarrow \quad \quad
\frac{\sin \alpha} {\sin a} &= \frac{\sin \beta} {\sin b} = \frac{\sin \gamma} {\sin c}.
\end{align*} 
\label{skript:kugel:satz:Sinussatz}
\index{Sinussatz, sphärisch}%
\end{satz}

%BILD SINUSSATZ BEWEIS
\begin{figure}[htbp]
\centering
\includegraphics[width=0.9\textwidth]{kugel/SinussatzB.jpg}
\caption{Sobald die Seite $a$ ihr Maximum erreicht, sinkt der Betrag
der Seite wieder bis auf 0. Aus diesem Grund ist sowohl beim Winkel,
als auch bei der Seite die Verwendung des Sinus vonnöten.}
\label{Beweissinus}
\end{figure}

Es fällt auf, dass der sphärische Sinussatz sehr ähnlich ist, zu
diesem in der ebenen Trigonometrie.
Der Grund weshalb man auch bei den Seiten den Sinuswert nehmen muss,
ist folgender:
Das Verhältnis zwischen den Seiten und Winkeln wird nicht unendlich
lange grösser je weiter man sich vom gesuchten Winkel entfernt. Auf
der Kugel steigt der Wert nur bis in die Mitte des Zweiecks an
(Abbildung \ref{Beweissinus}). Dort angelangt sinkt das Verhältnis
wieder solange, bis es beim Eckpunkt $A'$ angekommen ist.


\subsection{Seitenkosinussatz}
Es sei das Stück einer Kugel mit dem sphärischen Dreieck $\triangle{ABC}$
und dessen Winkeln $\alpha, \beta, \gamma$. Das ebene Dreieck
$\triangle{A'B'C'}$ steht dabei senkrecht auf der Kante $\overline{MC'}$.

%SEITENKOSINUSSATZ
\begin{figure}[htbp]
\centering
\includegraphics[width=0.4\textwidth]{kugel/Seitenkosinus.jpg}
\caption{Hilfsskizze um die Beziehungen auf der Kugel rund um den
Seitenkosinussatz zu illustrieren und aufzuzeigen}
\end{figure}

Die Strecken lassen sich nach den bekannten Regeln der Trigonometrie
beschreiben:
\begin{align*}
\overline{C'A'} = d\cdot {\tan(b)} \quad \quad \quad \quad \quad \quad 
\overline{MA'} = \frac{ d }{\cos(b)} \\
\overline{C'B'} = d\cdot {\tan(a)} \quad \quad \quad \quad \quad \quad 
\overline{MB'} = \frac{ d }{\cos(a)}
\end{align*} 

Um die Beziehung zwischen $a,b,c$ und $\gamma$ zu beweisen, berechnen
wir die Strecke $\overline{A'B'}$ auf zwei verschiedene Arten.
Als erstes mithilfe des Kosinussatzes im Dreieck $\triangle{A'B'C'}$:
\begin{align}
\overline{A'B'}^{ 2 } &= \overline{ C'B' }^{ 2 } + \overline{ C'A' }^{ 2 } - 2 \cdot \overline{C'B'} \cdot \overline{ C'A' } \cdot \cos(\gamma) \nonumber \\ 
\Rightarrow \quad \quad
\overline{A'B'}^{ 2 } &= d^{ 2 } \cdot \left(\left(\tan^{ 2 }(a) + \tan^{ 2 }(b)\right) - 2\cdot \tan(a) \cdot \tan(b) \cdot \cos(\gamma)\right).
\label {V3} 
\end{align}

Ebenso kann man die Strecke $\overline{A'B'}$ mithilfe des Kosinussatzes
im Dreieck $MA'B'$ bestimmen. Man erhält dabei die Formel \eqref{V4}
\begin{align*}
\overline{A'B'}^{2} &= \overline{MB'}^{2} + \overline{MA'}^{2} - 2\cdot \overline{MB'} \cdot \overline{MA'} \cdot \cos(c)
\end{align*}
\begin{align}
\Rightarrow \quad \quad
\overline{A'B'}^{ 2 } &= \left(\frac{ d }{\cos(a) }  \right)^{ 2 } + \left(\frac{ d }{\cos(b)}  \right)^{ 2 } - 2 \cdot \frac{ d }{\cos(a)} \cdot \frac{ d }{\cos(b)} \cdot \cos(c).
\label{V4}
\end{align}
Durch die Beziehung $\frac{1}{\cos^{2}(x)}=\tan^{2}(x)+1$ wird \eqref{V4} zu 
\begin{align}
\overline{ A'B'}^{ 2 } &= d^{ 2 } \cdot \left(\left(\tan^{ 2 }(a) + 1\right) + \left(\tan^{ 2 }(b) + 1\right) - \left(2 \cdot \frac{\cos(c)}{\cos(a) \cdot \cos(b)}\right)\right).
\label {V6}
\end{align}

Nach Gleichsetzen der beiden Gleichungen \eqref{V3} und \eqref{V6}
erhalten wir den vereinfachten Ausdruck
\begin{align*}
-2 \cdot \tan(a) \cdot \tan(b) \cdot cos(\gamma)
&=
-2+2 \cdot \frac{\cos©}{\cos(a) \cdot \cos(b)}.
\end{align*}


Umgeformt und unter Einhaltung der Definition
$\tan(a)=\frac{\sin(a)}{\cos(a)}$ (und der Multiplikation mit
$\frac{1}{2}$) ergibt sich
\begin{align*}
\frac{\sin(a)}{\cos(a)} \cdot \frac{\sin(b)}{\cos(b)} \cdot \cos(\gamma) &= -1 + \frac{\cos(c)}{\cos(a) \cdot \cos(b)}.
\end{align*}
Durch Vereinfachen erhalten wir den Seitenkosinussatz der Seite $c$
\begin{align*}
\cos(a) \cdot \cos(b) + \sin(a) \cdot \sin(b) \cdot \cos(\gamma) = \cos(c) .
\end{align*}

Um die Sätze für die Seiten $a$ und $b$ zu erhalten, vertauschen
wir die Variablen zyklisch und erhalten den Seitenkosinussatz dieser
Seiten.

\begin{satz}
Im sphärischen Dreieck ist der Kosinus einer Seite gleich der Summe
der Kosinusprodukte der beiden anderen Seiten und dem mit dem Kosinus
des eingeschlossenen Winkels multiplizierten Sinusprodukt dieser
Seiten.
\begin{align*}
{\cos a} &= {\cos b} \cdot {\cos c} + {\sin b} \cdot {\sin c} \cdot {\cos \alpha}\\
{\cos b} &= {\cos c} \cdot {\cos a} + {\sin c} \cdot {\sin a} \cdot {\cos \beta}\\
{\cos c} &= {\cos a} \cdot {\cos b} + {\sin a} \cdot {\sin b} \cdot {\cos \gamma}
\end{align*}
\label{skript:kugel:satz:Seitenkosinussatz}
\index{Seitenkosinussatz}%
\end{satz}

Mithilfe des Seitenkosinussatzes lassen sich die Dreiecksseiten
durch den jeweiligen gegenüberliegenden Winkel berechnen.


\subsection{Winkelkosinussatz}
Wenden wir den sphärischen Seitenkosinussatz auf dem Polardreieck
an, erhalten wir
\begin{align*}
{\cos a} &= {\cos b} \cdot {\cos c} + {\sin b} \cdot {\sin c} \cdot {\cos \alpha}
\end{align*}

%WINKELKOSINUSSATZ
\begin{figure}[htbp]
\centering
\includegraphics[width=0.4\textwidth]{kugel/PolarWinkelkosinus.jpg}
\caption{Hilfsskizze um die Beziehungen auf der Kugel zwischen dem
sphärischen Dreieck und dem Polardreieck mithilfe des Winkelkosinussatzes
zu illustrieren und aufzuzeigen}
\end{figure}

Durch die Beziehung zwischen dem Polardreieck und dem sphärischen
Dreieck, lässt sich der Seitenkosinussatz folgendermassen umformen
\begin{align*}
{\cos (\pi-\alpha)} &= {\cos (\pi-\beta)} \cdot {\cos (\pi-\gamma)} + {\sin(\pi-\beta)} \cdot {\sin(\pi-\gamma)} \cdot {\cos (\pi-a)}.
\end{align*}
Durch die Quadrantenbeziehung der trigonometrischen Funktionen im
Einheitskreis folgt
\begin{align*}
\sin (\pi-\alpha) &= \sin \alpha\\
\cos (\pi-\alpha) &= - \cos \alpha.
\end{align*}
Dies ergibt
\begin{align*}
{-\cos \alpha} &= {(-\cos \beta)} \cdot {(-\cos \gamma)} + {\sin \beta} \cdot {\sin \gamma} \cdot {(-\cos a)}.
\end{align*}
Mittels Vertauschen der Vorzeichen, erhalten wir den Winkelkosinussatz

\begin{satz}
Im sphärischen Dreieck ist der Kosinus eines Winkels gleich der
Summe aus dem negativen Produkt der Kosinusse der beiden anderen
Winkel und dem mit dem Kosinus der gegenüberliegenden Seite
multiplizierten Sinusprodukt der beiden anderen Winkel.
\begin{align*}
{\cos \alpha} &= {-\cos \beta} \cdot {\cos \gamma} + {\sin \beta} \cdot {\sin \gamma} \cdot {\cos a}\\
{\cos \beta} &= {-\cos \gamma} \cdot {\cos \alpha} + {\sin \gamma} \cdot {\sin \alpha} \cdot {\cos b}\\
{\cos \gamma} &= {-\cos \alpha} \cdot {\cos \beta} + {\sin \alpha} \cdot {\sin \beta} \cdot {\cos c}
\end{align*}
\label{skript:kugel:satz:Winkelkosinussatz}
\end{satz}
\index{Winkelkosinussatz}%
Auf der Kugel können wir demnach zwei Kosinussätze anwenden und
nicht nur einen wie in der Ebene. Dies ist darauf zurückzuführen,
dass wir uns nicht damit behelfen können, eine definierte
Innenwinkelsumme von $180^{\circ}$ vorzufinden. Doch mit beiden
Sätzen können wir alle Unbekannten auf der Kugel herausfinden.

\section{Dualität auf der Kugel}
Die Herleitung des Winkelkosinussatzes illustriert die Idee der
Dualität in der Geometrie. In der projektiven Geometrie lassen sich
Sätze dadurch dualisieren, dass die Begriffe Punkt und Geraden
vertauscht werden. Das Polardreieck präzisiert dies für die sphärische
Geometrie.
\index{projektive Geometrie}%
Daher lässt sich der Winkelkosinussatz durch Übergang zum Polardreieck
in den Seitenkosinussatz umwandeln. Dieser Vorgang ist umkehrbar
und zeigt ebenfalls die Dualität der Kugel auf.

\begin{satz}
Die sphärische Geometrie ist eine projektive Geometrie. In der
projektiven Geometrie lassen sich alle Sätze dualisieren. Sprich
die Begriffe Punkt und Geraden werden vertauscht; demzufolge auch
Längen und Winkel.
\label{skript:kugel:satz:Dualitaet}
\end{satz}
\index{Dualitaet}%

%DUALITÄT
\begin{figure}
\centering
\includegraphics[width=0.4\textwidth]{kugel/Dualitaet.jpg}
\caption{Sämtliche Seiten auf einer Kugel verhalten sich zu ihren
gegenüberliegenden Winkeln dual.}
\end{figure}%

Nimmt man nun den Punkt $A$, welcher auf der Geraden $b$ liegt, so
verläuft die duale Gerade $a$ durch den zur Geraden $b$ dualen Punkt
$B$.
Aber nicht nur die Beziehungen zwischen Punkten und Längen bleiben
erhalten. Auch die Winkel und Längen gehen ineinander über, wie im
Beweis des Winkelkosinussatzes.
Der Winkel $\gamma$ zwischen den beiden Seiten $a$ und $b$ entspricht
dem Abstand zwischen den zu der Geraden dualen Punkten $A$ und $B$.

\section{Navigation auf See}
Das Besondere an Seekarten ist die inhaltliche Ausrichtung. Anders
als die Landkarten muss sie Informationen enthalten, welche für den
Kapitän und seine Besatzung von grosser Bedeutung sind. Vor allem
in Küstennähe ist das Navigieren eines Schiffes besonders gefährlich.
So enthalten Seekarten Informationen über die Wassertiefen,
Bodenbeschaffenheiten, Gezeiten, Küstenlinien, Landzungen und
Windrichtungen.
Der Hauptunterschied dabei ist, dass auf der Landkarte feste
Positionen definiert und aufgezeigt werden. Das Einzige, das sich
bewegt, ist dabei der Reisende selbst. Bei der Seekarte ist das
anders. Es werden veränderliche Einwirkungen der Natur festgehalten
und die Schiffe auf See bewegen sich.
Dieser kleine Unterschied zeigt die Notwendigkeit auf, die Position
und den Kurs seines Schiffes auf See immer ermitteln zu können.

\subsection{Geografische Koordinaten}
Bereits der griechische Astronom Claudius Ptolemäus verwendete in
seiner {\em Geographike Hyphegesis} ein Gradnetz aus Längen- und
Breitengraden. Doch sein Werk  blieb lange verschollen. Im Mittelalter
beschäftigten sich die Gelehrten lieber mit dem Jenseits als mit
dem Diesseits und vergassen so einige Entdeckungen der Antike.
\index{Ptolemäus, Claudius}%
\index{Tordesillas, Vetrag von}%
\index{Ferro-Meridian}%
\index{Greenwich Meridian}%

Ptolemäus Vermächtnis wurde erst anfangs des 15. Jahrhunderts
wiederentdeckt und in die lateinische Sprache übersetzt. Das
ptolemäische Gradsystem setzte sich allmählich durch.
Mit dem Vertrag von Tordesillas 1494 gewann sein Gradnetz an
politischer Bedeutung und wurde zum Standard Gradsystem. Der dabei
verwendete Nullmeridian (Ferro-Meridian\footnote{%
Die Insel Ferro liegt in den Kanaren und war in der Antike der
damals westlichste bekannte Punkt der Erde. Dieser Referenzpunkt
bildete den Ferro-Meridian. Seine Festlegung beruht auf Claudio
Ptolemäus.}) wurde bis ins 19. Jahrhundert verwendet, als er dann
im Jahr 1884 durch den noch heute gültigen Greenwich Meridian\footnote{%
Der Greenwich Meridian war bereits seit 1738 in Grossbritannien in
Gebrauch. Er führt durch das {\em Royal Greenwich Observatory},
welches im Bezirk {\em Royal Borough of Greenwich}, südöstlich von
London liegt.}
ersetzt wurde.
\index{Greenwich Mean Time}%
\index{GMT}%
\index{Coordinated Universal Time}%
\index{UTC}%
Zur geografischen Ortsbestimmung und damit der Festlegung seines
eigenen Standortes auf einer Kugel sind Längen- und Breitengrad
nötig.
Die Koordinaten werden traditionell im Sexagesimalsystem angegeben
und setzen sich aus folgenden Komponenten zusammen:
\begin{center}
\renewcommand{\arraystretch}{1.5}
\begin{tabular}{ccc}
Grad $(^{\circ})$ & Bogenminuten (`) & Bogensekunden (``)
\end{tabular}
\end{center}

Die Erdoberfläche wurde in je 360 Breiten- und Längengrade eingeteilt.
Die Breitengrade haben zueinander einen Abstand von 111.31 km, dies
entspricht auch dem Abstand der Längengrade am Äquator, welcher mit
zunehmender Nähe zu den Polen abnimmt.
\begin{center}
\renewcommand{\arraystretch}{1.5}
\begin{tabular}{ccc}
\textbf{Gradmass} & \textbf{Umrechnung in Zeitmass} & \textbf{Umrechnung in Kilometer}  \\
$1^{\circ}$ & 4 Minuten & 111.31\,km \\
$0.25^{\circ}$ & 1 Minute & 27.78\,km \\
$0.004166^{\circ}$ & 1 Sekunde & 463\,m 
\label {V7}
\end{tabular}
\end{center}
 
Eine ganze Erdumdrehung beinhaltet $360 ^{\circ}$, was 1440 Minuten
entspricht. Umgerechnet in Kilometer erhält man bei einer Umdrehung
um den Äquator genau den Erdumfang von 40\,071\,km.
Nach einer vollen Umdrehung der Erde stellt sie sich wieder in ihrer
Ursprungsposition ein und ein neuer Tag beginnt. Dies zeigt, dass
die Koordinaten in direktem Zusammenhang mit der Zeit stehen. Diese
Erkenntnis wird später für die Navigation auf See von grosser
Bedeutung sein.


\subsection{Zeitzonen der Erde} \label{Zeitzonen} 
Wenn man nun die verschiedenen Zeitzonen der Erde betrachtet, macht
die Verschiebung von jeweils genau einer Stunde durchaus Sinn.
Zwischen den verschiedenen Zeitzonen liegen genau 15 Längengrade:
\begin{center}
\renewcommand{\arraystretch}{1.5}
\begin{tabular}{ccc}
\textbf{Anzahl Längengrade} & \textbf{Umrechnung in Minuten} & \textbf{Umrechnung in Kilometer}  \\
15 Längengrade à 4 Minuten & 60 Minuten Zeitverschiebung & ca. 1665\,km \\
\end{tabular}
\end{center}

Dabei ist die Zeitzone, in welcher Mitte sich der Greenwich Meridian
befindet, die {\em Greenwich Mean Time (GMT)}, welche bis 1928 als
Weltzeit galt. Im Jahr 1972 wurde diese umbenannt in die {\em
Coordinated Universal Time (UTC)} und wird von da an als Weltzeit
und Startpunkt für die Aufteilung der Zeitzonen verwendet. Der
Nullmeridian blieb derselbe.


\section{Der Breitengrad}
Die Breitengrade bilden die bereits genannten Kleinkreise auf der
Kugeloberfläche. Sie verlaufen in einem Abstand von etwa 111 km
parallel zum Äquator. Der Äquator stellt die Mitte zwischen Nord-
und Südhalbkugel dar und teilt die Erdkugel dabei in zwei gleich
grosse Hälften. Somit bildet er einen natürlichen Nullpunkt für die
Breitengrade.
Um zu wissen, auf welcher Halbkugel man sich befindet, wird von
nördlicher und südlicher Breite gesprochen. Für die Nordhalbkugel
werden positive und für die Südhalbkugel negative Werte geschrieben.
So ist auf den ersten Blick erkennbar auf welcher Halbkugel der
Erde sich der Standort befinden.

%BREITE
\begin{figure}[htbp]
\centering
\includegraphics[width=0.4\textwidth]{kugel/BreiteErdkugel.jpg}
\caption{Breitengrade auf der Erdoberfläche mit dem Äquator als
Nullmeridian welcher die Erde in nördliche- und südliche Breite
gleichermassen aufteilt.}
\end{figure}



\subsection{Geografische Breite $\phi$}
\begin{definition}
Die geografische Breite eines Standortes ist der Winkel am
Erdmittelpunkt zwischen der Ebene des Äquators und der Geraden zum
Standpunkt auf der Erdoberfläche.
\end{definition}

%BREITE
\begin{figure}[hbtp]
\centering
\includegraphics[width=0.4\textwidth]{kugel/GeografischeBreite.jpg}
\caption{Der Breitengrad ist nichts anderes, als der Winkel $\phi$
welcher den Abstand der Ebene des Äquators bis zum gesuchten
Breitengrad angibt.}
\end{figure}

\subsection{Navigation mit den Breitengraden}  \label{BreitengradM}
Da der Breitengrad bereits sehr früh ziemlich präzise bestimmt
werden konnte, nutzten bereits die Seefahrer um Christoph Kolumbus
den Breitengrad zur Navigation ihrer Flotten.
Dieser lässt sich ziemlich einfach aus dem höchsten Sonnenstand
oder mithilfe eines Fixsternes bestimmen. Dabei wird mit einem
Jakobsstab\footnote{%
Der Jakobsstab ist ein früheres astronomisches Instrument zur
Winkelmessung und wurde vor allem in der Seefahrt verwendet. Er ist
in der Nautik der Vorläufer des Sextanten.} (später Sextant\footnote{%
Der Sextant ist ein nautisches Messinstrument zur Winkelmessung von
Horizont und Fixstern (Gestirn)}) der Winkel zwischen dem Horizont
und dem höchsten Sonnenstand oder dem Fixstern gemessen. Den Winkel,
welchen man erhält, zieht man von 90° ab und erhält somit die
geografische Breite.

%BREITE
\begin{figure}[htbp]
\centering
\includegraphics[width=1\textwidth]{kugel/Breitengrad.jpg}
\caption{Mit dem Sextant wird gleichzeitig der Horizont und die
Sonne (Fixstern) angepeilt. Mit dem Winkel zwischen Horizont und
Gestirn lässt sich dann der Breitengrad ermitteln.}
\end{figure}

Wenn man sich auf der Nordhalbkugel befindet, ist der Polarstern
ein für die Navigation gut geeigneter Fixstern. Befindet sich ein
Schiff nun sehr nahe am Nordpol, steht dieser nahezu senkrecht am
Himmels bei $90^{\circ}$. Würde es sich aber nahe des Äquators
befinden, erscheint dieser am Horizont bei $0^{\circ}$ und wäre je
nachdem nicht mehr zu sehen.


\subsection{Korrekturbeiwert}
Da Breitengrade Kleinkreise sind, haben diese nicht immer den selben
Radius und Mittelpunkt. Daher segelt man am Äquator viel länger dem
Breitengrad entlang, um zum nächsten Längengrad zu kommen als in
der Nähe des Nord- oder Südpols.
Um die verminderte Strecke zu erhalten, müssen wir den Cosinus des
gemessenen Breitengrades berechnen und diesen mit der Abweichung
auf dem Äquator von 1 Sekunde multiplizieren.

Dies zeigt, je näher man den Polen ist, desto weniger weit muss man
segeln, um den nächsten Längengrad zu erreichen.

%KORREKTURBEIWERT
\begin{figure}[htbp]
\centering
\includegraphics[width=1\textwidth]{kugel/Korrekturbeiwert.jpg}
\caption{Die längsten Wege sind dort, wo der grösste Radius vorhanden
ist. Fährt man hingegen in der Nähe eines Pols, verkürzt sich der
Weg um den Cosinus des gemessenen Winkels $\phi$.}
\end{figure}

\section{Der Längengrad}
Die Längengrade bilden die bereits genannten Grosskreise auf der
Kugeloberfläche.
Sie schneiden den Äquator im rechten Winkel und haben dort einen
Abstand von etwa 111 km zueinander. Zugleich verbinden sie die
beiden Pole, Nord und Süd, miteinander. Anders als bei der geografischen
Breite ist in der Natur kein Längengrad gegeben, welcher den Nullpunkt
darstellen könnte.

%LÄNGENGRAD
\begin{figure}[hbtp]
\centering
\includegraphics[width=0.4\textwidth]{kugel/Laengengrad.jpg}
\caption{Bild}
\end{figure}


\subsection{Geografische Länge $\lambda$}
\begin{definition}
Die geografische Länge $\lambda$ ist der Winkel an der Erdachse zum
Nullmeridian.
\end{definition}

%LÄNGENGRADDEF
\begin{figure}[htbp]
\centering
\includegraphics[width=0.4\textwidth]{kugel/GeografischeLaenge.jpg}
\caption{Der Längengrad ist nichts anderes, als der Winkel $\lambda$
welcher den Abstand zwischen dem gesuchten Längengrad und dem
Nullmeridian angibt.}
\end{figure}

\subsection{Navigation mit den Längengraden}
Die geografische Länge eines Punktes lässt sich nicht so einfach
bestimmen wie seine Breite.
Für die Berechnung auf See wird eine Referenzzeit eines Ortes mit
bekannter Länge benötigt.
In der Zeit der Entdecker im 15. Jahrhundert gab es noch keine
mechanischen Uhren. Die Sonnenuhr war zudem ungeeignet, da diese
nur die Tageszeit am Standort mass und nicht die am Referenzort
selbst.

Die erste Pendeluhr wurde Mitte des 17.~Jahrhunderts erfunden, was
in der Schifffahrt aber auch nicht die Lösung brachte. Pendeluhren
auf einem Schiff sind ungeeignet, da das Pendel mit dem Wellengang
aus dem Takt gebracht wird und somit die Uhr falsch geht.
Zu ungenau und gegen äussere Erschütterungen sehr empfindlich waren
später die federgetriebene Uhren mit Unruh. Die verschiedenen
Klimazonen, welche mit dem Schiff durchquert wurden, stellten
ebenfalls ein grosses Problem dar. Das Metall zog sich zusammen
oder dehnte sich aus, was dazu führte, dass die Uhr unregelmässig
lief.

\subsection{Das Längenproblem}
Das sogenannte ``Längenproblem'' stellte nicht nur in der Navigation
auf See ein Problem dar, auch die Wirtschaft hatte darunter zu
leiden. Die Schiffe mussten bis zur gewünschten geografischen Breite
navigieren und segelten dann dem Breitengrad entlang, um auch
wirklich an der gewünschten Position anzukommen. Dabei waren die
Schiffe oft wochenlang unterwegs und segelten die ``Breiten ab'', um
die gewünschten Positionen zu erreichen. Dies führte zu erheblichen
Zeitverlusten und viel längeren Reisezeiten.


%BREITENSEGELN
\begin{figure}[htbp]
\centering
\includegraphics[width=0.4\textwidth]{kugel/Breitensegeln.jpg}
\caption{Die rote Linie zeigt den deutlich längeren Weg an, um ans
Ziel zu kommen. Es war aber bevor man den Längengrad bestimmen
konnte, der einzig sichere Weg, die Route zu finden durch das
``Breiten absegeln''. Erst mit der Möglichkeit den Längengrad auf See
zu berechnen, konnte die gelbe Route gesegelt werden.}
\end{figure}

\section{The Board of Longitude}
\index{Board of Longitude}%
Das Längenproblem beschäftigte alle grossen Seefahrernationen
Europas. Die fehlenden Längengrade bei der Navigation führten zu
vielen Schiffsunglücken. Nicht selten kam es vor, dass sich auf den
untergegangenen Schiffen Schätze in der Höhe von halben britischen
Staatshaushalten befanden. Der Verlust solcher Schiffe war enorm.
Bereits um 1600 hatte der König von Spanien ein Preisgeld ausgeschrieben
für denjenigen, welcher eine Lösung für das Problem präsentieren
konnte. Leider ohne Erfolg.

Nach einem tragischen Unglück im Jahr 1707, bei dem der siegreiche
Admiral Sir Cloudesley und seine 1\,450 Mann ihr Leben liessen, indem
sie auf die Scilly-Inseln kurz vor Land's End aufliefen und dabei
die 21 Schiffe sanken, rückte das Problem wieder in den Vordergrund.
\index{Cloudesley, Shovell}%
\index{Scilly-Inseln}%
Sieben Jahre später und mithilfe einer Petition von William Whiston
und Humphry Ditton, welche von Sir Isaac Newton und Edmond Halley
unterstützt wurde, reagierte das britische Parlament.
\index{Newton, Isaac}%
\index{Halley, Edmond}%
Es schrieb folgende Preisgelder für eine praktische und brauchbare
Lösung aus:
\begin{center}
\renewcommand{\arraystretch}{1.5}
\begin{tabular}{ccc}
\textbf{Preissumme £} & \textbf{Max. Längenfehler in Grad} & \textbf{Max. Längenfehler in km am Äquator}  \\
20\,000£ & max. $\frac{1}{2}^{\circ}$ & 55.5 km \\
15\,000£ & max. $\frac{2}{3}^{\circ}$ & 74km \\
10\,000£ & max. $1 ^{\circ}$ & 111 km 
\end{tabular}
\end{center}

Eine Abweichung von 111km am Äquator entspricht 60 Seemeilen.
Auf der Höhe des Ärmelkanals und damit in der Nähe von London
entspricht die Abweichung von $1 ^{\circ}$ noch 74\,km.
Das Preisgeld entsprach einer enorm hohen Summe für die damalige
Zeit. Der Kaufpreis für ein mittleres Schiff, welches zur See fahren
konnte, lag damals bei etwa 1\,500--2\,500£. Ein einzelner Arbeiter
lebte von 10£ im Jahr.

Würde man dieses Problem in der heutigen Zeit, mit einer maximalen
Abweichung von einem halben Grad lösen, erhielte man 2\,840\,000\,£.
Dies entspricht etwa einem Wert von 3\,600\,000 Schweizer Franken,
je nach Wechselkurs.
Damit die Lösungsvorschläge kontrolliert und verwaltet werden
konnten, wurde die Board of Longitude (Längenkommission) gegründet.
Ihr gehörten die bedeutendsten Astronomen und Mathematiker dieser
Zeit an, aber auch berühmte Persönlichkeiten aus Grossbritannien
wie der Präsident der Royal Society und damit niemand anderes als
Sir Issac Newton.

\subsection{John Harrison}
\index{Harrison, John}%
John Harrison brachte sich das Handwerk des Uhrmachers selbst bei.
Im Jahr 1713 und im Alter von 20 Jahren konstruierte er seine erste
Pendeluhr. Später folgten weitere Pendel- und Standuhren. Durch
seine Erfindungen der Grasshopper-Hemmung und des Rostpendels
erreichten seine Uhren eine aussergewöhnliche Genauigkeit für die
damalige Zeit. Die Abweichung pro Monat betrug dabei nur etwa eine
Sekunde.

%JOHN
\begin{figure}[htbp]
\centering
\includegraphics[width=0.3\textwidth]{kugel/JohnHarrison.jpg}
\caption{Portrait von John Harrison mit der H4, gemalt von P.~K.~Tassaert}
\end{figure}

Erst 13 Jahre nach der Ausschreibung für die Lösung des Längenproblems
tüftelte er an einer Konstruktion für eine Schiffsuhr und setzte
sich mit dem Längenproblem auseinander.
Namhafte Astronomen in ganz Europa suchten nach astronomischen
Lösungen für das Problem. Harrison jedoch setzte auf genaue Uhren
und war somit unabhängig davon, ob man den Mond sah oder nicht.
1728 folgte sein erstes Konzept für eine schiffstaugliche Uhr. 1735
präsentierte er sein erstes Modell, die H1. Die Testfahrt mit der
H1 an Bord von London nach Lissabon und zurück hielt die vorgeschriebene
Genauigkeit ein und übertraf diese sogar. Die Reisedauer hatte
jedoch nicht den vorgeschriebenen Bedingungen entsprochen.

Das Hauptproblem war jedoch, dass Harrison als nichtstudierter Laie
einem gelehrten Gremium dem Board of Longitude gegenüber stand.
Dies verzögerte seine Annahme um Jahrzehnte. Der Krieg verlief
ebenso zu Ungunsten Harrisons, denn seine weiteren Uhren, die H2
und H3, wurden nie getestet. Das Britische Empire wollte verhindern,
dass die Uhren in die Hände der zur dieser Zeit verfeindeten Spanier
gelangten.

Mit einem Durchmesser von 13\,cm und 1.45\,kg folgte im Jahr 1753
endlich der Durchbruch.
Die Taschenuhr H4 stellte alle anderen Uhren in den Schatten. Auf
einer 81-tägigen Fahrt nach Jamaika zeigte sie eine Abweichung von
nur 5 Sekunden.
Das Gremium entschied erneut gegen Harrison und beauftragte ihn
seine Uhr vor ihren Augen zu zerlegen, zu erklären und
Konstruktionszeichnungen anzufertigen.
Harrison erhielt vom britischen Parlament ein Preisgeld von 10\,000£,
um eine Kopie seiner Uhr H4 anzufertigen. Er musste beweisen, dass
seine Uhr nicht nur zufällig so genau lief.
\begin{figure}[!htb]
\centering
    \subfigure[H1, 1735]{\includegraphics[width=0.25\textwidth]{kugel/H1.jpg}} 
\quad \quad
\centering
    \subfigure[H2, 1737]{\includegraphics[width=0.25\textwidth]{kugel/H2.jpg}} 
\quad \quad
\centering
    \subfigure[H4, 1753]{\includegraphics[width=0.25\textwidth]{kugel/H4.jpg}} 
\caption{Harrisons Uhren im Laufe der Zeit} 
\end{figure}

Der bekannte Londoner Uhrmacher Larcum Kendall fertigte danach eine
Kopie der H4 an und zeigte somit, dass die Uhr den Anforderungen
entsprach.
Das Geld blieb für Harrison aber weiterhin aus. Erst nachdem der
britische König Georg III dem Parlament angedroht hatte, persönlich
zu erscheinen und Harrison das Geld zuzusprechen, schrieben diese
ihm 3 Jahre vor seinem Ableben weitere 8750£ zu.

Acht Monate vor seinem Tod und der Rückkehr der K1 (Kopie der H4)
ging die Vision Harrisons in Erfüllung. Es war nun definitiv bewiesen,
dass seine Uhr auf dem offenen Meer zur Bestimmung des Längengrades
taugte. James Cook, der berühmte britische Seefahrer und Entdecker,
welcher die Uhr K1 auf See mitnehmen und testen durfte, nannte die
Uhr liebevoll seinen {\em nie versagenden Führer}.
\index{Cook, James}%
Auf der Grundlage Harrisons Uhr H4 wurden die Schiffschronometer
noch lange Zeit gebaut.



\subsection{Tobias Mayer}
\index{Mayer, Tobias}%
Etwa zur gleichen Zeit wie John Harrison entwickelte Tobias
Mayer\footnote{%
Tobias Mayer (1723-1762) studierte nie an einer Universität und war
trotzdem ein annerkannter Wissenschaftler seiner Zeit in den Bereichen
Astronomie, Geo- und Kartografie, Mathematik und Physik.}
ebenso eine Lösung für das Längenproblem.

%MONDKARTE
\begin{figure}[!htb]
\centering
    \subfigure[Portrait, Kupferstich von Conrad Westermayr]{\includegraphics[width=0.25\textwidth]{kugel/TobiasMayer.jpg}}
\quad \quad
\centering
    \subfigure[Mondkarte]{\includegraphics[width=0.25\textwidth]{kugel/Mondkarte.jpg}}
\caption{Tobias Mayer und seine Mondkarte} 
\end{figure}

Seine Mondkarten galten ein halbes Jahrhundert lang als unübertroffen.
Der Ruhm galt aber hauptsächlich seinen Mondkarten, welche er im
Jahr 1755 in einer erweiterten Version dem britischen Parlament
vorlegte.
Mit ihnen konnte man die geografische Länge bis auf 5 Bogensekunden
genau bestimmen. Dies entsprach am Äquator $0.5 ^{\circ}$, was
wiederum einer Genauigkeit von 55\,565\,km entsprach.

Eine Lösung für das Längenproblem war gefunden. Die Publikation
seiner Mondtafeln fand 1767 unter dem Titel {\em Theoria lunae juxta
systema Newtonianum} in London statt, 5 Jahre nach Mayers Tod.
Seine Witwe schickte die publizierten Mondkarten über die Universität
Göttingen nach Grossbritannien. Sie erhielt von der britischen
Regierung eine Prämie in der Höhe von £ 3\,000.

Im Jahr 1935 wurde ein Krater auf der westlichen Mondvorderseite
nach dem deutschen Astronomen benannt. Er trägt fortan den Namen
T.~Mayer.

\section{Nautisches Dreieck (Astronomisches Dreieck)}
Um seine Koordinaten bestimmen zu können, genauer um den Längengrad
ohne moderne GPS-Geräte zu ermitteln, ziehen wir ein altbewährtes
Hilfsmittel aus der Seefahrt zu Hilfe --- das Nautische Dreieck.

Es dient zur Positionsbestimmung auf dem offenen Meer oder anderen
Gebieten, in denen keine Orientierungspunkte wie Landzungen oder
Gebirgsketten zu Hilfe genommen werden können.

Das Nautische Dreieck an der Himmelskugel hat folgende Eckpunkte:
\begin{itemize}
\item Himmelsnordpol\footnote{%
Der Himmelsnordpol ist von einem Schiff aus nicht ersichtlich, da
es sich um einen fiktiven Punkt handelt. Um diesem Problem auszuweichen,
verwenden wir den Polarstern, welcher nahezu senkrecht über dem
Himmelsnordpol steht.}
 ($N$) --- wir verwenden hier den Polarstern als unseren Nordpol
\item Gestirn ($S$) --- ein uns bekannter Stern, wir verwenden im
Beispiel die Sonne
\item Zenit ($Z$) --- der Himmelspunkt, welcher sich senkrecht über
uns befindet
\end{itemize}

%SKIZZE NAUTISCHES DREIECK
\begin{figure}[htbp]
\centering
\includegraphics[width=0.4\textwidth]{kugel/NautischesDreieck.jpg}
\caption{Das Nautische Dreieck}
\end{figure}

Dieses Kugeldreieck wird “nautisches Dreieck“ genannt und wird zur
Navigation auf See und dem Bestimmen seiner Koordinaten verwendet.

\subsection{Das Nautische Dreieck an der Himmelskugel}
%
%Bestimmung Seiten
\begin{figure}
\centering
\includegraphics[width=1\textwidth]{kugel/Bestimmung.jpg}
\caption{Das Nautische Dreieck an der Himmelskugel mit den zu
ermittelnden Längen}
\label{NautischeDreieck}
\end{figure}
%
%
Das Nautische Dreieck hat folgende Dreiecksseiten, welche zu berechnen
sind:
\begin{align*}
\overline{ZS} = 90^{\circ} - h \quad \quad \quad \quad \quad \quad 
\overline{NZ} = 90^{\circ} - \phi \\
\overline{NS} = 90^{\circ} - \delta \quad \quad \quad \quad \quad \quad 
\tau = t - e_\delta - \lambda 
\end{align*}
(siehe auch Abbildung \ref{NautischeDreieck})

\subsection{Bestimmung des Längengrades} \label{BestimmungL} 
Zur Bestimmung des Längengrades verwenden wir den Seitenkosinussatz:
\begin{align*}
\cos(c) = \cos(a)\cos(b) + \sin(a)\sin(b)\cos(\gamma).
\end{align*}
Auf unsere Seiten angewendet können wir einsetzen
\begin{align*}
\cos(\overline{ZS}) &= \cos(\overline{NZ}) \cos(\overline{NS}) + \sin(\overline{NZ}) \sin(\overline{NS}) \cos(\tau) \\
\Rightarrow \quad \quad
\cos(90^{\circ} - h) &= \cos(90^{\circ} - \phi) \cos(90^{\circ} - \delta) + \sin(90^{\circ} - \phi)\sin(90^{\circ} - \delta) \cos(\tau).
\end{align*}
Nach dem uns unbekannten Winkel $\tau$ aufgelöst ergibt die Gleichung
\begin{align*}
\tau &= \arccos 
\frac{ \cos(90^{\circ} - \phi) \cos(90^{\circ} - \delta) - \cos(90^{\circ} - h)} {\sin(90^{\circ} - \phi)\sin(90^{\circ} - \delta)}.
\end{align*}
Der Winkel $\tau$ setzt sich aus folgenden Komponenten zusammen
\begin{align*}
\cos (\tau) &= \cos (t - e_\delta - \lambda).
\end{align*}

Die zugehörigen bekannten Grössen sind:
\begin{itemize}
\item Zeit ($t$) $\Rightarrow$ Uhr
\item Rektazension ($e_\delta$) $\Rightarrow$ Sternatlas, Almanach, App 
\item Längengrad ($\lambda$) $\Rightarrow$ Der Längengrad unseres
Standorts ist die einzige Unbekannte
\end{itemize}


\subsection{Sternzeit}
\index{Sternzeit}%
Für die Positionsbestimmung auf der Erde ist die Sonne nicht der
beste Stern, welchen man auswählen kann. Sie ist nur am Tag zu sehen
und man kann nur eine Messung durchführen, was zu einem relativ
ungenauen Ergebnis führt. Zudem kann man die Messergebnisse nur
verwenden, wenn die Sonne zu diesem Zeitpunkt im Meridian stand.
Ansonsten ist das Messergebnis unbrauchbar. Aufgrund ihrer Grösse
am Himmel ist es auch schwierig einen Punkt anzupeilen. Hinzukommt
die Helligkeit der Sonne. Würde man sie direkt ansehen, führt das
eher zu Blindheit als zu einem Messergebnis.
Nichts desto trotz haben die Menschen schon jeher Aufzeichnungen
über die verschiedenen Sonnenstände an der Himmelskugel gemacht.
Auch heute noch werden nautische Karten verwendet, um den genauen
Sonnenstand zu ermitteln oder die Abweichung zu erhalten, welche
von der nicht exakten elliptischen Erdumlaufbahn entsteht.

Sterne eignen sich zur Bestimmung der Position dafür umso besser.
Sie befinden sich immer am selben Ort und bewegen sich nicht. Zudem
kann man mehrere Sterne messen, was zu einem genaueren Ergebnis
führt. Da die Erdumdrehung nicht exakt 24 Stunden dauert, wird
zwischen Sonnen- und Sternzeit unterschieden. Die Sonne braucht von
einem zum nächsten Meridiandurchgang 24 Stunden, genauso lange wie
unser Erdentag lang ist. Die Zeit zwischen zwei Meridiandurchgängen
eines Sterns ist kürzer, daher ist ihr Tag auch kürzer. Der Sterntag
ist somit etwa 4 Minuten kürzer als der Sonnentag. Dies lässt sich
damit begründen, dass sich die Sonne ein wenig langsamer als die
Sterne um die Erde bewegt. Dies lässt sich auf die Bewegung der
Erde um die Sonne abstützen:
\begin{center}
\begin{tabular}{ll}
Sonnenzeit: & 24:00:00 Stunden \\
Sternzeit: & 23:56:04 Stunden
\end{tabular}
\end{center}

Wenn man nun Sterne zur Positionsbestimmung verwenden möchte, muss
man die Sternzeit zuerst in die Sonnenzeit umrechnen, um auf das
richtige Ergebnis zu kommen. Auf dieses Thema gehen wir in dieser
Arbeit aber nicht genauer ein.


\subsection{Breitengrad bekannt? --- Breite nicht bekannt!}
In den vorherigen Berechnungen sind wir immer davon ausgegangen,
dass uns die Breite bekannt ist. Dies ist in Wirklichkeit aber nicht
so, denn auf dem offenen Meer hat man keinerlei Anhaltspunkte, wo
man sein könnte.

Im Abschnitt~\ref{BreitengradM} \nameref{BreitengradM} haben wir
aber gesehen, dass sich der Breitengrad relativ einfach und ohne
grossen Aufwand bestimmen lässt. Damit wir auf See eine verlässliche
Messung erhalten, benötigen wir mindestens zwei Sterne. Mit diesen
beiden Messungen kann man den Messfehler minimieren und den Standort
relativ genau bestimmen.

Da wir zwei Sterne haben, benötigen wir auch zwei Gleichungen mit
zwei Unbekannten
\begin{align*}
\cos(\overline{ZS_1}) &= \cos(\overline{NZ}) \cos(\overline{NS_1}) + \sin(\overline{NZ}) \sin(\overline{NS_1}) \cos(\tau_1) \\
\cos(\overline{ZS_2}) &= \cos(\overline{NZ}) \cos(\overline{NS_2}) + \sin(\overline{NZ}) \sin(\overline{NS_2}) \cos(\tau_2) \\
\\
\Rightarrow \quad \quad
\cos(90^{\circ} - h_1) &= \cos(90^{\circ} - \phi) \cos(90^{\circ} - \delta_1) + \sin(90^{\circ} - \phi)\sin(90^{\circ} - \delta_1) \cos(\tau_1) \\
\Rightarrow \quad \quad
\cos(90^{\circ} - h_2) &= \cos(90^{\circ} - \phi) \cos(90^{\circ} - \delta_2) + \sin(90^{\circ} - \phi)\sin(90^{\circ} - \delta_2) \cos(\tau_2)
\end{align*}

Diese Gleichungen lassen sich nur mit Hilfe eines Computers oder
viel Geduld und Zeit lösen. Die Hilfe des Computers ist heutzutage
kein Problem, jedoch gab es diesen zur Zeit der Entdecker und
Seefahrer nicht. Da kommt die Geduld und die Zeit ins Spiel, denn
diese war auf einem Schiff auf offener See reichlich vorhanden.


\section{Übungsaufgabe}
Um es den alten Seefahrern gleichzutun und seinen eigenen Standort
herauszufinden, können Sie in der folgenden Übungsaufgabe in ihre
Fussstapfen treten und die alten Berechnungsmethoden anwenden.
Mithilfe der Kulmination der Sonne das heisst, der Zeitpunkt wann
die Sonne ihren höchsten Stand am Himmel erreicht kann ein Zusammenhang
zwischen der Zeit und dem Längengrad hergestellt werden. Dies erlaubt
einfache Navigationsaufgaben lösen zu können.
\bigskip

%BILD MIT UHR
\begin{figure}
\centering
\includegraphics[width=0.5\textwidth]{kugel/Uhr.jpg}
\caption{Illustrierter Zusammenhang zwischen Zeit und Längengrad}
\label{Uhr}
\end{figure}

\textbf{1.} Erstellen Sie mit den vorhandenen Angaben das Nautische Dreieck und berechnen sie dessen Seiten und den Zeitwinkel $\tau$.

\begin{center}
\renewcommand{\arraystretch}{1.5}
\begin{tabular}{llll}
\textbf{Messwerte Sextant} & &\textbf{Aufzeichnungen Nautisches Jahrbuch} \\
Elevation der Sonne ($h$) &$26^{\circ}$ $18^{\prime}$ $00.000^{\prime \prime}$ & Deklination der Sonne ($\delta$)  \quad$17^{\circ}$ $16^{\prime}$ $01.200^{\prime \prime}$ \\
Geografische Breite ($\phi$) &$47^{\circ}$ $13^{\prime}$ $24.751^{\prime \prime}$ N
\end{tabular}
\end{center}


\textbf{2.} Ermitteln Sie die Koordinaten für den gesuchten Längengrad
$\lambda$ und finden Sie den gesuchten Ort auf einer Weltkarte.

\begin{center}
\renewcommand{\arraystretch}{1.5}
\begin{tabular}{ll}
\textbf{Aufzeichnungen Nautisches Jahrbuch} \\
Rektaszension der Sonne\footnotemark ($e_\delta$) &$03^{\circ}$ $03^{\prime}$ $14.400^{\prime \prime}$ \\
Kulmination der Sonne &13h 21min 18s \\
Meridiandurchgang der Sonne &11h 56min 29s
\end{tabular}
\end{center}
\footnotetext{%
Winkel zwischen dem Frühlingspunkt und dem Stundenkreis (Längengrad).}

Die Zeitdifferenz zur {\em Coordinated Universal Time (UTC)} beträgt $\Delta{t}=01.00$h. 

\bigskip

\textit{Hinweis.} Beziehen Sie die Zeit für die Ermittlung des
Längengrades mit ein und berücksichtigen Sie die Sommerzeit. 

\bigskip

\textit{Lösung.} \quad a) Der Lösungsansatz verbirgt sich hinter
dem Seitenkosinussatz welchen wir nach $\tau$ umformen
\begin{align*}
\tau &= \arccos 
\frac{ \cos(90^{\circ} - \phi) \cos(90^{\circ} - \delta) - \cos(90^{\circ} - h)} {\sin(90^{\circ} - \phi)\sin(90^{\circ} - \delta)}.
\end{align*}
Die gesuchten Werte können wir mithilfe der Informationen des
Sextanten und des nautischen Jahrbuchs berechnen. Es ergeben sich
die Werte
\begin{align*}
90^{\circ} - \phi &= 42.778^{\circ}
\\
90^{\circ} - \delta &= 72.733^{\circ}
\\
90^{\circ} - h &= 63.700^{\circ}.
\end{align*}
In die Formel eingesetzt erhalten wir
\begin{align*}
\tau = \arccos 
\frac{ \cos(42.778^{\circ}) \cos(72.733^{\circ}) - \cos(63.700^{\circ})} {\sin(42.778^{\circ})\sin(72.733^{\circ})}
\qquad \Rightarrow \qquad
\tau = 110.319^{\circ}.
\end{align*}

b) \quad  Durch den direkten Zusammenhang der Koordinaten und dem
Zeitmass lassen sich Gradmasse direkt in Stunden umrechnen
\begin{align*}
\tau &= 110.319^{\circ} \\
\Rightarrow \quad \quad
\tau &= 7.3546h
\end{align*}
Um den gesuchten Längengrad nun zu erhalten, müssen wir $\tau$ in
seine Komponenten Zeit ($t$), Rektaszension der Sonne ($e_\delta$)
und unseren gesuchten Längengrad ($\lambda$) zerlegen
\begin{align*}
\tau &= t - e_\delta - \lambda 
\end{align*}


Für die Berechnung von $t$ kommt die Uhr von Harrison ins Spiel.
Wir wissen, dass die Sonne in unserem Heimathafen London am Mittag
kulminiert. Das heisst um 12.00 Uhr steht sie im Zenit und ist somit
senkrecht über uns.
Befinden wir uns nicht am selben Standort, kulminiert die Sonne
nicht genau um 12.00 Uhr sondern früher oder später.
In unserem Fall, findet diese 13.355h nach Mitternacht statt. Für
diese Zeitangabe benötigten die Seefahrer die Uhr, um eine exakte
Zeitangabe zu treffen. Kleine Abweichungen in der Zeitmessung,
hätten Kilometerweite Abweichungen in der Standortbestimmung zur
Folge.
Um die Zeitdifferenz zu berechnen, subtrahieren wir die Kulmination
unseres Standortes mit dieser von London unseres Heimathafens und
erhalten
\begin{align*}
13.355 \text{h} - 12.000 \text{h} &= 1.3555 \text{h}.
\end{align*}

Um die Differenz der Zeit und somit die Zeitzonen der Erde noch miteinzubeziehen, müssen wir den Zeitunterschied zwischen den beiden Zeitzonen {\em Coordinated Universal Time (UTC)} und {\em Greenwich Mean Time (GMT)} miteinbeziehen.

Achtung: Es ist Sommer, also sind es zwei Stunden Zeitunterschied, sprich $\Delta{t}=02.00$h
\begin{align*}
1.3555 \text{h} - 2.000 \text{h} &= -0.645 \text{h}.
\end{align*}
Die Rektaszension der Sonne $e_\delta$ lässt sich aus dem Sternatlas herauslesen und ist in Stunden
\begin{align*}
03^{\circ} 03' 14.400'' &= 0.2036 \text{h}.
\end{align*}

Der Meridiandurchgang eines Tages ist nicht immer gleich, sprich
nicht jeder Sonnentag ist gleich lang. Dies muss berücksichtigt
werden, da sich kleinste Abweichungen direkt auf die Genauigkeit
des Längengrades niederschlagen.
Die Zeitdifferenz berechnet sich aus dem vollen Sonnentag von 12.00h
und dem aktuellen Meridiandurchgang
\begin{align*}
\text{12:00:00h} - \text{11:56:29h} = \text{0:03:31h} = 0.0586
\qquad \Rightarrow \qquad
00:03:31h = 0.0586.
\end{align*}
Dieser Betrag dazu addiert ergibt
\begin{align*}
-0.645 h + 0.0586 h = - 0.5863 h = \lambda.
\end{align*}
Umgerechnet in Bogenmass ergibt sich für $\lambda$
\begin{align*}
\lambda = - 8.7945^{\circ} = -8^{\circ} 47' 40.2''.
\end{align*}
(vergleiche Abbildung \ref{Uhr})

Daraus ergeben sich folgende Koordinaten, mit welchen wir unseren
Standort genau ermitteln können
\begin{align*}
8^{\circ}  47'  40.2'' \text{E} \\
47^{\circ}  13'  24.751'' \text{N}
\end{align*}




\printbibliography[heading=subbibliography]
\end{refsection}







