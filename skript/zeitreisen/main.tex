\chapter{Zeitreisen\label{chapter:thema}}
\lhead{Zeitreisen}
\begin{refsection}
\chapterauthor{Sascha Jecklin und Jonas Gründler}

\section{Einleitung}
Das Thema Zeitreisen fasziniert den Menschen schon seit langem. Bereits in der Hinduistischen Mythologie und in der Budhistischen Religion besch\"aftige man sich mit Reisen in der Zeit. Auch in der modernen Literatur und in der Science-fiction sind Zeitreisen ein verbreitetes Thema. Klassische Beispiele daf\"ur w\"aren: 

\begin{itemize}
    \item Time Machine, H.G.Wells, 1895 
    \item Das Ende der Ewigkeit, Isaac Asimov, 1955
    \item Back to the Future, 1985-1990
    \item Star Trek
    \item Doctor Who
    \item \ldots
\end{itemize}

Doch was sind Zeitreisen genau und sind Sie \"uberhaupt m\"oglich? Diese Frage versuchen Wir hier im Rahmen dieser Arbeit zu beantworten.
Grunds\"atzlich gibt zwei Arten von Zeitreisen:

\begin{itemize}
    \item In die Zukunft
    \item In die Vergangenheit
\end{itemize}

Wir werden uns in dieser Arbeit auf Zeitreisen in die Zukunft beschr\"anken. Nach aktuellem Stand der Wissenschaft sind Zeitreisen in die Vergangenheit nicht m\"oglich. Es existieren Theorien doch diese sind umstritten und spekulativ.

\section{Was ist eine Zeitreise}
Was versteht man unter einer Zeitreise? Es bezeichnet eine Bewegung in der Zeit, die vom gew\"ohnlichen Zeitablauf abweicht. In der Physik kann dieser Effekt durch die Zeitdilatation erreicht werden. Also, wenn die Zeit f\"ur einem Langsamer vergeht als f\"ur ein Bezugssystem. Dies kann durch verschiedene Arten erreicht werden. Wir beschr\"anken uns hier auf zwei Arten:

\begin{itemize}
    \item Geschwindigkeit
    \item Gravitation	
\end{itemize}

In den n\"achsten zwei Unterkapiteln untersuchen wir je einen dieser Faktoren

\subsection{Geschwindigkeit}

Eine M\"oglichkeit diesen eine Zeitdillatation zu erreichen ist durch hohe Geschwindigkeit. Wenn sich eine Person relativ zu einem Bezugssystem schnell bewegt, vergeht f\"ur diese Person die Zeit im Vergleich zum Bezugssystem langsamer. Die Ver\"anderung wird durch den Lorentzfaktor beschrieben, welcher sich aus der Lorentztransformation herleiten l\"asst. Er beschreibt das Verh\"altnis zwischen der Eigenzeit und der Zeit des Bezugssystems.

\begin{equation}
    \gamma=\frac{1}{\sqrt{1-\frac{v^2}{c^2}}} 
\end{equation}



Die Eigenzeit ist also:

\begin{equation}
    \tau
    =
    \int_{}^{}\frac{1}{\gamma}dt=\int_{}^{}\sqrt{1-\frac{v^2}{c^2}}dt
    =
    \frac{1}{c}\int_{}^{}\sqrt{g_{\mu\nu}\dot{x}^{\mu}(s)\dot{x}^{\nu}(s)}ds
\end{equation}

Diese Formel ist in dieser Form noch nicht sehr Anschaulich. Sie Beschreibt nur, dass eine Metrik mit den jeweiligen Basisvektoren "multipliziert" werden muss(Einsteinsche Summenkonvention). Von dem Ganzen die Wurzel ziehen und dann noch integrieren.
Durch das Einsetzen der Minkowski-Metrik, welche Raum und Zeit miteinander verbindet, und der Basisvektoren $t, x, y, z$ l\"asst sich eine Verst\"andliche Form herleiten. 

Minowski-Metrik:

\begin{equation}
    g_{\mu\nu}=
    \begin{pmatrix}
        -1 & 0 & 0 & 0 \\
        0 & 1 & 0 & 0 \\
        0 & 0 & 1 & 0 \\
        0 & 0 & 0 & 1
    \end{pmatrix}
\end{equation}

Standard Vierervektor:

\begin{list}{}{}
    \item \(x^{0}=ct\)
    \item \(x^{1}=x\)
    \item \(x^{2}=y\)
    \item \(x^{3}=z\)
\end{list}

Ein wenig Umstellen und vereinfachen und man kommt auf diese Form:

\begin{equation}
    \tau
    =
    \frac{1}{c}\int_{}^{}\sqrt{-(-c^2\dot{t}(s)^{2}+\dot{x}(s)^{2}+\dot{y}(s)^{2}+\dot{z}(s)^{2})}ds
\end{equation}

Je nachdem wie die Bewegung gew\"ahlt wird, fallen einer oder mehrere der Basisvektoren  $x, y, z$ weg.\\\\\\\\ Hier ein Beispiel bei welchem nur eine Geschwindigkeit in x-Richtung vorhanden ist(c=Lichtgeschwindigkeit, u beschreibt den Bruchteil):

\begin{list}{}{}
    \item $t(s)=1s, \dot{t}(s)=1$
    \item $x(s)=u*c*s, \dot{x}(s)=u*c$
    \item $y(s)=0, \dot{y}(s)=0$
    \item $z(s)=0, \dot{z}(s)=0$
\end{list}

\begin{align*}
    \tau
    &=
    \frac{1}{c}\int_{}^{}\sqrt{-(-c^2\dot{t}(s)^2+\dot{x}(s)^2)}ds 
    =
    \frac{1}{c}\int_{}^{}\sqrt{-(-c^2*1+(u*c)^{2}}ds\\
    &=
    \frac{s*\sqrt{c^2+(u*c)^{2}}}{c} 
    =
    s*\sqrt{1-\frac{u^2*c^2}{c^2}}
\end{align*}

Welches die einfachste Form einer Zeitdilatation darstellt.\\
\\
Hier ein Zahlenbeispiel bei welchem zuf\"allige Werte gew\"ahlt wurden:
$s=5000, u=0.2$ 

\begin{list}{}{}
    \item $t(s)=1s, \dot{t}(s)=1$
    \item $x(s)=0.2cs, \dot{x}(s)=0.2c$
    \item $y(s)=0, \dot{y}(s)=0$
    \item $z(s)=0, \dot{z}(s)=0$
\end{list}

\begin{align*}
    \tau
    &=
    \frac{1}{c}\int_{s_{a}}^{s_{b}}\sqrt{-(-c^2\dot{t}(s)^2+\dot{x}(s)^2)}ds
    &=
    \frac{1}{c}\int_{0}^{5000}\sqrt{-(-c^2+((0.2c)^2))}ds\\
    &=
    5000*\sqrt{1-\frac{u^2*c^2}{c^2}} = 4898.98
\end{align*}

Dieses Beispiel zeigt auch, dass eine relevante Zeitverlangsamung erst bei sehr hohen Geschwindigkeiten erreicht wird

\begin{figure}
    \centering
    \includegraphics[width=\hsize]{zeitreisen/Lorentzfaktor.jpg}
    \caption{Ver\"anderung des Lorentzfaktor in Abh\"angigkeit der Geschwindigkeit%
        \label{skript:geodaten:fig:transport}}
\end{figure}

\subsection{Gravitation}

	In diesem Kapitel beschäftigen wir uns mit dem Einfluss der Gravitation auf die Zeit. Gravitation muss durch eine Krümmung des Raumes beschrieben werden. Erkennen lässt sich die Gravitation z.B. an der Anziehungskraft der Erde. Alles wird mit $g=9.81\frac{m}{s^2}$ in Richtung Erdmittelpunkt beschleunigt. $F=\frac{KMm}{r^2}$
	Die Beschleunigung ist also unabhängig von der Masse.
	Ein beschleunigtes Koordinatensystem kann zwar durch eine Koordinatentransformation erreicht werden, doch diese ist keine Lorentztransformation und lässt sich nicht durch die die Minkowski-metrik darstellen.\\ Der Ansatz $ -c^2dt^2 + dx^2 + dy^2 + dz^2$ genügt nicht mehr. Wir müssen also Metriken und Transformationen zulassen, welche von der Minkowski-Metrik und der Standarttransformation abweichen.\\
	Eine Metrik, welche wir für unsere Zwecke benützen können, ist die Schwarzschild-Metrik. Sie berücksichtigt die Gravitation, durch eine Krümmung des Raumes dargestellt, und beschreibt so Gravitationsfelder in der nähe von massereichen Objekten. Diese Lösung wurde von Karl Schwarzschild nur wenige Monate nach der Präsentierung der Einsteinschen Feldgleichungen gefunden. Aus ihr lassen sich Bewegungsgleichungen von Körpern in einem Gravitationstrichter herleiten.\\\\
	Die Schwarzschildmetrik in Vektordarstellung:



	\begin{equation}
		g_{\mu\nu}=
		\begin{pmatrix}
		-\biggl(-1-\frac{r_{g}}{r}\biggr) & 0 & 0 & 0 \\
		0 & \frac{1}{\displaystyle1-\frac{r_{g}}{r}} & 0 & 0 \\
		0 & 0 & r^{2} & 0 \\
		0 & 0 & 0 & r^{2}\sin^{2}(\vartheta)
		\end{pmatrix}
	\end{equation}

	Der Vierervektor in Kugelkoordinaten:

	\begin{list}{}{}
		\item \(x^{0}=ct\)
		\item \(x^{1}=r\)
		\item \(x^{2}=\vartheta\)
		\item \(x^{3}=\varphi\)
	\end{list}

	Daraus lässt sich die Längenmessung zusammenstellen. Auf die Herleitung der Schwarzsdchild-Metrik verzichten wir an dieser Stelle.

	\begin{equation}
	ds^2
	=
	-\biggl(1-\frac{r_g}r\biggr)c^2dt^2
	+
	\frac{1}{\displaystyle 1-\frac{r_g}r}\,dr^2 
	+
	r^2d\vartheta^2 
	+ 
	r^2\sin^2(\vartheta)d\varphi
	\end{equation}

	$r_g$ beschreibt den Gravitationsradius(Ereignishorizont) des betrachteten Körpers.

	\subsubsection{Bedeutung von $R_{g}$ und Ereignishorizonte}


	Der Ereignishorizont beschreibt in der allgemeinen Relativitätstheorie eine Grenzfläche dar. Er beschreibt die Entfernung ab welcher das Licht nicht mehr aus der Gravitationstrichter entkommen kann. Teilchen die den Ereignishorizont passiert haben, können diesen nicht mehr verlassen. Alles ausserhalb dieses Radius hätte die Möglichkeit zu entkommen, die frage ist nur wie viel Energie benötigt wird. Da Licht nicht mehr entkommen kann nennt man diese Körper auch schwarze Löcher.\\
	Der Gravitationsradius lässt sich mit 

	\begin{equation}
	r_{g}= \frac{2MG}{c^2}
	\end{equation}

	berechnen.
	Der Gravitationsradius der Erde beträgt etwa 8.8mm. Wenn wir die Erde also unter  diese 8.8mm komprimieren, würde sie sich in ein schwarzes Loch verwandeln. Ihre Dichte wäre dann so gross, dass nichts mehr der Anziehung widerstehen kann.




	\printbibliography[heading=subbibliography]
	\end{refsection}

