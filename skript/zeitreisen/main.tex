\chapter{Zeitreisen\label{chapter:thema}}
\lhead{Zeitreisen}
\begin{refsection}
\chapterauthor{Sascha Jecklin und Jonas Gründler}
\section{Einleitung}
Das Thema Zeitreisen fasziniert den Menschen schon seit er sich der Zeit bewusst ist. Früh schon entstanden Träume, den Verlauf der Zeit manipulieren zu können. Erste schriftliche und bildliche Belege dafür gab es bereits in der hinduistischen Mythologie und in der buddhistischen Religion. Auch in der modernen Literatur und in der Filmindustrie sind Zeitreisen ein beliebtes Thema. Einige Beispiele daf\"ur sind: 
\begin{itemize}
    \item Time Machine, H.G.Wells, 1895 
    \item Das Ende der Ewigkeit, Isaac Asimov, 1955
    \item Back to the Future, 1985-1990
    \item Star Trek
    \item Doctor Who
    \item Interstellar

\end{itemize}
In diesem Kapitel beschäftigen wir uns mit der mathematischen Beschreibung von Zeitreisen, wie man solche Effekte erreichen kann und ob sie mit heutigen Mitteln erreichbar sind. Zu beginn soll gezeigt werden, wie man eine Zeitreise mit Hilfe der Eigenzeit beschreibt. Danach stellt sich die Frage, wie der Verlauf solcher Eigenzeiten beeinflusst werden kann. Es zeigt sich das dies mit Gravitation und Geschwindigkeit möglich ist. Ein besonderes Augenmerk gelten dabei schwarzen Löchern. Nach dem wir die nötige Mathematik eingeführt haben veranschaulichen wir das ganze mit einer Simulation. Natürlich wollen wir uns auch Gedanken über die Realisierbarkeit machen. Wie sich zeigen wird sind Zeitreisen keine reine Fiktion. Allerdings sind sie auch nicht in diesem Ausmass möglich wie man sich vielleicht erhofft.
\section{Was ist eine Zeitreise?}
Um diese Frage besser beantworten zu können starten wir mit einem Gedankenexperiment. Wir stellen uns vor zwei Autos fahren mit der gleichen Geschwindigkeit an einander vorbei. Fahren die beiden Autos z.B. 100km/h schnell ist die Geschwindigkeit mit der sie sich passieren gerade doppelt so hoch, also 200 km/h. In der klassischen Physik wäre das eine korrektes Ergebnis. In einem zweiten Anlauf sollen beide Fahrer Lichtgeschwindigkeit fahren. Spätestens jetzt macht sich der Fehler bemerkbar. Den die Geschwindigkeit in der sich die beiden Fahrzeuge kreuzen ist nicht etwas Doppelte Lichtgeschwindigkeit, sondern einfache Lichtgeschwindigkeit. Den wir wissen das dies die grösste Geschwindigkeit im Universum ist.
Beide Fahrer behaupten nun aber vehement ihre eigene Geschwindigkeitsmessung sei korrekt gewesen. Folglich müsste es Unterschiede in der Längen und Zeitmessung geben.
Wie wir in Kapitel \ref{chap:lichtkegel} festgehalten haben ist dies auch der Fall. Jeder Beobachter misst die gleiche Lichtgeschwindigkeit. Längenmessungen und Zeitmessungen müssen in einem solchen System angepasst werden. Genau dies tun wir mit der Lorentztransformation von Kapitel \ref{chap:lorenz}. Folglich gibt es nicht mehr eine allgemeine Zeitmessung für alle Objekte. Jedes Objekt hat seine eigene zeit, die Eigenzeit.
Integrieren wir entlang einer Weltenlinie erhalten wir für die Eigenzeit
\begin{equation}\label{Eigenzeit}
\tau
=
\int_{}^{}\frac{1}{\gamma}dt=\int_{}^{}\sqrt{1-\frac{v^2}{c^2}}dt
=
\frac{1}{c}\int_{}^{}\sqrt{g_{\mu\nu}\dot{x}^{\mu}(s)\dot{x}^{\nu}(s)}ds
\end{equation}
Da wir nun für jedes Objekt eine Eigenzeit festhalten, können wir auch Unterschiede dieser Zeiten angeben.
Eine Zeitreise beschreibt daher eine Bewegung durch die Zeit, abweichend von der eines Bezugssystems. 

Nicht nur hohe Geschwindigkeiten bringen Änderungen in der Zeitmessung. Wie wir sehen werden verursachen auch andere Effekte Dehnungen der Zeit. Man spricht dabei auch von Zeitdilatation.
Streng genommen bringt schon jede Geschwindigkeit eine Abweichung der Eigenzeit. Da die praktische Anwendung in diesem Kapitel im Vordergrund steht, interessieren uns in erster Linie natürlich signifikante Unterschiede. Erst doppelt oder sogar zehnmal schneller vergehenden Eigenzeiten w\"urden uns Effekte best\"atigen, welche man uns in Science Fiction Filmen und Romanen verspricht.

Es stellt sich noch die Frage ob das nun Reisen in die Zukunft oder in die Vergangenheit sind. Generell sind Reisen in die Vergangheit sehr komplex. Dort treten nämlich Kausalitätsprobleme auf. Was z.B geschieht wenn ein Zeitreisender seine Entstehung verhindern würde?
Mit unterschiedlichen schnell verlaufenden Eigenzeiten beschreibt man salopp ausgedrückt Reisen in die Zukunft oder ``weniger schnelles altern``. Ein Beobachter  A mit halb so grosser Eigenzeit wie eine Referenz B wäre quasi in die Zukunft gereist. Während für A ganz normal ein Jahr verstrich wäre für B zwei Jahre vergangen. Umgekehrt ist aus Sicht von B A weniger schnell gealtert. A ist damit aber nicht in die Vergangenheit gereist und kann trotz Zeit\"anderung immer nur älter sein als zu Beginn der Betrachtung
\section{Einfluss von Geschwindigkeit und Gravitation}
Da wir nun Wissen wie wir eine Zeitreise beschreiben können, müssen wir uns nun Gedanken machen wie wir Zeitdilation erreichen können. Im wesentlichen gelingt dies durch Geschwindigkeiten oder Gravitation. 
\subsection{Geschwindigkeit}
Eine M\"oglichkeit eine Zeitdillatation zu bewirken ist durch hohe Geschwindigkeit. Wenn sich eine Person relativ zu einem Bezugssystem schnell bewegt, vergeht f\"ur diese Person die Zeit im Vergleich zum Bezugssystem langsamer. Die Ver\"anderung wird durch den Lorentzfaktor beschrieben, welcher sich aus der Lorentztransformation herleiten l\"asst. Er beschreibt das Verh\"altnis zwischen der Eigenzeit und der Zeit des Bezugssystems.
\begin{equation}
    \gamma=\frac{1}{\sqrt{1-\displaystyle\frac{v^2}{c^2}}} 
\end{equation}
Das ist genau das $\gamma$ welches in der Eigenzeit Berechnung vorkommt.

Die Formel \ref{Eigenzeit} ist in dieser Form noch nicht sehr anschaulich. Sie beschreibt nur, dass eine Metrik mit den jeweiligen Basisvektoren "multipliziert" werden muss (Einsteinsche Summenkonvention). Von dem Ganzen die Wurzel ziehen und dann noch integrieren.
Durch das verwenden der Minkowski-Metrik, welche Raum und Zeit miteinander verbindet, und der Basisvektoren $t, x, y, z$ l\"asst sich eine verst\"andliche Form herleiten. 
Minowski-Metrik:
\begin{equation}
    g_{\mu\nu}=
    \begin{pmatrix}
        -1 & 0 & 0 & 0 \\
        0 & 1 & 0 & 0 \\
        0 & 0 & 1 & 0 \\
        0 & 0 & 0 & 1
    \end{pmatrix}
\end{equation}
Standard Vierervektor:
\begin{align*}
    x^{0}=c \cdot t,
    x^{1}=x,
    x^{2}=y,
    x^{3}=z,
\end{align*}
Ein wenig Umstellen und vereinfachen und man kommt auf diese Form:
\begin{equation}
    \tau
    =
    \frac{1}{c}\int_{}^{}\sqrt{-(-c^2\dot{t}(s)^{2}+\dot{x}(s)^{2}+\dot{y}(s)^{2}+\dot{z}(s)^{2})}ds
\end{equation}
Je nachdem wie die Bewegung gew\"ahlt wird, fallen einer oder mehrere der Koordinaten $x, y, z$ weg.
Hier ein Beispiel bei welchem nur eine Geschwindigkeit in x-Richtung vorhanden ist ($c=$Lichtgeschwindigkeit, u beschreibt den Bruchteil):
\begin{align*}
     t(s)=1\cdot s, \dot{t}(s)=1,
 	 x(s)=u\cdot c \cdot s, \dot{x}(s)=u\cdot c,
     y(s)=0, \dot{y}(s)=0,
     z(s)=0, \dot{z}(s)=0
\end{align*}
\begin{align*}
    \tau
    &=
    \frac{1}{c}\int_{}^{}\sqrt{-(-c^2\dot{t}(s)^2+\dot{x}(s)^2)}ds 
    =
    \frac{1}{c}\int_{}^{}\sqrt{-(-c^2 +(u\cdot c)^{2}}ds\\
    &=
    \frac{s\sqrt{c^2+(u\cdot c)^{2}}}{c} 
    =
    s\sqrt{1-\frac{u^2\cdot c^2}{c^2}}
\end{align*}
Welches die einfachste Form einer Zeitdilatation darstellt.
Hier ein Zahlenbeispiel bei welchem willkürliche Werte gew\"ahlt wurden:
$s=5000, u=0.2$ 
\begin{align*}
    t(s)=s, \dot{t}(s)=1,
    x(s)=0.2c \cdot s, \dot{x}(s)=0.2c,
    y(s)=0, \dot{y}(s)=0,
    z(s)=0, \dot{z}(s)=0,
\end{align*}
Diese Werte in die Gleichung eingesetzt ergeben
\begin{align*}
    \tau
    &=
    \frac{1}{c}\int_{s_{a}}^{s_{b}}\sqrt{-(-c^2\dot{t}(s)^2+\dot{x}(s)^2)}ds
    &=
    \frac{1}{c}\int_{0}^{5000}\sqrt{-(-c^2+((0.2c)^2))}ds\\
    &=
    5000\sqrt{1-\frac{u^2 c^2}{c^2}} = 4898.98
\end{align*}
Dieses Beispiel zeigt auch, dass eine relevante Zeitverlangsamung erst bei sehr hohen Geschwindigkeiten erreicht wird.
\begin{figure}[H]
    \centering
    \includegraphics[width=\hsize]{zeitreisen/Lorentzfaktor.jpg}
    \caption{Ver\"anderung des Lorentzfaktor in Abh\"angigkeit der Geschwindigkeit%
        \label{skript:geodaten:fig:transport}}
\end{figure}
\subsection{Gravitation}

	In diesem Kapitel beschäftigen wir uns mit dem Einfluss der Gravitation auf die Zeit. Einstein hat erkannt, dass die Gravitation durch eine Krümmung des Raumes beschrieben werden muss. Erkennen lässt sich die Gravitation z.B. an der Erdbeschleunigung. Alles wird mit $g=9.81\frac{m}{s^2}$ in Richtung Erdmittelpunkt beschleunigt. $F=\frac{KMm}{r^2}$. F lässt sich jedoch nach dem zweiten Newtonschen Gesetz auch als $F=m\cdot a$ schreiben. Eingesetzt und gekürzt zeigt sich, das die Beschleunigung nicht von der Masse des betrachteten Objektes abhängt.
	\begin{align*}
		m\cdot a = \frac{KMm}{r^2} \rightarrow a=\frac{KM}{r^2} 
	\end{align*}
	
	dieser Effekt lässt sich auch im Alltag betrachten, grosse fallende Objekte werden gleich Richtung Erdmittelpunkt beschleunigt wie Kleine.
	
	\begin{equation}
	\left.
	\begin{aligned}
	t'&=t\\
	x'&=x+\frac12gt^2
	\end{aligned}
	\right\}
	\qquad
	\Leftrightarrow
	\qquad
	\left\{
	\begin{aligned}
	t&=t'\\
	x&=x'-\frac12gt'^2
	\end{aligned}
	\right.
	\eqref{skript:gravitation:beschleunigt}
	\end{equation}
<<<<<<< Updated upstream
	Die Gleichungen oben sind zwar eine Koordinatentransformation, doch sie sind keine Lorentztransformation und enthält somit die Minkowski-Metrik nicht. Mit ihr lässt sich der Einfluss der Gravitation auf die Zeit nicht berechnen. 
	Der Ansatz $ -c^2dt^2 + dx^2 + dy^2 + dz^2$ genügt also nicht mehr. Wir müssen also Metriken und Transformationen zulassen, welche von der Minkowski-Metrik und der Standarttransformation abweichen.
	Doch welche Metrik beinhaltet den Einfluss der Gravitation auf die Zeit? 
	Die Lösung unseres Problems ist die Schwarzschild-Metrik, mit ihr lässt sich die Zeitveränderung in der Umgebung von Massereichen Objekten berechnen. Wir werden sie jedoch erst später vorstellen, wenn wir versuchen eine Zeitreise tatsächlich durchzuführen. %be patient;)
	
	
	
	
=======
	$r_g$ beschreibt den Gravitationsradius(Ereignishorizont) des betrachteten Körpers.
	\subsubsection{Bedeutung von $R_{g}$ und Ereignishorizonte}
	Der Ereignishorizont beschreibt in der allgemeinen Relativitätstheorie eine Grenzfläche dar. Er beschreibt die Entfernung ab welcher das Licht nicht mehr aus der Gravitationstrichter entkommen kann. Teilchen die den Ereignishorizont passiert haben, können diesen nicht mehr verlassen. Alles ausserhalb dieses Radius hätte die M\"oglichkeit zu entkommen, die frage ist nur wie viel Energie benötigt wird. Da Licht nicht mehr entkommen kann nennt man diese Körper auch schwarze Löcher.
	Der Gravitationsradius lässt sich mit \ref{Gravitationsradius} berechnen.
	\begin{equation} \label{Gravitationsradius}
		r_{g}= \frac{2MG}{c^2}
	\end{equation}
    Der Gravitationsradius der Erde beträgt etwa 8.8mm. Wenn wir die Erde also unter diese 8.8mm komprimieren, würde sie sich in ein schwarzes Loch verwandeln. Ihre Dichte wäre dann so gross, dass nichts mehr der Anziehung widerstehen kann.
	%do vlt no chli was... weiss aber nonig genau was
>>>>>>> Stashed changes
	
	\section{Realisierbarkeit}
	
	Wie erreichen wir nun eine grössere Zeitdilatation um in der Zeit zu reisen? Entweder mit grosser Geschwindigkeit oder mittels eines Massereichen Objektes. Zeitdilatation durch hohe Geschwindigkeiten zu erreichen ist sehr schwierig, da die benötigte Energie im Quadrat mit der Geschwindigkeit anwächst. Und wie (Figure 1) zeigt, muss für eine grosse Zeitverlangsamung annähernd Lichtgeschwindigkeit erreicht werden. Erst ab ca. $0.9c$ beginnt der Lorentzfaktor wirklich zu wachsen.
	Das bedeutet, dass wir uns auf den Effekt der Gravitation konzentrieren müssen.
	Doch wir lösen wir dieses Problem? Wir suchen also ein Massereiches Objekt in unserer Umgebung. Die Sonne genügt für unsere Zwecke leider nicht, ihr Schwarzschildradius ist erstens viel zu klein, und zweitens innerhalb des Körpers. Wir suchen also weiter\dots Schliesslich finden wir Sagittarrius A*, ein supermassives schwarzes Loch im Zentrum unserer Milchstrasse. Dieses sollte für unsere Zwecke genügen. 
	Doch hier beginnen die Probleme. 
	Sagittarrius A* ist 26'000 Lichtjahre von uns entfernt, das bedeutet mit Lichtgeschwindigkeit bräuchten Wir 26'000 Jahre bis wir am Ziel ankommen. Die Hoffnung wäre nun, dass sich durch eine hohe Fluggeschwindigkeit und die daraus resultierende Zeitdehnung, die benötigte Zeit signifikant reduziert. 
	Nehmen wir nun an wir fliegen mit 0.5-facher Lichtgeschwindigkeit ins Zentrum unserer Milchstrasse. Die Eigenflugzeit im Raumschiff wäre immer noch
	\begin{align*}
	\tau
	&= 
	\int_{}^{26000}\frac{1}{\gamma}dt=\int_{}^{26000}\sqrt{1-\frac{v^2}{c^2}}dt
	= 
	\frac{1}{c}\int_{0}^{26000}\sqrt{-(-c^2+(0.5c)^2)}ds\\
	&=
	\biggl[s\sqrt{1-\frac{(0.5c)^{2}}{c^2}}\biggr]_0^{26000}
	=
	22516.7.
	\end{align*}
	was also grosszügig gerundet immer noch 22500 Jahre dauern würde. 
	Wir müssen Mit unserem Projekt also warten, bis Geschwindigkeiten von 0.99c mit Raumschiffen Erreicht werden können. Doch auch mit 0.99c würde die Reise noch 3600 Jahre Dauern. Die Energie die benötigt würde, um ein Raumschiff auf diese Geschwindigkeit zu beschleunigen, vernachlässigen wir hier. Das ist ein anderes Problem welches noch gelöst werden muss.
	Wir verschieben unser Projekt also in hypothetische und stellen uns vor wir wären bereits da.
	
	\section{Am Ziel\dots Was nun?}
	
	Wenn wir nun also diese Reise auf uns genommen haben, und Sagittarrius A* wirklich erreichen, eröffnen sich uns zwei Möglichkeiten:
	\begin{itemize}
		\item eine gesteuerte Bahn
		\item wir lassen uns "fallen"
	\end{itemize}
	Eine gesteuerte Kreisbahn um das schwarze Loch bringt die besten Resultate, da wir eine Kreisbahn sehr nahe am Ereignishorizont wählen können. Das Ganze wäre jedoch sehr Energieaufwändig, da wir mit Steuerdüsen unseren Absturz in den Ereignishorizont verhindern müssen. Je nach Geschwindigkeit und Abstand zum schwarzen Loch wird mehr oder weniger Energie benötigt.
	
	Die zweite Möglichkeit wäre, dass wir uns, wenn wir angekommen sind "fallenlassen" und sehen was passiert. So können wir je nach Anfangsgeschwindigkeit, Abstand und Position verschiedene Bahnen erreicht werden. Vom Absturz in den Ereignishorizont bis zu einer Bahn auf der wir der Gravitation des schwarzen Lochs entkommen und uns in die weiten des Weltalls verabschieden. Beides sind eher unangenehme Szenarien. 
	Wie finden wir also eine Bahn möglichst nahe an der schwarzen Loch, ohne das eines dieser zwei Ereignisse auftritt?
	Lösung unseres Problem sind die Geodätengleichungen, welche uns aufgrund der Anfangsbedingungen eine Bahn berechnen lassen. Aus der Lösung dieser Gleichungen lässt sich dann auch die Eigenzeit unsd somit die Effizienz unserer Zeitreise berechnen.
	
	\section{Umsetzung mittels Geodätenbahnen}
	
	Mit den Geodäten lassen sich nun physikalisch korrekte Bahnen um ein schwarzes Loch herleiten. Dafür benötigen wir die in Kapitel 3.4 vorgestellte Geodätengleichung:	
	\begin{equation}
	\ddot{x}^{\alpha} + \Gamma^{\alpha}_{\mu\nu}\dot{x}^{\mu}\dot{x}^{\nu} = 0
	\end{equation}	
	Wir können nun in der nähe des schwarzen Lochs alle Antriebe ausschalten und uns treiben lassen. Die aktuelle Geschwindigkeit, Position, Richtung und Abstand stellen nun die Anfangsbedingungen dar. Daran dass die Gleichung $=0$ ist können wir erkennen, dass kein weiterer Energieinput benötigt wird. Die Gleichungen berücksichtigen also auch die Energieerhaltung, und so also wie sich das Raumschiff in Zukunft mit abgeschalteten Antrieben bewegen wird. Mit einer numerischen Lösung dieses Differentialgleichungssystems können wir nun Schritt für Schritt die neuen Faktoren ausrechnen. Wir werden im nächsten Kapitel sehen, dass dann die Zeit in der Simulation langsamer vergeht und wir so daraus die Zeitdilatation extrahieren können.
	Wir k\"onnen nun die Gleichung wie folgt umstellen:	
	\begin{equation}
	\ddot{x}^{\alpha} = -\Gamma^{\alpha}_{\mu\nu}\dot{x}^{\mu}\dot{x}^{\nu}
	\end{equation}
	Daraus lässt sich interpretieren, dass die Beschleunigungen, welche für die nächste Position wichtig sind, ein Produkt aus den Christoffelsymbole 2.Art und den zugehörigen Ableitungen sind. Die Christoffelsymbole sind die $\Gamma^{\alpha}_{\mu\nu}$. Sie beschreiben also, wie sich die im Raumschiff enthaltene Energie transformiert, und sich so die Beschleunigungen in die jeweiligen Richtungen, die Geschwindigkeiten und die Position verändern. Zusätzlich beschreiben sie auch welchen Einfluss das Gravitationsfeld des schwarzen Lochs auf unser Raumschiff hat.
	Doch woher kommen diese Symbole, und wie lassen sie sich berechnen?
	Sie lassen sich aus der jeweiligen gewählten Metrik berechnen. Wie bereits Abschnitt über Gravitation angetönt (:::label:::), benötigen Wir für unsere Zwecke die Schwarzschild-Metrik, da sie den Einfluss der Gravitation auf unser Raumschiff berücksichtigt. Die genaue Erklärung der Christoffelsymbole und der Schwarzschildmetrik folgen in den nächsten zwei Unterkapiteln.
	
	Nun können wir die Gleichung von oben in die vier Komponenten aufspalten.
	Sie ist entsprechend des gewählten Koordinatensystems in dieser Form:
	\begin{align*}
	\ddot{t}(s) = -\Gamma^{1}_{\mu\nu}\dot{x}^{\mu}\dot{x}^{\nu}\\
	\ddot{r}(s) = -\Gamma^{2}_{\mu\nu}\dot{x}^{\mu}\dot{x}^{\nu}\\
	\ddot{\vartheta}(s) = -\Gamma^{3}_{\mu\nu}\dot{x}^{\mu}\dot{x}^{\nu}\\
	\ddot{\varphi}(s) = -\Gamma^{4}_{\mu\nu}\dot{x}^{\mu}\dot{x}^{\nu}		
	\end{align*}
	
	Da wir uns in einer Ebene bewegen ist der Winkel$\vartheta = const$ und Somit sind seine Ableitungen $=0$ Die dritte Gleichung reduziert sich also auf $\ddot{\vartheta}(s)=0$ und fällt somit weg. Das Problem reduziert sich also auf 3 Dimensionen.
	
	\begin{align*}
	\ddot{t}(s) = -\Gamma^{1}_{\mu\nu}\dot{x}^{\mu}\dot{x}^{\nu}\\
	\ddot{r}(s) = -\Gamma^{2}_{\mu\nu}\dot{x}^{\mu}\dot{x}^{\nu}\\
	\ddot{\varphi}(s) = -\Gamma^{4}_{\mu\nu}\dot{x}^{\mu}\dot{x}^{\nu}		
	\end{align*}
	
	Uns sind nun alle Elemente bekannt. Wenn wir alles zusammensetzen bekommen wir Folgende Gleichungen:
	
	\begin{align*}
	\ddot t(s)
	&=
	-\frac{1}{1-\displaystyle\frac{r_g}{r}}\frac{r_g}{r}\frac{1}{r}\dot t(s)\,\dot r(s)
	\\
	\ddot r(s)
	&=
	-\biggl(1-\frac{r_g}{r}\biggr)\frac{r_g}{r}\frac1{2r}\dot t(s)^2
	+\frac{1}{1-\displaystyle\frac{r_g}{r}} \frac{r_g}{r}\frac1{2r}\dot r(s)^2
	- (r_g-r) \dot\varphi(s)^2
	\\
	\ddot \vartheta(s)
	&=
	0
	\\
	\ddot \varphi(s)
	&=
	-\frac2r \dot r(s)\,\dot\varphi(s)
	\end{align*}
	
	Dieses Differentialgleichungssystem werden wir im nächsten Kapitel numerisch mittels einer Simulation lösen und so endlich herausfinden wie gut unsere Zeitreise wirklich ist. 
	
	\subsection{Schwarzschild-Metrik}
	
	Die Metrik welche von Karl Schwarzschild gefunden wurde, ist ein Lösung der Einsteinschen Feldgleichungen und wurde nur wenige Monate nach deren Vorstellung gefunden.
	Die Schwarzschildmetrik in Vektordarstellung:
	
	\begin{equation}
	g_{\mu\nu}=
	\begin{pmatrix}
	-\biggl(-1-\frac{r_{g}}{r}\biggr) & 0 & 0 & 0 \\
	0 & \frac{1}{\displaystyle1-\frac{r_{g}}{r}} & 0 & 0 \\
	0 & 0 & r^{2} & 0 \\
	0 & 0 & 0 & r^{2}\sin^{2}(\vartheta)
	\end{pmatrix}
	\end{equation}
	Der dazugehörige Vierervektor, welcher das Koordinatensystem beschreibt:
	\begin{align*}
	x^{0}=c\cdot t
	x^{1}=r
	x^{2}=\vartheta
	x^{3}=\varphi
	\end{align*}
<<<<<<< Updated upstream
	Daraus folgt also folgende Metrik in Zeilenform, welche auch gleich die Längenmessung darstellt.
	
	\begin{equation}
	ds^2
	=
	-\biggl(1-\frac{r_g}r\biggr)c^2dt^2
	+
	\frac{1}{\displaystyle 1-\frac{r_g}r}\,dr^2 
	+
	r^2d\vartheta^2 
	+ 
	r^2\sin^2(\vartheta)d\varphi
	\end{equation}
	$r_g$ beschreibt den Gravitationsradius(Ereignishorizont) des betrachteten Körpers.
	
	Sie berücksichtigt auch die Krümmung des Raumes und ist so für unsere Zwecke bestens geeignet. Sie Beschreibt den Gravitationstrichter um ein ungeladenes, nicht rotierendes schwarzes Lochs, und ist somit die einfachste Lösung. Es gibt noch weiter Metriken, welche diese Faktoren auch berücksichtigen, dies würde jedoch den Rahmen dieser Arbeit sprengen.
	Die Herleitung sparen wir uns hier, dies wurde bereits im Kapitel (:::LABEL:::) erledigt. 

	\subsubsection{Bedeutung von $R_{g}$ und der Ereignishorizont}
	Der Ereignishorizont stellt in der allgemeinen Relativitätstheorie eine Grenzfläche dar. Er beschreibt die Entfernung ab welcher das Licht nicht mehr aus der Gravitationstrichter entkommen kann. Teilchen die den Ereignishorizont passiert haben, können diesen nicht mehr verlassen. Alles Teilchen oder Objekte die den Ereignishorizont passieren werden unweigerlich ins Zentrum stürzen. Alles ausserhalb dieses Radius hätte die Möglichkeit zu entkommen, die Frage ist nur wie viel Energie benötigt wird. Da sogar das Licht nicht mehr entkommen nennt man diese Objekte auch schwarze Löcher oder Singularitäten.
	Der Gravitationsradius lässt sich wie folgt berechnen.
	\begin{equation} \label{Gravitationsradius}
	r_{g}= \frac{2MG}{c^2}
	\end{equation}
	Der Gravitationsradius der Erde beträgt etwa 8.8mm. Wenn wir die Erde also unter diese 8.8mm komprimieren, wäre sie kleiner als der Gravitationsradius und wäre so nicht mehr sichtbar.
	
	\subsection{Christoffelsymbole}
	
	Die Berechnung der Christoffelsymbole aus der Schwarzschild-Metrik ist kompliziert und sehr Rechenaufwändig. Wir beschränken uns hier auf die Berechnung, die Herleitung wird in Kapitel (:::LABEL:::) ausführlich beschrieben. Es gibt Christoffelsymbole erster und zweiter Art, die Christoffelsymbole 2.Art lassen sich aus denen 1.Art berechnen.
=======
	Da wir uns in einer Ebene bewegen ist der Winkel$\vartheta = const$ und somit auch seine Ableitungen. Dadurch fällt der ganze $\ddot{\vartheta}$ Term weg. Das Problem reduziert sich also auf drei Dimensionen.	
>>>>>>> Stashed changes
	
	$g_{\mu\nu}$ ist hier die Schwarzschild-Metrik in Vektorform(:::LABEL:::).
	
	Die Christoffelsymbole 1.Art:
	
	\begin{align*}
		\Gamma_{\alpha,\mu\nu} 
		= 
		\frac{1}{2}\biggl(\frac{\partial g_{\nu\alpha}}{\partial x^{\mu}} 
		+
		\frac{\partial g_{\alpha\mu}}{\partial x^{\nu}}
		+
		\frac{\partial g_{\mu\nu}}{\partial x^{\alpha}}
		 \biggr)
	\end{align*}
	
	Die Christoffelsymbole 2.Art:
	
	\begin{align*}
	\Gamma^{\sigma}_{\mu\nu} 
	= 
	g^{\sigma\alpha}\Gamma_{\alpha,\mu\nu} 
	=
	\frac{1}{2}g^{\sigma\alpha}\biggl(\frac{\partial g_{\nu\alpha}}{\partial x^{\mu}} 
	+
	\frac{\partial g_{\alpha\mu}}{\partial x^{\nu}}
	+
	\frac{\partial g_{\mu\nu}}{\partial x^{\alpha}}
	\biggr)
	\end{align*}
	
	Sie werden Auch Zusammenhangskoeffizienten genannt.
	Folgende Christoffelsymbole entstehend nach einer Berechnung mit MAXIMA:
	
	\begin{equation}
	\begin{aligned}
	\Gamma^0_{01}
	&=
	\frac{1}{1-\displaystyle\frac{r_g}{r}}
	\frac{r_g}{r}
	\frac{1}{2r}
	\\
	\Gamma^1_{00}
	&=
	\biggl(1-\displaystyle\frac{r_g}{r}\biggr)
	\frac{r_g}{r}
	\frac{1}{2r}
	&
	\Gamma^1_{11}
	&=
	-\frac1{1-\displaystyle\frac{r_g}{r}}
	\frac{r_g}{r}
	\frac{1}{2r}
	&
	\Gamma^1_{22}
	&=
	r_g-r
	&
	\Gamma^1_{33}
	&=
	r_g-r
	\\
	\Gamma^2_{12}
	&=
	\frac1r
	\\
	\Gamma^3_{13}
	&=
	\frac1r
	\end{aligned}
	\end{equation}
		
	\section{Simulation}
	
	%Erklärung sourcecode
	
	\section{fazit}
	
	%üseri erkenntnis us de arbet
	
	


	\printbibliography[heading=subbibliography]
	\end{refsection}

