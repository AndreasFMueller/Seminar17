%
% uebersicht.tex -- Uebersicht ueber die Seminar-Arbeiten
%
% (c) 2015 Prof Dr Andreas Mueller, Hochschule Rapperswil
%
\chapter*{"Ubersicht}
\lhead{"Ubersicht}
\rhead{}
\label{skript:uebersicht}
Im zweiten Teil kommen die Teilnehmer des Seminars selbst zu Wort.
Sie zeigen Anwendungsbeispiele f"ur die im ersten
Teil entwickelte Theorie der gew"ohnlichen Differentialgleichungen.
Eine breite Vielfalt von Arbeiten vertieft einzelne Aspekte der Theorie,
untersucht spezielle Differentialgleichungen im Detail
oder erm"oglicht das bessere Verst"andnis interessanter Anwendungen.

{\em Melina Staub} und {\em Fabian Schmid} untersuchen als Beispiel
eines Raumes mit konstanter Krümmung die Geometrie der Kugeloberfläche
und zeigen, wie sich die ebene Trigonometrie auf diesen Fall verallgemeinern
lässt.
Sie illustrieren auch die praktische Bedeutung für die Vermessung auf
der Kugeloberfläche.

{\em Nadja Rutz} und {\em Ambroise Suter} befassen sich mit Minimalflächen.
Sie erläutern an Beispielen, wie sich Flächen minimalen Inhaltes durch
die mittlere Krümmung charakterisieren lassen.

{\em Jonas Gründler} und {\em Sasha Jecklin} gehen mit numerischen
Experimenten der Frage nach, ob man durch wahl einer geeigneten
Flugbahn um ein schwarzes Loch eine Zeitreise machen kann.
{\em Kevin Schmidiger} geht einer ähnlichen Fragestellung auf den Grund,
nämlich der Frage, wie stark die Zeitverschiebungen, die von der
speziellen und der allgemeinen Relativitätstheorie vorhergesagt werden,
die Genauigkeit des GPS beeinflussen.

{\em Hansruedi Patzen} und {\em Nico Vinzens} berechnen
die Kugelfunktionskoeffizienten des kosmischen Mikrowellenhintergrundes
und bestätigen damit die noch sehr junge Erkenntnis, dass das
Universum flach ist.

Klimamodelle müssen die Energieverteilung über lange Zeit berechnen,
dazu eignen sich spektrale Methoden, die auf Kugelfunktionen basieren,
besonders gut. {\em Peter Nötzli} versucht modellhaft zu zeigen, wie
ein solches Modell funktionieren kann.

Adaptive Optik beschäftigt sich mit dem Problem, dass die Wellenfront
zum Beispiel durch Inhomogenitäten der Atmosphäre gekrümmt sein.
{\em Matthias Schneider} beschreibt, wie ein adaptives optisches 
System diese Krümmung wieder entfernen und damit höhere Abbildungsschärfe
erreichen kann.

Die Bestätigung durch Arthur Eddington der Vorhersage der Lichtablenkung
durch die Sonne hat Einstein über Nacht berühmt gemacht.
Erst viel später konnte man diese Wirkung der Schwerkraft auch in
Bildern ferner Galaxien erkennen.
{\em Pascal Stump} berechnet, wie ein Bild durch einen Galaxiencluster
verzerrt wird.

