%
% friedmann.tex
%
% (c) 2017 Prof Dr Andreas Müller, Hochschule Rapperswil
%
\chapter{Friedmann-Gleichungen%
\label{skript:chapter:friedmann}}
\lhead{Friedmann-Gleichungen}
\rhead{}
Natürliche Systeme sind meistens so komplex, dass es praktisch nicht
möglich ist, jedes einzelne Detail physikalisch exakt wiederzugeben.
Es ist daher notwendig, vereinfachte Modelle zu verwenden, welche 
die Anzahl der zu berücksichtigenden Variablen reduzieren auf eine
Art, die immer noch gestattet, die gestellten Fragen mit ausreichender
Genauigkeit zu beantworten.

In diesem Kapitel betrachten wir zunächst ein paar Beispiele aus den
Naturwissenschaften, welche diesen Prozess der Modellbildung exemplarisch
vorstellen.
Anschliessend wird ein Modell für das Universum betrachtet, welches
das Universum derart vereinfacht, dass über die langfristige
Geschichte des Universums konkrete Aussagen zu machen gestattet.
Mit diesen sogenannten Friedmann-Gleichungen kann man sodann zum
Beispiel das Alter des Universums bestimmen.

\section{Friedmann-Gleichungen}
\rhead{Friedmann-Gleichungen}
Im Kapitel~\ref{skript:chapters:kosmologie} haben wir gelernt, dass
das homogene und isotrope Universum die Robertson-Walker-Metrik 
\[
ds^2
=
-c^2\,dt^2
+
a(t)^2\bigl(
dr^2 + S_\kappa(r)^2d\Omega^2
\bigr)
\]
hat.
Ausserdem muss die Metrik eine Lösung der Einstein-Gleichungen sein.
Auf deren rechten Seite steht der Energie-Impuls-Tensor des Universums.
Da das Universum homogen und isotrop ist, muss auch der Energie-Impuls-Tensor
isotrop und homogen sein.
Für einen mit Gas der Dichte $\varrho$ und Druck $p$ gefülltes Universum
ist der Energie-Impuls-Tensor durch
\[
T^{\mu\nu}
=
\begin{pmatrix}
\varrho c^2 & 0 & 0 & 0 \\
     0      & p & 0 & 0 \\
     0      & 0 & p & 0 \\
     0      & 0 & 0 & p \\
\end{pmatrix}
\]
beschrieben.

Die noch unbekannte Funktion $a(t)$ sollte sich also aus der
Einstein-Gleichung bestimmen lassen, sofern wir den Energie-Impuls-Tensor
des Universums berechnen können.
Dabei ist zu beachten, dass mit der Ausdehnung des Universums sich sowohl
die Dichte $\varrho$ als auch der Druck $p$ ändern werden.
Wir erwraten daher, ein System von Differentialgleichungen zu erhalten.
Die Einstein-Gleichungen werden einen Zusammenhang zwischen den 
Ableitungen von $a(t)$ und der Dichte herstellen,
andererseits müssen wir aus unserer Kenntnis des Verhaltens von
Materie und Strahlung herleiten, wie sich die Dichte mit dem
Streckungsfaktor  $a(t)$ des Universums ändert.
Dieses Differentialgleichungssystem wird die Entwicklung des homogenen
und isotropen Universums vollständig beschreiben.

Als erstes müssen wir jetzt die Differentialgleichung für $a(t)$
in Abhängigkeit von $\varrho$ ermitteln.
Dazu betrachten wir ausschliesslich die $00$-Komponente der
Einsteingleichungen, also
\begin{equation}
G_{00} = \frac{8\pi K}{c^2} T_{00}=\frac{8\pi K\varrho}{c^2}.
\label{skript:friedmann:einstein}
\end{equation}
Wir berechnen den Einstein-Tensor der Robertson-Walker-Metrik
wieder mit Maxima, und bekommen.
%(%i5)                       ratsimp(Einstein(1, 1))
%                                         2
%         2     d         2      2       d              2  d         2    2
%      3 S (r) (-- (a(t)))  - 2 c  S(r) (--- (S(r))) - c  (-- (S(r)))  + c
%               dt                         2               dr
%                                        dr
%(%o5) --------------------------------------------------------------------
%                                   2     2
%                                  S (r) a (t)
%(%i6)                    tex(ratsimp(Einstein(1, 1)))
\[
{{3\,S^2\left(r\right)\,\left({{d}\over{d\,t}}\,a\left(t\right)
 \right)^2-2\,c^2\,S\left(r\right)\,\left({{d^2}\over{d\,r^2}}\,S
 \left(r\right)\right)-c^2\,\left({{d}\over{d\,r}}\,S\left(r\right)
 \right)^2+c^2}\over{S^2\left(r\right)\,a^2\left(t\right)}}
\]
\begin{equation}
G_{00}
=
3\frac{\dot a(t)^2}{a(t)^2}
-\frac{2c^2}{a(t)^2}\frac{S''_\kappa(r)}{S_\kappa(r)}
-\frac{c^2}{a(t)^2}
\frac{S'_\kappa(r)^2-1}{S_\kappa(r)^2}
\end{equation}
Die rechte Seite können wir mit Hilfe der im
Kapitel~\ref{skript:chapter:kosmologie} bereitgestellten Formeln
\eqref{skript:robertson:ersteableitung} und
\eqref{skript:robertson:zweiteableitung} für die Ableitungen der Funktion
$S_\kappa(r)$ vereinfachen.
Wir erhalten 
\begin{align*}
G_{00}
&=
3\frac{\dot a(t)^2}{a(t)^2}
-\frac{2c^2}{a(t)^2}\frac{-\kappa\displaystyle \frac{S_\kappa(r)}{R_c^2}}{S_\kappa(r)}
-\frac{c^2}{a(t)^2}
\frac{1-\kappa \frac{\displaystyle S_\kappa(r)^2}{R_c^2}-1}{S_\kappa(r)^2}
\\
&=
3\frac{\dot a(t)^2}{a(t)^2}
+\frac{3c^2\kappa}{a(t)^2R_c^2}
\end{align*}
Die $00$-Komponente der Einstein-Gleichung liefert uns jetzt die Gleichung
\begin{equation}
\frac{\dot a(t)^2}{a(t)^2}
+
\frac{c^2\kappa}{a(t)^2R_c^2}
=
\frac{8\pi K}{3c^2}\varrho.
\end{equation}
Üblicherweise wird die Gleichung als Gleichung für $H=\dot a(t)/a(t)$
geschrieben, also
\begin{equation}
H^2
=
\biggl(
\frac{\dot a(t)}{a(t)}
\biggr)^2
=
\frac{8\pi K}{3c^2}\varrho
-
\frac{c^2\kappa }{a(t)^2R_c^2}
\label{skript:friedmann:friedmann}
\end{equation}

\section{Die Beschleunigungsgleichung}
\rhead{Die Beschleunigungsgleichung}
Durch Verjüngung der Einsteingleichung kann man auch den anderen
Komponenten des Einstein-Tensors und insbesonder der Druck-Komponenten 
des Energie-Impuls-Tensors Rechnung tragen.
Die rechte Seite der Einstein-Gleichung wird einfach nur die
Spur, also
\[
\frac{8\pi K}{c^2}
T^\mu_\mu = \frac{8\pi K}{c^4}(-\varrho c^2 + 3p).
\]
Für die linke Seite muss die Verjüngung des Einstein-Tensors ausgerechnet
werden.
%(%i16) s : ratsimp(sum(sum(ginverse     Einstein(i, j), i, 1, length(g)), j, 
%                                   i, j
%                                                                 1, length(g)))
%                         2
%             2          d                2     d         2
%(%o16) - (6 S (r) a(t) (--- (a(t))) + 6 S (r) (-- (a(t)))
%                          2                    dt
%                        dt
%                        2
%               2       d                2  d         2      2    2  2     2
%          - 4 c  S(r) (--- (S(r))) - 2 c  (-- (S(r)))  + 2 c )/(c  S (r) a (t))
%                         2                 dr
%                       dr
%(%i17)                              tex(s)
%\[
%-{{6\,S^2\left(r\right)\,a\left(t\right)\,\left({{d^2}\over{d\,t^2
% }}\,a\left(t\right)\right)+6\,S^2\left(r\right)\,\left({{d}\over{d\,
% t}}\,a\left(t\right)\right)^2-4\,c^2\,S\left(r\right)\,\left({{d^2
% }\over{d\,r^2}}\,S\left(r\right)\right)-2\,c^2\,\left({{d}\over{d\,r
% }}\,S\left(r\right)\right)^2+2\,c^2}\over{c^2\,S^2\left(r\right)\,a^2
% \left(t\right)}}
%\]
Der verjüngte Einstein-Tensor ist
\begin{align*}
G^\mu_\mu
&=
- \frac{6}{c^2} \frac{\ddot a(t)}{a(t)}
-\frac{6}{c^2} \frac{\dot a(t)^2}{a(t)^2}
+\frac{4}{a(t)^2} \frac{S''_\kappa(r)}{S_\kappa(r)}
+\frac{2S'_\kappa(r)^2-2}{a(t)^2 S_\kappa(r)^2}
\\
&=
-\frac{6}{c^2}\biggl(\frac{\ddot a(t)}{a(t)} + \frac{\dot a(t)^2}{a(t)^2}\biggr)
-\frac{6}{a(t)^2}\frac{\kappa }{R_c^2}.
\end{align*}
Die verjüngte Einsteingleichung wird dann
\begin{align*}
G^\mu_\mu
&=
\frac{8\pi K}{c^2}(-\varrho  c^2+ 3p)
\\
-\frac{6}{c^2}\biggl(\frac{\ddot a(t)}{a(t)} + \frac{\dot a(t)^2}{a(t)^2}\biggr)
-\frac{6}{a(t)^2}\frac{\kappa }{R_c^2}
&=
\frac{8\pi K}{c^2}(-\varrho c^2 + 3p)
\\
\frac{\ddot a(t)}{a(t)} + \frac{\dot a(t)^2}{a(t)^2}
&=
-\frac{4\pi K}{3c^2}(-\varrho c^2 + 3p)
-\frac{c^2}{a(t)^2}\frac{\kappa }{R_c^2}.
\end{align*}
Den zweiten Term auf der rechten Seite können wir mit Hilfe der Gleichung
\eqref{skript:friedmann:friedmann}
eliminieren und bekommen
\begin{align}
\frac{\ddot a(t)}{a(t)}
&=
-\frac{4\pi K}{3c^2}(-\varrho c^2 + 3p)
-\frac{c^2}{a(t)^2}\frac{\kappa }{R_c^2}
-\frac{8\pi K}{3c^2}\varrho c^2
+
\frac{c^2\kappa }{a^2R_c^2}
\notag
\\
\frac{\ddot a(t)}{a(t)}
&=
-\frac{4\pi K}{3c^2}(\varrho c^2 + 3p).
\label{skript:friedmann:beschleunigung}
\end{align}
Dies ist die {\em Beschleunigungsgleichung}, sie ist eine
Differentialgleichung zweiter Ordnung für den Skalenfaktor.
\index{Beschleunigungsgleichung}

Man kann die linke Seite auch durch $H$ und seine erste Ableitung
ausdrücken:
\begin{align*}
\dot H
&=
\frac{d}{dt}\frac{\dot a(t)}{a(t)}
=
\frac{\ddot a(t) a(t)-\dot a(t)^2}{a(t)^2}
=
\frac{\ddot a(t)}{a(t)} - \frac{\dot a(t)^2}{a(t)^2}
=
\frac{\ddot a(t)}{a(t)} - H^2
\\
\Rightarrow
\dot H+H^2
&=
\frac{\ddot a(t)}{a(t)}.
\end{align*}
Damit können wir die Beschleunigungsgleichung auch als Gleichung
für $H$ schreiben:
\begin{equation}
\dot H+H^2
=
-\frac{4\pi K}{3c^2}(\varrho c^2 + 3p).
\end{equation}
Natürlich muss $\varrho$ und $p$ in Abhängigkeit von $H$ ausgedrückt
werden, bevor diese Form der Beschleunigungsgleichung zur Berechnung
der Geschichte des Universums genutzt werden kann.

\section{Zustandsgleichungen}
\rhead{Zustandsgleichungen}
\subsection{Ideales Gas}

\subsection{Strahlung}

\subsection{Staub}

\subsection{Dunkle Energie}

\section{Dunkle Energie}
\rhead{Dunkle Energie}

\section{Geschichte des Universums}
\rhead{Geschichte des Universums}


