%
% speziell.tex -- Spezielle Relativitätstheorie
%
% (c) 2017 Prof Dr Andreas Müller, Hochschule Rapperswil
%

\chapter{Spezielle Relativitätstheorie%
\label{skript:chapter:spezielle}}
\lhead{Spezielle Relativitätstheorie}
\rhead{}
Im neunzehnten Jahrhundert hat James Clark Maxwell die Theorien
über Elektriztität und Magnetismus zu einer einheitlichen Theorie
der Elektrodynamik zusammengefasst.
\index{Maxwell, James Clark}
\index{Elektrodynamik}
Diese Theorie ist die Grundlage aller Phänomene, mit denen sich
ein Elektroingenieur täglich herumschlägt.
Sie hat jedoch eine seltsame Eigenschaft, die schon sehr früh
aufgefallen ist.
Die Formeln der Mechanik von Galilei und Newton nicht ändern,
wenn man eine Koordinatentransformation der Form
\begin{equation}
\begin{aligned}
t'&=t\\
x'&=x+vt
\end{aligned}
\label{skript:kruemmung:galileitransformation}
\end{equation}
\index{Galiei-Transformation}
durchführt.
Diese Koordinatentransformation entspricht einer gleichförmigen
Bewegung des $(t',x')$-Koordinatensystems gegenüber dem 
$(t,x)$-Koordinatensystem.
Sie wird auch Galilei-Transformation genannt und wiederspiegelt die
Erfahrungstatsache, dass es in einem abfahrenden Zug schwierig ist
zu entscheiden, ob sich nun der Zug oder der Bahnhof in Bewegung setzt.

Die Gleichungen der Elektrodynamik verändern sich jedoch.
Es stellte sich daher die Frage, ob die Gleichungen der Elektrodynamik
nur einen Teilaspekt der Realität darstellen, oder ob die Gleichungen
der Mechanik nur eine Näherung sind, die für Geschwindigkeiten nahe
der Lichtgeschwindigkeiten nicht mehr zulässig sein würden.
Im letzten Fall wäre die
Galilei-Transformation~\eqref{skript:kruemmung:galileitransformation}
für solche Geschwidigkeiten auch nur eine Näherung, die durch
eine exaktere Formel ersetzt werden müsste, mit weitreichenden
Folgen für die Mechanik bei sehr hohen Geschwindigkeiten.

Es hat sich herausgestellt, dass tatsächlich die klassische Mechanik
angepasst werden muss.
Einstein hat diesen Schritt 1905 in seiner speziellen Relativitätstheorie
vollzogen.
Ziel dieses Abschnittes ist zu zeigen, welche Auswirkungen seine
Erkenntnis auf die Mechanik aber auch auf unser Weltbild hat.

\section{Lichtkegel}
\rhead{Lichtkegel}
Die Elektrodynamik sagt die Ausbreitungsgeschwindigkeit von
elektromagnetishen Wellen voraus.
Stellt man sich vor, dass elektromagnetische Wellen von einem
Medium geleitet werden, das man Äther nannte, dann müsste die
Geschwindigkeit von der Bewegung des Beobachters relativ zu
diesem Äther abhängen.
In hochgenauen Experimenten konnten Michelson und Morley und später
viele andere keine solche Abhängigkeit feststellen.
Dies steht zwar in Einklang mit der Theorie der Elektrodynamik,
wiederspricht der Galilei-Transformation, welche eine Veränderung
der Ausbreitungsgeschwindigkeit um $v$ vorhersagen würde.

Einstein hat die experimentell sehr gut bestätigte Konstanz der
Lichtgeschwindigkeit daher als Ausgangspunkt seiner Theorie genommen.
Entscheidend für die Physik ist, ob zwei Punkte sich mit elektromagnetischen
Wellen beeinflussen können.
Es ist daher nicht mehr ausreichend, nur Punkte miteinander zu vergleichen,
es muss auch immer die Zeit berücksichtigt werden, zu der sie verglichen
werden.
Raum und Zeit verschmelzen so zu einem einzigen vierdimensionalen
Kontinuum mit den Koordinaten $(t,x,y,z)$, welche wir die Raumzeit
nennen.
Quadrupel $(t,x,y,z)$ heissen auch {\em Ereignisse}.
\index{Ereignis}
Zwei Ereignisse $(t_1,x_1,y_1,z_1)$ und $(t_2,x_2,y_2,z_2)$ können
kausal voneinander abhängen, wenn ein Lichtsignal sich vom einen Ereignis
zum anderen Ereignis ausbreiten kann.
Dazu ist notwendig, dass der ``Abstand''
\begin{equation}
s^2
=
-c^2(t_1-t_2)^2
+
\mathstrut
\underbrace{
(x_1-x_2)^2
+
(y_1-y_2)^2
+
(z_1-z_2)^2}_{\text{Raumabstand}}
\label{skript:kruemmung:raumzeitabstand}
\end{equation}
gleich $0$ ist.
Breitet sich die Wirkung langsamer als mit Lichtgeschwindigket aus,
wird die zeitliche Differenz noch grösser sein, und damit der
Ausdruck~\eqref{skript:kruemmung:raumzeitabstand} negativ ist.

\begin{figure}
\centering
\includegraphics[width=\hsize]{chapters/3d/lichtkegel.jpg}
\caption{Lichtkegel ausgehend vom Nullpunkt zur Zeit $t=0$.
Alle Punkte innerhalb des Kegels mit $t>0$ liegen in der Zukunft des
Beobachters im Nullpunkt, solche mit $t<0$ in seiner Vergangenheit.
Punkte ausserhalb des Kegels können vom Nullpunkt aus nicht
beeinflusst werden.
Nur Punkte innerhalb des Kegels mit $t<0$ können Einfluss haben
auf den Nullpunkt.
\label{skript:kruemmung:fig:lichtkegel}}
\end{figure}

Das Vorzeichen des Ausdrucks~\eqref{skript:kruemmung:raumzeitabstand}
beschreibt also, ob zwei Ereignisse sich gegenseitig beeinflussen
können. 
Ist er negativ, dann ist eine Beeinflussung mit einer Wirkung möglich,
die sich weniger schnell als Licht ausbreitet.
Ist er gleich $0$, ist nur eine Beeinflussung mit einer Wirkung möglich,
die sich mit Lichtgeschwindigkeit ausbreitet.
Ist er positiv, ist keine gegenseitige Beeinflussung möglich.
Dies führt auf eine Unterteilung des Raumes in drei Gebiete
(Abbildung~\ref{skript:kruemmung:fig:lichtkegel}).
Die Fläche mit der Gleichung
\[
-c^2t^2+x^2+y^2+z^2=0
\]
heisst der Lichtkegel.
Ereignisse innerhalb des Lichtkegels mit $t>0$ können vom Nullpunkt aus
beeinflusst werden.
Ereignisse innerhalb des Lichtkegels imt $t<0$ können auf den Nullpunkt
Einfluss nehmen.
Alle anderen Ereignisse können den Nullpunkt weder beeinflussen
noch von ihm beeinflusst werden.
Sie müssen daher als die Gegenwart des Nullpunktes bezeichnet
werden.

In diesem Bild gibt es also keinen sinnvollen Begriff der Gleichzeitigkeit
mehr.
Wäre das Universum nicht aus einem Punkt entstanden, wie das Big Bang-Modell
besagt, dann wäre bereits die Frage ``Wann ist das Universum entstanden''
unsinnig, denn es gibt keinen sinnvollen Art und Weise, wie etwas gleichzeitig
im ganzen Universum stattfinden könnte.
Diese Beobachtung hat zu Einsteins Zeit aber keine grossen Wellen
geworfen, denn die meisten Phyisker gingen davon aus, dass das Universum
ewig und statisch ist, dass es also gar keinen Anfang braucht.
Auch Einstein ging davon aus, was ihn später in seiner allgemeinene
Relativitätstheorie dazu gebracht hat, einen nicht weiter erklärbaren
Term hinzuzufügen, einzig um zu erreichen, dass das Universum statisch 
sein kann.

\section{Lorentztransformation}
\rhead{Lorentztransformation}
Der Ausdruck~\eqref{skript:kruemmung:raumzeitabstand} ist nicht
invariant bei einer Galilei-Transformation.
Damit stellt sich die Frage, welche Transformationen denn
invariant wären.
Dies wären die Koordinaten-Änderungen, die zulässig sind, wenn
man das grundlegende Gesetz~\eqref{skript:kruemmung:raumzeitabstand}
der Kausalität erhalten will.

Zunächst sind Koordinatentransformationen, die nur die Raumkoordinaten
$x$, $y$ und $z$ beeinflussen, und den räumlichen Abstand nicht
ändern, zulässig.
Diese Transformationen sind aber wohlbekannt.
In der Linearen Algebra lernt man, dass genau die orthogonalen
Matrizen $A\in \textrm{O}(3)$, also Matrizen mit $A^tA=E$ diese
Eigenschaft haben.
Diese entsprechen räumlichen Drehungen und Spiegelungen.

Wenn sich aber zwei Koordinatensystem gegeneinander verschieben,
so wie bei der Galilei-Transformation, dann müssen auch die auch
die Zeitkoordinaten in die Transformation involviert sein.
Wir suchen also eine Koordinatentransformation, die nur die
$t$- und $x$-Koordinaten betrifft.
Wir können Sie in der Form
\begin{equation}
\begin{linsys}{3}
t'&=& a_{11}t&+&a_{12}x\\
x'&=& a_{21}t&+&a_{22}x
\end{linsys}
\end{equation}
ansetzen.
Wir verlangen jetzt aber, dass der
Ausdruck~\eqref{skript:kruemmung:raumzeitabstand}
unverändert bleibt.
Da die $y$- und $z$-Koordinaten ohnehin nicht involviert sind, bedeutet
das, dass
\begin{align*}
-c^2t^2 + x^2
&=
-c^2t'^2 + x'^2
\\
&=
-c^2(a_{11}t+a_{12}x)^2 + (a_{21}t+a_{22}x)^2
\\
&=
-c^2a_{11}^2t^2 -2c^2a_{11}a_{12}tx -c^2a_{12}^2x^2
+a_{21}^2t^2+2a_{21}a_{22}tx+a_{22}^2x^2
\\
&=
-(c^2a_{11}^2 - a_{21}^2)t^2
+2(-c^2a_{11}a_{12}+a_{21}a_{22})tx
+(-c^2a_{12}^2 + a_{22}^2)x^2
\end{align*}
Da dies für beliebige Werte von $x$ und $t$ gelten muss, müssen die
Koeffizienten der beiden Seiten übereinstimmen. 
Wir erhalten damit für die Koeffizienten $a_{ik}$ die Gleichungen
\[
\begin{aligned}
c^2a_{11}^2-a_{21}^2&=c^2
&&\Rightarrow&
a_{11}^2
-
\biggl(\frac{a_{21}}{c}\biggr)^2
&=1
\\
-c^2a_{12}^2+a_{22}^2&=1
&&\Rightarrow
&
a_{22}^2 - (ca_{12})^2&=1
\\
a_{21}a_{22}&=c^2a_{11}a_{12}
\end{aligned}
\]
Da die Differenz der Quadrate jeweils $1$ ist, können die Basen
als Werte des hyperbolischen Kosinus und Sinus dargestellt werden:
\[
\begin{aligned}
a_{11}&=\cosh\beta  &&&\frac{a_{21}}{c}&=\sinh\beta
\\
a_{22}&=\cosh\gamma &&&ca_{12}&=\sinh\gamma
\end{aligned}
\]
Die dritte Gleichung lautet dann
\begin{align*}
a_{21}a_{22}=c\sinh\beta\cosh\gamma
&=
c^2a_{11}a_{12}=c^2\cosh\beta\frac1c\sinh\gamma
\\
c\sinh\beta\cosh\gamma
&=
c\cosh\beta\sinh\gamma
\\
\tanh\beta&=\tanh\gamma,
\end{align*}
es folgt $\beta=\gamma$.
Damit haben wir die gesuchte Transformation gefunden, es ist
\begin{equation}
\begin{pmatrix}
a_{11}&a_{12}\\
a_{21}&a_{22}
\end{pmatrix}
=
\begin{pmatrix}
 \cosh\beta&\frac1c\sinh\beta\\
c\sinh\beta&\cosh\beta
\end{pmatrix}
\end{equation}
Um die physikalische Bedeutung des Parameters $\beta$ zu verstehen, 
betrachten wir die Punkte $x=0$ für beliebige $t$.
Sie werten abgebildet auf
\[
x' = c\sinh\beta t
\]
Im $(t',x')$-Koordinatensystem bewegt sich also der Nullpunkt des 
$(t,x)$-Koordinatensystems mit der Geschwindigkeit
\[
v=c\sinh\beta
\]
Der Parameter $\sinh\beta$ hat also die Bedeutung einer in Einheiten
der Lichtgeschwindigkeit gemessenen Geschwindigkeit.
Da also $v/c=\sinh\beta$ ist, kann man die Transformation auch
als
\begin{equation}
\begin{pmatrix}
a_{11}&a_{12}\\
a_{21}&a_{22}
\end{pmatrix}
=
\begin{pmatrix}
\displaystyle\sqrt{1-\biggl(\frac{v}{c}\biggr)^2}
	&\displaystyle\frac{v}{c^2} \\
v
	&\displaystyle\sqrt{1-\biggl(\frac{v}{c}\biggr)^2}
\end{pmatrix}
\label{skript:kruemmung:lorentz}
\end{equation}
schreiben.
Die Transformation 
\eqref{skript:kruemmung:lorentz}
heisst {\em Lorentz-Transformation}.
\index{Lorentz-Transformation}


\section{Energie und Impuls}
\rhead{Energie und Impuls}

