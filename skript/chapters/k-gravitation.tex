%
% k-gravitatzion.tex
%
% (c) 2017 Prof Dr Andreas Müller, Hochschule Rapperswil
%
\section{Gravitation%
\label{skript:kruemmung:sectipn:gravitation}}
Albert Einstein hat erkannt, dass die Wirkung der Gravitation 
durch die Krümmung des Raumes beschrieben werden muss.

\subsection{Physikalische Bedeutung der Krümmung}

\subsection{Einstein-Tensor}

\subsection{Feldgleichungen}

\subsection{Schwarzschild-Metrik}
Die Feldgleichungen schränken die Metriken ein, die in einem Raumzeit-Gebiet
überhaupt möglich sind.
Damit stellt sich automatisch die Frage, wie die Einsteinsche Theorie 
das Gravitationsfeld in der Umgegung eines Sterns beschreiben kann.
Schon wenige Monate nachdem Einstein seine allgemeine Relativitätstheorie
veröffentlich hat, hat Karl Schwarzschild eine Lösung der Einsteinschen
Feldgleichungen gefunden.
Daraus lassen sich die Bewegungsgleichungen eines Körpers in der Nähe
eines Sterns ableiten und es sollte möglich sein, den Unterschied zwischen
der Newtonschen Gravitations-Theorie und Einsteinschen  zu quantifizieren
und damit Tests der allgmeinen Relativitätstheorie zu ermöglichen.

Karl Schwarzschild suchte eine Metrik, die sich mit der Zeit nicht ändert,
also
\[
\frac{\partial g_{\mu\nu}}{\partial t}=0,
\]
und die ausserdem kugelsymmetrisch sein soll.
Diese Metrik sollte eine erste Approximation für die Gravitation in
der Umgebung eines Sterns sein.
Natürlich berücksichtigt dieses Modell weder, dass sich Sterne mit
der Zeit entwickeln, noch die Tatsache, dass sich Sterne normalerweise
um eine Achse drehen, dass man also gar nicht eine rotationssymmetrische
Lösung erwarten darf.

Die Längenmessung
\begin{equation}
ds^2
=
-c^2\,dt^2 + dr^2 + r^2 d\Omega^2
\qquad
d\Omega^2 = d\vartheta^2 + \sin^2\vartheta\,d\varphi^2
\label{skript:kruemmung:euklid}
\end{equation}
im Raum mit Kugelkoordinaten ist natürlich eine solche Metrik,
doch da dies nur eine andere Parametrisierung des euklidischen
Raumes ist, ist diese Geometrie flach.
Sie kann also sich nicht ein Modell eines Sternes sein.

Man kann eine Lösung der Feldgleichungen finden, indem man den
einzelnen Termen der euklischen Metrik~\eqref{skript:kruemmung:euklid}
zunächst unbestimmte Faktoren hinzufügt, die nur von $r$ abhängt,
und dann mit Hilfe der Feldgleichungen dafür Differentialgleichungen
herleitet.
Wir wollen diesen beschwerlichen Weg nicht gehen, und uns mit dem
Resultate zufriedenstellen, es lautet
\begin{equation}
ds^2
=
-\biggl(1-\frac{2M}r\biggr)c^2\,dt^2
+\frac1{\displaystyle 1-\frac{2M}r}\,dr^2 + r^2\,d\Omega^2.
\end{equation}
Man kann nachrechnen, zum Beispiel mit Hilfe der früher vorgestellen
Maxima-Programme, dass der Einstein-Tensor für diese Metrik überall
verschwindet.

\subsection{Robertson-Walker-Metrik}

