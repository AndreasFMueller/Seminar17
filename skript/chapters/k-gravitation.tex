%
% k-gravitatzion.tex
%
% (c) 2017 Prof Dr Andreas Müller, Hochschule Rapperswil
%
\section{Gravitation%
\label{skript:kruemmusng:sectipn:gravitation}}
Albert Einstein hat erkannt, dass die Wirkung der Gravitation 
durch die Krümmung des Raumes beschrieben werden muss.
\index{Einstein, Albert}

%\subsection{Speziellue Relativitätstheorie}
%Im neunzehnten Jahrhundert hat James Clark Maxwell die Theorien
%über Elektriztität und Magnetismus zu einer einheitlichen Theorie
%der Elektrodynamik zusammengefasst.
%\index{Maxwell, James Clark}
%\index{Elektrodynamik}
%Diese Theorie ist die Grundlage aller Phänomene, mit denen sich
%ein Elektroingenieur täglich herumschlägt.
%Sie hat jedoch eine seltsame Eigenschaft, die schon sehr früh
%aufgefallen ist.
%Die Formeln der Mechanik von Galilei und Newton nicht ändern,
%wenn man eine Koordinatentransformation der Form
%\begin{equation}
%\begin{aligned}
%t'&=t\\
%x'&=x+vt
%\end{aligned}
%\label{skript:kruemmung:galileitransformation}
%\end{equation}
%\index{Galiei-Transformation}
%durchführt.
%Diese Koordinatentransformation entspricht einer gleichförmigen
%Bewegung des $(t',x')$-Koordinatensystems gegenüber dem 
%$(t,x)$-Koordinatensystem.
%Sie wird auch Galilei-Transformation genannt und wiederspiegelt die
%Erfahrungstatsache, dass es in einem abfahrenden Zug schwierig ist
%zu entscheiden, ob sich nun der Zug oder der Bahnhof in Bewegung setzt.
%
%Die Gleichungen der Elektrodynamik verändern sich jedoch.
%Es stellte sich daher die Frage, ob die Gleichungen der Elektrodynamik
%nur einen Teilaspekt der Realität darstellen, oder ob die Gleichungen
%der Mechanik nur eine Näherung sind, die für Geschwindigkeiten nahe
%der Lichtgeschwindigkeiten nicht mehr zulässig sein würden.
%Im letzten Fall wäre die
%Galilei-Transformation~\eqref{skript:kruemmung:galileitransformation}
%für solche Geschwidigkeiten auch nur eine Näherung, die durch
%eine exaktere Formel ersetzt werden müsste, mit weitreichenden
%Folgen für die Mechanik bei sehr hohen Geschwindigkeiten.
%
%Es hat sich herausgestellt, dass tatsächlich die klassische Mechanik
%angepasst werden muss.
%Einstein hat diesen Schritt 1905 in seiner speziellen Relativitätstheorie
%vollzogen.
%Ziel dieses Abschnittes ist zu zeigen, welche Auswirkungen seine
%Erkenntnis auf die Mechanik aber auch auf unser Weltbild hat.
%
%\subsubsection{Lichtkegel}
%Die Elektrodynamik sagt die Ausbreitungsgeschwindigkeit von
%elektromagnetishen Wellen voraus.
%Stellt man sich vor, dass elektromagnetische Wellen von einem
%Medium geleitet werden, das man Äther nannte, dann müsste die
%Geschwindigkeit von der Bewegung des Beobachters relativ zu
%diesem Äther abhängen.
%In hochgenauen Experimenten konnten Michelson und Morley und später
%viele andere keine solche Abhängigkeit feststellen.
%Dies steht zwar in Einklang mit der Theorie der Elektrodynamik,
%wiederspricht der Galilei-Transformation, welche eine Veränderung
%der Ausbreitungsgeschwindigkeit um $v$ vorhersagen würde.
%
%Einstein hat die experimentell sehr gut bestätigte Konstanz der
%Lichtgeschwindigkeit daher als Ausgangspunkt seiner Theorie genommen.
%Entscheidend für die Physik ist, ob zwei Punkte sich mit elektromagnetischen
%Wellen beeinflussen können.
%Es ist daher nicht mehr ausreichend, nur Punkte miteinander zu vergleichen,
%es muss auch immer die Zeit berücksichtigt werden, zu der sie verglichen
%werden.
%Raum und Zeit verschmelzen so zu einem einzigen vierdimensionalen
%Kontinuum mit den Koordinaten $(t,x,y,z)$, welche wir die Raumzeit
%nennen.
%Quadrupel $(t,x,y,z)$ heissen auch {\em Ereignisse}.
%\index{Ereignis}
%
%\begin{figure}
%\centering
%\includegraphics[width=\hsize]{chapters/3d/lichtkegel.jpg}
%\caption{Lichtkegel ausgehend vom Nullpunkt zur Zeit $t=0$.
%Alle Punkte innerhalb des Kegels mit $t>0$ liegen in der Zukunft des
%Beobachters im Nullpunkt, solche mit $t<0$ in seiner Vergangenheit.
%Punkte ausserhalb des Kegels können vom Nullpunkt aus nicht
%beeinflusst werden.
%Nur Punkte innerhalb des Kegels mit $t<0$ können Einfluss haben
%auf den Nullpunkt.
%\label{skript:kruemmung:fig:lichtkegel}}
%\end{figure}
%
%\subsubsection{Lorenztransformation}
%\subsubsection{Energie und Impuls}
%
\input{chapters/k-speziell.tex}


\subsection{Physikalische Bedeutung der Krümmung}

\subsection{Einstein-Tensor}

\subsection{Feldgleichungen}

\input{chapters/k-schwarzschild.tex}

\subsection{Robertson-Walker-Metrik}
Die Robertson-Walker-Metrik wurde entwickelt, um ein homogenes,
isotropes und expandierendes Universum zu modellieren.
Wir beginnen mit einem flachen Universum mit der
Metrik
\[
ds^2
=
-c^2\,dt^2 + dr^2 + r^2\,d\vartheta^2 + r^2\sin^2\vartheta \,d\varphi^2
=
-c^2\,dt^2 + dr^2 + d\Omega^2.
\]
Wenn dieses Universum expandiert, ändert sich die Längenmessung für die
Raum-Koordinaten, nicht aber für die Zeitkoordinate.
Wir können dies mit einem zeitabhängigen Skalierungsfaktor $a(t)$ 
modellieren:
\[
ds^2
=
-c^2\,dt^2 + a(t)\,dr^2 + a(t)r^2\,d\vartheta^2 + a(t)r^2 \sin^2\vartheta\,d\varphi^2
=
-c^2\,dt^2 + a(t)\,dr^2 + a(t)r^2\,d\Omega^2.
\]
Natürlich bleibt dies ein flaches Universum.

Wir müssen aber auch zulassen, dass das Universum gekrümmt ist.
Dies bedeutet, dass der Umfang eines Kreises um den Nullpunkt
nicht proportional mit $r$ wächst.
Alternativ kann man auch sagen, dass die $r$-Koorindate eines Punktes 
eines Kreises vom Umfang $2\pi R$ nicht unbedingt $R$ sein muss, also
$r \ne R$.
Wir müssen also den Term $dr^2$ ersetzen durch etwas, was mit zunehmendem
$r$ grösser oder kleiner als $1$ sein wird.
\[
ds^2
=
-c^2\,dt^2
+ a(t)^2 \biggl(
\frac{R^2}{R^2-kr^2} dr^2
+
r^2\, d\Omega^2
\biggr)
\]

Der Faktor $a(t)$ heisst der Skalierungsfaktor, er beschreibt, wie stark
das Universum sich zur Zeit $t$ bereits gestreckt hat.

Der Einstein-Tensor dieser Metrik kann in einer sehr langwierigen oder
noch besser mechanischen Rechnung ermittelt werden.
Er ist diagonal und hat die folgenden Diagonal-Elemente:
\begin{align*}
G_{00}
&=
\frac{3}{4}\biggl(\frac{\dot a(t)}{a(t)}\biggr)^2
\\
G_{11}
&=
-\ddot a(t) +\frac{\dot a(t)^2}{4a(t)}
\\
G_{22}
&=
G_{11} r^2
\\
G_{33}
&=
G_{11} r^2\sin^2\vartheta
\end{align*}
In den Ausdrücken für $G_{22}$ und $G_{33}$ fällt auf, dass sie gegenüber
$G_{11}$ nur die zusätzlichen Faktoren enthalten, die auch in der Definition
der Robertson-Walker-Metrik evident sind.
Um die Feldgleichungen aufzustellen reicht es daher, mit $G_{00}$ und
$G_{11}$ zu arbeiten.
Dies wird später auf die Friedmann-Gleichungen führen.



