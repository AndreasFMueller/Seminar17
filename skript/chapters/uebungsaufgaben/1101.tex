Für die Wärmeleitungsgleichung
\begin{equation}
\frac{\partial T}{\partial t}=\frac{\partial^2 T}{\partial x^2}
\label{skript:1101:pdgl}
\end{equation}
auf einem Kreis (d.~h.~$T(x,t)$ ist periodisch  in $x$)
soll eine Lösung in Form einer Fourier-Reihe
\begin{equation}
T(x,t)
=
\frac{a_0(t)}{2} + \sum_{k=1}^\infty\bigl(a_k(t)\cos kx + b_k(t)\sin kx\bigr)
\label{skript:1101:ansatz}
\end{equation}
gefunden werden.
Ausserdem soll die folgende, ebenfalls in Form einer Fourier-Reihe
gegebene Anfangsbedingung für $t=0$ erfüllt sein:
\begin{equation}
T(x,0)
=
\frac{c_0}2 + \sum_{k=1}^\infty \bigl(c_k\cos kx+d_k\sin kx\bigr).
\label{skript:1101:anfangsbedingung}
\end{equation}
\begin{teilaufgaben}
\item
Stellen Sie gewöhnliche Differentialgleichungen für die Koeffizienten 
von $a_k(t)$ und $b_k(t)$ auf.
\item
Finden Sie Anfangsbedingungen für $a_k(t)$ und $b_k(t)$.
\item
Lösen Sie die Differentialgleichungen.
\end{teilaufgaben}


\begin{loesung}
\begin{teilaufgaben}
\item
Wir müssen die Fourier-Reihe~\eqref{skript:1101:ansatz} für
$T(x,t)$ in die Differentialgleichung~\eqref{skript:1101:pdgl}
einsetzen.
Dazu berechnen wir erst die Ableitungen nach $t$ und $x$:
\begin{align*}
\frac{\partial T}{\partial t}
&=
\frac{\dot a_0(t)}{2}
+
\sum_{k=1}^\infty\bigl(\dot a_k(t)\cos kx+\dot b_k(t)\sin kx\bigr)
\\
\frac{\partial^2 T}{\partial x^2}
&=
\phantom{\frac{\dot a_0(t)}{2}}
\mathstrut
-
\sum_{k=1}^\infty k^2\bigl(a_k(t)\cos kx+\dot b_k(t)\sin kx\bigr)
\end{align*}
Die Differentialgleichung verlangt, dass diese beiden Terme übereinstimmen,
also folgen die gewöhnlichen Differentialgleichungen
\begin{align}
\dot a_0(t)&=0
\label{skript:1101:konstant}
\\
\dot a_k(t)&=-k^2 a_k(t)&
\dot b_k(t)&=-k^2 b_k(t)
\label{skript:1101:ab}
\end{align}
Man beachte, dass die erste Gleichung als Speziallfall in der ersten
Gleichung der zweiten Zeile enthalten ist.
\item
Die Anfangsbedingungen finden wir, indem wir in der
Fourier-Reihe~\eqref{skript:1101:ansatz}
$t=0$ einsetzen und mit der 
Anfangsbedingung~\eqref{skript:1101:anfangsbedingung}
vergleichen.
Wir erhalten
\begin{align*}
T(x,0)
&=
\frac{a_0(0)}{2}+\sum_{k=1}^\infty \bigl(a_k(0)\cos kx+b_k(0)\sin kx\bigr)
=
\frac{c_0}2 + \sum_{k=1}^\infty \bigl(c_k\cos kx+d_k\sin kx\bigr)
\end{align*}
und daraus durch Koeffizientenvergleich
\begin{align*}
a_k(0)&=c_k&&k\ge 0\\
b_k(0)&=d_k&&k>0
\end{align*}
als Anfangsbedingungen für die Funktionen $a_k(t)$ und $b_k(t)$.
\item
Die Differentialgleichung~\eqref{skript:1101:konstant} hat eine 
Konstante als Lösung, also $a_0(t)=c_0$.

Die Differentialgleichungen~\eqref{skript:1101:ab} können mit der
Exponentialfunktion gelöst werden, es folgt
\begin{align*}
a_k(t)&=c_ke^{-k^2 t}
&
&\text{und}&
b_k(t)&=d_ke^{-k^2 t}.
\qedhere
\end{align*}
\end{teilaufgaben}
\end{loesung}

