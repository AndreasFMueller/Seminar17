Die Sinus-Kurve kann parametrisiert werden durch
\[
c\colon \mathbb R\to \mathbb R^2 : t\mapsto (t,\sin t).
\]
\begin{teilaufgaben}
\item
Man berechne den Krümmungsradius in der Stelle $c(t)$.
\item
Man bestimme den Mittelpunkt des Krümmungskreises, der sich im Punkt $c(t)$
an die Sinus-Kurve anschmiegt.
\end{teilaufgaben}


\begin{loesung}
Wir brauchen die Ableitungen von $c(t)$
\begin{align*}
\dot c(t)
&=
(1,\cos t)
\\
\ddot c(t)
&=
(0,-\sin t)
\end{align*}
Wegen $|\dot c(t)|=\sqrt{1+\cos^2t}\ge 1$ ist dies sicher keine Bogenlängenparameter.
\begin{teilaufgaben}
\item
Die Formeln für die Krümmung sind
\begin{align*}
\kappa(t)
&=
\frac{\det(\dot c(t),\ddot c(t))}{|\dot c(t)|^3}
=
\frac{\left|\begin{matrix}1&0\\\cos t&-\sin t\end{matrix}\right|}{(1+\cos^2t)^\frac32}
=
\frac{-\sin t}{(1+\cos^2t)^\frac32}
\end{align*}
Der Krümmungsradius ist daher
\[
\varrho(t)
=
\frac{|\sin t|}{(1+\cos^2t)^\frac32}.
\]
\item
Wir brauchen die Tangente und die Normale.
Dazu müssen wir $\dot c(t)$ und $\ddot c(t)$ orthonormalisieren:
\begin{align*}
e_1(t)
&=
\frac{\dot c(t)}{|\dot c(t)|}
=
\frac1{\sqrt{1+\cos^2t}}
\begin{pmatrix}1\\ \cos t\end{pmatrix}
\\
e_2(t)
&=
\frac{\ddot c(t) - (e_1(t)\cdot \ddot c(t))\;\ddot c(t)}{|\ddot c(t) - (e_1(t)\cdot \ddot c(t))\;\ddot c(t)|}
\\
&=
\frac{1}{\dots}\left(
\begin{pmatrix}0\\-\sin t\end{pmatrix}
-
\left(
\frac{1}{\sqrt{1+\cos^2t}}
\begin{pmatrix}1\\\cos t\end{pmatrix}
\cdot
\begin{pmatrix}0\\-\sin t\end{pmatrix}
\right)
\frac{1}{\sqrt{1+\cos^2t}}
\begin{pmatrix}1\\\cos t\end{pmatrix}
\right)
\\
&=
\frac{1}{\dots}
\left(
\begin{pmatrix}0\\-\sin t\end{pmatrix}
+
\frac{\sin t\cos t}{1+\cos^2t}
\begin{pmatrix}1\\\cos t\end{pmatrix}
\right)
\\
&=
\frac{1}{\dots}
\frac{\sin t\cos t}{1+\cos^2t}
\left(
\begin{pmatrix}0\\-\sin t(1+\cos^2t)\end{pmatrix}
+
\begin{pmatrix}1\\\cos t\end{pmatrix}
\right)
\\
&=
\frac{1}{\dots}
\frac{\sin t\cos t}{1+\cos^2t}
\begin{pmatrix}
1
\\
-\sin t(1+\cos^2t)
+
\cos t\end{pmatrix}
\end{align*}
Der Mittelpunkt $m(t)$ des Krümmungskreises ist
\[
m(t)
=
c(t) + e_2(t)/\kappa(t)
\qedhere
\]
\end{teilaufgaben}
\end{loesung}

