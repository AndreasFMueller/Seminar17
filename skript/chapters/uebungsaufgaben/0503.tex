Im Large Hadron Collider LHC am CERN in Genf werden Protonen
mit der Masse $m_p=1.6726219\cdot 10^{-27}\text{kg}$
auf $v=0.999999991c$ beschleunigt.
\begin{teilaufgaben}
\item Welche scheinbare Masse erhält das Proton?
\item Wieviel Zeit vergeht für das Proton, wenn es 2 Stunden im LHC
zirkuliert?
\item Im LHC zirkulieren Protonen in zwei Strahlen bestehend aus je
2808 Paketen von jeweils $N=100\text{ Milliarden}$ Protonen, genannt Bunches.
Wieviel Energie enthält ein Bunch, und vieviel die beiden Strahlen?
\item Mit welcher Geschwindigkeit müsste ein Tesla Model S mit 2t Gewicht
fahren, um die gleiche kinetische Energie zu haben wie ein einzelner Bunch
Protonen im LHC?
\end{teilaufgaben}

\begin{hinweis}
$c=2.9979\cdot 10^8\text{m/s}$
\end{hinweis}

\begin{loesung}
Zur Geschwindigkeit  $v$ gehört der Faktor
\[
\gamma
=
\frac1{\displaystyle\sqrt{1-\frac{v^2}{c^2}}}
=
7453.56.
\]
\begin{teilaufgaben}
\item Die scheinbare Masse wird um den Faktor $1/\gamma$ grösser, also
\[
M
=
\frac{m_p}{\gamma}
=
7453.56
\cdot
1.6726219\cdot 10^{-27}\text{kg}
=
1.2467\cdot 10^{-22}\text{kg}.
\]
\item Die Zeit wird für ein Proton um den Faktor $1/\gamma$ gedehnt,
also ist die Zeit 
\[
\tau = \frac{7200\text{s}}{7453.56}=0.96591\text{s}
\]
\item 
Die Energie eines bewegten Protons ist $Mc^2$, gegenüber der
Ruheenergie müssen zur Beschleunigung eines Protons daher
\begin{align*}
E_p
&=
(M-m_p)c^2 
=
\biggl(\frac{1}{\gamma}-1\biggr)m_pc^2
=
(7453.56-1)\cdot
1.6726219\cdot 10^{-27}\text{kg}
\cdot
(2.9979\cdot 10^{8}\text{m/s})^2
\\
&=
1.1204\cdot 10^{-6}\text{J}
\end{align*}
aufgewendet werden.
Da ein Bunch 
\[
E_{\text{bunch}}
=
nE_p
=
10^{11}\cdot1.1204\cdot 10^{-6}
=
112.04\text{kJ}
\]
Energie enthält, ist die Gesamtenergie der beiden Strahlen
\[
E_{\text{total}}
=
2\cdot 2808\cdot E_{\text{bunch}}
=
629\text{MJ}.
\]
\item
Mit einer Masse von $m_{\text{Tesla}}=2000\text{kg}$ folgt
\begin{align*}
E_{\text{total}}
&=
\frac12m_{\text{Tesla}}v_{\text{Tesla}}^2
\\
v_{\text{Tesla}}
&=\sqrt{
\frac{2E_{\text{bunch}}}{m_{\text{Tesla}}}
}
=
\sqrt{\frac{2\cdot 112.04\cdot 10^3}{2000}}\frac{\text{m}}{\text{s}}
=
10.1\frac{\text{m}}{\text{s}}=36.36\text{km/h}.
\end{align*}
Wenn der LHC in Betrieb ist, enhalten die im Beschleuniger
gespeicherten Protonen also gleichviel Energie, wie man aufwenden müsste,
um 5616 Tesla Model S auf 36.36km/h zu beschleunigen.
\qedhere
\end{teilaufgaben}
\end{loesung}





