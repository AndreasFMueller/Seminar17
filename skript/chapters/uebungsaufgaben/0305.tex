Ein Fadenpendel der Masse $m$ und Länge $l$ wird beschrieben durch
die Auslenkung $\varphi(t)$ in Abhängigkeit von der Zeit. 
Die kinetische Energie ist 
\[
E_{\text{kin}}
=
\frac12ml^2\dot\varphi(t)^2,
\]
die potentielle Energie ist
\[
E_{\text{pot}}
=
-mgl\cos\varphi(t).
\]
Die Bewegung eines Fadenpendels minimiert das Wirkungsintegral
\[
\int_{t_0}^{t_1}
E_{\text{kin}}-E_{\text{pot}}
\,dt
=
\int_{t_0}^{t_1}
\frac12ml^2\dot\varphi(t)^2
+
mgl\cos\varphi(t)
\,dt.
\]
Finden Sie die Bewegungsgleichung.

\begin{loesung}
Wir müssen die Euler-Lagrange-Gleichungen aufstellen für die
Funktion
\[
L(\varphi, \dot\varphi)
=
\frac12ml^2\dot\varphi^2
+
mgl\cos\varphi.
\]
Die Ableitungen sind
\begin{align*}
\frac{\partial L}{\partial\varphi}
&=
-mgl\sin\varphi
\\
\frac{\partial L}{\partial\dot\varphi}
&=
ml^2\dot\varphi
\\
\frac{d}{dt}
\frac{\partial L}{\partial\dot\varphi}
&=
ml^2\ddot\varphi.
\end{align*}
Damit kann man jetzt die Euler-Lagrange-Gleichungen aufstellen:
\begin{align*}
0=
\frac{d}{dt}
\frac{\partial L}{\partial\dot\varphi}
-
\frac{\partial L}{\partial\varphi}
&=
ml^2\ddot\varphi
+
mgl\sin\varphi
\\
\Rightarrow\qquad
\ddot\varphi
&=
-\frac{g}{l}\sin\varphi.
\qedhere
\end{align*}
\end{loesung}


