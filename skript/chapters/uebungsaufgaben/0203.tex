Bestimmen Sie unter Verwendung der Metrik in Polarkoordinaten die
Länge eines Kreisbogens vom Punkt $r=a$, $\varphi=0$ bis
zum Punkt $r=a$, $\varphi=\alpha$.

\begin{loesung}
Die Metrik in Polarkoordinaten hat die Koeffizienten
\[
g_{\mu\nu}
=
\begin{pmatrix}
1&0\\0&r^2
\end{pmatrix}.
\]
Die Kurvenlänge einer Kurve $r(t), \varphi(t)$ bestimmt man daher mit
der Formel
\begin{align*}
l
&=
\int_{t_0}^{t_1}
\sqrt{g_{\mu\nu} \dot x^\mu \dot x^\nu}\,dt
=
\int_{t_0}^{t_1}
\sqrt{\dot r(t)^2 + r(t)^2\dot\varphi(t)^2}\,dt.
\end{align*}
Als Parameter können wir den Winkel $\varphi$ verwenden.
Ausserdem ist $r(t)=r(\varphi)=a$ und damit $\dot r(\varphi)=0$.
Damit bekommen wir für die Länge
\begin{align*}
l
&=
\int_0^{\alpha} \sqrt{r(\varphi)^2\dot \varphi^2}\,d\varphi
=
\int_0^{\alpha}
a
\,d\varphi
=
\bigl[a\varphi\bigr]_0^{\alpha}
=
a\alpha.
\end{align*}
Dabei haben wir verwendet, dass $\dot\varphi=1$ ist.
\end{loesung}

