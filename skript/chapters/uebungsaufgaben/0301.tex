Betrachten Sie die Metrik
\[
g_{\mu\nu}
=
\begin{pmatrix}
1+x&0\\
0  &1
\end{pmatrix}
\]
in $x$-$y$-Koordinaten in einer Umgebung des Nullpunktes.
\begin{teilaufgaben}
\item
Berechnen Sie die Geodätengleichungen?
\item
Finden Sie eine Geodäte mit
\begin{equation}
\begin{aligned}
x(0)&=0&\dot x(0)&=1\\
y(0)&=0&\dot y(0)&=1
\end{aligned}
\label{uebung:0301:anfangswerte}
\end{equation}
\end{teilaufgaben}

\begin{loesung}
\begin{teilaufgaben}
\item
Die maschinelle Berechnung der Christoffel-Symbole mit Hilfe von
Maxima zeigt, dass alle Christoffel-Symbole zweiter Art verschwinden
ausser
\[
\Gamma^{1}_{11}=\frac1{2(x+1)}.
\]
Die Geodätengleichungen sind daher
\begin{align*}
\ddot x&=-\frac1{2(x+1)}\dot x^2
\\
\ddot y&=0
\end{align*}
Die beiden Differentialgleichungen sind enkoppelt, und können daher
unabhängig voneinander gelöst werden.
Eine erste Konsequenz davon ist, dass die Koordinatenlinien
Geodäten sind.

Die zweite Differentialgleichung lässt sich sofort lösen, es gilt
\[
y(t) = y(0) + \dot y(0)\cdot t.
\]
Die erste Differentialgleichung ist etwas komplizierter.
Man kann aber eine Lösung mit einem Exponentialansatz finden.
Dazu vereinfacht man die Gleichung erst, indem man $X(t)=x(t)+1$ schreibt.
Die Differentialgleichung wird dann zu
\begin{equation}
-2\ddot X(t) X(t) = \dot X(t)^2.
\label{uebung:0301:Dgl}
\end{equation}
Eine offensichtliche Lösung ist $X(t)=\operatorname{const}$, sie gehört
zur Anfangsbedingung $\dot X(0)=0$ und führt auf eine Koordinatenlinie,
wir haben bereits darauf hingewiesen, dass die Koordinatenlinien
Lösungen der Geodätengleichungen sind.

Wir suchen eine weitere Lösung und versuchen einen Potenzansatz
in der Form $X(t)=Ct^\alpha$.
Wir berechnen die Ableitungen
\begin{align*}
\dot  X(t) &= C\alpha t^{\alpha - 1}
\\
\ddot X(t) &= C\alpha(\alpha - 1) t^{\alpha - 2}.
\end{align*}
Setzt man dies in die Differentialgleichung~\eqref{uebung:0301:Dgl}
ein, erhält man
\[
-2C^2\alpha(\alpha - 1) t^{2\alpha-2} = C^2\alpha^2 t^{2\alpha - 2}.
\]
Division durch $C^2\alpha t^{2\alpha-2}$ ergibt die Gleichung
\[
-2(\alpha-1)=\alpha
\qquad\Rightarrow\qquad
2=3\alpha
\qquad\Rightarrow\qquad
\alpha=\frac23.
\]
Somit bekommt man die Lösung $X(t)=Ct^{\frac23}$.
Die Lösung $x(t)$ ist $x(t)=Ct^{\frac23}-1$.
\item
Gesucht wird eine Lösung mit Anfangspunkt $x=x_0$ und $\dot x=v_0$.
Nach der eben gefundenen Lösung $x(t)=Ct^{\frac23}$ muss zur Zeit $t_0$ die
Gleichungen
\begin{equation}
\begin{aligned}
x_0&=Ct_0^{\frac23}-1&&\text{und}& v_0 &= C\frac23 t_0^{-\frac13},
\end{aligned}
\end{equation}
Bildet man den Quotienten findet man
\[
\frac{x_0+1}{v_0}
=
\frac32t_0
\qquad\Rightarrow\qquad
t_0=\frac{2(x_0+1)}{3v_0}.
\]
Durch Einsetzen von $t_0$ in die erste Gleichung liefert $C$
\[
C
=
(x_0+1)t_0^{-\frac23}
=
(x_0+1)\biggl(\frac{2(x_0+1)}{3v_0}\biggr)^{-\frac23}
\]
Die allgemeine Lösung ist daher
\[
x(t)
=
(x_0+1)\biggl(\frac{2(x_0+1)}{3v_0}\biggr)^{-\frac23}
\biggl( t + \frac{2(x_0+1)}{3v_0} \biggr)^{\frac23} - 1.
\]
Setzen wir die Anfangswerte~\eqref{uebung:0301:anfangswerte} ein, erhalten
wir
\begin{align*}
y(t)&=t\\
x(t)&=\biggl(\frac{2}{3}\biggr)^{-\frac23}\biggl(t + \frac23\biggr)^{\frac23} - 1.
\end{align*}
Das Resultat kann durch Einsetzen in die Geodätengleichung überprüft werden.
\qedhere
\end{teilaufgaben}
\end{loesung}

