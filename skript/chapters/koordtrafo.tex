%
% koordtrafo.tex -- Koordinatentransformationen
%
% (c) 2017 Prof Dr Andreas Müller, Hochschule Rapperswil
%
\section{Koordinatentransformation}
\rhead{Koordinatentransformation}
Für die Tensorrechnung ist entscheidend, dass für jede Grösse das Verhalten
bei Koordinatentransformation bekannt ist.
In diesem Abschnitt sollen die Transformationsregeln für die
Christoffelsymbole zusammengestellt werden.
Die Zusammenhangskoeffizienten sind keine Tensoren, vielmehr gilt für sie das
Transformationsgesetz
\[
\bar\Gamma^\varrho_{\beta\gamma}
\frac{\partial y^\alpha}{\partial x^\varrho}
=
\Gamma^\alpha_{\mu\nu}
\frac{\partial y^\mu}{\partial x^\beta}
\frac{\partial y^\nu}{\partial x^\gamma}
+
\frac{\partial^2}{\partial x\partial x}
\frac{\partial y}{\partial x},
\]
wie wir im Folgenden nachweisen wollen.
Dazu führen wir die Rechnung zunächst für die Christoffelsymbole
erster Art durch, die Christoffelsymbole zweiter Art unterschieden sich
ja nur durch das Hochziehen eines Index.

\subsection{Transformationsformel für die Ableitungen von $g_{\mu\nu}$}
Es ist bereits bekannt, dass $g_{\mu\nu}$ ein kovarianter Tensor ist.
Wir müssen die Ableitung von $\bar g_{\mu\nu}$ nach $y^\alpha$ durch
die metrischen Koeffizienten $g_{\sigma\tau}$ und ihre Ableitungen
nach $x^\sigma$ ausdrücken.
Insbesondere gilt das Transformationsgesetz
\begin{equation*}
\bar g_{\mu\nu}
=
g_{\varrho\sigma}
\frac{\partial x^\varrho}{\partial y^\mu}
\frac{\partial x^\sigma}{\partial y^\nu}
\end{equation*}
für die metrischen Koeffizienten. 
Damit kann man jetzt auch die Ableitung 
\begin{align}
\frac{\partial}{\partial y^\alpha}\bar g_{\mu\nu}
&=
\frac{\partial}{\partial y^\alpha} 
\biggl(
g_{\varrho\sigma}
\frac{\partial x^\varrho}{\partial y^\mu}
\frac{\partial x^\sigma}{\partial y^\nu}
\biggr)
=
\frac{\partial x^\tau}{\partial y^\alpha}
\frac{\partial g_{\varrho\sigma} }{\partial x^\tau}
\frac{\partial x^\varrho}{\partial y^\mu}
\frac{\partial x^\sigma}{\partial y^\nu}
+
g_{\varrho\sigma}
\frac{\partial}{\partial y^\alpha} 
\biggl(
\frac{\partial x^\varrho}{\partial y^\mu}
\frac{\partial x^\sigma}{\partial y^\nu}
\biggr)
\notag
\\
&=
\frac{\partial g_{\varrho\sigma}}{\partial x^\tau}
\frac{\partial x^\tau}{\partial y^\alpha}
\frac{\partial x^\varrho}{\partial y^\mu}
\frac{\partial x^\sigma}{\partial y^\nu}
+
g_{\varrho\sigma}
\biggl(
\frac{\partial^2 x^\varrho}{\partial y^\alpha\partial y^\mu}
\frac{\partial x^\sigma}{\partial y^\nu}
+
\frac{\partial x^\varrho}{\partial y^\mu}
\frac{\partial^2 x^\sigma}{\partial y^\alpha\partial y^\nu}
\biggr)
\label{skript:trafo:gabl}
\end{align}
berechnen.
Die zusätzlichen Terme mit zweiten Ableitungen der Koordinatentransformation
auf der rechten Seite zeigen, dass die Ableitungen der metrischen Koeffizienten
keine Tensoren sein können.

\subsection{Transformationsformel für die Christoffelsymbole erster Art}
Die Christoffelsymbole erster Art sind definiert durch
\[
\Gamma_{\alpha,\mu\nu}
=
\frac12\biggl(
\frac{\partial g_{\nu\alpha}}{\partial x^\mu}
+
\frac{\partial g_{\alpha\mu}}{\partial x^\nu}
-
\frac{\partial g_{\mu\nu}}{\partial x^\alpha}
\biggr).
\]
Die Transformationformeln~\eqref{skript:trafo:gabl}
für die Ableitungen des metrischen Tensors können jetzt dazu verwendet
werden, die Transformationsformeln für die Christoffelsymbole erster Art
zu ermitteln.
Wir wenden die Definition auf $\bar g$ an:
\begin{align}
\bar\Gamma_{\alpha,\mu\nu}
&=
\frac12\biggl(
\frac{\partial \bar g_{\nu\alpha}}{\partial y^\mu}
+
\frac{\partial \bar g_{\alpha\mu}}{\partial y^\nu}
-
\frac{\partial \bar g_{\mu\nu}}{\partial y^\alpha}
\biggr)
\notag
\\
&=
\underbrace{
\frac12\biggl(
\frac{\partial g_{\sigma\tau}}{\partial x^\varrho}
+
\frac{\partial g_{\tau\varrho}}{\partial x^\sigma}
-
\frac{\partial g_{\varrho\sigma}}{\partial x^\tau}
\biggr)
}_{\displaystyle = \Gamma_{\tau,\varrho\sigma}}
\frac{\partial x^\tau}{\partial y^\alpha}
\frac{\partial x^\varrho}{\partial y^\mu}
\frac{\partial x^\sigma}{\partial y^\nu}
\notag
\\
&\quad
+\frac12g_{\sigma\tau}
\biggl(
\frac{\partial^2 x^\sigma}{\partial y^\mu\partial y^\nu}
\frac{\partial x^\tau}{\partial y^\alpha}
+
{\color{blue}
\frac{\partial x^\sigma}{\partial y^\nu}
\frac{\partial^2 x^\tau}{\partial y^\mu\partial y^\alpha}}
\biggr)
%&&
%\begin{pmatrix}
%\mu    &\nu    &\alpha &\sigma &\tau   &\varrho
%\end{pmatrix}
\notag
\\
&\quad
+\frac12g_{\tau\varrho}
\biggl(
{\color{red}
\frac{\partial^2 x^\tau}{\partial y^\nu\partial y^\alpha}
\frac{\partial x^\varrho}{\partial y^\mu}}
+
\frac{\partial x^\tau}{\partial y^\alpha}
\frac{\partial^2 x^\varrho}{\partial y^\nu\partial y^\mu}
\biggr)
%&&
%\begin{pmatrix}
%\nu    &\alpha &\mu    &\tau   &\varrho&\sigma
%\end{pmatrix}
\notag
\\
&\quad
-\frac12g_{\varrho\sigma}
\biggl(
{\color{blue}
\frac{\partial^2 x^\varrho}{\partial y^\alpha\partial y^\mu}
\frac{\partial x^\sigma}{\partial y^\nu}}
+
{\color{red}
\frac{\partial x^\varrho}{\partial y^\mu}
\frac{\partial^2 x^\sigma}{\partial y^\alpha\partial y^\nu}}
\biggr)
\notag
%&&
%\begin{pmatrix}
%\alpha &\mu    &\nu    &\varrho&\sigma &\tau
%\end{pmatrix}
\intertext{Die farbig hervorgehobenen Terme heben sich nach geeigneter
Umbenennung von Summationsindizes weg.
Es bleibt dann
}
\bar\Gamma_{\alpha,\mu\nu}
&=
\Gamma_{\tau,\varrho\sigma}
\frac{\partial x^\tau}{\partial y^\alpha}
\frac{\partial x^\varrho}{\partial y^\mu}
\frac{\partial x^\sigma}{\partial y^\nu}
+
g_{\sigma\tau}
\frac{\partial x^\tau}{\partial y^\alpha}
\frac{\partial^2 x^\sigma}{\partial y^\mu\partial y^\nu}.
\label{skript:trafo:chr1}
\end{align}

\subsection{Transformationsformel für die Christoffelsymbole zweiter Art}
Für die Christoffelsymbole zweiter Art müssen wir nur noch auf der
linken Seite von \eqref{skript:trafo:chr1}
mit $\bar g^{\beta\alpha}$ multiplizieren.
Da $g^{\mu\nu}$ ein kontravarianter Tensor ist, gilt
\begin{equation}
\bar g^{\alpha\beta}
\frac{\partial x^\mu}{\partial y^\alpha}
\frac{\partial x^\nu}{\partial y^\beta}
=
g^{\mu\nu}
\qquad\Leftrightarrow\qquad
\bar g^{\alpha\beta}
=
g^{\mu\nu}
\frac{\partial y^\alpha}{\partial x^\mu}
\frac{\partial y^\beta}{\partial x^\nu}.
\label{skript:trafo:ginv}
\end{equation}
Wir multiplizieren das Transformationsgesetz~\eqref{skript:trafo:chr1}
mit~\eqref{skript:trafo:ginv} und erhalten
\begin{align}
\bar\Gamma^\beta_{\mu\nu}
=
\bar g^{\alpha\beta}
\bar\Gamma_{\alpha,\mu\nu}
&=
\frac{\partial y^\alpha}{\partial x^\eta}
\frac{\partial y^\beta}{\partial x^\zeta}
g^{\eta\zeta}
\Gamma_{\tau,\varrho\sigma}
\frac{\partial x^\tau}{\partial y^\alpha}
\frac{\partial x^\varrho}{\partial y^\mu}
\frac{\partial x^\sigma}{\partial y^\nu}
+
\frac{\partial y^\alpha}{\partial x^\eta}
\frac{\partial y^\beta}{\partial x^\zeta}
g^{\eta\zeta}
g_{\sigma\tau}
\frac{\partial x^\sigma}{\partial y^\alpha}
\frac{\partial^2 x^\tau}{\partial y^\mu\partial y^\nu}
\notag
\\
&=
\delta^\tau_\eta
\frac{\partial y^\beta}{\partial x^\zeta}
g^{\eta\zeta}
\Gamma_{\tau,\varrho\sigma}
\frac{\partial x^\varrho}{\partial y^\mu}
\frac{\partial x^\sigma}{\partial y^\nu}
+
\delta^\sigma_\eta
\frac{\partial y^\beta}{\partial x^\zeta}
g^{\eta\zeta}
g_{\sigma\tau}
\frac{\partial^2 x^\tau}{\partial y^\mu\partial y^\nu}
\notag
\\
&=
\frac{\partial y^\beta}{\partial x^\zeta}
g^{\tau\zeta}
\Gamma_{\tau,\varrho\sigma}
\frac{\partial x^\varrho}{\partial y^\mu}
\frac{\partial x^\sigma}{\partial y^\nu}
+
\frac{\partial y^\beta}{\partial x^\zeta}
g^{\sigma\zeta}
g_{\sigma\tau}
\frac{\partial^2 x^\tau}{\partial y^\mu\partial y^\nu}
\notag
\\
&=
\Gamma^\zeta_{\varrho\sigma}
\frac{\partial y^\beta}{\partial x^\zeta}
\frac{\partial x^\varrho}{\partial y^\mu}
\frac{\partial x^\sigma}{\partial y^\nu}
+
\frac{\partial y^\beta}{\partial x^\zeta}
\delta^\zeta_\tau
\frac{\partial^2 x^\tau}{\partial y^\mu\partial y^\nu}
\notag
\\
&=
\Gamma^\zeta_{\varrho\sigma}
\frac{\partial y^\beta}{\partial x^\zeta}
\frac{\partial x^\varrho}{\partial y^\mu}
\frac{\partial x^\sigma}{\partial y^\nu}
+
\frac{\partial y^\beta}{\partial x^\zeta}
\frac{\partial^2 x^\zeta}{\partial y^\mu\partial y^\nu}.
\label{skript:geodaeten:gmma2trafo}
\end{align}
Bei einer linearen Koordinatentransformation verschwinden die zweiten
Ableitungen und damit der zweite Term in~\eqref{skript:geodaeten:gmma2trafo}.
Nur für lineare Koordinatentransformationen verhalten sich die
Christoffelsymbole zweiter Art also wie ein Tensor.


