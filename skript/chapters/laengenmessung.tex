%
% laengenmessung.tex -- Längenmessung in einer Fläche oder in einem Raum
%
% (c) 2017 Prof Dr Andreas Müller, Hochschule Rapperswil
%
\chapter{Längenmessung
\label{skript:kruemmung:section:laengenmessung}}
\lhead{Längenmessung}
\rhead{}

Die Physik seit Galileo und Newton machte die Annahme, dass die
Geometrie des Raumes durch ein dreidimensionales rechtwinkliges
Koordinatensystem mit der Längenmessungsformel
\begin{equation}
l=\sqrt{\Delta x^2+\Delta y^2+\Delta z^2}
\label{skript:kruemmung:pytagoras}
\end{equation}
adäquat beschrieben wird.
Diese Annahme entsprach zwar der Erfahrung, doch es gab keine
Begründung dafür.
Der Philosoph Emanuel Kant konnte sich zwar keine andere Geometrie
vorstellen, doch mit den Erkenntnissen von Bolyai, Lobaschevski
und Gauss wurde klar, dass es durchaus denkbare andere Geometrien
gibt.
Bernhard Riemann hat dann auch Methoden entwickelt, wie man die
Geometrie studieren kann, indem man ausschliesslich die Längenmessung
innerhalb des Raumes verwendet.
Damit ist die Geometrie unseres Raumes nicht länger einfach das
Resultat einer axiomatischen Beschreibung, wie Euklid sie gegeben hat,
vielmehr ist sie zu einer experimentellen Wissenschaft geworden.

Da wir nicht weiter annehmen wollen, dass sich der Raum mit Hilfe
eines rechtwinkligen Koordinatensystems mit der
Längenmessung~\eqref{skript:kruemmung:pytagoras}
adäquat beschreiben lässt,
müssen wir automatisch beliebige Koordinatensysteme zulassen.
Je nach Wahl eines Koordinatensystems werden dann Vektoren, die
wir für die Beschreibung der physikalischen Gesetze benötigen,
verschiedenen Koordinaten haben.
Da alle diese Koordinatensystem gleichberechtigt sind, müssen
wir Vektoren auf einheitliche Art umrechnen können.
Ausserdem müssen die Naturgesetze so formuliert sein, dass
sie in jedem beliebigen Koordinatensystem gleich aussehen.

In diesem Kapitel müssen wir daher einen Formalismus für
Koordinatensysteme und Längenmessung entwickeln.

\section{Koordinatensystem}
\rhead{Koordinatensysteme}
In diesem Kapitel beginnen wir damit, den Begriff der Längenmessung
auf beliebige Koordinatensysteme auszudehnen.
Ein Punkt wird beschrieben durch seine Koordinaten, die wir mit
$x^\mu$ bezeichnen, wobei $\mu$ von $1$ bis $n$ läuft, $n$
ist die Dimension des Raumes. 
Die etwas ungewohnte Schreibweise für die Indizes
hat einen tieferen Grund in der Tensorrechnung und wird später
verständlich werden.
Im Folgenden werden $k$-te Potenzen einer Koordinate $x^\mu$ durch
Klammern wie in $(x^\mu)^k$ deutlich gemacht.

Uns interessiert vor allem die Beschreibung von Bahnkurven, wir möchten
zum Beispiel den Absturz eines Astronauten in ein schwarzes Loch berechnen
können.
Eine Bahnkurve erhält man, indem man die Koordinaten von der Zeit
abhängen lässt.
Eine Kurve wird also beschrieben durch Funktionen $x^\mu(t)$.

In der physikalischen Beschreibung werden meistens Vektoren wie
Geschwindigkeit und Beschleunigung verwendet.
Sie sind die Ableitung der Koordinaten eines Bahnpunktes nach
der Zeit.
Dies lässt sich direkt auch auf die Koordinaten übertragen,
wir erhalten für den Geschwindigkeitsvektor
\[
v^{\mu}(t) = \frac{dx^\mu(t)}{dt} = \dot x^\mu(t).
\]
Die Beschleunigung ist die zweite Ableitung
\[
a^\mu(t)
=
\frac{dv^\mu(t)}{dt}
=
\ddot x^\mu(t)
\]
nach dem Parameter $t$.
Natürlich haben wir diese Konstrukte in weniger verallgemeinerungsfähiger
Notation bereits im Kapitel~\ref{skript:chapter:kurven} angetroffen.

Bis jetzt arbeiten wir nur mit einem einzigen Koordinatensystem.
Die Komponenten $\dot x^\mu$ der Geschwindigkeit hängen offenbar
von der Wahl des Koordinatensystems ab.
Ein anderes Koordinatensystem ist gegeben durch
Koordinatenumrechnungsfunktionen
\[
y^1(x^1,\dots,x^n),\;\dots,\;y^n(x^1,\dots,x^n).
\]
In den $y$-Koordinaten wird die Kurve $x^\mu(t)$ durch die Funktionen
\[
t\mapsto y^\alpha(t)=y^\alpha(x^1(t),\dots,x^n(t))
\]
beschrieben werden.
Der Tangentenvektor ist die Ableitung 
\begin{equation}
\frac{dy^\alpha(t)}{dt}
=
\frac{\partial y^\alpha}{\partial x^1}\frac{dx^1(t)}{dt}
+\dots+
\frac{\partial y^\alpha}{\partial x^n}\frac{dx^n(t)}{dt}
=
\sum_{\mu=1}^n
\frac{\partial y^\alpha}{\partial x^\mu}\frac{dx^\mu(t)}{dt}
\label{skript::summenformel}
\end{equation}
nach $t$, nach der Kettenregel für Funktionen mehrere Variablen.
Wir können zwei Erkenntnisse aus der Formel~\eqref{skript::summenformel}
gewinnen.

Es ist zwar nicht möglich, vom Koordinatensystem unabhängige
Geschwindigkeitskomponenten zu definieren, doch immerhin kann man
angeben, wie die Geschwindigkeitskomponenten zwischen verschiedenen
Koordinatensystemen umzurechnen sind.
Wir leiten daraus ab, dass wir in Zukunft von jedem Objekt, welches
eine koordinatenunabhängige Bedeutung haben soll, zwar nicht
annehmen können, dass dessen Komponenten koordinatenunabhängig sind,
aber wir können fordern, dass wir immer eine Koordinatenumrechnungsformel
analog zu \eqref{skript::summenformel} aufstellen können, die nur die
ersten partiellen Ableitungen verwendet.
Solche Grössen werden wir in Kapitel~\ref{skript:chapter:geodaeten}
genauer definieren.

Die zweite Lehre ist, dass es sehr häufig vorkommen wird, dass wir
wie in 
\eqref{skript::summenformel}
über gleiche Indizes summieren müssen, die in einem Ausdruck sowohl
oben als auch unten vorkommen.
In der Tat kommt es praktisch nicht vor, dass ein oberer und unterer
Index übereinstimmt {\em ohne} dass darüber summiert werden muss.
Wir werden daher in Zukunft in dieser Situation das Summenzeichen weglassen.
Dies ist die {\em Einsteinsche Summenkonvention}.
\index{Summenkonvention!Einsteinsche}%
Mit der Summenkonvention lässt sich~\eqref{skript::summenformel} kompakter
als
\begin{equation}
\frac{dy^\alpha(t)}{dt}
=
\frac{\partial y^\alpha}{\partial x^\mu}\frac{dx^\mu(t)}{dt}
\end{equation}
schreiben.
Man beachte, dass der Index $\mu$ im Nenner der partiellen Ableitung nach
$x^\mu$ als unterer Index gilt.

%
% Abschnitt über die Metrik
%
\section{Metrik%
\label{skript:laengenmessung:section:metrik}}
\rhead{Metrik}
Die Länge einer Kurve können wir durch Zerlegen in viele kleine Teilstücke
approximieren.
Dazu unterteilen wir den Definitionsbereich $[a,b]$ des Kurvenparameters $t$
mit Hilfe der Teilpunkte
\[
a = t_0 < t_1 < \dots < t_{i-1} < t_i < t_{i+1} < \dots < t_n = b.
\]
In einem rechtwinkligen Koordinatensystem können wir dann die Länge
der Kurve als Summe der Länge der Strecken zwischen aufeinanderfolgenden
Punkten $x^\nu(t_i)$ und $x^\mu(t_{i+1})$ mit der Summe
\begin{align*}
l
&\simeq
\sum_{i=1}^n \sqrt{\sum_{\mu} (x^{\mu}(t_i)-x^{\mu}(t_{i-1}))^2}
\\
&\rightarrow
\int_{t_0}^{t_n} \sqrt{\sum_{\mu}\biggl(\frac{dx^{\mu}(t)}{dt}\biggr)^2}\,dt
\end{align*}
annähern.
Unter der Wurzel sehen wir die Quadratsummen wieder, die für den
Satz des Pythagoras charakteristisch sind.
Machen wir die Unterteilung beliebig fein, wird diese Summe in ein
Integral übergehen, wie in der zweiten Zeile angedeutet.

In einem beliebigen Koordinatensystem funktioniert dies jedoch nicht mehr.
Wir müssen zulassen, dass die Koordinaten entlang verschiedener
Achsen nicht mehr direkt der Länge entsprechen, dass wir also
entlang der Achsen Skalierungsfaktoren haben.
Weiter ist damit zu rechnen, dass auch gemischte Termen auftauchen, ein
Ausdruck der Tatsache, dass Koordinatenrichtungen nicht senkrecht aufeinander
stehen müssen.
Die allgemeinstmögliche Form einer Längenmessungsformel ist daher
\begin{equation}
l
=
\int_{t_0}^{t_1}
\sqrt{\sum_{\mu,\nu} g_{\mu\nu}(x^1(t),\dots,x^n(t)) \frac{dx^{\mu}(t)}{dt}\frac{dx^{\nu}(t)}{dt}}\,dt.
\label{skript:kruemmung:metrikformel}
\end{equation}
Die Zahlen $g_{\mu\nu}(x^1,\dots,x^n)$ hängen im Allgemeinen auch noch von
den Koordinaten ab, wir werden diese Abhängigkeit im Folgenden aber
normalerweise nicht explizit ausdrücken.

Da in der Formel~\eqref{skript:kruemmung:metrikformel} immer beide Terme
\[
g_{\mu\nu} \frac{dx^\mu(t)}{dt} \frac{dx^\nu(t)}{dt}
\qquad\text{und}\qquad
g_{\nu\mu} \frac{dx^\nu(t)}{dt} \frac{dx^\mu(t)}{dt}
=
g_{\nu\mu} \frac{dx^\mu(t)}{dt} \frac{dx^\nu(t)}{dt}
\]
vorkommen, können wir zum Beispiel durch Zusammenfassen in neue Komponenten
\[
g'_{\mu\nu} = \frac12(g_{\mu\nu}+g_{\nu\mu})
\]
erreichen, dass für die Koeffizienten $g_{\mu\nu}$ die Symmetriebedingung
\[
g_{\mu\nu}=g_{\nu\mu}
\]
gilt.

\begin{definition}
Die Zahlen $g_{\mu\nu}$ heissen der metrische Tensor.
Er ist symmetrisch, d.~h.~$g_{\mu\nu}=g_{\nu\mu}$.
\index{Tensor!metrischer}%
\index{Tensor!symmetrischer}%
\end{definition}

Die Schreibweise~\eqref{skript:kruemmung:metrikformel} ist nicht
sehr handlich.
Wesentlich an der Notation ist einzig, wie die Koeffizienten $g_{\mu\nu}$
mit den Koordinateninkrementen $dx^\mu$ kombiniert werden müssen.
Wir können daher formal auch schreiben:
\begin{equation}
ds^2
=
g_{\mu\nu}\,dx^\mu\,dx^\nu.
\label{skript:kruemmung:laengenelement}
\end{equation}
Man nennt \eqref{skript:kruemmung:laengenelement} das {\em Längenelement}
der Metrik $g_{\mu\nu}$.
\index{Längenelement}%
Diese Notation ist konsistent, denn die Längen\-berechnung ist
\begin{equation}
\int\,ds
=
\int \sqrt{g_{\mu\nu}\,dx^\mu\,dx^\nu}
=
\int \sqrt{\mathstrut g_{\mu\nu}\,\dot x^\mu\,\dot x^\nu}\,dt,
\label{skript:kruemmung:laengenformel}
\end{equation}
wozu man einzig die Konvention braucht, dass man in Integralen
$dx^\mu=\dot x^\mu\,dt$ schreiben kann.

%
% Beispiel für Längenmessung in einem Koordinatensystem mit nicht
% orthogonalen Achsen.
%
\begin{beispiel}
Wir untersuchen die Längenmessung mit nicht orthogonalen Achsen.
Statt des gewöhnli\-chen Koordinatensystems in einer Ebene verwenden
wir die Koordinaten $x'=x$ und $y'=x+y$.
In den ungestrichenen Koordinaten ist der Abstand zwischen zwei
Punkten durch den Satz von Pythagoras
\begin{equation}
l^2 = \Delta x^2 + \Delta y^2
\label{skript:kruemmung:p2}
\end{equation}
gegeben.
Um den Abstand in $x'$-$y'$-Koordinaten auszudrücken, müssen wir diese
erst wieder in $x$-$y$-Koordinaten umrechnen. 
Man findet
\[
x=x'
\qquad\text{und}\qquad
y=y'-x=y'-x'.
\]
Eingesetzt in die Formel \eqref{skript:kruemmung:p2} finden wir
\begin{align*}
l^2
&=
\Delta x^2 + \Delta y^2
=
\Delta x^{\prime 2}
+
(\Delta y'- \Delta x')^2
=
\Delta x^{\prime 2}
+
\Delta y^{\prime 2}-2\Delta x'\Delta y' + \Delta x^{\prime 2}
\\
&= 2 \Delta x^{\prime 2} - 2 \Delta x'\Delta y'+\Delta y^{\prime 2}.
\end{align*}
Die zugehörigen Koeffizienten $g_{\mu\nu}$ sind
\[
g_{11} = 2,\quad
g_{12}=g_{21}=-1\quad\text{und}\quad
g_{22}=1.
\]
Als Matrix geschrieben ist
\[
(g_{\mu\nu})
=
\begin{pmatrix}
 2&-1\\
-1& 1
\end{pmatrix}.
\]
Diese Koeffizienten beschreiben die gleiche Längenmessung in den 
gestrichenen Koordinaten wie der Satz von Pythagoras in den ursprünglichen
Koordinaten.
Man kann sie auch in der Form
\[
ds^2
=
dx^2+dy^2
=
2\,dx'^2-2\,dx'\,dy'+dy'^2
\]
als Längenelemente beschreiben.
\end{beispiel}

%Die Formel \eqref{skript:kruemmung:metrikformel} ist etwas unhandlich.
%Wir können aber wieder die Einsteinsche Summenkonvention verwenden,.
%um das Summenzeichen los zu werden..
%Ausserdem kann man die Ableitungen nach $t$ auch mit einem Punkt abkürzen.
%Damit lässt sich die Längenmessung daher etwas kompakter als
%\[
%l=\int_{t_0}^{t_1} \sqrt{g_{\mu\nu}\dot x^{\mu}(t) \dot x^{\nu}(t)}\,dt
%\]
%schreiben.

\section{Beispiele}
\rhead{Beispiele}
Wir betrachten drei für die Anwendungen wichtige Bespiele von
Koordinatensystemen des gewöhnlichen dreidimensionalen Raumes
und berechnen die zugehörigen metrischen Tensoren.

\subsection{Polarkoordinaten}
Punkte in der Ebene können statt in rechtwinkligen $x$-$y$-Koordinaten
auch mit Hilfe von Polarkoordinaten $(r,\varphi)$ nach der Umrechnungsregel
\begin{align*}
x&=r\cos\varphi\\
y&=r\sin\varphi
\end{align*}
beschrieben werden.
Um den metrischen Tensor zu bestimmen, müssen die Ableitungen von $x$ 
und $y$ nach $t$ durch Ableitungen von $r$ und $\varphi$ nach $t$ 
ausdrücken.
Die Produktregel liefert:
\begin{align*}
\dot x&= \dot r\cos \varphi - r\dot\varphi \sin\varphi 
\\
\dot y&= \dot r\sin\varphi + r\dot\varphi\cos\varphi.
\end{align*}
Eingesetzt in den Satz von Pythagoras folgt
\begin{align*}
\dot x^2 + \dot y^2
&=
\dot r^2\cos^2\varphi -2r\dot r\dot\varphi\cos\varphi\sin\varphi +r^2\dot \varphi^2\sin^2\varphi
+
\dot r^2\sin^2\varphi +2r\dot r\dot\varphi\sin\varphi\cos\varphi +r^2\dot\varphi^2\cos^2\varphi
\\
&=
\dot r^2(\cos^2\varphi+\sin^2\varphi)+ r^2\dot\varphi^2(\sin^2\varphi+\cos^2\varphi)
\\
&=\dot r^2 + r^2\dot\varphi^2.
\end{align*}
Die gemischten Terme haben sich weggehoben.
Man liest daraus für die Koeffizienten des metrischen Tensors
\[
g_{11}=1,\qquad g_{12}=g_{21}=0\qquad\text{und}\qquad g_{22}=r^2
\]
ab.
Das Längenelement in Polarkoordinaten ist daher
\[
ds^2
=
dr^2 + r^2\,d\varphi^2.
\]
\index{Längenelement!in Polarkoordinaten}%
In Matrix Form ist
\[
(g_{\mu\nu})
=
\begin{pmatrix}
1&0\\0&r^2
\end{pmatrix}.
\]
Die Diagonalform der Matrix bedeutet, dass sich die Koordinatenlinien
in Polarkoordinaten orthogonal schneiden.

Die metrischen Koeffizienten $g_{\mu\nu}$ hängen zwar von den Koordinaten ab,
doch bedeutet das noch nicht, dass ein gekrümmter Raum vorliegt.
Diese $g_{\mu\nu}$ beschreiben ja die gleiche Längenmessung wie der Satz
des Pythagoras in $x$-$y$-Koordinaten.

\subsection{Zylinderkoordinaten}
Zylinderkoordinaten beschreiben die Punkte des dreidimensionalen
Raumes mit Polarkoordinaten in der $x$-$y$-Ebene und der $z$-Koordinate.
Da wir die Metrik in der $x$-$y$-Ebene schon durch $(r,\varphi)$
ausgedrückt haben, können wir auch die Metrik in Polarkoordinaten
bekommen, indem wir die $z$-Koordinaten ergänzen:
\[
\dot x^2+\dot y^2 +\dot z^2
=
\dot r^2 + r^2\dot\varphi^2 + \dot z^2.
\]
Der zugehörige metrische Tensor hat daher die Koeffizienten
\[
g_{11}=1,\qquad
g_{22}=r^2
\qquad\text{und}\qquad
g_{33}=1,
\]
alle anderen Koeffizienten sind $0$.
Das Längenelement in Zylinderkoordinaten ist
\[
ds^2
=
dr^2+r^2\,d\varphi^2 + dz^2.
\]
\index{Längenelement!in Zylinderkoordinaten}%
Die Matrixform ist
\[
(g_{\mu\nu})
=
\begin{pmatrix}
1&  0&0\\
0&r^2&0\\
0&  0&1
\end{pmatrix}.
\]

\subsection{Zylinderoberfläche}
Beschränken wir uns auf die Punkte im Abstand $1$ zur $z$-Achse, erhalten
wir eine Zylinderfläche, welche mit Koordinaten $\varphi$ und $z$
beschrieben werden kann.
Die Längenmessung in dieser Fläche wird durch den metrischen
Tensor der Zylinderkoordinaten beschrieben, in dem wir $r=1$ einsetzen.
Wir erhalten
\[
\dot\varphi^2+\dot z^2.
\]
Das Längenelement auf der Zylinderoberfläche $r=1$ mit den Koordinaten
$\varphi$ und $z$ ist daher
\[
ds^2
=
d\varphi^2+dz^2,
\]
\index{Längenelement!auf der Zylinderoberfläche}%
die Koeffizienten des metrischen Tensors sind
\[
g_{11}=1,\qquad
g_{12}=g_{21}=0
\qquad\text{und}\qquad
g_{22}=1,
\]
oder
\[
(g_{\mu\nu})
=
\begin{pmatrix}
1&0\\0&1
\end{pmatrix}
\]
in Matrixform.
Dies ist der metrische Tensor einer Ebene mit rechtwinkligen Koordinaten.
Daraus können wir bereits ablesen, dass die Zylinderoberfläche sich durch
Längenmessung nicht von einer Ebene unterscheiden lässt.

\subsection{Kugelkoordinaten}
Kugelkoordinaten beschreiben die Punkte eines dreidimensionalen Raumes
durch die Entfernung $r$ vom Nullpunkt, die geographische Länge
$\varphi$, die von
der $x$-Koordinate gemessen wird, und durch die geographische Breite
$\vartheta$,
die als Winkel von der $z$-Achse gemessen wird.
Die Umrechnung in kartesische Koordinaten erfolgt mit den Formeln
\begin{align*}
x&= r\sin\vartheta\cos\varphi\\
y&= r\sin\vartheta\sin\varphi\\
z&= r\cos\vartheta
\end{align*}
Wir bestimmen wieder die Koeffizienten des metrischen Tensors.
Dazu leiten wir zunächst nach $t$ ab und erhalten
die Geschwindigkeitskomponenten
\begin{align*}
\dot x
&=
\dot r\sin\vartheta\cos\varphi
+
r\dot\vartheta \cos\vartheta\cos\varphi
-
r\dot\varphi \sin\vartheta\sin\varphi
\\
\dot y
&=
\dot r\sin\vartheta\sin\varphi
+
r\dot\vartheta\cos\vartheta\sin\varphi
+
r\dot\varphi\sin\vartheta\cos\varphi
\\
\dot z
&=
\dot r\cos\vartheta
-
r\dot\vartheta \sin\vartheta.
\end{align*}
Für die Berechnung der Länge müssen wir das Skalarprodukt dieses
Vektors mit sich selbst bilden.
Wegen der verschiedenen Vorzeichen des letzten Terms im Ausdruck für
$\dot x$ und $\dot y$  werden sich viele Terme gegenseitig wegheben.
Die etwas mühselige Rechnung ergibt für den Ausdruck
$\dot x^2 + \dot y^2 + \dot z^2$ 
\[
\begin{array}{clclclcl}
 &
\dot r^2\sin^2\vartheta\cos^2\varphi
	&+&r^2\dot\vartheta^2\cos^2\vartheta\cos^2\varphi
		&+&r^2\dot\varphi^2\sin^2\vartheta\sin^2\varphi
			&+&2r\dot r\dot\vartheta\sin\vartheta\cos\vartheta\cos^2\varphi
\\
+&
\dot r^2\sin^2\vartheta\sin^2\varphi
	&+&r^2\dot\vartheta^2\cos^2\vartheta\sin^2\varphi
		&+&r^2\dot\varphi^2\sin^2\vartheta\cos^2\varphi
			&+&2r\dot r\dot\vartheta\sin\vartheta\cos\vartheta\sin^2\varphi
\\
+&
\dot r^2\cos^2\vartheta
	&+&r^2\dot\vartheta^2\sin^2\vartheta
		& &
			&-&2r\dot r\dot\vartheta \sin\vartheta \cos\vartheta.
\\
\end{array}
\]
In den Spalten ergänzen sich in den ersten beiden Zeilen jeweils
$\cos^2\varphi$ und $\sin^2\varphi$ zu $1$.
In den ersten beiden Spalten lassen sich danach auch noch
$\cos^2\vartheta$ und $\sin^2\vartheta$ zu $1$ zusammenfassen,
während sich die Terme in der vierten Spalte aufheben.
Damit bekommt man
\begin{equation}
\dot x^2 + \dot y^2 + \dot z^2
=
\dot r^2+r^2\dot\vartheta^2 + r^2\dot\varphi^2\sin^2\vartheta
\label{skript:kruemmung:kugelkoordinaten}
\end{equation}
für die Längenmessung, und
\[
g_{11}=1,\qquad
g_{22}=r^2
\qquad\text{und}\qquad
g_{33}= r^2\sin^2\vartheta
\]
für die nicht verschwindenden Komponenten des metrischen Tensors,
oder
\[
(g_{\mu\nu})
=
\begin{pmatrix}
1&  0&       0          \\
0&r^2&       0          \\
0&  0&r^2\sin^2\vartheta
\end{pmatrix}
\]
in Matrixform.
Alternativ kann man dies auch als
\begin{equation}
ds^2
=
dr^2 + r^2\,d\vartheta^2 + r^2\sin^2\vartheta\,d\varphi^2,
\label{skript:laengenmessung:kugelkoordinaten}
\end{equation}
also als Längenelement schreiben.
\index{Längenelement!in Kugelkoordinaten}%

\subsection{Kugeloberfläche}
Aus dem metrischen Tensor der Kugelkoordinaten lässt sich durch festhalten
des Radius der metrische Tensor einer Kugeloberfläche ableiten.
Aus dem Ausruck \eqref{skript:kruemmung:kugelkoordinaten}
erhalten wir
\[
\dot\vartheta^2+\dot\varphi^2\sin^2\vartheta.
\]
Der $\sin$-Term deutet an, dass wir hier nicht mehr direkt eine flache
Metrik haben.
Den Nachweis können wir aber erst führen, wenn wir den Begriff der
Krümmung zur Verfügung haben.
Für die nicht verschwindenden Komponenten des metrischen Tensors
finden wir
\[
g_{11} = 1
\qquad\text{und}\qquad
g_{22}=\sin^2\vartheta
\]
oder
\[
(g_{\mu\nu})
=
\begin{pmatrix}
1&0\\
0&\sin^2\vartheta
\end{pmatrix}
\]
in Matrixform.
Etwas übersichtlicher ist 
\begin{equation}
ds^2
=
d\vartheta^2 + \sin^2\vartheta\,d\varphi^2,
\label{skript:laengenmessung:kugeloberflaeche}
\end{equation}
das Längenelement auf der Kugeloberfläche.
\index{Längenelement!auf der Kugeloberfläche}%

Man erkennt, dass der Koeffizient $g_{22}$ bei $\vartheta \in\{0,\pi\}$
verschwindet.
Man spricht von einer Singularität der Längenmessung an den Polen der Kugel.
Dies ist jedoch nur ein Artefakt der Tatsache, dass die Kugelkoordinaten
bei den Polen nicht mehr eindeutig sind.
Zur Beschreibung der Pole sind alle möglichen Werte der geographischen
Länge gleichermassen geeignet.

Die Metrik 
\eqref{skript:laengenmessung:kugelkoordinaten}
des mit Kugelkoordinaten beschriebenen dreidimensionalen Raumes
ist nahe verwandt mit der Metrik
\eqref{skript:laengenmessung:kugeloberflaeche}
der Kugeloberfläche.
Es ist jedoch ganz offensichtlich, dass die Kugeloberfläche sich ganz
offensichtlich durch ihre Krümmung vom dreidimensionalen Raum unterscheidet.
Die Ähnlichkeit ist also nicht ausreichend, um die Krümmung festzustellen,
dazu müssen weitergehende Hilfsmittel in Kapitel~\ref{skript:chapter:kruemmung}
entwickelt werden.

\section{Übungsaufgaben}
\rhead{Übungsaufgaben}
\uebungsaufgabe{0202}
\uebungsaufgabe{0203}
\uebungsaufgabe{0201}

