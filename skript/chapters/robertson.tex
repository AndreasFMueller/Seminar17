%
% robertson.tex -- Robertson Walker Metrik
%
% (c) 2017 Prof Dr Andreas Müller, Hochschule Rapperswil
%
\chapter{Kosmologie%
\label{skript:chapter:kosmologie}}
\lhead{Kosmologie}
\rhead{}
Die Einstein-Gleichungen ermöglichen erstmals, die Entwicklung
des ganzen Universums zu modellieren.
Dazu sind allerdings die Kenntnis der grossräumigen
Masseverteilung notwendig.
Es stellt sich heraus, dass diese im grossen isotrop und homogen ist.
Dies bedeutet, dass auch die Krümmung im Universum homogen und isotrop
ist.
Die Robertson-Walker-Metrik erfüllt diese Voraussetzung.

\section{Expansion des Universums}
\rhead{Expansion}
Zu Beginn des 20.~Jahrhunderts hatte man noch keine klare Vorstellung
über die Grösse des Universums.
Allgemein wurde angenommen, dass unsere Milchstrasse im wesentlichen das
ganze Universum ist.
Die seit längerer Zeit bekannten Nebel wurden als Teil der Milchstrasse
angesehen.
Auch wurde das Universum im wesentlichen als unveränderlich angesehen.

In den frühen zwanziger Jahren des 20.~Jahrhunderts gelang es dann
Edwin Hubble, die Entfernung von Andromeda und anderen heute als
Galaxien bekannten Nebeln zu bestimmen.
Er verwendete dafür sogenannte Cepheiden-Veränderliche.
Zwischen der absoluten Helligkeit dieser veränderlichen
Sterne und der Periode der Helligkeitsschwankungen besteht
eine einfache Beziehung.
Hubble fand Cepheiden-Veränderliche in Aufnahmen der Andromeda-Galaxie
und war damit in der Lage, deren absolute Helligkeit zu bestimmen.
Aus der beobachteten Helligkeit konnte er die Entfernung ableiten.
Es zeigte sich, dass diese Galaxien weit ausserhalb unserer Milchstrasse
befinden.

Hubble und sein Assistent Milton Humason massen mit dem neuen
2.5m-Hooker-Teleskop neben der Entfernung auch die Spektren und
damit die Rotverschiebung von 46 Galaxien.
Es zeigte sich, dass sich weiter entfernte Galaxien schneller 
entfernen.
Sie schlossen daraus, dass das Universum expandieren muss, sie konnten
sogar die Geschwindigkeit bestimmen, mit der dies geschieht.

Der belgische Physiker
Georges Lemaître hatte auf der Basis von Einsteins allgemeiner
Relativitätstheorie die Expansion des Universums bereits vor
Hubbles Entdeckung vorhergesagt.
Doch erst mit die Messungen von Hubble und Humason haben bewiesen,
dass Einsteins Theorie die Entwicklung des Universums korrekt
vorhersagen kann.

\section{Masseverteilung im Universum}
\rhead{Masseverteilung im Universum}
Lange Zeit galt die Erde als das Zentrum des Universums.
Nikolaus Kopernikus hat mit dem heliozentrischen Modell der Erde
diese privilegierte Stellung genommen.
Wilhelm Herschel war der erste Astronom, der sich darüber Gedanken
gemacht hat, wo innerhalb unserer Milchstrasse unser Sonnensystem
einzuordnen wäre.
Heute wissen wir, dass die Erde eher am Rande der Milchstrasse liegt.
Doch auch die Milchstrasse ist nichts besonderes, sie ist eine eher
unterdurchschnittliche Galaxie in einem durchschnittlichen
Galaxienhaufen.
Man kann diese Beobachtungen zusammenfassen im sogenannten
{\em kosmologischen Prinzip}, welches besagt, dass an unserem
Standort im Universum nichts speziell ist.

\subsection{Isotropie}
Isotropie bedeutet, dass es im Universum keine bevorzugte Richtung
gibt.
Auf kurze Distanzen stimmt dies ganz offensichtlich nicht.
In einer Kugel von wenigen Metern Durchmesser um den Beobachter
gibt es eine objektiv bevorzugte Richtung, die Richtung des Schwerfeldes
der Erde zum Erdmittelpunkt,
Aber auch in einer Kugel von etwa einer Milliarde Kilometer um
den Beobachter gibt es eine bevorzugte Richtung, nämlich die
Richtung zum hellsten und massivsten Stern, der Sonne.
Selbst in einer Kugel von etwa 3 Mpc\footnote{Mpc = 1 Million parsec,
ein parsec entspricht $30.857\cdot 10^{16}\text{m} = 3.26\text{Lichtjahre}$.}
kann man eine bevorzugte Richtung finden, nämlich die Richtung zur grössten
und hellsten Galaxie des lokalen Haufens, der Andromeda-Galaxie M31.
Erst in einer Kugel mit einem Radius von etwa 100Mpc gibt es keine
bevorzugte Richtung mehr. 

\subsection{Homogenität}

\subsection{Energie-Impuls-Tensor}

\section{Robertson-Walker-Metrik}
\rhead{Robertson-Walker-Metrik}
Die Robertson-Walker-Metrik wurde entwickelt, um ein homogenes,
isotropes und expandierendes Universum zu modellieren.
Wir beginnen mit einem flachen Universum mit der
Metrik
\[
ds^2
=
-c^2\,dt^2 + dr^2 + r^2\,d\vartheta^2 + r^2\sin^2\vartheta \,d\varphi^2
=
-c^2\,dt^2 + dr^2 + d\Omega^2.
\]
Wenn dieses Universum expandiert, ändert sich die Längenmessung für die
Raum-Koordinaten, nicht aber für die Zeitkoordinate.
Wir können dies mit einem zeitabhängigen Skalierungsfaktor $a(t)$ 
modellieren:
\[
ds^2
=
-c^2\,dt^2 + a(t)\,dr^2 + a(t)r^2\,d\vartheta^2 + a(t)r^2 \sin^2\vartheta\,d\varphi^2
=
-c^2\,dt^2 + a(t)\,dr^2 + a(t)r^2\,d\Omega^2.
\]
Natürlich bleibt dies ein flaches Universum.

Wir müssen aber auch zulassen, dass das Universum gekrümmt ist.
Dies bedeutet, dass der Umfang eines Kreises um den Nullpunkt
nicht proportional mit $r$ wächst.
Alternativ kann man auch sagen, dass die $r$-Koordinate eines Punktes 
eines Kreises vom Umfang $2\pi R$ nicht unbedingt $R$ sein muss, also
$r \ne R$.
Wir müssen also den Term $dr^2$ ersetzen durch etwas, was mit zunehmendem
$r$ grösser oder kleiner als $1$ sein wird.
\[
ds^2
=
-c^2\,dt^2
+ a(t)^2 \biggl(
\frac{R^2}{R^2-kr^2} dr^2
+
r^2\, d\Omega^2
\biggr)
\]

Der Faktor $a(t)$ heisst der Skalierungsfaktor, er beschreibt, wie stark
das Universum sich zur Zeit $t$ bereits gestreckt hat.

Der Einstein-Tensor dieser Metrik kann in einer sehr langwierigen oder
noch besser mechanischen Rechnung ermittelt werden.
Er ist diagonal und hat die folgenden Diagonal-Elemente:
\begin{align*}
G_{00}
&=
\frac{3}{4}\biggl(\frac{\dot a(t)}{a(t)}\biggr)^2
\\
G_{11}
&=
-\ddot a(t) +\frac{\dot a(t)^2}{4a(t)}
\\
G_{22}
&=
G_{11} r^2
\\
G_{33}
&=
G_{11} r^2\sin^2\vartheta
\end{align*}
In den Ausdrücken für $G_{22}$ und $G_{33}$ fällt auf, dass sie gegenüber
$G_{11}$ nur die zusätzlichen Faktoren enthalten, die auch in der Definition
der Robertson-Walker-Metrik evident sind.
Um die Feldgleichungen aufzustellen reicht es daher, mit $G_{00}$ und
$G_{11}$ zu arbeiten.
Dies wird später auf die Friedmann-Gleichungen führen.

\section{Steady State Universum}
\rhead{Steady State Universum}
Die Schlussfolgerung, dass das Universum vor relativ kurzer Zeit
aus einem heissen, dichten 
Zustand hervorgegangen sein müsse, war nicht von Anfang an akzeptiert.
Als alternative Hypothese wurde vorgeschlagen, dass im Universum 
zwar expandiert, dies aber schon seit ewigen Zeiten tut, und dass die
enstehenden
Lücken zwischen den Galaxien dadurch gefüllt werden, dass ständig
neue Materie entstand.
Dazu müsste pro Kubikkilometer und Jahr ein neues Wasserstoffatom
entstehen.

Diese Steady-State genannte Hypothese widerspricht jedoch zwei
im Folgenden diskutierten Beobachtungen.

\subsection{Olbers Paradoxon}
Wenn das Universum unendlich alt und gross ist, dann wird man in jeder
beliebigen Richtung nicht beliebig weit sehen können, sondern wird
irgendwann auf einen Stern treffen.
Der ganze Himmel müsste daher die Helligkeit von Sternen haben.

Nimmt man dagegen an, dass das Universum relativ jung ist, dann kann
man wegen der Endlichkeit der Lichtgeschwindigkeit nur soweit sehen,
wie das Alter des Universums erlaubt.

\subsection{Der kosmische Mikrowellenhintergrund}





