%
% robertson.tex -- Robertson Walker Metrik
%
% (c) 2017 Prof Dr Andreas Müller, Hochschule Rapperswil
%
\chapter{Robertson-Walker-Metrik%
\label{skript:chapter:robertson}}
\lhead{Robertson-Walker-Metrik}
\rhead{}
Die Robertson-Walker-Metrik wurde entwickelt, um ein homogenes,
isotropes und expandierendes Universum zu modellieren.
Wir beginnen mit einem flachen Universum mit der
Metrik
\[
ds^2
=
-c^2\,dt^2 + dr^2 + r^2\,d\vartheta^2 + r^2\sin^2\vartheta \,d\varphi^2
=
-c^2\,dt^2 + dr^2 + d\Omega^2.
\]
Wenn dieses Universum expandiert, ändert sich die Längenmessung für die
Raum-Koordinaten, nicht aber für die Zeitkoordinate.
Wir können dies mit einem zeitabhängigen Skalierungsfaktor $a(t)$ 
modellieren:
\[
ds^2
=
-c^2\,dt^2 + a(t)\,dr^2 + a(t)r^2\,d\vartheta^2 + a(t)r^2 \sin^2\vartheta\,d\varphi^2
=
-c^2\,dt^2 + a(t)\,dr^2 + a(t)r^2\,d\Omega^2.
\]
Natürlich bleibt dies ein flaches Universum.

Wir müssen aber auch zulassen, dass das Universum gekrümmt ist.
Dies bedeutet, dass der Umfang eines Kreises um den Nullpunkt
nicht proportional mit $r$ wächst.
Alternativ kann man auch sagen, dass die $r$-Koorindate eines Punktes 
eines Kreises vom Umfang $2\pi R$ nicht unbedingt $R$ sein muss, also
$r \ne R$.
Wir müssen also den Term $dr^2$ ersetzen durch etwas, was mit zunehmendem
$r$ grösser oder kleiner als $1$ sein wird.
\[
ds^2
=
-c^2\,dt^2
+ a(t)^2 \biggl(
\frac{R^2}{R^2-kr^2} dr^2
+
r^2\, d\Omega^2
\biggr)
\]

Der Faktor $a(t)$ heisst der Skalierungsfaktor, er beschreibt, wie stark
das Universum sich zur Zeit $t$ bereits gestreckt hat.

Der Einstein-Tensor dieser Metrik kann in einer sehr langwierigen oder
noch besser mechanischen Rechnung ermittelt werden.
Er ist diagonal und hat die folgenden Diagonal-Elemente:
\begin{align*}
G_{00}
&=
\frac{3}{4}\biggl(\frac{\dot a(t)}{a(t)}\biggr)^2
\\
G_{11}
&=
-\ddot a(t) +\frac{\dot a(t)^2}{4a(t)}
\\
G_{22}
&=
G_{11} r^2
\\
G_{33}
&=
G_{11} r^2\sin^2\vartheta
\end{align*}
In den Ausdrücken für $G_{22}$ und $G_{33}$ fällt auf, dass sie gegenüber
$G_{11}$ nur die zusätzlichen Faktoren enthalten, die auch in der Definition
der Robertson-Walker-Metrik evident sind.
Um die Feldgleichungen aufzustellen reicht es daher, mit $G_{00}$ und
$G_{11}$ zu arbeiten.
Dies wird später auf die Friedmann-Gleichungen führen.



