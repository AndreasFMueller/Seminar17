%
% kruemmung.tex
%
% (c) 2017 Prof Dr Andreas Müller, Hochschule Rapperswil
%
\chapter{Krümmung\label{skript:chapter:kruemmung}}
\lhead{Krümmung}
\rhead{}
Das Konzept der Krümmung einer Kurve wird bereits im ersten Semester
in der Analysis eingeführt.
Die Kurve kann zum Beispiel durch die Länge entlang der Kurve
parametrisiert werden.
Durch Festlegung eines Nullpunkts der Längenmessung erhält die
Kurve damit ein Koordinatensystem, welches die Punkte auf der
Kurve eindeutig identifizieren lässt.
Es stellt sich heraus, dass Krümmung eine Eigenschaft der Einbettung
einer Kurve in einen höherdimensionalen Raum ist.
Durch Messungen allein innerhalb der Kurve, also durch Verwendung
des eben beschriebenen Koordinatensystems lassen sich verschiedene
gekrümmte Kurven nicht unterscheiden.

Dies ändert in höheren Dimension.
Eine Fläche kann zwar immer noch auf ganz verschiedene Art in
einen dreidimensionalen Raum eingebettet werden.
Ein Stück Papier kann man sich zum Beispiel als Ausschnitt einer
Ebene vorstellen, man kann es aber auch zu einem Zylinder oder
Kegel zusammenrollen.
Durch Messungen allein innerhalb der Fläche, sind diese verschiedenen
Einbettungen nicht unterscheidbar.
Eine Ameise auf der Fläche könnte keinen Unterschied feststellen.
Eine Kugeloberfläche dagegen liesse sich sehr wohl durch Messungen
allein innerhalb der Fläche unterscheiden.
Misst man zum Beispiel von einem Kreis um einen Punkt den Umfang,
dann stellt man fest, dass der auf einer Ebene der Umfang immer $2\pi r$ ist.
Auf einer Kugeloberfläche dagegen ist der Umfang kleiner.
Der Radius wird ja nicht in gerader Linie in einer Eben gemessen,
sondern entlang einer gekrümmten Linie, die sich von der Tangentialebene
entfernt.

Die Physik seit Galileo und Newton machte die Annahme, dass die
Geometrie des Raumes durch ein dreidimensionales rechtwinkliges
Koordinatensystem mit der Längenmessungsformel
\begin{equation}
l=\sqrt{\Delta x^2+\Delta y^2+\Delta z^2}
\label{skript:kruemmung:pytagoras}
\end{equation}
adäquat beschrieben wird.
Diese Annahme entsprach zwar der Erfahrung, doch es gab keine
Begründung dafür.
Der Philosoph Emanuel Kant konnte sich zwar keine andere Geometrie
vorstellen, doch mit den Erkenntnissen von Bolyai, Lobaschevski
und Gauss wurde klar, dass es durchaus denkbare andere Geometrien
gibt.
Bernhard Riemann hat dann auch Methoden entwickelt, wie man die
Geometrie studieren kann, indem man ausschliesslich die Längenmessung
innerhalb des Raumes verwendet.
Damit ist die Geometrie unseres Raumes nicht länger einfach das
Resultat einer axiomatischen Beschreibung, wie Euklid sie gegeben hat,
vielmehr ist sie zu einer experimentellen Wissenschaft geworden.

In diesem Kapitel wird daher untersucht, wie die Geometrie des Raumes
mit der Längenmessung zusammenhängt und wie Krümmung charakterisiert
werden kann.
Dazu wird der Begriff der Geodäten verwendet, der auch zum Beispiel
in der Vermessung eine Rolle spielt.
Auch die Ausbreitung des Lichts in einem Medium ist einer solchen
Beschreibung zugänglich.
Das Licht wählt immer den Weg mit der geringsten Laufzeit, nicht
unbedingt den geometrisch kürzesten Weg.
Daher können sich Lichtstrahlen in einem inhomogenen Medium
krümmen.
Aus der Perspektive des Lichtes ist aber nicht die Bahn gekrümmt,
es folgt der in dieser Geometrie geradest möglichen Bahn.
Nicht die Bahn ist gekrümmt, sondern die Längenmessung weicht von
der üblichen \eqref{skript:kruemmung:pytagoras} ab, der Raum ist
gekrümmt.

Schon aus diesen Beispielen wird klar, dass man sich einen gekrümmten
Raum nicht unbedingt als etwas vorstellen muss, was ``krumm'' in einen
grösseren Raum eingebettet ist. 
Es ist zwar so, das auf diese Weise gekrümmte Räume entstehen können,
aber das allgemeine Verständnis sagt einfach nur, dass die Längenmessung
nicht so funktioniert wie \eqref{skript:kruemmung:pytagoras} suggeriert.

Wir werden wie folgt vorgehen.
Im ersten Abschnitt wird die Krümmung in einer Dimension definiert,
es geht hier vor allem darum daran zu erinnern, wie die zweiten
Ableitungen eingehen.
Im zweiten Abschnitt wird der Begriff der Längenmessung auf einer
Fläche und allgemein in einem höherdimensionalen Raum beschrieben
und an einigen Bespielen studiert.
Der kürzeste Weg zwischen zwei Punkten heisst Geodäte, die Gleichung
einer Geodäten wird im dritten Abschnitt hergeleitet.
Dazu entwickeln wir einen umfangreichen Formalismus, der später auch
zur Berechnung der Krümmung nützlich sein wird.
Im vierten Abschnitt untersuchen wir, wie der Versuch, einen Tangentialvektor
entlang einer Kurve zu transportieren, in gewissen Fällen vom 
Weg abhängt.
Wir werden dies als ein Anzeichen von Krümmung werten.
Es folgt die Definition einer Grösse, mit der man die Krümmung messen
kann.

In den folgenden Abschnitten geben wir einen Ausblick darauf, wie 
die moderne Physik unseren Raum als einen gekrümmten Raum beschreibt.
An einfachen Modellen soll gezeigt werden, wie man sich die Gravitationswirkung
als die Wirkung eines gekrümmten Raumes vorstellen kann, wie schwarze Löcher
beschrieben werden können, und wie alle diese Dinge tatsächlich gemessen
werden können.

%
% k-id.tex -- Krümmung eines eindimensionalen Raumes, Einbettung
%
% (c) 2017 Prof Dr Andreas Müller, Hochschule Rapperswil
%
\section{Krümmung einer Kurve
\label{skript:kruemmung:section:kurve}}
\rhead{Krümmung einer Kurve}


%
% k-laenge.tex -- Längenmessung in einer Fläche oder in einem Raum
%
% (c) 2017 Prof Dr Andreas Müller, Hochschule Rapperswil
%
\section{Längenmessung
\label{skript:kruemmung:section:laengenmessung}}
\rhead{Längenmessung}
Da wir nicht weiter annehmen wollen, dass sich der Raum mit Hilfe
eines rechtwinkligen Koordinatensystems adäquat beschreiben lässt,
müssen wir automatisch beliebige Koordinatensysteme zulassen.
Je nach Wahl eines Koordinatensystems werden dann Vektoren, die
wir für die Beschreibung der physikalischen Gesetze benötigen,
verschiedenen Koordinaten haben.
Da alle diese Koordinatensystem gleichberechtigt sind, müssen
wir Vektoren auf einheitliche Art umrechnen können.
Ausserdem müssen die Naturgesetzt so formuliert sein, dass
sie in jedem beliebigen Koordinatensystem gleich aussehen.

In diesem Abschnitt beginnen wir damit, den Begriff der Längenmessung
auf beliebige Koordinatensysteme auszudehnen.
Ein Punkt wird beschrieben durch seine Koordinaten, die wir mit
$x^\mu$ bezeichnen, wobei $\mu$ von $1$ bis $n$ läuft, $n$
ist die Dimension. 
Die etwas ungewohnte Schreibweise für die Indizes wie Exponenten
hat einen tieferen Grund in der Tensorrechnung, und wird später
verständlich werden.

\subsection{Vektoren und Koordinatentransformation}
Eine Koordinatentransformation zwischen zwei Koordinatensystemen ist
eine Abbildung, die die Koordinaten $x^\mu$ des einen Koordinatensystems
in die Koordinaten $y^\nu$ des anderen Koordinatensystems umrechnet.
Man kann also schreiben
\begin{equation}
y^{\nu}=y^{\nu}(x^1,\dots,x^n).
\label{skript:kruemmung:umrechnung}
\end{equation}
Uns interessiert vor allem die Beschreibung von Bahnkurven, wir möchten
ja zum Beispiel den Absturz in ein schwarzes Loch berechnen können.
Eine Bahnkurve erhält man, indem man die Koordinaten mit der Zeit
varieren lässt.
Eine Kurve wird also beschrieben durch Funktionen $x^\mu(t)$.

In der physikalischen Beschreibung werden meistens Vektoren wie
Geschwindigkeit und Beschleunigung verwendet.
Sie sind die Ableitung der Koordinaten eines Bahnpunktes nach
der Zeit.
Dies lässt sich direkt auch auf die Koordinaten übertragen,
wir erhalten für den Geschwindigkeitsvektor
\[
v^{\mu} = \frac{dx^\mu(t)}{dt}.
\]

Wie sieht der Geschwindigkeitsvektor in $y^{\nu}$ Koordinaten aus?
Dazu setzen wir die Bahnkurve in die Umrechnungsformeln
\eqref{skript:kruemmung:umrechnung} ein.
Die Kettenregel liefert
\begin{align*}
y^{\nu}(t)&=y^{\nu}(x^1(t),\dots,x^n(t))
\\
u^{\nu}
=
\frac{dy^{\nu}(t)}{dt}
&=
\frac{\partial y^{\nu}}{\partial x^1}\frac{dx^1(t)}{dt}
+\dots+
\frac{\partial y^{\nu}}{\partial x^n}\frac{dx^n(t)}{dt}
=
\sum_{\mu=1}^n
\frac{\partial y^{\nu}}{\partial x^{\mu}}\frac{dx^{\mu}(t)}{dt}
\end{align*}
Auf der rechten Seite steht eine Summe von Termen, in denen
der Summationsindex sowohl oben als auch unten auftritt.
Diese Art von Summe kommt in der zu entwickelnden Theorie sehr
häufig vor, daher schreiben wir in Zukunft die Summe nicht mehr.
Diese {\em Einsteinsche Summenkonvention} bedeutet also, dass in
einem Term, in dem der gleiche Index oben und unten vorkommt,
über die möglichen Werte dieses Index summiert werden muss.
\index{Summenkonvention!Einsteinsche}

Die Koeffizienten 
\[
\alpha_\mu^\nu=\frac{\partial y^{\nu}}{\partial x^{\mu}}
\]
dienen also der Umrechnung der Komponenten eines Vektors vom
$x^{\mu}$-Koordinatensystem ins $y^{\nu}$-Koordinatensystem.
Man kann die Rechnung auch in Matrixform schreiben:
\[
\begin{pmatrix}y^1\\\vdots\\y^n\end{pmatrix}
=
\begin{pmatrix}
\frac{\partial y^1}{\partial x^1}&\dots&\frac{\partial y^1}{\partial x^n}\\
\vdots&\ddots&\vdots\\
\frac{\partial y^n}{\partial x^1}&\dots&\frac{\partial y^n}{\partial x^n}
\end{pmatrix}
\begin{pmatrix}x^1\\\vdots\\x^n\end{pmatrix}.
\]
Die Summationskonvention lässt sich zum Beispiel dadurch rechtfertigen,
dass bei der Matrizenmultiplikation die Summation ja auch nicht
explizit hingeschrieben wird.

\subsection{Metrik}
In einem rechtwinkligen Koordinatensystem können wir die Länge
einer Kurve durch Zerlegen in beliebig kleine Teilstücke 
\begin{align*}
l
&\simeq
\sum_{i=1}^n \sqrt{\sum_{\mu} (x^{\mu}(t_i)-x^{\mu}(t_{i-1}))^2}
\\
&\rightarrow
\int_{t_0}^{t_n} \sqrt{\sum_{\mu}\biggl(\frac{dx^{\mu}(t)}{dt}\biggr)^2}\,dt
\end{align*}
bestimmen.
Unter der Wurzel sehen wir die Quadratsummen wieder, die für den
Satz des Pythagoras charakteristisch sind.

In einem beliebigen Koordinatensystem funktioniert dies jedoch nicht
mehr.
Wir müssen zulassen, dass die Koordinaten entlang verschiedener
Achsen nicht mehr direkt der Länge entsprechen, dass wir als
entlang der Achsen Skalierungsfaktoren haben.
Weiter ist damit zu rechnen, dass auch gemischte Termen auftauchen.
Die allgemeinstmögliche Form einer Längenmessungsformel ist daher
\begin{equation}
l
=
\int_{t_0}^{t_1}
\sqrt{\sum_{\mu,\nu} g_{\mu\nu} \frac{dx^{\mu}(t)}{dt}\frac{dx^{\nu}(t)}{dt}}\,dt.
\label{skript:kruemmung:metrikformel}
\end{equation}
Die Zahlen $g_{\mu\nu}$ können dabei auch noch von den Koordinaten
abhängen.
Wir verlangen ausserdem, dass $g_{\mu\nu}=g_{\nu\mu}$ ist, denn
diese beiden Terme sind in \eqref{skript:kruemmung:metrikformel}
auf symmetrische Art und Weise vertreten.

\begin{definition}
Die Zahlen $g_{\mu\nu}$ heissen der metrische Tensor.
\index{Tensor!metrischer}
\end{definition}

%
% Beispiel für Längenmessung in einem Koordinatensystem mit nicht
% orthogonalen Achsen.
%
\begin{beispiel}
Wir untersuchen die Längenmessung mit nicht orthogonalen Achsen.
Statt des gewöhnlichen Koordinatensystems in einer Ebene verwenden
wir die Koordinaten $x'=x$ und $y'=x+y$.
In den ungestrichenen Koordinaten ist der Abstand zwischen zwei
Punkten durch den Satz von Pytagoras
\begin{equation}
l^2 = \Delta x^2 + \Delta y^2
\label{skript:kruemmung:p2}
\end{equation}
gegeben.
Um den Abstand in $x'$-$y'$-Koordinaten auszudrücken, müssen wir diese
erst wieder in $x$-$y$-Koordinaten umrechnen. 
Man findet
\[
x=x'
\qquad\text{und}\qquad
y=y'-x=y'-x'.
\]
Eingesetzt in die Formel \eqref{skript:kruemmung:p2} finden wir
\begin{align*}
l^2
&=
\Delta x^2 + \Delta y^2
=
\Delta x^{\prime 2}
+
(\Delta y'- \Delta x')^2
=
\Delta x^{\prime 2}
+
\Delta y^{\prime 2}-2\Delta x'\Delta y' + \Delta x^{\prime 2}
\\
&= 2 \Delta x^{\prime 2} - 2 \Delta x'\Delta y'+\Delta y^{\prime 2}.
\end{align*}
Die zugehörigen Koeffizienten $g_{\mu\nu}$ sind
\[
g_{11} = 2,\quad
g_{12}=g_{21}=-1\quad\text{und}\quad
g_{22}=1.
\]
Diese Koeffizienten beschreiben die gleiche Längenmessung in den 
gestrichenen Koordinaten wie der Satz von Pythagoras in den ursprünglichen
Koordinaten.
\end{beispiel}

Die Formel \eqref{skript:kruemmung:metrikformel} ist etwas unhandlich.
Wir können aber wieder die Einsteinsche Summenkonvention verwenden,
um das Summenzeichen los zu werden.
Ausserdem kann man die Ableitungen nach $t$ auch mit einem Punkt abkürzen.
Damit lässt sich die Längenmessung daher etwas kompakter als
\[
l=\int_{t_0}^{t_1} \sqrt{g_{\mu\nu}\dot x^{\mu}(t) \dot x^{\nu}(t)}\,dt
\]
schreiben.

\subsection{Beispiele}
Wir betrachten drei für die Anwendungen wichtige Bespiele von
Koordinatensystemen des gewöhnlichen dreidimensionalen Raumes
und berechnen die zugehörigen metrischen Tensoren.

\subsubsection{Polarkoordinaten}
Punkte in der Ebene können statt in rechtwinkligen $x$-$y$-Koordinaten
auch mit Hilfe von Polarkoordinaten $(r,\varphi)$0nach der Umrechnungsregel
\begin{align*}
x&=r\cos\varphi\\
y&=r\sin\varphi
\end{align*}
beschrieben werden.
Um den metrischen Tensor zu bestimmen, müssen die Ableitungen von $x$ 
und $y$ nach $t$ durch Ableitungen von $r$ und $\varphi$ nach $t$ 
ausdrücken.
Die Produktregel liefert:
\begin{align*}
\dot x&= \dot r\cos \varphi - r\dot\varphi \sin\varphi 
\\
\dot y&= \dot r\sin\varphi + r\dot\varphi\cos\varphi.
\end{align*}
Eingesetzt in den Satz von Pythagoras folgt
\begin{align*}
\dot x^2 + \dot y^2
&=
\dot r^2\cos^2\varphi -2r\dot r\dot\varphi\cos\varphi\sin\varphi +r^2\dot \varphi^2\sin^2\varphi
+
\dot r^2\sin^2\varphi +2r\dot r\dot\varphi\sin\varphi\cos\varphi +r^2\dot\varphi^2\cos^2\varphi
\\
&=
\dot r^2(\cos^2\varphi+\sin^2\varphi)+ r^2\dot\varphi^2(\sin^2\varphi+\cos^2\varphi)
\\
&=\dot r^2 + r^2\dot\varphi^2.
\end{align*}
Die gemischten Terme haben sich weggehoben.
Man liest daraus für die Koeffizienten des metrischen Tensors
\[
g_{11}=r^2,\qquad g_{12}=g_{21}=0\qquad\text{und}\qquad g_{22}=r^2\dot\varphi^2
\]
ab.
Die Koeffizienten hängen zwar von den Koordinaten ab, doch bedeutet
das noch nicht, dass ein gekrümmter Raum vorliegt.
Diese $g_{\mu\nu}$ beschreiben ja die gleiche Lägenmessung wie der Satz
des Pythagoras in $x$-$y$-Koordinaten.

\subsubsection{Zylinderkoordinaten}
Zylinderkoordinaten beschreiben die Punkte des dreidimensionalen
Raumes mit Polarkoordinaten in der $x$-$y$-Ebene und der $z$-Koordinate.
Da wir die Metrik in der $x$-$y$-Ebene schon durch $(r,\varphi)$
ausgedrückt haben, können wir auch die Metrik in Polarkoordinaten
bekommen, indem wir die $z$-Koordinaten ergänzen:
\[
\dot x^2+\dot y^2 +\dot z^2
=
\dot r^2 + r^2\dot\varphi^2 + \dot z^2.
\]
Der zugehörige metrische Tensor hat daher die Koeffizienten
\[
g_{11}=\dot r^2,\qquad
g_{22}=r^2\dot\varphi^2
\qquad\text{und}\qquad
g_{33}=1,
\]
alle anderen Koeffizienten sind $0$.

\subsubsection{Zylinderoberfläche}
Beschränken wir uns auf die Punkte im Abstand $1$ zur $z$-Achse, erhalten
wir eine Zylinderfläche, welche mit Koordinaten $\varphi$ und $z$
beschrieben werden kann.
Die Längenmessung in dieser Fläche wird durch den metrischen
Tensor der Zylinderkoordinaten beschrieben, in dem wir $r=1$ einsetzen.
Wir erhalten
\[
\dot\varphi^2+\dot z^2.
\]
Dies ist der metrische Tensor einer Ebene mit rechtwinkligen Koordinaten.

\subsubsection{Kugelkoordinaten}
Kugelkoordinaten beschreiben die Punkte eines dreidimensionalen Raumes
durch die Entfernung $r$ vom Nullpunkt, die geographische Länge
$\varphi$, die von
der $x$-Koordinate gemessen wird, und durch die geograpische Breite
$\vartheta$,
die als Winkel von der $z$-Achse gemessen wird.
Die Umrechnung in kartesische Koordinaten erfolgt mit den Formeln
\begin{align*}
x&= r\sin\vartheta\cos\varphi\\
y&= r\sin\vartheta\sin\varphi\\
z&= r\cos\vartheta
\end{align*}
Wir bestimmen wieder die Koeffizienten des metrischen Tensors.
Dazu leiten wir zunächst nach $t$ ab.
\begin{align*}
\dot x
&=
\dot r\sin\vartheta\cos\varphi
+
r\dot\vartheta \cos\vartheta\cos\varphi
-
r\dot\varphi \sin\vartheta\sin\varphi
\\
\dot y
&=
\dot r\sin\vartheta\sin\varphi
+
r\dot\vartheta\cos\vartheta\sin\varphi
+
r\dot\varphi\sin\vartheta\cos\varphi
\\
\dot z
&=
\dot r\cos\vartheta
-
r\dot\vartheta \sin\vartheta
\end{align*}
Bei der Berechnung der Länge werden sich wieder viele Terme
wegen der verschiedenen Vorzeichen des letzten Terms im
Ausdruck für $\dot x$ und $\dot y$ wegheben.
Für den Ausdruck $ \dot x^2 + \dot y^2 + \dot z^2$ findet man
\[
\begin{array}{clclclcl}
 &
\dot r^2\sin^2\vartheta\cos^2\varphi
	&+&r^2\dot\vartheta^2\cos^2\vartheta\cos^2\varphi
		&+&r^2\dot\varphi^2\sin^2\vartheta\sin^2\varphi
			&+&2r\dot r\dot\vartheta\sin\vartheta\cos\vartheta\cos^2\varphi
\\
+&
\dot r^2\sin^2\vartheta\sin^2\varphi
	&+&r^2\dot\vartheta^2\cos^2\vartheta\sin^2\varphi
		&+&r^2\dot\varphi^2\sin^2\vartheta\cos^2\varphi
			&+&2r\dot r\dot\vartheta\sin\vartheta\cos\vartheta\sin^2\varphi
\\
+&
\dot r^2\cos^2\vartheta
	&+&r^2\dot\vartheta^2\sin^2\vartheta
		& &
			&-&2r\dot r\dot\vartheta \sin\vartheta \cos\vartheta
\\
\end{array}
\]
In den Spalten ergänzen sich in den ersten beiden Zeilen jeweils
$\cos^2\varphi$ und $\sin^2\varphi$ zu $1$.
In den ersten beiden Spalten lassen sich danach auch noch
$\cos^2\vartheta$ und $\sin^2\vartheta$ zu $1$ zusammenfassen,
während sich die Terme in der vierten Spalte wegheben.
Damit bekommt man
\begin{equation}
\dot x^2 + \dot y^2 + \dot z^2
=
\dot r^2+r^2\dot\vartheta^2 + r^2\dot\varphi^2\sin^2\vartheta
\label{skript:kruemmung:kugelkoordinaten}
\end{equation}
für die Längenmessung, und
\[
g_{11}=\dot r^2,\qquad
g_{22}=r^2\dot\vartheta^2
\qquad\text{und}\qquad
g_{33}= r^2\dot\varphi^2\sin^2\vartheta
\]
für die nicht verschwindenden Komponenten des metrischen Tensors.

\subsubsection{Kugeloberfläche}
Aus dem metrischen Tensor der Kugelkoordinaten lässt sich durch festhalten
des Radius der metrische Tensor einer Kugeloberfläche ableiten.
Aus dem Ausruck \eqref{skript:kruemmung:kugelkoordinaten}
erhalten wir
\[
\dot\vartheta^2+\dot\varphi^2\sin^2\vartheta.
\]
Der $\sin$-Term deutet an, dass wir hier nicht mehr direkt eine flache
Metrik haben.
Den Nachweis können wir aber erst führen, wenn wir den Begriff der
Krümmung zur Verfügung haben.
Für die nicht verschwindenden Komponenten des metrischen Tensors
finden wir
\[
g_{11} = 1
\qquad\text{und}\qquad
g_{22}=\sin^2\vartheta.
\]
Man erkennt, dass der Koeffizient $g_{22}$ bei $\vartheta \in\{0,\pi\}$
verschwindet.
Man spricht von einer Singularität der Längenmessung.
Dies ist jedoch nur ein Artefakt der Tatsache, dass die Kugelkoordinaten
bei den Polen nicht mehr eindeutig sind.
Zur Beschreibung der Pole sind alle möglichen Werte der geographischen
Länge gleichermassen geeignet.



%
% k-geodaeten.tex -- Gleichung der Geodäten, Christoffel-Symbole
%
% (c) 2017 Prof Dr Andreas Müller, Hochschule Rapperswil
%
\section{Geodäten
\label{skript:kruemmung:section:geodaeten}}
\rhead{Geodäten}
In der Ebene ist die kürzeste Verbindung zwischen zwei Punkten
eine Gerade.
Auf einem Zylinder oder Kegel kann man die kürzeste Verbindung finden,
indem man die Fläche in eine Ebene abrollt, und dann dort verwendet,
dass die kürzeste Verbindung in der Ebene eine Gerade ist.
Dies zeigt dass die kürzsten Verbindung nichts mit der speziellen
Einbettung einer Fläche zu tun hat, sondern eine Eigenschaft ist,
die sich allein aus der Längenmessung in der Fläche ist, man nennt
dies auch eine intrinsische Eigenschaft.

Auf einer Kugeloberfläche kann man die kürzesten Verbindungen ebenfalls
direkt angeben, es sind die Grosskreise.
Man kann dabei so argumentieren: von allen Schnitten der Kugeloberfläche
mit Ebenen durch die zwei gegeben Punkte ist der Grosskreis derjenige
mit der kleinsten Krümmung, und daher die ``direkteste'' Verbindung.
Dieses Argument ist allerdings nicht ganz exakt, denn man vergisst dabei,
dass es noch viele weitere Kurven gibt, die die beiden Punkte verbinden.
Es ist auch nicht wirklich auf noch allgemeinere Situationen übertragbar,
denn es nützt aus, dass die Kugeloberfläche homogen ist, in jedem Punkt
ist die ``Krümmung'' (ein im Moment noch nicht definierter Begriff) 
gleich gross.

In diesem Abschnitt suchen wir daher nach einer allgemeinen Methode,
die kürzeste Verbindung, die sogenannten Geodäten zu finden.

\subsection{Paralleltransport}
Die Beispiele von kürzesten Verbindungen suggerieren, dass die kürzeste
Verbindung auch die ``geradeste'' ist, sie weicht möglichst wenig von
der Richtung des aktuellen Tangentialvektors ab.
Das Problem bei dieser Interpretation ist allerdings, dass wir Vektoren
in zwei verschiedenen Punkten der Fläche nicht unmittelbar vergleichen
können.
Auf der Kugeloberfläche liegen die Tangentialvektoren an eine Kurve in
verschiedenen Punkten zum Beispiel in verschiedenen Tangentialebenen
an die Kugel.

Wir müssen also zunächst in der Lage sein, Vektoren in zwei verschiedenen
Punkten miteinander zu vergleichen.
Wir können das tun, indem wir einen Vektor entlang einer Kurve transportieren,
wobei wir versuchen, in so parallel zu sich selbst wie möglich zu sich
selbst zu halten.
Auch dies ist im Moment noch ein etwas schwammiger Begriff. 
Es ist aber klar, dass die Komponenten des transportierten Vektors
sowohl von der Transportrichtung wie auch vom ursprünglichen Vektor
abhöngen.
Seien $A^\mu$ die Komponenten eines Vektors, die Richtung mit Komponenten
$\Delta x^\nu$ transport werden soll.
Dann wird der transportierte die Form
\[
\tilde A^\alpha
=
A^\alpha - \Gamma_{\mu\nu}^\alpha A^\mu \Delta x^\nu
\]
haben.
Die Wahl des Vorzeichnes von $\Gamma_{\mu\nu}^\alpha$ ist im Wesentlichen
eine Frage der Konvention.
Die instantane Änderung zur Zeit $t=0$ entlang der Kurve ist
\[
\frac{d}{dt}\tilde A^\alpha\bigg|_{t=0}
=
\Gamma_{\mu\nu}^\alpha A^\mu \dot x^\nu.
\]

Bis jetzt haben wir die Metrik nicht verwendet.
Wir möchten dass der Paralleltransport die Länge des Vektors beim
Transport entlang einer Kurve nicht verändert.
Die Länge des Vektors wird durch $g_{\mu\nu}\tilde A^\mu \tilde A^\nu$
gegeben.
Die Ableitung entlang der Kurve $x^\mu(t)$ ist
\[
\frac{d}{dt} g_{\mu\nu}\tilde A^\mu \tilde A^\nu\bigg|_{t=0}
=
\frac{\partial g_{\mu\nu}}{\partial x^\alpha}A^\mu A^\nu\dot x^\alpha
-
g_{\mu\nu}A^\mu\Gamma_{\alpha\beta}^\nu A^\alpha \dot x^\beta
-
g_{\mu\nu}\Gamma_{\alpha\beta}^\mu A^\alpha \dot x^\beta A^\nu
=
0.
\]
In diesem Ausdruck kommen in allen Termen $A^\mu$ und $\dot x^\beta$ mit
verschiedenen Indizes vor.
Damit wir diese Faktoren ausklammern können, bennen wir die Indizes um,
so dass sie in allen Termen gleich sind.
Wir erhalten dann die Gleichung
\[
\biggl(
\frac{\partial g_{\mu\nu}}{\partial x^\alpha}
-
g_{\mu\beta}\Gamma_{\nu\alpha}^\beta
-
g_{\beta\nu}\Gamma_{\mu\alpha}^\beta
\biggr)
A^\mu A^\nu\dot x^\alpha
=
0
\]
F"ur die Koeffizienten $\Gamma$.
Diese Gleichung muss f"ur jede beliebige Wahl von $A^\mu$ und jede
beliebige Richtung der Kurve $\dot x^\alpha$ erfüllt sein, wenn der
Klammerausdruck verschwindget für alle Werte der freien, also nicht durch
die Summationskonvention als Laufindizes ausgezeichnten, Indizes.
Wir erhalten also
\[
\frac{\partial g_{\mu\nu}}{\partial x^\alpha}
-
g_{\mu\beta}\Gamma_{\nu\alpha}^\beta
-
g_{\beta\nu}\Gamma_{\mu\alpha}^\beta
=
0,
\]
ein System von Gleichungen für die $n^3$ Grössen $\Gamma_{\mu\nu}^\alpha$.
Wir können die $\Gamma$ auch allein auf der linken Seite haben:
\[
g_{\mu\beta}\Gamma_{\nu\alpha}^\beta
+
g_{\beta\nu}\Gamma_{\mu\alpha}^\beta
=
\frac{\partial g_{\mu\nu}}{\partial x^\alpha}
\]
Durch zyklische Vertauschung der drei Indizes $\mu$, $\nu$ und $\alpha$
erhalten wir drei Gleichungen
\[
\def\arraystretch{2.0}
\begin{linsys}{3}
g_{\mu\beta}\Gamma_{\nu\alpha}^\beta &+& g_{\beta\nu}\Gamma_{\mu\alpha}^\beta
& &
&=&
\displaystyle
\frac{\partial g_{\mu\nu}}{\partial x^\alpha}
\\
& &g_{\nu\beta}\Gamma_{\alpha\mu}^\beta &+& g_{\beta\alpha}\Gamma_{\nu\mu}^\beta
&=&
\displaystyle
\frac{\partial g_{\nu\alpha}}{\partial x^\mu}
\\
g_{\beta\mu}\Gamma_{\alpha\nu}^\beta
& &
&+&g_{\alpha\beta}\Gamma_{\mu\nu}^\beta 
&=&
\displaystyle
\frac{\partial g_{\alpha\mu}}{\partial x^\nu}
\end{linsys}
\]
Da sowohl $g$ als auch $\Gamma$ in den unteren Indizes symmetrisch
sind, können wir die Gleichungen weiter vereinfachen:
\[
\def\arraystretch{2.0}
\begin{linsys}{3}
g_{\mu\beta}\Gamma_{\nu\alpha}^\beta
	&+& g_{\nu\beta}\Gamma_{\mu\alpha}^\beta
		& &
&=&
\displaystyle
\frac{\partial g_{\mu\nu}}{\partial x^\alpha}
\\
	& &g_{\nu\beta}\Gamma_{\mu\alpha}^\beta
		&+& g_{\alpha\beta}\Gamma_{\nu\mu}^\beta
&=&
\displaystyle
\frac{\partial g_{\nu\alpha}}{\partial x^\mu}
\\
g_{\mu\beta}\Gamma_{\nu\alpha}^\beta
	& &
		&+&g_{\alpha\beta}\Gamma_{\nu\mu}^\beta 
&=&
\displaystyle
\frac{\partial g_{\alpha\mu}}{\partial x^\nu}
\end{linsys}
\]
Subtrahieren wir die erste Zeile von der Summe der letzten beiden,
heben sich ersten Terme weg, es bleibt
\[
g_{\alpha\beta}\Gamma_{\nu\mu}^\beta
=
\biggl(
\frac{\partial g_{\nu\alpha}}{\partial x^\mu}
+
\frac{\partial g_{\alpha\mu}}{\partial x^\nu}
-
\frac{\partial g_{\mu\nu}}{\partial x^\alpha}
\biggr).
\]
Bezeichen wir die inverse Matrix von $g_{\alpha\beta}$ mit
$g^{\alpha\beta}$, dann k"onnen wir nach $\Gamma_{\mu\nu}^\beta$ aufl"osen:
\[
g^{\sigma\alpha}
\biggl(
\frac{\partial g_{\nu\alpha}}{\partial x^\mu}
+
\frac{\partial g_{\alpha\mu}}{\partial x^\nu}
-
\frac{\partial g_{\mu\nu}}{\partial x^\alpha}
\biggr)
=
g^{\sigma\alpha}
g_{\alpha\beta}\Gamma_{\mu\nu}^\beta
=
\delta^\sigma_\beta\Gamma_{\mu\nu}^\beta
=
\Gamma_{\mu\nu}^\sigma.
\]

\begin{definition}
Sei $g_{\mu\nu}$ ein metrischer Tensor. 
Dann heissen die
\[
\Gamma_{\alpha,\mu\nu}
=
\frac12
\biggl(
\frac{\partial g_{\nu\alpha}}{\partial x^\mu}
+
\frac{\partial g_{\alpha\mu}}{\partial x^\nu}
-
\frac{\partial g_{\mu\nu}}{\partial x^\alpha}
\biggr)
\]
die {\em Christoffelsymbole 1.~Art}
und
\[
\Gamma_{\mu\nu}^\sigma
=
g^{\sigma\alpha} \Gamma_{\alpha,\mu\nu}
=
\frac12
g^{\sigma\alpha}
\biggl(
\frac{\partial g_{\nu\alpha}}{\partial x^\mu}
+
\frac{\partial g_{\alpha\mu}}{\partial x^\nu}
-
\frac{\partial g_{\mu\nu}}{\partial x^\alpha}
\biggr)
\]
heissen {\em Christoffelsymbole 2.~Art} oder {\em Zusammenhangskoeffizienten}.
\end{definition}

\subsection{Beispiele}
In den nachfolgenden Beispielen wollen wir die Christoffelsymbole erster
und zweiter Art für Zylinder- und Kugeloberfläche berechnen.

\subsubsection{Zylinderkoordinaten}
Da die Komponenten des metrischen Tensors sind konstant, damit verschwinden
alle Ableitungen
\[
\frac{\partial g_{\mu\nu}}{\partial x^\alpha}=0
\qquad\Rightarrow\qquad
\Gamma_{\alpha,\mu\nu}=0
\qquad\Rightarrow\qquad
\Gamma_{\mu\nu}^\alpha=0.
\]

\subsubsection{Kugelkoordinaten}
Die Kugeloberfläche verwendet die Koordinaten $(\vartheta,\varphi)$.
Zun"achst brauchen wir die Ableitungen der Komponenten des metrischen
Tensors
\begin{align*}
\frac{\partial g_{11}}{\partial \vartheta} &=0
&
\frac{\partial g_{12}}{\partial \vartheta} &=0
&
\frac{\partial g_{21}}{\partial \vartheta} &=0
&
\frac{\partial g_{22}}{\partial \vartheta} &=\sin2\vartheta
\\
\frac{\partial g_{11}}{\partial \varphi} &=0
&
\frac{\partial g_{12}}{\partial \varphi} &=0
&
\frac{\partial g_{21}}{\partial \varphi} &=0
&
\frac{\partial g_{22}}{\partial \varphi} &=0
\\
\end{align*}
Daraus können wir die Christoffelsymbole erster Art ableiten:
\begin{align*}
 \Gamma_{1,11}
&=
\frac12\biggl(\frac{\partial g_{11}}{\partial \vartheta}
	+ \frac{\partial g_{11}}{\partial \vartheta}
	- \frac{\partial g_{11}}{\partial \vartheta}\biggr)=0,
&\Gamma_{1,12}
&=
\frac12\biggl(\frac{\partial g_{11}}{\partial \varphi}
	+ \frac{\partial g_{21}}{\partial \vartheta}
	- \frac{\partial g_{12}}{\partial \vartheta}\biggr)=0,
\\
\Gamma_{1,21}
&=
\frac12\biggl(\frac{\partial g_{12}}{\partial \vartheta}
	+ \frac{\partial g_{11}}{\partial \varphi}
	- \frac{\partial g_{21}}{\partial \vartheta}\biggr)=0,
&\Gamma_{1,22}
&=
\frac12\biggl(\frac{\partial g_{12}}{\partial \varphi}
	+ \frac{\partial g_{12}}{\partial \varphi}
	- \frac{\partial g_{22}}{\partial \vartheta}\biggr)=-\frac12\sin2\vartheta,
\\
\Gamma_{2,11}
&=
\frac12\biggl(\frac{\partial g_{12}}{\partial \vartheta}
	+ \frac{\partial g_{12}}{\partial \vartheta}
	- \frac{\partial g_{11}}{\partial \varphi}\biggr)=0,
&\Gamma_{2,12}
&=
\frac12\biggl(\frac{\partial g_{12}}{\partial \varphi}
	+ \frac{\partial g_{22}}{\partial \vartheta}
	- \frac{\partial g_{12}}{\partial \varphi}\biggr)=\frac12\sin2\vartheta,
\\
\Gamma_{2,21}
&=
\frac12\biggl(\frac{\partial g_{12}}{\partial \varphi}
	+ \frac{\partial g_{22}}{\partial \vartheta}
	- \frac{\partial g_{21}}{\partial \varphi}\biggr)=\frac12\sin2\vartheta,
&\Gamma_{2,22}
&=
\frac12\biggl(\frac{\partial g_{22}}{\partial \varphi}
	+ \frac{\partial g_{22}}{\partial \varphi}
	- \frac{\partial g_{22}}{\partial \varphi}\biggr)=0.
\end{align*}
Die inverse Matrix von $g_{\mu\nu}$ hat die nicht verschwindenen
Komponenten
\[
g^{11} = 1
\qquad\text{und}\qquad
g^{22} = \frac1{\sin^2\vartheta}
\]
und die Christoffelsymbole 2.~Art
\begin{align*}
 \Gamma_{11}^1
&=0,
&\Gamma_{12}^1
&=0,
&\Gamma_{21}^1
&=0,
&\Gamma_{22}^1
&=-\frac12\sin2\vartheta,
\\
 \Gamma_{11}^2
&=0,
&\Gamma_{12}^2
&=\cot\vartheta,
&\Gamma_{21}^2
&=\cot\vartheta,
&\Gamma_{22}^2
&=0.
\end{align*}

\subsection{Geodätengleichung}

\subsection{Variationsprinzip}







%
% k-kruemmung.tex -- Krümmungstensor, Ricci, Einstein
%
% (c) 2017 Prof Dr Andreas Müller, Hochschule Rapperswil
%
\section{Krümmung
\label{skript:kruemmung:section:kruemmung}}

\begin{figure}
\centering
\includegraphics[width=\hsize]{chapters/3d/flach.jpg}
\caption{Beim Paralleltransport eines Vektors entlang einer Kurve in
einer Ebene dreht sich der Vektor nicht
\label{skript:kruemmung:transportflach}}
\end{figure}
\begin{figure}
\centering
\includegraphics[width=\hsize]{chapters/3d/sphere.jpg}
\caption{Drehung eines Tangentialvektors beim Transport entlang eines
Dreiecks auf der Kugeloberfläche.
Der Flächeninhalt des Dreiecks ist ein Achtel der Kugeloberfläche,
also $4\pi/8=\pi/2$, der Drehwinkel ist $\pi/2$.
\label{skript:kruemmung:transportkugel}}
\end{figure}

Transportiert man einen Vektor in der Ebene mit dem üblichen euklidischen
Koordinatensysteme parallel, dann ändert seine Richtung nicht,
denn die Christoffelsymbole verschwinden alle
(Abbildung~\ref{skript:kruemmung:transportflach}).
Auf einer Kugeloberfläche sieht das ganz anders aus.
Transportiert man einen Vektor tangential an den Äquator zunächst entlang 
des Äquators über einen Winkel von $90^\circ$, dann auf einem Längenkreis
bis zum Nordpol und wieder zurück zum Ausgangspunkt.
Wie in Abbildung~\ref{skript:kruemmung:transportkugel}
sichtbar, dreht sich der Vektor
dabei um $90^\circ$. 
Der Unterschied rührt natürlich daher, dass die Kugeloberfläche gekrümmt
ist.
Offenbar ist die Änderung der Richtung eines Tangentialvektors beim
Paralleltransport entlang eines geschlossenen Weges ein Mass für die
Krümmung einer Fläche.

In diesem Kapitel wollen wir zeigen, wie aus dem Konzept des
Paralleltransportes ein mathematisch wohldefiniertes Mass für die
Krümmung gewonnen werden kann.

\subsection{Krümmungstensor}
Wir möchten Berechnen, wie sich ein Vektor beim Paralleltransport entlang
einer geschlossenen Kurve ändert.
In dieser Form ist das Problem sicher zu kompliziert, die Wahl
einer geschlossenen Kurve beinhaltet viel zu viele Freiheitsgrade.

Zu zwei Tangentialvektoren $u^\mu$ und $v^\mu$ und in einem Punkt
$P$ können wir immer eine
Fläche finden, die aus Geodäten besteht, die alle vom Punkt $P$ ausgehen
und dort eine Richtung haben, die eine Linearkombination der beiden
Tangentialvektoren ist.
Mit diesem Trick können wir das Problem auf eine zweidimensionale
Fläche reduzieren.
Und statt eine beliebige Kurve zuzulassen, können wir uns weiter
auf einen Polygonzug beschränken, bei dem wir den Geodäten folgen,
die als Tangentialrichtung die Richtung der beiden gegebenen
Tangentialvektoren haben.

Wenn wir einen Vektor $x^\mu$ entlang einer solchen Kurve parallel
transportieren, dann aber die Kurve auf einen Punkt zusammenschrumpfen
lassen, dann entsteht im Grenzwert ein Vektor $y^\mu$,
der die Verschiebung des Vektors $x^\mu$ beschreibt.
Dieser Vektor muss linear von $u^\mu$, $v^\mu$ und $x^\mu$ abhängen,
wir erwarten also, dass in jedem Punkt Zahlen
$R^\alpha\mathstrut_{\mu\nu\sigma}$ geben muss, mit denen man
$y^\mu$ berechnen kann:
\[
y^\alpha = R^\alpha\mathstrut_{\mu\nu\sigma}u^\mu v^\nu x^\sigma.
\]

\subsection{Ricci-Krümmung}





