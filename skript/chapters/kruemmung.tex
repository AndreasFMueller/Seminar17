%
% kruemmung.tex
%
% (c) 2017 Prof Dr Andreas Müller, Hochschule Rapperswil
%
\chapter{Krümmung\label{skript:chapter:kruemmung}}
\lhead{Krümmung}
\rhead{}

In höheren Dimension.
Eine Fläche kann zwar immer noch auf ganz verschiedene Art in
einen dreidimensionalen Raum eingebettet werden.
Ein Stück Papier kann man sich zum Beispiel als Ausschnitt einer
Ebene vorstellen, man kann es aber auch zu einem Zylinder oder
Kegel zusammenrollen.
Durch Messungen allein innerhalb der Fläche, sind diese verschiedenen
Einbettungen nicht unterscheidbar.
Eine Ameise auf der Fläche könnte keinen Unterschied feststellen.
Eine Kugeloberfläche dagegen liesse sich sehr wohl durch Messungen
allein innerhalb der Fläche unterscheiden.
Misst man zum Beispiel von einem Kreis um einen Punkt den Umfang,
dann stellt man fest, dass der auf einer Ebene der Umfang immer $2\pi r$ ist.
Auf einer Kugeloberfläche dagegen ist der Umfang kleiner.
Der Radius wird ja nicht in gerader Linie in einer Eben gemessen,
sondern entlang einer gekrümmten Linie, die sich von der Tangentialebene
entfernt.

In diesem Kapitel wird daher untersucht, wie die Geometrie des Raumes
mit der Längenmessung zusammenhängt und wie Krümmung charakterisiert
werden kann.
In der Einleitung zu Kapitel~\ref{skript:chapter:geodaeten} wurde
als Anwendungsbeispiel die Ausbreitung von Licht genannt.
Schon daraus wird klar, dass man sich einen gekrümmten
Raum nicht unbedingt als etwas vorstellen muss, was ``krumm'' in einen
grösseren Raum eingebettet ist. 
Es ist zwar so, das auf diese Weise gekrümmte Räume entstehen können,
aber das allgemeine Verständnis sagt einfach nur, dass die Längenmessung
nicht so funktioniert wie \eqref{skript:kruemmung:pytagoras} suggeriert.

%
% k-kruemmung.tex -- Krümmungstensor, Ricci, Einstein
%
% (c) 2017 Prof Dr Andreas Müller, Hochschule Rapperswil
%
\section{Krümmung
\label{skript:kruemmung:section:kruemmung}}

\begin{figure}
\centering
\includegraphics[width=\hsize]{chapters/3d/flach.jpg}
\caption{Beim Paralleltransport eines Vektors entlang einer Kurve in
einer Ebene dreht sich der Vektor nicht
\label{skript:kruemmung:transportflach}}
\end{figure}
\begin{figure}
\centering
\includegraphics[width=\hsize]{chapters/3d/sphere.jpg}
\caption{Drehung eines Tangentialvektors beim Transport entlang eines
Dreiecks auf der Kugeloberfläche.
Der Flächeninhalt des Dreiecks ist ein Achtel der Kugeloberfläche,
also $4\pi/8=\pi/2$, der Drehwinkel ist $\pi/2$.
\label{skript:kruemmung:transportkugel}}
\end{figure}

Transportiert man einen Vektor in der Ebene mit dem üblichen euklidischen
Koordinatensysteme parallel, dann ändert seine Richtung nicht,
denn die Christoffelsymbole verschwinden alle
(Abbildung~\ref{skript:kruemmung:transportflach}).
Auf einer Kugeloberfläche sieht das ganz anders aus.
Transportiert man einen Vektor tangential an den Äquator zunächst entlang 
des Äquators über einen Winkel von $90^\circ$, dann auf einem Längenkreis
bis zum Nordpol und wieder zurück zum Ausgangspunkt.
Wie in Abbildung~\ref{skript:kruemmung:transportkugel}
sichtbar, dreht sich der Vektor
dabei um $90^\circ$. 
Der Unterschied rührt natürlich daher, dass die Kugeloberfläche gekrümmt
ist.
Offenbar ist die Änderung der Richtung eines Tangentialvektors beim
Paralleltransport entlang eines geschlossenen Weges ein Mass für die
Krümmung einer Fläche.

In diesem Kapitel wollen wir zeigen, wie aus dem Konzept des
Paralleltransportes ein mathematisch wohldefiniertes Mass für die
Krümmung gewonnen werden kann.

\subsection{Krümmungstensor}
Wir möchten Berechnen, wie sich ein Vektor beim Paralleltransport entlang
einer geschlossenen Kurve ändert.
In dieser Form ist das Problem sicher zu kompliziert, die Wahl
einer geschlossenen Kurve beinhaltet viel zu viele Freiheitsgrade.

Zu zwei Tangentialvektoren $u^\mu$ und $v^\mu$ und in einem Punkt
$P$ können wir immer eine
Fläche finden, die aus Geodäten besteht, die alle vom Punkt $P$ ausgehen
und dort eine Richtung haben, die eine Linearkombination der beiden
Tangentialvektoren ist.
Mit diesem Trick können wir das Problem auf eine zweidimensionale
Fläche reduzieren.
Und statt eine beliebige Kurve zuzulassen, können wir uns weiter
auf einen Polygonzug beschränken, bei dem wir den Geodäten folgen,
die als Tangentialrichtung die Richtung der beiden gegebenen
Tangentialvektoren haben.

Wenn wir einen Vektor $x^\mu$ entlang einer solchen Kurve parallel
transportieren, dann aber die Kurve auf einen Punkt zusammenschrumpfen
lassen, dann entsteht im Grenzwert ein Vektor $y^\mu$,
der die Verschiebung des Vektors $x^\mu$ beschreibt.
Dieser Vektor muss linear von $u^\mu$, $v^\mu$ und $x^\mu$ abhängen,
wir erwarten also, dass in jedem Punkt Zahlen
$R^\alpha\mathstrut_{\mu\nu\sigma}$ geben muss, mit denen man
$y^\mu$ berechnen kann:
\[
y^\alpha = R^\alpha\mathstrut_{\mu\nu\sigma}u^\mu v^\nu x^\sigma.
\]

\subsection{Ricci-Krümmung}




