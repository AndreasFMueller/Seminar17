%
% modellgelichungen.tex -- Beispiele von mathematischen Modellen
%
% (c) 2017 Prof Dr Andreas Müller, Hochschule Rapperswil
%
\chapter{Modellgleichungen%
\label{skript:chapter:modellgleichungen}}
\lhead{Modellgleichungen}
\rhead{}
Das Modell des idealen Gases erklärt ein sehr kompliziertes physikalisches
System bestehend aus einer sehr grossen zahl einzelner Teilchen mit einer
einfachen Gleichung zwischen einer kleinen Zahl einfach zu interpretierender
Variablen.
Es ist damit der Friedmann-Gleichung nicht unähnlich, welche ein ganzes
Universum mit der gewöhnlichen Differentialgleichung für nur eine
einzige Funktion, nämlich den Skalenfaktor $a(t)$ beschreibt.
Die Thermodynamik ist spezialisiert darauf, komplexe Systeme durch
statistische Grössen zu beschreiben.
Die Temperatur ist zum Beispiel die mittlere Energie eines Teilchens.
Die Entropie ist ein Mass für die Wahrscheinlichkeit eines Zustands.

Für die Elektronen in einem metallischen Festkörper kann ähnlich
einfaches Modell gefunden werden, welches die physikalischen
Eigenschaften des Leiters als die Eigenschaften eines Gases von
Elektronen beschreibt.
Natürlich müssen dafür die Quanteneigenschaften der Elektronen,
vor allem das Pauli-Prinzip berücksichtigt werden.
Das Bändermodell ist eine Verfeinerungen dieses Modells, welches
das Verhalten von Halbleitern zu verstehen erlaubt.

Die Friedmann-Gleichungen haben gezeigt, dass die aktuelle Energiedichte
des Universums die weitere Expansion beschreibt.
In unserer einfachen Näherung haben wir dabei nur gerade die
Materiedichte berücksichtigt.
Die Expansion des Universum verändert aber auch das
Volumen, welches der Materie im Universum zur Verfügung steht,
und beeinflusst damit auch dessen Druck und Temperatur.
Bei hoher Temperatur oder hoher mittlerer Energie pro Teilchen
ist es nicht mehr zulässig, die Materie im
Universum als ein ideales Gas zu modellieren.
Reaktionen zwischen den Teilchen beginnen eine nicht vernachlässigbare
Rolle zu spielen.
Im ganz frühen Universum muss die Temperatur so hoch gewesen sein,
dass zum Beispiel Elektronen und Protonen zu Neutronen reagieren konnten.

Um das ganz frühe Universum zu verstehen, muss man auch solche
Reaktionsmodelle für leichte Atome oder Elementarteilchen verstehen.
Zusammen mit den Friedmann-Gleichungen entsteht so aber ein
Modell, welches beschreiben kann, in welcher Konzentration die
verschiedenen Elemente unmittelbar nach dem Urknall entstanden sind.
Durch spektroskopische Beobachtung das heutigen Universums und
Messung der Verteilung der Elemente kann man also das Big-Bang-Modell
falsifizieren.

Dieses Kapitel beschreibt zunächst als einfachstes Modell ohne
Rekationen das ideale Gas.
Das Massenwirkungsgesetz der Chemie zeigt, wie sich eine einfache
Chemische Reaktion beschreiben lässt.
Ähnliche Gesetze für Elementarteilchen beschreiben den Prozess 
der Nukleosynthese im frühen Universum.

\section{Ideales Gas}
\rhead{Ideales Gas}
Ein ideales Gas besteht aus gleichartigen Teilchen, die durch elastische
Stösse Energie austauschen können.
Ein solches Gas wird durch die Gasgesetze beschrieben.
Bei konstanter Temperatur sind Druck und Volumen umgekehrt
proportional.
Bei konstantem Volumen sind Druck und Temperatur proportional.
Beide Gesetze lassen sich in der einzigen Formel 
\[
pV=nRT
\]
zusammenfassen.
Darin ist  $R$ eine universelle Konstante, und $n$ ist die Gasmenge
gemessen in Molen.

Die mittlere kinetische Energie eines Teilchens ist auch bekannt
als die Temperatur.


\section{Massenwirkungsgesetz}
\rhead{Massenwirkungsgesetz}

\section{Nukleosynthese}
\rhead{Nukleosynthese}

