%%
% k-geodaeten.tex -- Gleichung der Geodäten, Christoffel-Symbole
%
% (c) 2017 Prof Dr Andreas Müller, Hochschule Rapperswil
%
\chapter{Geodäten
\label{skript:chapter:geodaeten}}
\lhead{Geodäten}
\rhead{}
In der Ebene ist die kürzeste Verbindung zwischen zwei Punkten
eine Gerade.
Auf einem Zylinder oder Kegel kann man die kürzeste Verbindung finden,
indem man die Fläche in eine Ebene abrollt, und dann dort verwendet,
dass die kürzeste Verbindung in der Ebene eine Gerade ist.
Dies zeigt, dass die kürzesten Verbindung nichts mit der speziellen
Einbettung einer Fläche zu tun hat, sondern eine Eigenschaft ist,
die allein durch Längenmessung in der Fläche bestimmt ist.
Man nennt dies auch eine intrinsische Eigenschaft.

Auf einer Kugeloberfläche kann man die kürzesten Verbindungen ebenfalls
direkt angeben, es sind die Grosskreise (Im Kapitel~\ref{chapter:kugel}
findet man eine ausführliche Diskussion der Geometrie der Kugeloberfläche).
\begin{figure}
\centering
\includegraphics[width=0.7\hsize]{chapters/3d/geodaete.jpg}
\caption{Ein Ausschnitt aus einem Grosskreis (rot) ist die kürzestmögliche
Verbindung zwischen zwei Punkten auf einer Kugeloberfläche.
Alle anderen Kurven (blau) in der Kugeloberfläche haben grössere Länge.
\label{skript:kruemmung:fig:geodaete}}
\end{figure}
Man kann dabei so argumentieren: von allen Schnitten der Kugeloberfläche
mit Ebenen durch die zwei gegeben Punkte ist der Grosskreis derjenige
mit der kleinsten Krümmung, und daher die ``direkteste'' Verbindung.
Dieses Argument ist allerdings nicht ganz exakt, denn man vergisst dabei,
dass es noch viele weitere Kurven gibt, die die beiden Punkte verbinden
(Abbildung~\ref{skript:kruemmung:fig:geodaete}).
Es ist auch nicht wirklich auf noch allgemeinere Situationen übertragbar,
denn es nützt aus, dass die Kugeloberfläche homogen ist, in jedem Punkt
ist die ``Krümmung'' (ein im Moment noch nicht definierter Begriff) 
gleich gross.

Auch die Ausbreitung des Lichts in einem Medium ist einer solchen
Beschreibung zugänglich.
Das Licht wählt immer den Weg mit der geringsten Laufzeit, nicht
unbedingt den geometrisch kürzesten Weg.
Daher können sich Lichtstrahlen in einem inhomogenen Medium
krümmen.
Aus der Perspektive des Lichtes ist aber nicht die Bahn gekrümmt,
es folgt der in dieser Geometrie geradest möglichen Bahn.
Nicht die Bahn ist gekrümmt, sondern die Längenmessung weicht von
der üblichen \eqref{skript:kruemmung:pytagoras} ab, der Raum ist
gekrümmt.

In diesem Abschnitt suchen wir daher nach einer allgemeinen Methode,
die kürzeste Verbindung, die sogenannten Geodäten, zu finden.

\section{Paralleltransport und kovariante Ableitung}
\rhead{Paralleltransport und kovariante Ableitung}
\begin{figure}
\centering
\includegraphics[width=0.7\hsize]{chapters/3d/transport.jpg}
\caption{Paralleltransport eines Vektors entlang einer Kurve auf
einer Kugeloberfläche%
\label{skript:geodaten:fig:transport}}
\end{figure}
Die obigen Beispiele von kürzesten Verbindungen suggerieren, dass die kürzeste
Verbindung auch die ``geradeste'' ist, sie weicht möglichst wenig von
der Richtung des aktuellen Tangentialvektors ab.
Das Problem bei dieser Interpretation ist allerdings, dass wir Vektoren
in zwei verschiedenen Punkten der Fläche nicht unmittelbar vergleichen
können.
Auf der Kugeloberfläche liegen die Tangentialvektoren an eine Kurve in
verschiedenen Punkten zum Beispiel in verschiedenen Tangentialebenen
an die Kugel.

Wir müssen also zunächst in der Lage sein, Vektoren in zwei verschiedenen
Punkten miteinander zu vergleichen.
Wir können das tun, indem wir einen Vektor entlang einer Kurve transportieren,
wobei wir versuchen, in so parallel zu sich selbst wie möglich zu sich
selbst zu halten (Abbildung~\ref{skript:geodaten:fig:transport}).
Auch dies ist im Moment noch ein etwas schwammiger Begriff. 
Es ist aber klar, dass die Komponenten des transportierten Vektors
sowohl von der Transportrichtung wie auch vom ursprünglichen Vektor
abhängen.

\subsection{Kovariante Ableitung}
Seien $A^\mu$ die Komponenten eines Vektors, die Richtung mit Komponenten
$\Delta x^\nu$ transport werden soll.
Wenn wir uns entlang einer Kurve in diese Richtung bewegen, werden sich
die Komponenten $A^\mu$ einfach deshalb ändern, weil die Komponenten von
den Koordinaten abhängig sind.
Bei einem paralleltransportierten Vektor wird ein Teil dieser Änderung
auf geeignete Art kompensiert sein müssen.
Die Kompensation wird linear vom Vektor $A^\mu$ und von der
Bewegungsrichtung abhängen.
Wir schreiben dafür
\[
\nabla_\nu A^\mu\cdot \Delta x^\nu
=
\biggl(\frac{\partial A^\mu}{\partial x^\nu}
+
\Gamma^\mu_{\alpha\nu}A^\alpha\biggr)\Delta x^\nu.
\]
Die Funktionen $\Gamma^\mu_{\alpha\nu}$ sind vorerst noch nicht bekannt,
wir werden sie noch bestimmen müssen.

\begin{definition}
Die {\em kovariante Ableitung} eines Vektorfeldes $A^\mu$ ist der 
\index{kovariante Ableitung}%
\index{Ableitung, kovariante}%
Ausdruck
\begin{equation}
\nabla_\nu A^\mu
=
\frac{\partial A^\mu}{\partial x^\nu}
+
\Gamma^\mu_{\alpha\nu}A^\alpha,
\label{skript:geodaeten:kovarianteableitung}
\end{equation}
die Koeffizienten $\Gamma^\mu_{\alpha\nu}$ heissen Christoffelsymbole
zweiter Art oder Zusammenhangskoeffizienten.
Eine weit verbreitete, aber in diesem Buch nicht verwendete,
alternative Schreibweise für die kovariante Ableitung ist
$
\nabla_\nu A^\mu
=
A^\mu\mathstrut_{;\nu}
$.
\end{definition}

Die kovariante Ableitung soll ausdrücken, wie sich die Komponenten eines
Vektors entlang einer Kurve ändern, unabhängig von der konkreten Wahl
des Koordinatensystems.
Paralleltransport soll heissen, dass sich entlang einer Kurve die
Komponenten eines Vektors nicht ändern,
dass also $\nabla_\nu A^\mu=0$.
Insbesondere möchten wir später den Tangentialvektor einer Kurve entlang
der Kurve selbst transportieren, wir erwarten, dass dies die 
``geradestmögliche'' Kurve ergibt.

Die kovariante Ableitung des Tangentialvektors $\dot x^\mu$ entlang
einer Kurve wird sein
\[
\nabla_\nu\dot x^\mu
=
\frac{\partial \dot x^\mu}{\partial x^\nu}
+
\Gamma^\mu_{\alpha\nu}\dot x^\alpha
\]
und die zeitliche Änderung entlang der Kurve wird
\[
\nabla_\nu\dot x^\mu\cdot \dot x^\nu
=
\frac{\partial \dot x^\mu}{\partial x^\nu}\dot x^\nu
+
\Gamma^\mu_{\alpha\nu}\dot x^\alpha \dot x^\nu.
\]
Im letzten Ausdruck kommen $x^\alpha$ und $x^\nu$ in symmetrischer
Art und Weise vor.
Dies suggeriert, dass wir die $\Gamma^\mu_{\alpha\nu}$ möglicherweise
sogar symmetrisch wählen können, dass also
$\Gamma^\mu_{\alpha\nu}=\Gamma^\mu_{\nu\alpha}$
gewählt werden kann.
Wir fordern daher zusätzlich, dass die Zusammenhangskoeffizienten
symmetrisch sein sollen.
Es wird sich gleich zeigen, dass dies tatsächlich möglich ist.

\subsection{Paralleltransport}
Bis jetzt haben wir die Metrik nicht verwendet.
Wir möchten, dass der Paralleltransport die Länge des Vektors
paralleltransportierten Vektors $\tilde A^\mu$
beim Transport entlang einer Kurve nicht verändert.
Die Länge des Vektors wird durch $g_{\mu\nu}\tilde A^\mu \tilde A^\nu$
gegeben.
Die Ableitung entlang der Kurve $x^\mu(t)$ ist
\[
\frac{d}{dt} g_{\mu\nu}\tilde A^\mu \tilde A^\nu\bigg|_{t=0}
=
\dot x^\alpha
\frac{\partial g_{\mu\nu}}{\partial x^\alpha}A^\mu A^\nu
-
g_{\mu\nu}A^\mu\Gamma_{\alpha\beta}^\nu A^\alpha \dot x^\beta
-
g_{\mu\nu}\Gamma_{\alpha\beta}^\mu A^\alpha \dot x^\beta A^\nu
=
0.
\]
In diesem Ausdruck kommen in allen Termen $A^\mu$ und $\dot x^\beta$ mit
verschiedenen Indizes vor.
Damit wir diese Faktoren ausklammern können, benennen wir die Indizes um,
so dass sie in allen Termen gleich benannt sind.
Wir erhalten dann die Gleichung
\[
\biggl(
\frac{\partial g_{\mu\nu}}{\partial x^\alpha}
-
g_{\mu\beta}\Gamma_{\nu\alpha}^\beta
-
g_{\beta\nu}\Gamma_{\mu\alpha}^\beta
\biggr)
A^\mu A^\nu\dot x^\alpha
=
0
\]
für die Koeffizienten $\Gamma$.
Diese Gleichung muss für jede beliebige Wahl von $A^\mu$ und jede
beliebige Richtung der Kurve $\dot x^\alpha$ erfüllt sein.
Es folgt, dass der Klammerausdruck für alle Werte der freien, also nicht durch
die Summationskonvention als Laufindizes ausgezeichneten, Indizes verschwindet.
Wir erhalten also
\begin{equation}
\frac{\partial g_{\mu\nu}}{\partial x^\alpha}
-
g_{\mu\beta}\Gamma_{\nu\alpha}^\beta
-
g_{\beta\nu}\Gamma_{\mu\alpha}^\beta
=
0,
\label{skript:geodaeten:gammaidentitaet}
\end{equation}
ein System von Gleichungen für die $n^3$ Grössen $\Gamma_{\mu\nu}^\alpha$.
Wir können die $\Gamma$ auch allein auf der linken Seite haben:
\[
g_{\mu\beta}\Gamma_{\nu\alpha}^\beta
+
g_{\beta\nu}\Gamma_{\mu\alpha}^\beta
=
\frac{\partial g_{\mu\nu}}{\partial x^\alpha}
\]
Durch zyklische Vertauschung der drei Indizes $\mu$, $\nu$ und $\alpha$
erhalten wir drei Gleichungen
\[
\def\arraystretch{2.0}
\begin{linsys}{3}
g_{\mu\beta}\Gamma_{\nu\alpha}^\beta &+& g_{\beta\nu}\Gamma_{\mu\alpha}^\beta
& &
&=&
\displaystyle
\frac{\partial g_{\mu\nu}}{\partial x^\alpha}
\\
& &g_{\nu\beta}\Gamma_{\alpha\mu}^\beta &+& g_{\beta\alpha}\Gamma_{\nu\mu}^\beta
&=&
\displaystyle
\frac{\partial g_{\nu\alpha}}{\partial x^\mu}
\\
g_{\beta\mu}\Gamma_{\alpha\nu}^\beta
& &
&+&g_{\alpha\beta}\Gamma_{\mu\nu}^\beta 
&=&
\displaystyle
\frac{\partial g_{\alpha\mu}}{\partial x^\nu}
\end{linsys}
\]
Da sowohl $g$ als auch $\Gamma$ in den unteren Indizes symmetrisch
sind, können wir die Gleichungen weiter vereinfachen:
\[
\def\arraystretch{2.0}
\begin{linsys}{3}
g_{\mu\beta}\Gamma_{\nu\alpha}^\beta
	&+& g_{\nu\beta}\Gamma_{\mu\alpha}^\beta
		& &
&=&
\displaystyle
\frac{\partial g_{\mu\nu}}{\partial x^\alpha}
\\
	& &g_{\nu\beta}\Gamma_{\mu\alpha}^\beta
		&+& g_{\alpha\beta}\Gamma_{\nu\mu}^\beta
&=&
\displaystyle
\frac{\partial g_{\nu\alpha}}{\partial x^\mu}
\\
g_{\mu\beta}\Gamma_{\nu\alpha}^\beta
	& &
		&+&g_{\alpha\beta}\Gamma_{\nu\mu}^\beta 
&=&
\displaystyle
\frac{\partial g_{\alpha\mu}}{\partial x^\nu}
\end{linsys}
\]
Subtrahieren wir die erste Zeile von der Summe der letzten beiden,
heben sich ersten Terme weg, es bleibt
\[
g_{\alpha\beta}\Gamma_{\nu\mu}^\beta
=
\biggl(
\frac{\partial g_{\nu\alpha}}{\partial x^\mu}
+
\frac{\partial g_{\alpha\mu}}{\partial x^\nu}
-
\frac{\partial g_{\mu\nu}}{\partial x^\alpha}
\biggr).
\]
Bezeichen wir die inverse Matrix von $g_{\alpha\beta}$ mit
$g^{\alpha\beta}$, dann können wir nach $\Gamma_{\mu\nu}^\beta$ auflösen:
\[
g^{\sigma\alpha}
\biggl(
\frac{\partial g_{\nu\alpha}}{\partial x^\mu}
+
\frac{\partial g_{\alpha\mu}}{\partial x^\nu}
-
\frac{\partial g_{\mu\nu}}{\partial x^\alpha}
\biggr)
=
g^{\sigma\alpha}
g_{\alpha\beta}\Gamma_{\mu\nu}^\beta
=
\delta^\sigma_\beta\Gamma_{\mu\nu}^\beta
=
\Gamma_{\mu\nu}^\sigma.
\]

\begin{definition}
Sei $g_{\mu\nu}$ ein metrischer Tensor. 
Dann heissen die
\[
\Gamma_{\alpha,\mu\nu}
=
\frac12
\biggl(
\frac{\partial g_{\nu\alpha}}{\partial x^\mu}
+
\frac{\partial g_{\alpha\mu}}{\partial x^\nu}
-
\frac{\partial g_{\mu\nu}}{\partial x^\alpha}
\biggr)
\]
die {\em Christoffelsymbole erster Art}
und
\begin{equation}
\Gamma_{\mu\nu}^\sigma
=
g^{\sigma\alpha} \Gamma_{\alpha,\mu\nu}
=
\frac12
g^{\sigma\alpha}
\biggl(
\frac{\partial g_{\nu\alpha}}{\partial x^\mu}
+
\frac{\partial g_{\alpha\mu}}{\partial x^\nu}
-
\frac{\partial g_{\mu\nu}}{\partial x^\alpha}
\biggr)
\label{skript:definition:christoffel2}
\end{equation}
heissen {\em Christoffelsymbole zweiter Art} oder
{\em Zusammenhangskoeffizienten}.
Die auf diese Weise aus der Metrik gewonnenen Zusammenhangskoeffizienten
sind auch als der
{\em Levi-Cività-Zusammenhang} bekannt.
\index{Levi-Cività-Zusammenhang}%
\end{definition}

%Die Christoffelsymbole erster und zweiter Art bilden keinen Tensor.
%Im allgemeinen gilt
%\[
%
%\]

\section{Rechenregeln für die kovariante Ableitung}
\rhead{Rechenregeln für die kovariante Ableitung}
Die kovariante Ableitung eines Vektorfeldes wurde in 
\eqref{skript:geodaeten:kovarianteableitung}
für ein Vektorfeld $A^\mu$ definiert.
Später wurden die Koeffizienten $\Gamma^\alpha_{\mu\nu}$ so bestimmt,
dass der Paralleltransport, der als geometrische Vorstellung der
Formel~\eqref{skript:geodaeten:kovarianteableitung} zu Grunde lag,
Skalarprodukte erhält.
Für zwei Vektorfelder, die beide verschwindende
kovariante Ableitung in eine bestimmte Richtung haben,
verschwindet dann auch die Ableitung in diese Richtung.
Doch wie verhält sich die Ableitung eines Skalarprodukts, wenn
die einzelnen Faktoren nicht verschwindende kovariante Ableitung haben?
Ziel dieses Abschnittes ist, Rechenregeln für die kovariante Ableitung
zusammenzustellen.

\subsection{Ableitung eines Skalarproduktes}
Wir betrachten zwei Vektorfelder $A^\mu$ und $B^\mu$ und möchten wissen,
wie sich der Wert des Skalarproduktes $g_{\mu\nu}A^\mu B^\nu$ entlang
einer Kurve mit Tangentenrichtung $C^\mu$ ändert.
Seien $A^\mu$  und $B^\nu$ beliebige Vektorfelder,
dann gilt entlang einer Kurve $x^\alpha(s)$ mit
Richtungsvektor $C^\alpha = dx^\alpha/ds$ im Punkt mit Parameterwert $s=0$
\begin{align*}
\frac{d}{ds}(g_{\mu\nu}A^\mu B^\nu)\bigg|_{s=0}
&=
\frac{\partial g_{\mu\nu}}{\partial x^\alpha}A^\mu B^\nu
\frac{dx^\alpha(0)}{ds}
+
g_{\mu\nu}\frac{\partial A^\mu}{\partial x^\alpha}C^\alpha B^\nu
\frac{dx^\alpha(0)}{ds}
+
g_{\mu\nu}A^\mu \frac{\partial B^\nu}{\partial x^\alpha}C^\alpha
\frac{dx^\alpha(0)}{ds}
\\
&=
\biggl(
\frac{\partial g_{\mu\nu}}{\partial x^\alpha} A^\mu B^\nu
+
g_{\mu\nu}
\biggl(\nabla_\alpha A^\mu - \Gamma^\mu_{\beta\alpha}A^\beta \biggr)
B^\nu
+
g_{\mu\nu}A^\mu \biggl(\nabla_\alpha B^\nu-\Gamma^\nu_{\beta\alpha}B^\beta\biggr)
\biggr)C^\alpha
\\
&=
C^\alpha
\biggl(
g_{\mu\nu}\nabla_\alpha A^\mu B^\nu
+
g_{\mu\nu}A^\mu\nabla_\alpha B^\nu
+
\biggl(
\frac{\partial g_{\mu\nu}}{\partial x^\alpha}
-g_{\beta\nu}\Gamma^\beta_{\mu\alpha}
-g_{\mu\beta}\Gamma^\beta_{\nu\alpha}
\biggr)A^\mu B^\nu
\biggr)
\end{align*}
Der innere Klammerausdruck in der letzten Gleichung ist aber identisch
mit \eqref{skript:geodaeten:gammaidentitaet}, er verschwindet also.
Damit können wir die Komponenten Ableitung in Richtung $C^\mu$ auch
so schreiben:
\[
\frac{\partial}{\partial x^\alpha}
(g_{\mu\nu}A^\mu B^\nu)
=
g_{\mu\nu}\nabla_\alpha A^\mu B^\nu
+
g_{\mu\nu}A^\mu\nabla_\alpha B^\nu,
\]
dies ist die Produktregel für die kovariante Ableitung.

\begin{satz}
Für zwei beliebige Vektorfelder sind die Komponenten der Ableitung
des Skalarproduktes 
\begin{equation}
\frac{\partial}{\partial x^\alpha}
(g_{\mu\nu}A^\mu B^\nu)
=
g_{\mu\nu}\nabla_\alpha A^\mu B^\nu
+
g_{\mu\nu}A^\mu\nabla_\alpha B^\nu.
\label{skript:geodaeten:produktregel}
\end{equation}
\end{satz}

\subsection{Kovariante Ableitung kontravarianter Vektoren}
Bis jetzt können wir nur die kovariante Ableitung von Vektorkomponenten
$A^\mu$ berechnen, eine kovariante Ableitung von Komponenten $B_\nu$
ist noch gar nicht definiert.
Wir können aber jeden Vektor $B_\nu$ also das Resultat einer Operation
$B_\nu=B^\alpha g_{\alpha\nu}$ betrachten.
Das Skalarprodukt $A^\mu B_\mu=A^\mu B^\nu g_{\mu\nu}$ erfüllt dann
wieder die Produktregel~\eqref{skript:geodaeten:produktregel},
es folgt
\[
\frac{\partial}{\partial x^\alpha}A^\mu B_\mu
=
\frac{\partial}{\partial x^\alpha}g_{\mu\nu} A^\mu B^\nu
=
g_{\mu\nu} \nabla_\alpha A^\mu B^\nu
+
g_{\mu\nu} A^\mu \nabla_\alpha B^\nu
=
\nabla_\alpha A^\mu B_\mu
+
A^\mu g_{\mu\nu} \nabla_\alpha B^\nu.
\]
Wenn die kovariante Ableitung für $B_\nu$ überhaupt definiert werden 
kann, dann muss gelten
\begin{align}
\nabla_\alpha B_\nu
&=
g_{\mu\nu}\nabla_\alpha B^\mu
\notag
\\
&=
g_{\mu\nu}\frac{\partial B^\mu}{\partial x^\alpha}
+
g_{\mu\nu}\Gamma^\mu_{\alpha\beta}B^\beta
\notag
\\
&=
g_{\mu\nu}\frac{\partial g^{\mu\beta}B_\beta}{\partial x^\alpha}
+
g_{\mu\nu}\Gamma^\mu_{\alpha\beta}g^{\beta\sigma}B_\sigma
\notag
\\
&=
g_{\mu\nu}\frac{\partial g^{\mu\beta}}{\partial x^\alpha} B_\beta
+
\underbrace{g_{\mu\nu}g^{\mu\beta}}_{\displaystyle=\delta_\nu^\beta}\frac{\partial B_\beta}{\partial x^\alpha}
+
g_{\mu\nu}\Gamma^\mu_{\alpha\beta}g^{\beta\sigma}B_\sigma
\notag
\\
&=
\frac{\partial B_\nu}{\partial x^\alpha}
+
g_{\mu\nu}\biggl(
\frac{\partial g^{\mu\sigma}}{\partial x^\alpha}
+\Gamma^\mu_{\alpha\beta}g^{\beta\sigma}
\biggr)B_\sigma
\label{skript:geodaeten:kovkov1}
\end{align}
Die Ableitungen der inversen Matrix $g^{\mu\sigma}$ berechnen wir
durch Ableiten der Bedingung $\delta_\nu^\sigma=g_{\nu\mu}g^{\mu\sigma}$:
\begin{equation}
0
=
\frac{\partial}{\partial x^\alpha}\delta_\nu^\sigma
=
\frac{\partial g_{\mu\nu}}{\partial x^\alpha} g^{\mu\sigma}
+
g_{\mu\nu}
\frac{\partial g^{\mu\sigma}}{\partial x^\alpha}
\qquad\Rightarrow\qquad
g_{\mu\nu}
\frac{\partial g^{\mu\sigma}}{\partial x^\alpha}
=
-
\frac{\partial g_{\mu\nu}}{\partial x^\alpha} g^{\mu\sigma}.
\label{skript:geodaeten:ginvabl}
\end{equation}
Setzen wir jetzt~\eqref{skript:geodaeten:ginvabl}
in~\eqref{skript:geodaeten:kovkov1} ein, erhalten wir
\begin{align*}
\nabla_\alpha B_\nu
&=
\frac{\partial B_\nu}{\partial x^\alpha}
+
\biggl(
-\frac{\partial g_{\mu\nu}}{\partial x^\alpha} g^{\mu\sigma}
+
g_{\mu\nu}
\Gamma^\mu_{\alpha\beta}g^{\beta\sigma}
\biggr)B_\sigma
\end{align*}
In der Klammer ersetzen wir $\Gamma^\mu_{\alpha\beta}$ durch
\eqref{skript:definition:christoffel2} und erhalten
\begin{align*}
-\frac{\partial g_{\mu\nu}}{\partial x^\alpha} g^{\mu\sigma}
+
g_{\mu\nu}
\Gamma^\mu_{\alpha\beta}
g^{\beta\sigma}
&=
-\frac{\partial g_{\mu\nu}}{\partial x^\alpha} g^{\mu\sigma}
+
g_{\mu\nu}
\frac12
g^{\mu\gamma}
\biggl(
\frac{\partial g_{\alpha\gamma}}{\partial x^\beta}
+
\frac{\partial g_{\beta\gamma}}{\partial x^\alpha}
-
\frac{\partial g_{\alpha\beta}}{\partial x^\gamma}
\biggr)
g^{\beta\sigma}
\\
&=
\biggl(
-\frac{\partial g_{\beta\nu}}{\partial x^\alpha}
+
\delta_\nu^\gamma
\frac12
\biggl(
\frac{\partial g_{\alpha\gamma}}{\partial x^\beta}
+
\frac{\partial g_{\beta\gamma}}{\partial x^\alpha}
-
\frac{\partial g_{\alpha\beta}}{\partial x^\gamma}
\biggr)
\biggr)
g^{\beta\sigma}
\\
&=
\frac12
\biggl(
-2\frac{\partial g_{\beta\nu}}{\partial x^\alpha}
+
\frac{\partial g_{\alpha\nu}}{\partial x^\beta}
+
\frac{\partial g_{\beta\nu}}{\partial x^\alpha}
-
\frac{\partial g_{\alpha\beta}}{\partial x^\nu}
\biggr)
g^{\beta\sigma}
\\
&=
-\frac12
\biggl(
\frac{\partial g_{\beta\nu}}{\partial x^\alpha}
+
\frac{\partial g_{\alpha\beta}}{\partial x^\nu}
-
\frac{\partial g_{\alpha\nu}}{\partial x^\beta}
\biggr)
g^{\beta\sigma}
=
\Gamma_{\beta,\alpha\nu}g^{\beta\sigma}
\\
&=-\Gamma^\sigma_{\alpha\nu}.
\end{align*}
Die rechte Seite der kovarianten Ableitung für einen kovarianten Vektor
kann daher mit den bereits bekannten Christoffel-Symbolen zweiter Art
ausgedrückt werden.

\begin{definition}
Die Komponenten der kovariante Ableitung $\nabla_\alpha B_\nu$ eines
kovarianten Vektorfeldes $B_\nu$ in Richtung der Koordinaten $x^\alpha$
ist
\begin{equation}
\nabla_\alpha B_\nu
=
\frac{\partial B_\nu}{\partial x^\alpha}
-\Gamma^{\sigma}_{\alpha\nu}B_\sigma,
\label{skript:geodaeten:kovabl2}
\end{equation}
diese wird auch $\nabla_\alpha B_\nu=B_{\nu;\alpha}$ geschrieben.
\end{definition}

\subsection{Kovariante Ableitung von Tensoren beliebiger Stufe}
Wir möchten die kovariante Ableitung jetzt auf beliebige Tensoren
verallgemeinern.
Sie soll natürlich linear bleiben und die Produktregel, die für jede
Art von Ableitung gilt, muss ebenfalls erhalten bleiben.
Aus diesen Voraussetzungen lässt sich eine allgemeine Formel für
die kovariante Ableitung eines beliebigen Tensors herleiten.

Ein Tensor der Form $t_{\mu\nu}$ kann immer geschrieben werden
als Linearkombination von Tensoren der Form $a_\mu b_\nu$, es reicht
also wegen der Linearität der kovarianten Ableitung, wenn wir solche
Produkte ableiten können.
Dafür muss aber die Produktregel gelten.

Auf einem Produkt $a_\mu b_\nu$ muss die kovariante Ableitung die Form
\begin{align*}
\nabla_\alpha (a_\mu b_\nu)
&=
(\nabla_\alpha a_\mu) b_\nu
+
a_\mu(\nabla_\alpha b_\nu)
\\
&=
\biggl(
\frac{\partial a_\mu}{\partial x_\alpha}
-
\Gamma^\sigma_{\alpha\mu}a_\sigma
\biggr)b_\nu
+
a_\mu
\biggl(
\frac{\partial b_\nu}{\partial x_\alpha}
-
\Gamma^\sigma_{\alpha\nu}b_\sigma
\biggr)
\\
&=
\frac{\partial a_\mu b_\nu}{\partial x^\alpha}
-\Gamma^\sigma_{\alpha\mu}a_\sigma b_\nu
-\Gamma^\sigma_{\alpha\nu}a_\mu b_\sigma.
\end{align*}
Auf analoge Art können auch obere Indizes behandelt werden.
Dies führt uns auf die folgende allgemeine Definition für die 
koveriante Ableitung von Tensonren beliebiger Stufe.

\begin{definition}
Sei $A$ ein beliebiger Tensor, von dem wir nur die Indizes
$\mu$ und $\nu$ in der Form
$A^{\dots \mu\dots}\mathstrut_{\dots\nu\dots}$
explizit ausschreiben wollen.
Die kovariante Ableitung von $A$ ist
\begin{equation}
\nabla_\alpha A^{\dots \mu\dots}\mathstrut_{\dots\nu\dots}
=
\frac{\partial A^{\dots \mu\dots}\mathstrut_{\dots\nu\dots}}{\partial x^\alpha}
+
\dotsc
+
\Gamma^\mu_{\sigma\alpha}
A^{\dots \sigma\dots}\mathstrut_{\dots\nu\dots}
+
\dotsc
-
\Gamma^\sigma_{\alpha\nu}
A^{\dots \mu\dots}\mathstrut_{\dots\sigma\dots}
+
\dotsc,
\label{skript:geodaten:kovablbel}
\end{equation}
wobei für jeden kontravarianten Index ein Term mit positivem
Vorzeichen und für jeden kovarianten Index ein Term mit negativem
Vorzeichen hinzukommen.
\end{definition}

Aus der Formel~\eqref{skript:geodaten:kovablbel} können wir jetzt eine
ganze Reihen von Schlussfolgerungen über das Rechnen mit der kovarianten
ableiten.
In den folgenden Abschnitten soll gezeigt werden, dass die kovariante
Ableitung mit den Tensoroperationen Verjüngung und Index hoch-
bzw.~herunterziehen verträglich ist.

\subsubsection{Kovariante Ableitung und Verjüngung%
\label{skript:geodaeten:kovablverj}}
Wir betrachten jetzt einen Tensor der Form $T^\mu_\nu$, seine kovariante
Ableitung ist nach \eqref{skript:geodaeten:kovabl2}
\begin{equation}
\nabla_\alpha T^\mu_\nu
=
\frac{\partial T^\mu\nu}{\partial x^\alpha}
+\Gamma^\mu_{\sigma\alpha} T^\sigma_\nu
-\Gamma^\sigma_{\alpha\nu} T^\mu_\sigma.
\end{equation}
Verjüngen wir die beiden Indizes $\mu$ und $\nu$ wird daraus 
\begin{equation}
\nabla_\alpha T^\mu_\mu
=
\frac{\partial T^\mu\mu}{\partial x^\alpha}
+\Gamma^\mu_{\sigma\alpha} T^\sigma_\mu
-\Gamma^\sigma_{\alpha\mu} T^\mu_\sigma.
\label{skript:geodaeten:kovablverj1}
\end{equation}
Die letzten zwei Terme sind bis auf die Benennung der Indizes, über die
summiert wird, gleich, sie heben sich daher weg.
Wir schliessen daher, dass 
\begin{equation}
\nabla T^\mu_\mu = \frac{\partial}{\partial x^\alpha}T^\mu_\mu
\label{skript:geodaeten:kovablverj2}
\end{equation}
ist.

Man beachte, dass in der Notation von
\eqref{skript:geodaeten:kovablverj1}
und
eine Zweideutigkeit steckt.
Es ist daraus nämlich klar, ab zuerst die Verjüngung gebildet werden muss
und dann die kovariante Ableitung oder Umgekehrt.
Da die Verjüngung ein Skalar ist, ist die kovariante Ableitung einfach
nur die partielle Ableitung.
Die Schlussfolgerung~\eqref{skript:geodaeten:kovablverj2} besagt daher
einfach nur, dass man auch erst die kovariante Ableitung bilden und
dann erst verjüngen kann, dass die Zweideutigkeit der Notation also
kein Problem ist.

\subsubsection{Kovariante Ableitung der Metrik}
Mit Hilfe der Formel~\eqref{skript:geodaten:kovablbel} erlaubt uns nun
auch, die kovariante Ableitung des metrischen Tensors zu berechnen.
Die Rechnung ergibt
\begin{align*}
\nabla_\alpha g_{\mu\nu}
&=
\frac{\partial g_{\mu\nu}}{\partial x^\alpha}
-\Gamma^\sigma_{\alpha\mu}g_{\sigma\nu}
-\Gamma^\sigma_{\alpha\nu}g_{\mu\sigma}
=
\frac{\partial g_{\mu\nu}}{\partial x^\alpha}
-\Gamma_{\nu,\alpha\mu}
-\Gamma_{\mu,\alpha\nu}
\\
&=
\frac{\partial g_{\mu\nu}}{\partial x^\alpha}
-
\frac12\biggl(
\frac{\partial g_{\mu\nu}}{\partial x^\alpha}
+
\frac{\partial g_{\alpha\nu}}{\partial x^\mu}
-
\frac{\partial g_{\alpha\mu}}{\partial x^\nu}
\biggr)
-
\frac12\biggl(
\frac{\partial g_{\nu\mu}}{\partial x^\alpha}
+
\frac{\partial g_{\alpha\mu}}{\partial x^\nu}
-
\frac{\partial g_{\alpha\nu}}{\partial x^\mu}
\biggr)
=0.
\end{align*}
Die kovariante Ableitung des metrischen Tensors verschwindet.
Dies ist nicht überraschend, die kovariante Ableitung war ja mit dem Ziel
konstruiert worden, dass sie ausdrücken kann, dass sich Längen und
Winkel, die mit dem metrischen Tensor konstruiert werden, nicht ändern.

\subsubsection{Index Herunterziehen und kovariante Ableitung}
Sie $A^\mu$ ein kontravarianter Vektor.
Der zugehörige kovariante Vektor $A_\nu=A^\mu g_{\mu\nu}$ entsteht
durch Herunterziehen des Index mit Hilfe der Metrik.
Aus Abschnitt \ref{skript:geodaeten:kovablverj} wissen wir, dass es nicht
auf die Reihenfolge von kovarianter Ableitung und Verjüngung ankommt.
Wir schliessen daher aus der Rechnung
\begin{align*}
\nabla_\alpha A_\nu
&=
\nabla_\alpha(A^\mu g_{\mu\nu})
=
\nabla_\alpha A^\mu g_{\mu\nu} + A^\mu\nabla_\alpha g_{\mu\nu}
=
\nabla_\alpha A^\mu g_{\mu\nu},
\end{align*}
dass herunterziehen eines Index ebenfalls mit der kovarianten Ableitung
vertauscht.

\section{Beispiele}
%\rhead{Beispiele}
In den nachfolgenden Beispielen wollen wir die Christoffelsymbole erster
und zweiter Art für Zylinder- und Kugeloberfläche berechnen.

\rhead{Beispiele}

\subsection{Polarkoordinaten}
Die nicht verschwinden Komponenten
des metrischen Tensors in Polarkoordinaten $(r,\varphi)$
sind $g_{11}=1$ und $g_{22}=r^2$.
Davon brauchen wir die Ableitungen
\[
\begin{aligned}
\frac{\partial g_{11}}{\partial x^1} &=0,&
\frac{\partial g_{12}}{\partial x^1} &=0,&
\frac{\partial g_{21}}{\partial x^1} &=0,&
\frac{\partial g_{22}}{\partial x^1} &=2r,&
\\
\frac{\partial g_{11}}{\partial x^2} &=0,&
\frac{\partial g_{12}}{\partial x^2} &=0,&
\frac{\partial g_{21}}{\partial x^2} &=0,&
\frac{\partial g_{22}}{\partial x^2} &=0
\end{aligned}
\]
Die Christoffelsymbole erster Art sind daher 
\[
\begin{aligned}
\Gamma_{1,11} &=  0,&
\Gamma_{1,12} &=  0,&
\Gamma_{1,21} &=  0,&
\Gamma_{1,22} &= -r,
\\
\Gamma_{2,11} &=  0,&
\Gamma_{2,12} &=  r,&
\Gamma_{2,21} &=  r,&
\Gamma_{2,22} &=  0
\end{aligned}
\]
und die Christoffelsymbole zweiter Art sind
\begin{equation}
\begin{aligned}
\Gamma_{11}^1 &= 0,&
\Gamma_{12}^1 &= 0,&
\Gamma_{21}^1 &= 0,&
\Gamma_{22}^1 &=-r,
\\
\Gamma_{11}^2 &= 0,&
\Gamma_{12}^2 &= \frac1{r},&
\Gamma_{21}^2 &= \frac1{r},&
\Gamma_{22}^2 &= 0.
\end{aligned}
\label{skript:geodaeten:christoffel:polar}
\end{equation}

\subsection{Zylinderkoordinaten}
Da die Komponenten des metrischen Tensors sind konstant, damit verschwinden
alle Ableitungen
\[
\frac{\partial g_{\mu\nu}}{\partial x^\alpha}=0
\qquad\Rightarrow\qquad
\Gamma_{\alpha,\mu\nu}=0
\qquad\Rightarrow\qquad
\Gamma_{\mu\nu}^\alpha=0.
\]

\subsection{Kugelkoordinaten}
Die Kugeloberfläche verwendet die Koordinaten $(\vartheta,\varphi)$.
Zunächst brauchen wir die Ableitungen der Komponenten des metrischen
Tensors
\begin{align*}
\frac{\partial g_{11}}{\partial \vartheta} &=0,
&
\frac{\partial g_{12}}{\partial \vartheta} &=0,
&
\frac{\partial g_{21}}{\partial \vartheta} &=0,
&
\frac{\partial g_{22}}{\partial \vartheta} &=\sin2\vartheta,
\\
\frac{\partial g_{11}}{\partial \varphi} &=0,
&
\frac{\partial g_{12}}{\partial \varphi} &=0,
&
\frac{\partial g_{21}}{\partial \varphi} &=0,
&
\frac{\partial g_{22}}{\partial \varphi} &=0.
\\
\end{align*}
Daraus können wir die Christoffelsymbole erster Art ableiten:
\begin{align*}
 \Gamma_{1,11}
&=
\frac12\biggl(\frac{\partial g_{11}}{\partial \vartheta}
	+ \frac{\partial g_{11}}{\partial \vartheta}
	- \frac{\partial g_{11}}{\partial \vartheta}\biggr)=0,
&\Gamma_{1,12}
&=
\frac12\biggl(\frac{\partial g_{11}}{\partial \varphi}
	+ \frac{\partial g_{21}}{\partial \vartheta}
	- \frac{\partial g_{12}}{\partial \vartheta}\biggr)=0,
\\
\Gamma_{1,21}
&=
\frac12\biggl(\frac{\partial g_{12}}{\partial \vartheta}
	+ \frac{\partial g_{11}}{\partial \varphi}
	- \frac{\partial g_{21}}{\partial \vartheta}\biggr)=0,
&\Gamma_{1,22}
&=
\frac12\biggl(\frac{\partial g_{12}}{\partial \varphi}
	+ \frac{\partial g_{12}}{\partial \varphi}
	- \frac{\partial g_{22}}{\partial \vartheta}\biggr)=-\frac12\sin2\vartheta,
\\
\Gamma_{2,11}
&=
\frac12\biggl(\frac{\partial g_{12}}{\partial \vartheta}
	+ \frac{\partial g_{12}}{\partial \vartheta}
	- \frac{\partial g_{11}}{\partial \varphi}\biggr)=0,
&\Gamma_{2,12}
&=
\frac12\biggl(\frac{\partial g_{12}}{\partial \varphi}
	+ \frac{\partial g_{22}}{\partial \vartheta}
	- \frac{\partial g_{12}}{\partial \varphi}\biggr)=\frac12\sin2\vartheta,
\\
\Gamma_{2,21}
&=
\frac12\biggl(\frac{\partial g_{12}}{\partial \varphi}
	+ \frac{\partial g_{22}}{\partial \vartheta}
	- \frac{\partial g_{21}}{\partial \varphi}\biggr)=\frac12\sin2\vartheta,
&\Gamma_{2,22}
&=
\frac12\biggl(\frac{\partial g_{22}}{\partial \varphi}
	+ \frac{\partial g_{22}}{\partial \varphi}
	- \frac{\partial g_{22}}{\partial \varphi}\biggr)=0.
\end{align*}
Die inverse Matrix von $g_{\mu\nu}$ hat die nicht verschwindenden
Komponenten
\[
g^{11} = 1
\qquad\text{und}\qquad
g^{22} = \frac1{\sin^2\vartheta}
\]
und die Christoffelsymbole 2.~Art
\begin{equation}
\begin{aligned}
 \Gamma_{11}^1
&=0,
&\Gamma_{12}^1
&=0,
&\Gamma_{21}^1
&=0,
&\Gamma_{22}^1
&=-\frac12\sin2\vartheta,
\\
 \Gamma_{11}^2
&=0,
&\Gamma_{12}^2
&=\cot\vartheta,
&\Gamma_{21}^2
&=\cot\vartheta,
&\Gamma_{22}^2
&=0.
\end{aligned}
\label{skript:kruemmung:christoffelkugel}
\end{equation}


\section{Geodätengleichung}
\rhead{Geodätengleichung}
Die Rolle der Geraden in der Ebene müssen diejenigen Kurven auf der
Fläche übernehmen, die so gerade wie möglich sind. 
Eine Gerade ist dadurch charakterisiert, dass sie überall die gleiche
Richtung hat. 
Diese Formulierung ist aber nur möglich, weil Tangentialvektoren in
beliebigen Punkten unmittelbar miteinander vergleichen können.
Für einen allgemeinen Raum müssen wir diesen Vergleich mit Hilfe
des Paralleltransportes durchführen.
Die Forderung an die Kurve wird dann, dass der Tangentialvektor
an die Kurve durch Paralleltransport wieder in einen Tangentialvektor
an die Kurve übergeht.

Etwas formaler setzen wir den Tangentialvektor $\dot x^\mu$ an die
Kurve in die Gleichung \eqref{skript:kruemmung:ableitung} ein und
erhalten die Differentialgleichung 
\begin{equation}
\ddot x^\alpha+\Gamma_{\mu\nu}^\alpha \dot x^\mu\dot x^\nu=0.
\label{skript:kruemmung:geodatengleichung}
\end{equation}
\index{Geodäte}%

\begin{definition}
Eine Kurve in einem Riemannschen Raum heisst eine Geodäte, wenn sie
die Differentialgleichung~\eqref{skript:kruemmung:geodatengleichung}
erfüllt.
\end{definition}

In den nachfolgenden Beispielen wollen wir die Geodäten für diejenigen
Räume berechnen, für die wir die Christoffelsymbole bereits bestimmt
haben.

%
% XXX Geschwindigkeitsinvarianz
%

\subsection{Flache Räume}
In flachen Räumen wie der Ebene oder der Zylinderoberfläche verschwinden
alle Christoffelsymbole.
Die Geodätengleichung ist daher nur noch
$\ddot x^\alpha=0$, was gleichbedeutend ist mit
\[
x^\alpha(t)=x^\alpha(0) + t \dot x^\alpha(0),
\]
also einer Geradengleichung.
In flachen Räumen sind die Geodäten Geraden.

\subsection{Polarkoordinaten}
Die Geodätengleichungen in Polarkoordinaten erhält man durch
Einsetzen der Christoffelsymbole von
\eqref{skript:geodaeten:christoffel:polar}
in
\eqref{skript:kruemmung:geodatengleichung}.
Die Gleichungen lauten
\begin{equation}
\begin{aligned}
\ddot r &= r\dot \varphi^2,
\\
\ddot \varphi &= -2\frac{\dot r}{r}\dot\varphi.
\end{aligned}
\label{skript:geodaeten:dglpolar}
\end{equation}
Radiale Geraden haben $\dot\varphi=0$, aus der ersten Gleichung folgt dann
$\ddot r=0$ oder $\dot r=\operatorname{const}$.
Geodäten durch den Nullpunkt sind also mit konstanter Geschwindigkeit
durchlaufene Geraden.

Wir dürfen natürlich davon ausgehen, dass jede beliebige andere Gerade
ebenfalls eine Geodäte ist.
Um dies nachzuprüfen betrachten wir eine Gerade senkrecht auf die
$r=0$-Achse, in $x$-$y$-Koordinaten können wir sie als
$t\mapsto (x_0,t)$ beschreiben.
Die zugehörigen Polarkoordinaten sind
\[
\begin{aligned}
r(t)&=\sqrt{x_0^2+t^2}
&&\text{und}&
\varphi(t)&=\arctan\frac{t}{x_0}.
\end{aligned}
\]
Wir prüfen durch einsetzen, ob diese Funktionen die
Geodätendifferentialgleichung~\eqref{skript:geodaeten:dglpolar}
erfüllen.
Die ersten und zweiten Ableitungen von $r(t)$ und $\varphi(t)$ sind
\begin{align*}
\dot r(t)
&=\frac{t}{r},
&
\ddot r(t)
&=
\frac{1}{r}-\frac{t^2}{r^3}
=
\frac{r^2-t^2}{r^3}
=
\frac{x_0^2}{r^3},
\\
\dot\varphi(t)
&=
\frac{x_0}{r^2},
&
\ddot\varphi(t)
&=
-\frac{2tx_0}{r^4}.
\end{align*}
Setzen wir dies in die
Differentialgleichungen~\eqref{skript:geodaeten:dglpolar}
ein, erhalten wir
\begin{align*}
r\dot\varphi^2
&=
r\frac{x_0^2}{r^4}
=
\frac{x_0^2}{r^3}
=
\ddot r,
\\
-2\frac{\dot r}{r}\dot\varphi
&=
-2\frac{t}{r}\frac1{r}\frac{x_0}{r^2}
=
-\frac{2tx_0}{r^4}
=
\ddot\varphi.
\end{align*}
Wie erwartet erfüllt also die Gerade die Geodätengleichung.

\subsection{Kugeloberfläche}
Die Christoffelsymbole zweiter Art für die Kugeloberfläche haben wir
in \eqref{skript:kruemmung:christoffelkugel} berechnet.
Statt die Differentialgleichung der Geodäten direkt zu lösen, versuchen
wir nur, die Parameterdarstellung eines Grosskreises in die
Differentialgleichung einzusetzen und damit zu zeigen, dass die
Grosskreise Lösungen der Geodätengleichung sind.
Da es durch jeden Punkt der Kugeloberflächen und zu jeder Tangentialrichtung
einen Grosskreis gibt, und die Lösungen der Geodätengleichung eindeutig
sind, können wir schliessen, dass alle Geodäten Grosskreise sind.

Der Äquator der Kugel hat eine besonders einfache Parametrisierung mit
\[
\begin{aligned}
x^1(t)=\vartheta(t)&=\frac{\pi}2,
&\qquad&&
x^2(t)=\varphi(t)&=t
\end{aligned}
\]
mit den Ableitungen
\[
\begin{aligned}
\dot x^1(t)&=0,
&\qquad&&
\dot x^2(t)&=1.
\end{aligned}
\]
Die zweiten Ableitungen der Koordinaten verschwinden, da die ersten
Ableitungen konstant sind.
Wir setzen die ersten Ableitungen in die Geodätengleichung ein.
Es bleiben nur die Terme mit $\mu=\nu=2$ stehen, da $\dot x^1=0$ ist,
also
\begin{align}
\ddot x^1
&=
-\Gamma_{\mu\nu}^1\dot x^\mu\dot x^\nu
=
-\Gamma_{22}^1=\frac12\sin2\theta
=
\frac12\sin\biggl( 2\cdot\frac{\pi}2\biggr)
=
\frac12\sin\pi
=
0,
\label{skript:kruemmung:geodaete:breitenkreis}
\\
\ddot x^2
&=
-\Gamma_{\mu\nu}^2\dot x^\mu\dot x^\nu
=
-\Gamma_{22}^2=0.
\notag
\end{align}
Die Geodätengleichung ist für den Äquator erfüllt, der Äquator ist
eine Geodäte.

An dieser Stelle könnten wir mit den Rechnungen eigentlich aufhören, 
denn jeder andere Grosskreis entsteht aus dem Äquator durch eine
Drehung des dreidimensionalen Raumes.
Die Eigenschaft einer Kurve, eine Geodäte zu sein, hängt aber nur von
der Metrik ab, die sich bei einer solchen Drehung nicht ändert.
Jeder andere Grosskreis ist also automatisch auch eine Geodäte.
Trotzdem wollen wir die Frage für Breitenkreise und Meridiane
gesondert untersuchen.

Ein Breitenkreis ist nach der gleichen Rechnung keine Geodäte.
Er ist charakterisiert durch $x^1(t)=\vartheta(t)\ne\frac{\pi}2$,
die Differentialgleichung
\eqref{skript:kruemmung:geodaete:breitenkreis}
wird damit zu
\[
\ddot x^1
=
-\Gamma_{\mu\nu}^1\dot x^\mu\dot x^\nu
=
-\Gamma_{22}^1
=
\frac12\sin2\theta
\ne
0,
\]
die Differentialgleichung der Geodäten ist nicht erfüllt.
Die rechte Seite ist sogar konstant, dies besagt dass der Breitenkreis
im Bezug zu einer Geodäten so gekrümmt ist, dass er ganz
auf einer Seite des Grosskreises mit gleichem Anfangspunkt und
Anfangsgeschwindigkeit liegt.

Als nächstes betrachten wir die Meridiane der Kugel.
Ein Meridian hat die Parameterdarstellung
\[
\begin{aligned}
x^1(t)=\vartheta(t)&=t,
&\qquad&&
x^2(t)=\varphi(t)&=0.
\end{aligned}
\]
Die Tangentialvektoren sind daher
\[
\begin{aligned}
\dot x^1(t)&=1,
&\qquad&&
\dot x^2(t)&=0.
\end{aligned}
\]
Wiederum verschwinden
die zweiten Ableitungen von $x^\mu$.
Wir setzen dies jetzt in die Geodäten\-gleichung ein und erhalten
\begin{align*}
\ddot x^1
&=
-\Gamma_{\mu\nu}\dot x^\mu \dot x^\nu
=
-\Gamma_{22}^1\dot x^2 \dot x^2
=
0,
\\
\ddot x^2
&=
-\Gamma_{\mu\nu}^2\dot x^\mu \dot x^\nu
=
-\Gamma_{11}^2\dot x^1\dot x^1
=
-\Gamma_{11}^2
=
0.
\end{align*}
Die Differentialgleichungen für eine Geodäte sind für Meridiane erfüllt,
damit ist gezeigt, dass Meridiane Geodäten sind.

Die Berechnung für einen beliebigen Grosskreis ist dagegen sehr 
kompliziert und nicht sehr instruktiv.

\section{Geodäten und Variationsprinzipien%
\label{skript:geodaeten:section:variationsprinzip}}
\rhead{Variationsprinzipien}
Wir möchten in diesem Abschnitt verstehen, dass die Geodäten tatsächlich 
die kürzesten Kurven sind.
Dazu müssen wir für eine beliebige mit dem Parameter $t$ parametrisierte
Kurve $x^\alpha(t)$ die Länge berechnen.
Dies kann mit dem Integral
\begin{equation}
l=\int_{t_0}^{t_1} \sqrt{g_{\mu\nu}(x^\alpha(t)) \dot x^\mu(t)\dot x^\nu(t)}\,dt
\label{skript:variation:laenge}
\end{equation}
geschehen.
Gesucht wird jetzt unter allen möglichen Kurven, die zwei Punkte miteinander
verbinden, diejenige für die das Integral \eqref{skript:variation:laenge}
minimal wird.

\subsection{Euler-Lagrange-Gleichung%
\label{skript:geodaeten:subsection:Euler-Lagrange-Gleichung}}
\begin{figure}
\includegraphics{chapters/tikz/lagrange.pdf}
\caption{Nachbarkurven der Lösungskurve eines Variationsproblems
für die Herleitung der Euler-Lagrange-Gleichung.
\label{skript:geodaeten:fig:lagrange}}
\end{figure}
Wir lösen gleich ein etwas allgemeineres Problem.
Gegeben ist eine Funktion $F(x^\alpha, \dot x^\alpha, t)$, welche
von den Koordinaten $x^\mu$, den Geschwindigkeiten $\dot x^\mu$ und
der Zeit abhängt.
Gesucht ist eine Kurve $x^\alpha(t)$, welche 
den Punkt $x^\alpha(t_0)$ mit dem Punkt $x^\alpha(t_1)$ verbindet,
und das Integral
\begin{equation}
\int_{t_0}^{t_1} F(x^\alpha(t), \dot x^\alpha(t), t)\,dt
\label{skript:variation:funktional}
\end{equation}
minimiert.
Das Problem, eine kürzeste Kurve zu finden, ist der Fall
\[
F(x^\alpha(t),\dot x^\alpha(t),t)
=
\sqrt{g_{\mu\nu}(x^\alpha(t)) \dot x^\mu(t)\dot x^\nu(t)},
\]
wie man durch Vergleich mit \eqref{skript:variation:laenge}
erkennen kann.

Wenn $x^\alpha(t)$ das Integral \eqref{skript:variation:funktional}
minimiert, dann wird es grösser, wenn man die Kurve etwas ändert.
Wir wählen also eine Kurve $\eta^\alpha(t)$ und berechnen, die
um $\eta^\mu(t)$ verschobene Kurve
(Abbildung~\ref{skript:geodaeten:fig:lagrange})
wird ein grösseres Integral
\eqref{skript:variation:funktional} ergeben.
Wir bilden daher
\[
I(s)
=
\int_{t_0}^{t_1}
F(x^\alpha(t) + s\eta^\alpha(t), \dot x^\alpha(t) + s \dot \eta^\alpha(t), t)
\,dt.
\]
Dieses Integral soll minimal werden für $s=0$, also muss die Ableitung
dort verschwinden.
Wir leiten daher $I(s)$ nach $s$ ab:
\begin{align*}
\frac{dI(s)}{ds}
&=
\int_{t_0}^{t_1}
\frac{\partial F}{\partial x^\alpha}(x^\alpha(t)+s\eta^\alpha(t),
\dot x^\alpha(t) + s\dot\eta^\alpha(t), t) \eta^\alpha(t)
+
\frac{\partial F}{\partial \dot x^\alpha}(x^\alpha(t) + s\eta^\alpha(t),
\dot x^\alpha(t) + s\dot\eta^\alpha(t), t) \dot\eta^\alpha(t)
\,dt.
\end{align*}
Wir brauchen nur den Wert an der Stelle $s=0$:
\begin{align*}
\frac{dI(0)}{ds}
&=
\int_{t_0}^{t_1}
\frac{\partial F}{\partial x^\alpha}(x^\alpha(t), \dot x^\alpha(t), t)
\eta^\alpha(t)
+
\frac{\partial F}{\partial \dot x^\alpha}(x^\alpha(t),
\dot x^\alpha(t), t)
\dot\eta^\alpha(t)
\,dt.
\\
\intertext{
Den zweiten Term können wir mit partieller Integration vereinfachen
und erhalten}
&=
\int_{t_0}^{t_1}
\frac{\partial F}{\partial x^\alpha}(x^\alpha(t), \dot x^\alpha(t), t)
\eta^\alpha(t)\,dt
+
\biggl[
\frac{\partial F}{\partial \dot x^\alpha}(x^\alpha(t),
\dot x^\alpha(t), t)
\eta^\alpha(t)
\biggr]_{t_0}^{t_1}
\\
&\qquad\qquad
-
\int_{t_0}^{t_1}
\frac{d}{dt}\frac{\partial F}{\partial \dot x^\alpha}(x^\alpha(t),
\dot x^\alpha(t), t)
\eta^\alpha(t)
\,dt.
\intertext{Da die verschobene Kurve immer noch die beiden Punkte
$x^\alpha(t_0)$ und $x^\alpha(t_1)$ verbindet, ist $\eta^\alpha(t_0)=0$
und $\eta^\alpha(t_1)=0$, und damit verschwindet der mittlere Term.
Die rechte Seite wird damit zu}
&=
\int_{t_0}^{t_1}
\biggl(
\frac{\partial F}{\partial x^\alpha}(x^\alpha(t), \dot x^\alpha(t), t)
-
\frac{d}{dt}\frac{\partial F}{\partial \dot x^\alpha}(x^\alpha(t),
\dot x^\alpha(t), t)
\biggr)
\eta^\alpha(t)
\,dt
\end{align*}
Das Integral auf der rechten Seite verschwindet nur dann für jede
beliebige Wahl der Verschiebung $\eta^\alpha(t)$, wenn die grosse Klammer
verschwindet.
Eine Lösung des Minimalproblems erfüllt also notwendigerweise die
sogenannten {\em Euler-Lagrange-Gleichungen}
\index{Euler-Lagrange-Gleichungen}%
\[
\frac{\partial F}{\partial x^\alpha}(x^\alpha(t), \dot x^\alpha(t), t)
-
\frac{d}{dt}\frac{\partial F}{\partial \dot x^\alpha}(x^\alpha(t), \dot x^\alpha(t), t)=0.
\]
Etwas kompakter geschrieben lauten diese
\begin{equation}
\frac{\partial F}{\partial x^\alpha}
-
\frac{d}{dt}\frac{\partial F}{\partial \dot x^\alpha}
=0.
\label{skript:variation:euler-lagrange}
\end{equation}

\subsection{Parametrisierung%
\label{skript:geodaeten:subsection:Parametrisierung}}
Wir vermuten, dass Geodäten die kürzesten Verbindungen sind.
Mit den Euler-Lagrange-Gleichungen können wir dies nachprüfen.
Wenn die Vermutung stimmt, müssten die Euler-Lagrange-Gleichungen 
\eqref{skript:variation:euler-lagrange}
für die Funktion
\[
L(x^\alpha, \dot x^\beta) =\sqrt{g_{\mu\nu}(x^\alpha)\dot x^\mu\dot x^\nu}
\]
die Differentialgleichungen der Geodäten liefern.
Wegen der Wurzel ist allerdings damit zu rechnen, dass die
Ableitungen, die wir für die Euler-Lagrange-Gleichungen brauchen sehr
kompliziert werden.

Die Geodäten-Gleichungen haben aber noch eine zusätzliche Eigenschaft.
Wir sind nämlich davon ausgegangen, dass der Tangentialvektor (der
Geschwindigkeitsvektor) immer die gleiche Länge hat, dass also
\[
\frac{d}{dt}L=0.
\]
Unter Verwendung dieser Eigenschaft können wir die Euler-Lagrange-Gleichung
wesentlich vereinfachen.
Dazu schreiben wir zunächst
\[
F=\frac12 L^2
\]
und berechnen die partiellen Ableitungen mit der Kettenregel:
\[
\begin{aligned}
\frac{\partial F}{\partial x^\alpha}
&=
L
\frac{\partial L}{\partial x^\alpha}
&&\text{und}
&
\frac{\partial F}{\partial \dot x^\alpha}
&=
L\frac{\partial L}{\partial\dot x^\alpha}
\end{aligned}
\]
Damit können wir jetzt die Euler-Lagrange-Gleichungen für $F$ aufstellen:
\begin{align*}
\frac{d}{dt}
\frac{\partial F}{\partial \dot x^\alpha}
&=
\frac{dL}{dt} \frac{\partial L}{\partial\dot x^\alpha}
+
L\frac{d}{dt}\frac{\partial L}{\partial\dot x^\alpha}
\\
\Rightarrow\qquad
\frac{d}{dt}
\frac{\partial F}{\partial \dot x^\alpha}
-
\frac{\partial F}{\partial x^\alpha}
&=
\underbrace{\frac{dL}{dt}}_{\displaystyle =0}
\frac{\partial L}{\partial\dot x^\alpha}
+
L\frac{d}{dt}\frac{\partial L}{\partial\dot x^\alpha}
-
L
\frac{\partial L}{\partial x^\alpha}
=
L\biggl(
\frac{d}{dt}\frac{\partial L}{\partial\dot x^\alpha}
-
\frac{\partial L}{\partial x^\alpha}
\biggr)
\end{align*}
Die Klammer auf der rechten Seite ist die Euler-Lagrange-Gleichung für
$L$.
Wenn wir also $dL/dt=0$ voraussetzen, dann ist die Euler-Lagrange-Gleichung
für $L$ gleichbedeutend mit der Euler-Lagrange-Gleichung für $F$.

\subsection{Lagrange-Gleichung für Geodäten}
Wir suchen jetzt also die Euler-Lagrange-Gleichungen für die Funktion
\[
F(x^\alpha, \dot x^\alpha) = \frac12 g_{\mu\nu} \dot x^\mu \dot x^\nu.
\]
Wir erwarten, dass die Euler-Lagrange-Gleichungen zu dieser Funktion $F$
die Gleichungen~\eqref{skript:kruemmung:geodatengleichung}
der der Geodäten ergeben.

Wir berechnen daher die Ableitungen
\begin{align*}
\frac{\partial F}{\partial x^\alpha}
&=
\frac12\frac{\partial g_{\mu\nu}}{\partial x^\alpha} \dot x^\mu \dot x^\nu,
\\
\frac{\partial F}{\partial \dot x^\alpha}
&=
\frac12 g_{\mu\nu}\frac{\partial \dot x^\mu}{\partial \dot x^\alpha}\dot x^\nu
+
\frac12 g_{\mu\nu}\dot x^\mu \frac{\partial \dot x^\nu}{\partial \dot x^\alpha}
=
\frac12 g_{\mu\nu}\delta^\mu_\alpha \dot x^\nu
+
\frac12 g_{\mu\nu}\dot x^\mu \delta^\nu_\alpha 
=
\frac12 g_{\alpha\nu}\dot x^\nu
+
\frac12 g_{\mu\alpha}\dot x^\mu 
\\
\frac{d}{dt}\frac{\partial F}{\partial \dot x^\alpha}
&=
\frac12\frac{\partial g_{\alpha\nu}}{\partial x_{\beta}}\dot x^\beta\dot x^\nu
+
\frac12\frac{\partial g_{\mu\alpha}}{\partial x_{\beta}}\dot x^\mu\dot x^\beta
+
\frac12g_{\alpha\nu}\ddot x^\nu
+
\frac12g_{\mu\alpha}\ddot x^\mu.
\end{align*}
Damit wird die Euler-Lagrange-Gleichung
\begin{align*}
0=
\frac{d}{dt}\frac{\partial F}{\partial \dot x^\alpha}
-
\frac{\partial F}{\partial x^\alpha}
&=
\frac12
\frac{\partial g_{\alpha\nu}}{\partial x_{\beta}}\dot x^\beta\dot x^\nu
+
\frac12
\frac{\partial g_{\mu\alpha}}{\partial x_{\beta}}\dot x^\mu\dot x^\beta
+
\frac12
g_{\alpha\nu}\ddot x^\nu
+
\frac12
g_{\mu\alpha}\ddot x^\mu
-
\frac12\frac{\partial g_{\mu\nu}}{\partial x^\alpha} \dot x^\mu \dot x^\nu.
\\
\intertext{Darin ersetzen wir im ersten Term auf der rechten Seite
$\beta$ durch $\mu$ und im zweiten Term $\beta$ durch $\nu$.
Ausserdem fassen wir die zwei Terme mit zweiten Ableitungen von $x^\nu$
zusammen:}
&=
g_{\alpha\nu}\ddot x^\nu
+
\frac12
\biggl(
\frac{\partial g_{\alpha\nu}}{\partial x_\mu}
+
\frac{\partial g_{\mu\alpha}}{\partial x_\nu}
-
\frac{\partial g_{\mu\nu}}{\partial x^\alpha}
\biggr)\dot x^\mu\dot x^\nu.
\\
\intertext{%
Wir multiplizieren mit $g^{\sigma\alpha}$ und verwenden,
dass $g^{\sigma\alpha}g_{\alpha\nu} = \delta_\nu^\alpha$, da $g^{\sigma\alpha}$
die zu $g_{\mu\nu}$ inverse Matrix ist.
Dann folgt auch
$g^{\sigma\alpha}g_{\alpha\nu}\ddot x^\nu
=
\delta_\nu^\alpha\ddot x^\nu
=
\ddot x^\alpha$, und
wir bekommen für die Differentialgleichung der Geodäten:}
0&=
\ddot x^\sigma
-
g^{\sigma\alpha}\frac12\biggl(
\frac{\partial g_{\mu\nu}}{\partial x^\alpha}
-
\frac{\partial g_{\alpha\nu}}{\partial x_{\mu}}
-
\frac{\partial g_{\mu\alpha}}{\partial x_{\nu}}
\biggr)
\dot x^\mu\dot x^\nu
&=
\ddot x^\sigma
-
\Gamma_{\mu\nu}^\sigma \dot x^\mu\dot x^\nu.
\end{align*}
Damit ist gezeigt, dass die Lösungen der Geodäten-Gleichung tatsächlich
kürzeste Verbindungen sind.

%
% koordtrafo.tex -- Koordinatentransformationen
%
% (c) 2017 Prof Dr Andreas Müller, Hochschule Rapperswil
%
\section{Koordinatentransformation}
Der bisher entwickelte Formalismus zielt darauf ab, dass die Formeln der
Theorie in jedem beliebigen Koordinatensystem aufgestellt werden können.
Sie müssen also derart formuliert sein, dass sie bei Koordinatenwechsel
der Form nach unverändert bleiben.
Das heisst natürlich nicht, dass zum Beispiel die metrischen Koeffizienten
$g_{\mu\nu}$ in jedem Koordinatensystem gleich aussehen müssen, sondern
nur, dass zum Objekt $g_{\mu\nu}$ in einem Koordinatensystem sofort
das Transformationsgesetz für die Transformation in ein anderes
Koordinatensystem angegeben werden kann.
Man nennt dies Eigenschaft die {\em allgemeine Kovarianz}.
\index{Kovarianz, allgemeine}
\index{allgemeine Kovarianz}
Nur allgemein kovariante Gleichungen kommen als von der speziellen
Wahl des Koordinatensystems unabhängige Naturgesetze in Frage.

Bis jetzt haben wir uns nicht darum gekümmert, wie sich eine Grösse
bei Wechsel des Koordinatensystems verändert.
In diesem Abschnitt soll dies nachgeholt werden.
Wir werden in diesem Abschnitt daher untersuchen, wie sich verschiedene
Grössen beim Wechsel vom Koordinatensystem $x^1,\dots,x^n$ auf ein
anderes Koordinatensystem $y^1,\dots,y^n$ transformieren.
Wir nehmen dabei im Allgemeinen an, dass sich die Koordinaten $y^\alpha$
als Funktionen
\[
y^\alpha = y^\alpha(x^1,\dots,x^n)
\]
der Koordinaten $x^1,\dots,x^n$ ausdrücken lassen.
Grössen im $y$-Koordinatensystem werden im Folgenden jeweils mit
einem Querstrich bezeichnet.

\subsection{Vektoren}

\subsubsection{Tangentialvektoren}
Als einfachstes Beispiel eines Vektors betrachten wir den Tangentialvektor
einer Kurve $x^\mu(s)$.
Die Komponenten des Tangentialvektors sind in diesem Koordinatensystem
gegeben durch die Ableitungen
\[
v^\mu
=
\frac{x^\mu(s)}{ds}.
\]
Wenn der Vektor eine vom Koordinatensystem unabhängige Bedeutung haben soll,
dann muss er sich im $y$-Kooordinatensystem ebenfalls durch Ableitung nach
$s$ finden lassen.
Wir berechnen daher 
\begin{align}
\bar v^\alpha
=
\frac{d}{ds}y^\alpha(x^1(s),\dots,x^n(s))
&=
\frac{\partial y^\alpha(x^1,\dots,x^n)}{\partial x^\mu}\frac{x^\mu(s)}{ds}
=
\frac{\partial y^\alpha}{\partial x^\mu} v^\mu
\label{skript:trafo:kontra}
\end{align}
mit der Kettenregel für Funktionen mehrer Veränderlicher.
Wir lesen daraus ab, dass sich die Komponenten $v^\mu$ mit Hilfe der
partiellen Ableitungen $\partial y^\alpha/\partial x^\mu$ transformieren.

Man kann auch sagen, dass die Kettenregel für Funktionen mehrere Variablen
in der Formel~\eqref{skript:trafo:kontra}
die Rechtfertigung für die Einsteinsche Summationskonvention ist.

\subsubsection{Linearformen}
Bei der Berechnung der Arbeit entlang eines Weges wird in der klassischen
Physik das Skalarprodukt des Tangentialvektors $v^\mu$ mit dem Kraftvektor
gebildet und integriert.
Etwas allgemeiner geht es also um Koeffizienten $A_\mu$ einer Linearform,
für welche der Skalar $B=A_\mu\dot x^\mu$ eine koordinatensystemunabhängige
Bedeutung haben soll.
Es soll also gelten
$B=\bar A_\alpha \dot y^\alpha$.
Setzen wir die Transformationsformel für $\dot y^\alpha$ ein, erhalten
wir
\begin{equation}
B=\bar A_\alpha\dot y^\alpha
=
\bar A_\alpha \frac{\partial y^\alpha}{\partial x^\mu}\dot x^\mu
=
A_\mu\dot x^\mu.
\end{equation}
Dies ist nur möglich, wenn für $\bar A_\alpha$ gilt
\begin{equation}
\bar A_\alpha \frac{\partial y^\alpha}{\partial x^\mu}
= 
A_\mu.
\label{skript:trafo:kov}
\end{equation}

\subsubsection{Kovariante und kontravariante Vektoren}
Man erkennt, dass sich die Transformationsformel~\eqref{skript:trafo:kov}
für $A_\mu$ von der Transformationsformel~\eqref{skript:trafo:kontra}
für $\dot x^\mu$ durch die Position der partiellen Ableitungen unterscheiden.
Man nennt den Vektor $\dot x^\mu$ einen {\em kontravarianten Vektor},
\index{Vektor, kontravarianter}
die Koeffizienten $A_\mu$ der Linearform dagegen einen {\em kovarianten Vektor}.
\index{Vektor, kovarianter}
Der Formalismus trägt dem Unterschied zwischen kovarianten und kontravarianten
Vektoren dadurch Rechnung, dass für die Komponenten kovarianter Vektoren
immer tiefgestellte Indizes verwendet werden, während für die Komponenten
kontravarianter Vektoren hochgestellte Indizes zu verwenden sind.

Sei $\bar f(y^1,\dots,y^n)$ eine beliebige Funktion der Koordinaten.
Man kann die Funktion natürlich auch in den Koordinaten $x^1,\dots,x^n$
ausdrücken, indem man 
\[
f(x^1,\dots,x^n) = \bar f(y^1(x^1,\dots,x^n),\dots, y^n(x^1,\dots,x^n))
\]
setzt.
Die partiellen Ableitungen von $f$ nach den Koordinaten können wir
als kovarianten Vektor betrachten, denn es gilt
\[
\frac{\partial f}{\partial x^\mu}
=
\frac{\partial f}{\partial y^\alpha}\frac{\partial y^\alpha}{\partial x^\mu}
=
\frac{\partial y^\alpha}{\partial x^\mu}
\frac{\partial f}{\partial y^\alpha}
\]
nach der Kettenregel.
Schreiben wir dies in Operatorform, wir wegen
\[
\frac{\partial}{\partial x^\mu}
=
\frac{\partial y^\alpha}{\partial x^\mu}
\frac{\partial}{\partial y^\alpha}
\]
klar, dass wir die partiellen Ableitungsoperatoren als kovariante Vektoren
betrachten können.

In der Beschreibung der Metrik haben wir die symbolischen Grössen $dx^\mu$
verwendet, die zunächst nur in einem Integral eine Bedeutung haben.
Wendet man auf diese Grössen die Variablentransformation an, dann
folgt
\[
dy^\alpha
=
\frac{\partial y^\alpha}{\partial x^\mu}\,dx^\mu,
\]
die Grössen $dx^\mu$ können daher wie ein kontravarianter Vektor behandelt
werden.

\subsection{Metrik}
Die Koeffizienten $g_{\mu\nu}$ des metrischen Tensors dienen dazu,
das Längenelement entlang einer Kurve zu berechnen.
Wird eine mit $s$ parametrisierte Kurve wird durch die Funktionen
$x^\mu(s)$ beschrieben, dann ist die Länge der Kurve durch das
Integral
\begin{equation}
l = \int_a^b
\sqrt{g_{\mu\nu}(x^\alpha(s))\,\frac{dx^\mu}{ds}\,\frac{dx^\nu}{ds}}
\,ds
\label{skript:koord:metrik}
\end{equation}
gegeben.
Die Länge der Kurve ist eine grösse, die vom Koordinatensystem 
unabhängig definiert werden kann.
Die Formel~\eqref{skript:koord:metrik} darf also ihre Form nicht verändern,
wenn man die Kurvenlänge in einem anderen Koordinatensystem berechnen will.

Die allgemeine Kovarianz fordert, dass es $\bar g_{\alpha\beta}(y^1,\dots,y^n)$
gibt derart, dass die Kurvenlänge mit der Formel
\begin{equation}
l = \int_a^b
\sqrt{\bar g_{\alpha\beta}(y^\sigma(s))\,\frac{dy^\alpha}{ds}\,\frac{dy^\beta}{ds}}
\,ds
\label{skript:koord:metrikbar}
\end{equation}
berechnet werden kann.

Damit die Längenmessung mit der Formel
\eqref{skript:koord:metrikbar} auf das gleiche Resultat wie 
\eqref{skript:koord:metrik} führt, müssen die Integranden übereinstimmen.
Setzt man \eqref{skript:trafo:kontra} in \eqref{skript:koord:metrikbar}
ein, erhält man
\begin{equation*}
\bar g_{\alpha\beta}
\frac{dy^\alpha}{ds}\frac{dy^\beta}{ds}
=
\bar g_{\alpha\beta}
\frac{\partial y^\alpha}{\partial x^\mu}
\frac{\partial y^\beta}{\partial x^\nu}
\frac{dx^\mu}{ds}\frac{dx^\nu}{ds}
=
g_{\mu\nu}
\frac{dx^\mu}{ds}\frac{dx^\nu}{ds}.
\end{equation*}
Dies ist nur möglich, wenn gilt
\begin{equation}
\bar g_{\alpha\beta}
\frac{\partial y^\alpha}{\partial x^\mu}
\frac{\partial y^\beta}{\partial x^\nu}
=
g_{\mu\nu}.
\end{equation}
Die metrischen Koeffizienten $g_{\mu\nu}$ sind daher in beiden
Indizes kovariant.

\subsection{Tensoren}
In Erweiterung der eben diskutierten Beispiele nennen wir die Grössen
$
A^{\mu\nu}\mathstrut_{\alpha\beta\gamma}
$
einen zweifach kontravarianten und dreifach kovarianten Tensor, wenn
\[
\bar 
A^{\mu\nu}\mathstrut_{\varrho\sigma\tau}
\frac{\partial y^\delta}{\partial x^\mu}
\frac{\partial y^\epsilon}{\partial x^\nu}
=
A^{\delta\epsilon}\mathstrut_{\alpha\beta\gamma}
\frac{\partial y^\alpha}{\partial x^\varrho}
\frac{\partial y^\beta}{\partial x^\sigma}
\frac{\partial y^\gamma}{\partial x^\tau}
\]
gilt.
Geometrische Aussagen oder Naturgesetze sind nur dann von der Wahl
eines Koordinatensystems unabhängig, wenn sie mit Hilfe von Tensoren
beschrieben werden.

Die Transformationsgesetze sind so gestaltet, dass sich weitere Tensoren
daraus bilden.
Setzt man einen kovarianten und einen kovarianten gleich und summiert,
erhält man
\[
A^{\delta\nu}\mathstrut_{\delta\beta\gamma}
=
B^\nu\mathstrut_{\beta\gamma},
\]
genannt die Verjüngung von $A$.
Dazu beachten wir zunächst, dass
\begin{equation}
\frac{\partial y^\alpha}{\partial x^\mu}
\frac{\partial x^\mu}{\partial y^\beta}
=
\frac{\partial y^\alpha}{\partial y^\beta}
=
\delta^\alpha_\beta
\label{skript:trafo:delta}
\end{equation}
ist.
Wir müssen nachprüfen, ob für die Verjüngung die Transformationsregeln
\begin{align}
\bar B^\nu\mathstrut_{\beta\gamma}
\frac{\partial y^\zeta}{\partial x^\nu}
&=
B^\zeta\mathstrut_{\sigma\tau}
\frac{\partial y^\sigma}{\partial x^\beta}
\frac{\partial y^\tau}{\partial x^\gamma}
\label{skript:trafo:verjuengung}
\\
\intertext{gelten. Dazu wenden wir zunächst die Definition der
Verjüngung an und erhalten}
\bar A^{\mu\nu}\mathstrut_{\mu\beta\gamma}
\frac{\partial y^\zeta}{\partial x^\nu}
&=
A^{\varrho\zeta}\mathstrut_{\varrho\sigma\tau}
\frac{\partial y^\sigma}{\partial x^\beta}
\frac{\partial y^\tau}{\partial x^\gamma}
\notag
\\
\intertext{Die Verjüngung können wir auch mit Hilfe von $\delta^\alpha_\beta$
schreiben, welches wir dann unter Verwendung von
\eqref{skript:trafo:delta}
durch partielle Ableitungen der Transformationsmatrizen
ausdrücken.}
\bar A^{\mu\nu}\mathstrut_{\alpha\beta\gamma}
\frac{\partial y^\zeta}{\partial x^\nu}
\delta_\mu^\alpha
&=
A^{\eta\zeta}\mathstrut_{\varrho\sigma\tau}
\frac{\partial y^\sigma}{\partial x^\beta}
\frac{\partial y^\tau}{\partial x^\gamma}
\delta^\varrho_\eta
\notag
\\
\bar A^{\mu\nu}\mathstrut_{\alpha\beta\gamma}
\frac{\partial y^\delta}{\partial x^\mu}
\frac{\partial y^\zeta}{\partial x^\nu}
\frac{\partial x^\alpha}{\partial y^\delta}
&=
A^{\eta\zeta}\mathstrut_{\varrho\sigma\tau}
\frac{\partial y^\sigma}{\partial x^\beta}
\frac{\partial y^\tau}{\partial x^\gamma}
\frac{\partial y^\varrho}{\partial x^\alpha}
\frac{\partial x^\alpha}{\partial y^\eta}
\notag
\\
\intertext{Auf der linken Seite können wir die Transformationsformeln
für $\bar A$ anwenden und erhalten}
%\bar A^{\mu\nu}\mathstrut_{\alpha\beta\gamma}
%\frac{\partial y^\zeta}{\partial x^\nu}
%\frac{\partial y^\delta}{\partial x^\mu}
A^{\delta\zeta}\mathstrut_{\varrho\sigma\tau}
\frac{\partial y^\varrho}{\partial x^\alpha}
\frac{\partial y^\sigma}{\partial x^\beta}
\frac{\partial y^\tau}{\partial x^\gamma}
\frac{\partial x^\alpha}{\partial y^\delta}
&=
A^{\eta\zeta}\mathstrut_{\varrho\sigma\tau}
\frac{\partial y^\varrho}{\partial x^\alpha}
\frac{\partial y^\sigma}{\partial x^\beta}
\frac{\partial y^\tau}{\partial x^\gamma}
\frac{\partial x^\alpha}{\partial y^\eta}
\notag
\end{align}
Bis auf den Namen der Summationsvariablen $\delta$ auf der linken
Seite und $\eta$ auf der rechten Seite stimmen die beiden Seiten 
überrein, so dass die Verjüngung $B$ tatsächlich wie in
\eqref{skript:trafo:verjuengung} transformiert werden muss.

\subsection{Christoffel-Symbole}
Die Zusammenhangskoeffizienten sind keine Tensoren, vielmehr gilt für sie das
Transformationsgesetz
\[
\bar\Gamma^\varrho_{\beta\gamma}
\frac{\partial y^\alpha}{\partial x^\varrho}
=
\Gamma^\alpha_{\mu\nu}
\frac{\partial y^\mu}{\partial x^\beta}
\frac{\partial y^\nu}{\partial x^\gamma}
+
\frac{\partial^2}{\partial x\partial x}
\frac{\partial y}{\partial x},
\]
wie wir im Folgenden nachweisen wollen.
Dazu führen wir die Rechnung zunächst für die Christoffelsymbole
erster Art durch, die Christoffelsymbole zweiter Art unterschieden sich
ja nur durch das Hochziehen eines Index.

\subsubsection{Transformationsformel für die Ableitungen von $g_{\mu\nu}$}
Es ist bereits bekannt, dass $g_{\mu\nu}$ ein kovarianter Tensor ist.
Wir müssen die Ableitung von $\bar g_{\mu\nu}$ nach $y^\alpha$ durch
die metrischen Koeffizienten $g_{\sigma\tau}$ und ihre Ableitungen
nach $x^\sigma$ ausdrücken.
Insbesondere gilt das Transformationsgesetz
\begin{equation*}
\bar g_{\mu\nu}
=
g_{\varrho\sigma}
\frac{\partial x^\varrho}{\partial y^\mu}
\frac{\partial x^\sigma}{\partial y^\nu}
\end{equation*}
für die metrischen Koeffizienten. 
Damit kann man jetzt auch die Ableitung 
\begin{align}
\frac{\partial}{\partial y^\alpha}\bar g_{\mu\nu}
&=
\frac{\partial}{\partial y^\alpha} 
\biggl(
g_{\varrho\sigma}
\frac{\partial x^\varrho}{\partial y^\mu}
\frac{\partial x^\sigma}{\partial y^\nu}
\biggr)
=
\frac{\partial x^\tau}{\partial y^\alpha}
\frac{\partial}{\partial x^\tau}
\biggl(
g_{\varrho\sigma}
\frac{\partial x^\varrho}{\partial y^\mu}
\frac{\partial x^\sigma}{\partial y^\nu}
\biggr)
\notag
\\
&=
\frac{\partial g_{\varrho\sigma}}{\partial x^\tau}
\frac{\partial x^\tau}{\partial y^\alpha}
\frac{\partial x^\varrho}{\partial y^\mu}
\frac{\partial x^\sigma}{\partial y^\nu}
+
g_{\varrho\sigma}
\frac{\partial x^\tau}{\partial y^\alpha}
\frac{\partial}{\partial x^\tau}
\biggl(
\frac{\partial x^\varrho}{\partial y^\mu}
\frac{\partial x^\sigma}{\partial y^\nu}
\biggr)
\notag
\\
&=
\frac{\partial g_{\varrho\sigma}}{\partial x^\tau}
\frac{\partial x^\tau}{\partial y^\alpha}
\frac{\partial x^\varrho}{\partial y^\mu}
\frac{\partial x^\sigma}{\partial y^\nu}
+
g_{\varrho\sigma}
\frac{\partial x^\tau}{\partial y^\alpha}
\biggl(
\frac{\partial^2 x^\varrho}{\partial y^\tau\partial y^\mu}
\frac{\partial x^\sigma}{\partial y^\nu}
+
\frac{\partial x^\varrho}{\partial y^\mu}
\frac{\partial^2 x^\sigma}{\partial y^\tau\partial y^\nu}
\biggr)
\label{skript:trafo:gabl}
\end{align}
Die zusätzlichen Terme mit zweiten Ableitungen der Koordinatentransformation
auf der rechten Seite zeigen, dass die Ableitungen der metrischen Koeffizienten
keine Tensoren sein können.

\subsubsection{Transformationsformel für die Christoffelsymbole erster Art}
Die Christoffelsymbole erster Art sind definiert durch
\[
\Gamma_{\alpha,\mu\nu}
=
\frac12\biggl(
\frac{\partial g_{\nu\alpha}}{\partial x^\mu}
+
\frac{\partial g_{\alpha\mu}}{\partial x^\nu}
-
\frac{\partial g_{\mu\nu}}{\partial x^\alpha}
\biggr).
\]
Die Transformationformeln~\eqref{skript:trafo:gabl}
für die Ableitungen des metrischen Tensors können jetzt dazu verwendet
werden, die Transformationsformeln für die Christoffelsymbole erster Art
zu ermitteln.
Wir wenden die Definition auf $\bar g$ an:
\begin{align}
\bar\Gamma_{\alpha,\mu\nu}
&=
\frac12\biggl(
\frac{\partial \bar g_{\nu\alpha}}{\partial y^\mu}
+
\frac{\partial \bar g_{\alpha\mu}}{\partial y^\nu}
-
\frac{\partial \bar g_{\mu\nu}}{\partial y^\alpha}
\biggr)
\notag
\\
&=
\frac12\biggl(
\frac{\partial g_{\sigma\tau}}{\partial x^\varrho}
+
\frac{\partial g_{\tau\varrho}}{\partial x^\sigma}
-
\frac{\partial g_{\varrho\sigma}}{\partial x^\tau}
\biggr)
\frac{\partial x^\tau}{\partial y^\alpha}
\frac{\partial x^\varrho}{\partial y^\mu}
\frac{\partial x^\sigma}{\partial y^\nu}
\notag
\\
&\quad
+\frac12g_{\sigma\tau}
\frac{\partial x^\varrho}{\partial y^\mu}
\biggl(
{\color{blue}
\frac{\partial^2 x^\sigma}{\partial y^\varrho\partial y^\nu}
\frac{\partial x^\tau}{\partial y^\alpha}}
+
\frac{\partial x^\sigma}{\partial y^\nu}
\frac{\partial^2 x^\tau}{\partial y^\varrho\partial y^\alpha}
\biggr)
%&&
%\begin{pmatrix}
%\mu    &\nu    &\alpha &\sigma &\tau   &\varrho
%\end{pmatrix}
\notag
\\
&\quad
+\frac12g_{\tau\varrho}
\frac{\partial x^\sigma}{\partial y^\nu}
\biggl(
\frac{\partial^2 x^\tau}{\partial y^\sigma\partial y^\alpha}
\frac{\partial x^\varrho}{\partial y^\mu}
+
{\color{red}
\frac{\partial x^\tau}{\partial y^\alpha}
\frac{\partial^2 x^\varrho}{\partial y^\sigma\partial y^\mu}}
\biggr)
%&&
%\begin{pmatrix}
%\nu    &\alpha &\mu    &\tau   &\varrho&\sigma
%\end{pmatrix}
\notag
\\
&\quad
-\frac12g_{\varrho\sigma}
\frac{\partial x^\tau}{\partial y^\alpha}
\biggl(
{\color{red}
\frac{\partial^2 x^\varrho}{\partial y^\tau\partial y^\mu}
\frac{\partial x^\sigma}{\partial y^\nu}}
+
{\color{blue}
\frac{\partial x^\varrho}{\partial y^\mu}
\frac{\partial^2 x^\sigma}{\partial y^\tau\partial y^\nu}}
\biggr)
\notag
%&&
%\begin{pmatrix}
%\alpha &\mu    &\nu    &\varrho&\sigma &\tau
%\end{pmatrix}
\intertext{Die farbig hervorgehobenen Terme heben sich nach geeigneter
Umbenennung von Summationsindizes weg.
Es bleibt dann
}
\bar\Gamma_{\alpha,\mu\nu}
&=
\Gamma_{\tau,\varrho\sigma}
\frac{\partial x^\tau}{\partial y^\alpha}
\frac{\partial x^\varrho}{\partial y^\mu}
\frac{\partial x^\sigma}{\partial y^\nu}
+
\frac12
g_{\sigma\tau}
\frac{\partial x^\varrho}{\partial y^\mu}
\frac{\partial x^\sigma}{\partial y^\nu}
\frac{\partial^2 x^\tau}{\partial y^\varrho\partial y^\alpha}
+
\frac12
g_{\tau\varrho}
\frac{\partial x^\sigma}{\partial y^\nu}
\frac{\partial x^\varrho}{\partial x^\mu}
\frac{\partial^2 x^\tau}{\partial y^\sigma\partial y^\alpha}
\label{skript:trafo:chr1}
\end{align}



\subsubsection{Transformationsformel für die Christoffelsymbole zweiter Art}
Für die Christoffesymbole zweiter Art müssen wir nur noch auf der
linken Seite mit $\bar g^{\beta\alpha}$ multiplizieren.
Da $g^{\mu\nu}$ ein kontravarianter Tensor ist, gilt
\begin{equation}
\bar g^{\alpha\beta}
\frac{\partial x^\mu}{\partial y^\alpha}
\frac{\partial x^\nu}{\partial y^\beta}
=
g^{\mu\nu}
\qquad\Leftrightarrow\qquad
\bar g^{\alpha\beta}
=
g^{\mu\nu}
\frac{\partial y^\alpha}{\partial x^\mu}
\frac{\partial y^\beta}{\partial x^\nu}.
\label{skript:trafo:ginv}
\end{equation}
Wir multiplizieren das Transformationsgesetz~\label{skript:trafo:chr1}
mit~\eqref{skript:trafo:ginv} und erhalten
\begin{align*}
\bar g^{\alpha\beta}
\bar\Gamma_{\alpha,\mu\nu}
&=
\frac{\partial y^\alpha}{\partial x^\eta}
\frac{\partial y^\beta}{\partial x^\zeta}
g^{\eta\zeta}
\Gamma_{\tau,\varrho\sigma}
\frac{\partial x^\tau}{\partial y^\alpha}
\frac{\partial x^\varrho}{\partial y^\mu}
\frac{\partial x^\sigma}{\partial y^\nu}
+
\frac12
\frac{\partial y^\alpha}{\partial x^\eta}
\frac{\partial y^\beta}{\partial x^\zeta}
g^{\eta\zeta}
g_{\sigma\tau}
\frac{\partial x^\varrho}{\partial y^\mu}
\frac{\partial x^\sigma}{\partial y^\nu}
\frac{\partial^2 x^\tau}{\partial y^\varrho\partial y^\alpha}
+
\frac12
\frac{\partial y^\alpha}{\partial x^\eta}
\frac{\partial y^\beta}{\partial x^\zeta}
g^{\eta\zeta}
g_{\tau\varrho}
\frac{\partial x^\sigma}{\partial y^\nu}
\frac{\partial x^\varrho}{\partial x^\mu}
\frac{\partial^2 x^\tau}{\partial y^\sigma\partial y^\alpha}
\\
&=
\delta^\tau_\eta
\frac{\partial y^\beta}{\partial x^\zeta}
g^{\eta\zeta}
\Gamma_{\tau,\varrho\sigma}
\frac{\partial x^\varrho}{\partial y^\mu}
\frac{\partial x^\sigma}{\partial y^\nu}
+
\frac12
\frac{\partial y^\alpha}{\partial x^\eta}
\frac{\partial y^\beta}{\partial x^\zeta}
g^{\eta\zeta}
g_{\sigma\tau}
\frac{\partial x^\varrho}{\partial y^\mu}
\frac{\partial x^\sigma}{\partial y^\nu}
\frac{\partial^2 x^\tau}{\partial y^\varrho\partial y^\alpha}
+
\frac12
\frac{\partial y^\alpha}{\partial x^\eta}
\frac{\partial y^\beta}{\partial x^\zeta}
g^{\eta\zeta}
g_{\tau\varrho}
\frac{\partial x^\sigma}{\partial y^\nu}
\frac{\partial x^\varrho}{\partial x^\mu}
\frac{\partial^2 x^\tau}{\partial y^\sigma\partial y^\alpha}
\\
&=
\frac{\partial y^\beta}{\partial x^\zeta}
g^{\tau\zeta}
\Gamma_{\tau,\varrho\sigma}
\frac{\partial x^\varrho}{\partial y^\mu}
\frac{\partial x^\sigma}{\partial y^\nu}
+
\frac12
\frac{\partial y^\alpha}{\partial x^\eta}
\frac{\partial y^\beta}{\partial x^\zeta}
g^{\eta\zeta}
g_{\sigma\tau}
\frac{\partial x^\varrho}{\partial y^\mu}
\frac{\partial x^\sigma}{\partial y^\nu}
\frac{\partial^2 x^\tau}{\partial y^\varrho\partial y^\alpha}
+
\frac12
\frac{\partial y^\alpha}{\partial x^\eta}
\frac{\partial y^\beta}{\partial x^\zeta}
g^{\eta\zeta}
g_{\tau\varrho}
\frac{\partial x^\sigma}{\partial y^\nu}
\frac{\partial x^\varrho}{\partial x^\mu}
\frac{\partial^2 x^\tau}{\partial y^\sigma\partial y^\alpha}
\\
&=
\Gamma^\zeta_{\varrho\sigma}
\frac{\partial y^\beta}{\partial x^\zeta}
\frac{\partial x^\varrho}{\partial y^\mu}
\frac{\partial x^\sigma}{\partial y^\nu}
+
\frac12
\frac{\partial y^\alpha}{\partial x^\eta}
\frac{\partial y^\beta}{\partial x^\zeta}
g^{\eta\zeta}
g_{\sigma\tau}
\frac{\partial x^\varrho}{\partial y^\mu}
\frac{\partial x^\sigma}{\partial y^\nu}
\frac{\partial^2 x^\tau}{\partial y^\varrho\partial y^\alpha}
+
\frac12
\frac{\partial y^\alpha}{\partial x^\eta}
\frac{\partial y^\beta}{\partial x^\zeta}
g^{\eta\zeta}
g_{\tau\varrho}
\frac{\partial x^\sigma}{\partial y^\nu}
\frac{\partial x^\varrho}{\partial x^\mu}
\frac{\partial^2 x^\tau}{\partial y^\sigma\partial y^\alpha}
\end{align*}





\section{"Ubungsaufgaben}
\rhead{"Ubungsaufgaben}
\uebungsaufgabe{0301}
\uebungsaufgabe{0302}
\uebungsaufgabe{0303}
\uebungsaufgabe{0304}
\uebungsaufgabe{0305}

