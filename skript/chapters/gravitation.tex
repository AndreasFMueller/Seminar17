%
% k-gravitatzion.tex
%
% (c) 2017 Prof Dr Andreas Müller, Hochschule Rapperswil
%
\chapter{Gravitation%
\label{skript:kruemmusng:sectipn:gravitation}}
\lhead{Gravitation}
\rhead{}
Albert Einstein hat erkannt, dass die Wirkung der Gravitation 
durch die Krümmung des Raumes beschrieben werden muss.
\index{Einstein, Albert}
In den folgenden Abschnitten geben wir einen Ausblick darauf, wie
die moderne Physik unseren Raum als einen gekrümmten Raum beschreibt.
An einfachen Modellen soll gezeigt werden, wie man sich die Gravitationswirkung
als die Wirkung eines gekrümmten Raumes vorstellen kann, wie schwarze Löcher
beschrieben werden können, und wie alle diese Dinge tatsächlich gemessen
werden können.

\section{Äquivalenzprinzip}
Schon Galileo Galilei hat bemerkt, dass im Gravitationsfeld der Erde
jeder Körper unabhängig von seiner Masse die gleiche Beschleunigung $g$
erfährt.
Auch das Newtonsche Gravitationsgesetz beschreibt den Betrag der Kraft
zwischen zwei Massen $M$ und $m$ als
\[
F=\frac{GMm}{r^2}.
\]
Ein Teilchen der Masse $m$ wird daher nach dem zweiten Newtonschen Gesetz
$F=ma$ immer mit den Betrag
\[
\frac{GM}{r^2}
\]
haben.
Die Beschleunigung eines Teilchens ist daher unabhängig von der Masse.
Diese {\em Äquivalenzprinzpi} genannte Beobachtung wurde von
Lor\'and Eötvös 1909 mit sehr grosser Genauigkeit nachgemessen und bestätigt.
In neuerer Zeit wurde die Genauigkeit nochmals um den Faktor $10^4$
verbessert.

Eine Folge des Äquivalenzprinzips ist auch, dass sich ein konstant
beschleunigtes Labor und ein Labor in einem homogenen Gravitationsfeld
nicht unterscheiden lassen.
%
% TODO Beschleunigung und Gravitation im Lift-Experiment
%

\section{Newtonsches Gravitationsgesetz}
Als ersten Schritt in Richtung auf eine allgemeine Gravitationstheorie
zeigen wir, wie auch die Bahnen in einem Newtonschen Gravitationsfeld 
als Geodäten eines Zusammenhangs schreiben lassen.
Wer verwenden dazu aber nicht die Metrik und den daraus abgeleiteten 
Zusammenhang, dies hat erst die Einsteinsche Theorie geschafft.

Das Newtonsche Gravitationsgesetz beschreibt die Beschleunigung, die
auf ein Teilchen in einem Gravitationsfeld wirkt, welches von einer
Masse $m$ erzeugt wird.
Der Betrag der Kraft ist umgekehrt proportional zum Quadrat der
Entfernung.
Der Kraftvektor kann deshalb als
\[
\vec F = -\frac{GM}{r^2}\cdot\frac{\vec r}{r}
\]
geschrieben werden.
Die Bewegungsgleichung ist daher 
\begin{equation}
\ddot x^k = -\frac{GM}{r}\cdot\frac{x^k}{r},
\qquad\text{mit}\qquad r = \sqrt{(x^1)^2+(x^2)^2+(x^3)^2},
\label{skript:gravitation:bewegungsgleichung}
\end{equation}
es kommt nur die Beschleunigung und die Position vor.

In den Geodätengleichungen 
\[
\ddot x^\mu = -\Gamma^\mu_{\alpha\nu}\dot x^\alpha\dot x^\nu
\]
kommt dagegen auf der rechten Seite auch die Geschwindigkeit vor.
Es ist daher nicht unmittelbar klar, wie die Bewegungsgleichung als
Geodätengleichung geschrieben werden klar.

Wir beschreiben die Bewegung wieder in einem Vierdimensionalen Raum,
also als Kurve
\[
t\mapsto (t,x^1(t),x^2(t),x^3(t)),
\]
also mit $x^0(t)=t$.
Für eine solche Parametrisierung gilt $\dot x^0(t)=1$.
In der Geodätengleichung können wir daher die Terme mit mit Index $0$
separat behandelt werden:
\[
\ddot x^\mu
=
-\Gamma^\mu_{00}
-\Gamma^\mu_{k0}\dot x^k -\Gamma^\mu_{0l}\dot x^l
- \Gamma^\mu_{kl}\dot x^k\dot x^l
\]
Da in der Bewegungsgleichung~\eqref{skript:gravitation:bewegungsgleichung}
auf der rechten Seite keine Geschwindigkeiten vorkommen, darf nur der
erste Term stehen bleiben, also
\begin{equation}
\Gamma^k_{00} = \frac{GM}{r^2}\cdot \frac{x^k}{r},\qquad k=1,\dots,3.
\label{skript:gravitation:newtonzusammenhang}
\end{equation}
Da die erste Komponente der Geodäte $x^0(t)=t$ ist und seine zweite
Ableitung $\ddot x^0(t)=0$, muss auch $\Gamma^0_{\alpha\nu}=0$ sein.
Die einzigen nicht verschwindenden Zusammenhangskoeffizienten sind
\eqref{skript:gravitation:newtonzusammenhang}.

Man kann für den Zusammenhang~\eqref{skript:gravitation:newtonzusammenhang}
auch die Definition des Riemann-Krümmungstensors anwenden.
Die Rechnung zeigt aber, dass zwar einzelne Komponenten des Riemann-Tensors
nicht verschwinden, der Ricci-Tensor und damit auch der Einstein-Tensor
verschwinden dagegen identisch.
Diese Beschreibung der Gravitation führt also nicht auf das Modell
eines gekrümmten Raumes.

\section{Näherung für schwache Felder}
Wir möchten jetzt eine Metrik finden, die einerseits möglichst nahe
an der Minkovski-Metrik ist, deren zugehörige Geodäten andererseits die
Bewegung in einem schwachen Gravitationsfeld korrekt wiedergibt.
Als Metrik verwenden wir daher
\[
g_{\mu\nu}=\eta_{\mu\nu} + h_{\mu\nu},
\]
wobei wir annehmen, dass $h_{\mu\nu}$ sehr klein ist.
Die zugehörigen Christoffelsymbole sind nach einiger, vorzugsweise
maschineller Rechnung
\[
\Gamma^{k}_{00}
=
-\frac{1}{2(h_{kk} + 1)}\frac{\partial h_{00}}{\partial x^k}.
\]
Da $h_{kk}$ im Vergleich zu $1$ sehr klein soll, bleibt
\[
\Gamma^{k}_{00}
=
-\frac{1}{2}\frac{\partial h_{00}}{\partial x^k}.
\]
Damit die Bahnen der newtonschen Gravitationstheorie entstehen, müssen
die $-\Gamma^k_{00}$ die Beschleunigungskomponenten des Gravitationsfelds
entstehen.
Diese bilden den Gradienten des Potentials des Gravitationsfeldes
\[
-\Gamma^k_{00} = -\frac{\partial\varphi}{\partial x^k}.
\]
Es folgt daher, dass $h_{00}=-2\varphi$ ist.
In erster Näherung beschreibt daher die Metrik
\begin{equation}
\begin{aligned}
g_{00} &= -1 -2\varphi,
&&&
g_{kk} &= 1,\quad k=1,2,3.
\end{aligned}
\label{skript:gravitation:naeherung}
\end{equation}

\section{Physikalische Bedeutung der Krümmung}

\section{Einstein-Tensor}

\section{Feldgleichungen}

