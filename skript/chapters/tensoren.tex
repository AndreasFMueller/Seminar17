%
% tensoren.tex -- TEnsoren
%
% (c) 2017 Prof Dr Andreas Müller, Hochschule Rapperswil
%
\section{Tensoren%
\label{skript:section:tensoren}}
\rhead{Tensoren}
Der bisher entwickelte Formalismus zielt darauf ab, dass die Formeln der
Theorie in jedem beliebigen Koordinatensystem aufgestellt werden können.
Sie müssen derart formuliert sein, dass sie bei Koordinatenwechsel
der Form nach unverändert bleiben.
Das heisst natürlich nicht, dass zum Beispiel die metrischen Koeffizienten
$g_{\mu\nu}$ in jedem Koordinatensystem gleich aussehen müssen, sondern
nur, dass zum Objekt $g_{\mu\nu}$ in einem Koordinatensystem sofort
das Transformationsgesetz für die Transformation in ein anderes
Koordinatensystem angegeben werden kann.
Man nennt dies Eigenschaft die {\em allgemeine Kovarianz}.
\index{Kovarianz, allgemeine}%
\index{allgemeine Kovarianz}%
Nur allgemein kovariante Gleichungen kommen als von der speziellen
Wahl des Koordinatensystems unabhängige Naturgesetze in Frage.

Bis jetzt haben wir uns nicht darum gekümmert, wie sich eine Grösse
bei Wechsel des Koordinatensystems verändert.
In diesem Abschnitt soll dies nachgeholt werden.
Wir werden in diesem Abschnitt daher untersuchen, wie sich verschiedene
Grössen beim Wechsel vom Koordinatensystem $x^1,\dots,x^n$ auf ein
anderes Koordinatensystem $y^1,\dots,y^n$ transformieren.
Wir nehmen dabei im Allgemeinen an, dass sich die Koordinaten $y^\alpha$
als Funktionen
\[
y^\alpha = y^\alpha(x^1,\dots,x^n)
\]
der Koordinaten $x^1,\dots,x^n$ ausdrücken lassen.
Grössen im $y$-Koordinatensystem werden im Folgenden jeweils mit
einem Querstrich bezeichnet.

\subsection{Vektoren}

\subsubsection{Tangentialvektoren}
\index{Tangentialvektor}%
Als einfachstes Beispiel eines Vektors betrachten wir den Tangentialvektor
einer Kurve $x^\mu(s)$.
Die Komponenten des Tangentialvektors sind in diesem Koordinatensystem
gegeben durch die Ableitungen
\[
v^\mu
=
\frac{x^\mu(s)}{ds}.
\]
Wenn der Vektor eine vom Koordinatensystem unabhängige Bedeutung haben soll,
dann muss er sich im $y$-Kooordinatensystem ebenfalls durch Ableitung nach
$s$ finden lassen.
Wir berechnen daher 
\begin{align}
\bar v^\alpha
=
\frac{d}{ds}y^\alpha(x^1(s),\dots,x^n(s))
&=
\frac{\partial y^\alpha(x^1,\dots,x^n)}{\partial x^\mu}\frac{x^\mu(s)}{ds}
=
\frac{\partial y^\alpha}{\partial x^\mu} v^\mu
\label{skript:trafo:kontra}
\end{align}
mit der Kettenregel für Funktionen mehrerer Veränderlicher.
Wir lesen daraus ab, dass sich die Komponenten $v^\mu$ mit Hilfe der
partiellen Ableitungen $\partial y^\alpha/\partial x^\mu$ transformieren.

Man kann auch sagen, dass die Kettenregel für Funktionen mehrere Variablen
in der Formel~\eqref{skript:trafo:kontra}
die Einsteinsche Summationskonvention rechtfertigt.

\subsubsection{Linearformen}
\index{Linearform}%
Bei der Berechnung der Arbeit entlang eines Weges wird in der klassischen
Physik das Skalarprodukt des Tangentialvektors $v^\mu$ mit dem Kraftvektor
gebildet und integriert.
Etwas allgemeiner geht es also um Koeffizienten $A_\mu$ einer Linearform,
für welche der Skalar $B=A_\mu\dot x^\mu$ eine koordinatensystemunabhängige
Bedeutung haben soll.
\index{Linearform}%
Es soll also gelten
$B=\bar A_\alpha \dot y^\alpha$.
Setzen wir die Transformationsformel für $\dot y^\alpha$ ein, erhalten
wir
\begin{equation}
B=\bar A_\alpha\dot y^\alpha
=
\bar A_\alpha \frac{\partial y^\alpha}{\partial x^\mu}\dot x^\mu
=
A_\mu\dot x^\mu.
\end{equation}
Dies ist nur möglich, wenn für $\bar A_\alpha$ gilt
\begin{equation}
\bar A_\alpha \frac{\partial y^\alpha}{\partial x^\mu}
= 
A_\mu.
\label{skript:trafo:kov}
\end{equation}

\subsubsection{Kovariante und kontravariante Vektoren}
Man erkennt, dass sich die Transformationsformel \eqref{skript:trafo:kov}
für $A_\mu$ von der Transformationsformel \eqref{skript:trafo:kontra}
für $\dot x^\mu$ durch die Position der partiellen Ableitungen unterscheiden.
Man nennt den Vektor $\dot x^\mu$ einen {\em kontravarianten Vektor},
\index{Vektor, kontravarianter}%
\index{kontravariant}%
die Koeffizienten $A_\mu$ der Linearform dagegen einen {\em kovarianten Vektor}.
\index{kovariant}%
\index{Vektor, kovarianter}%
Der Formalismus trägt dem Unterschied zwischen kovarianten und kontravarianten
Vektoren dadurch Rechnung, dass für die Komponenten kovarianter Vektoren
immer tiefgestellte Indizes verwendet werden, während für die Komponenten
kontravarianter Vektoren hochgestellte Indizes zu verwenden sind.

\subsubsection{Ableitungsoperatoren und Integration}
Sei $\bar f(y^1,\dots,y^n)$ eine beliebige Funktion der Koordinaten.
Man kann die Funktion natürlich auch in den Koordinaten $x^1,\dots,x^n$
ausdrücken, indem man 
\[
f(x^1,\dots,x^n) = \bar f(y^1(x^1,\dots,x^n),\dots, y^n(x^1,\dots,x^n))
\]
setzt.
Die partiellen Ableitungen von $f$ nach den Koordinaten können wir
als kovarianten Vektor betrachten, denn es gilt
\[
\frac{\partial f}{\partial x^\mu}
=
\frac{\partial f}{\partial y^\alpha}\frac{\partial y^\alpha}{\partial x^\mu}
=
\frac{\partial y^\alpha}{\partial x^\mu}
\frac{\partial f}{\partial y^\alpha}
\]
nach der Kettenregel.
Schreiben wir dies in Operatorform, wird wegen
\[
\frac{\partial}{\partial x^\mu}
=
\frac{\partial y^\alpha}{\partial x^\mu}
\frac{\partial}{\partial y^\alpha}
\]
klar, dass wir die partiellen Ableitungsoperatoren als kovariante Vektoren
betrachten können.

In der Beschreibung der Metrik haben wir die symbolischen Grössen $dx^\mu$
verwendet, die zunächst nur in einem Integral eine Bedeutung haben.
Wendet man auf diese Grössen die Variablentransformation an, dann
folgt
\[
dy^\alpha
=
\frac{\partial y^\alpha}{\partial x^\mu}\,dx^\mu,
\]
die Grössen $dx^\mu$ können daher wie ein kontravarianter Vektor behandelt
werden.

\subsubsection{Inverse Transformation}
Die Koordinaten $x^\mu$ können auch durch $y^\alpha$ ausgedrückt werden.
Insbesondere ist
\begin{align}
x^\mu
&=
x^\mu(y^1,\dots,y^n)
\notag
\\
&=
y^\mu(y^1(x^1,\dots,x^n),\dots,y^n(x^1,\dots,x^n))
\notag
\\
\Rightarrow\qquad
\delta^\mu_\nu
=
\frac{\partial x^\mu}{\partial x^\nu}
&=
\frac{\partial x^\mu}{\partial y^\alpha}(y^1,\dots,y^n)
\frac{\partial y^\alpha}{\partial x^\nu}(x^1,\dots,x^n)
=
\frac{\partial x^\mu}{\partial y^\alpha}
\frac{\partial y^\alpha}{\partial x^\nu}.
\label{label:trafo:t}
\end{align}
Die Matrizen $\partial x^\mu/\partial y^\alpha$ und
$\partial x^\alpha/\partial y^\nu$ sind also zueinander invers.
Dasselbe gilt, wenn wir die Funktionen $x^\mu(y^1,\dots,y^n)$
als Argumente in den Funktionen $y^\mu(x^1,\dots,x^n)$ einsetzen:
\begin{align}
\delta^\alpha_\beta
=
\frac{\partial y^\alpha}{\partial y^\beta}
&=
\frac{\partial y^\alpha}{\partial x^\mu}
\frac{\partial x^\mu}{\partial y^\beta}.
\label{label:trafo:tinv}
\end{align}
Auch hier ist darauf zu achten, dass die Ableitungen am selben
Punkt zu nehmen sind.

\subsection{Metrik}
Die Koeffizienten $g_{\mu\nu}$ des metrischen Tensors dienen dazu,
das Längenelement entlang einer Kurve zu berechnen.
Wird eine mit $s$ parametrisierte Kurve durch die Funktionen
$x^\mu(s)$ beschrieben, dann ist die Länge der Kurve durch das
Integral
\begin{equation}
l = \int_a^b
\sqrt{g_{\mu\nu}(x^\alpha(s))\,\frac{dx^\mu}{ds}\,\frac{dx^\nu}{ds}}
\,ds
\label{skript:koord:metrik}
\end{equation}
gegeben.
Die Länge der Kurve ist eine geometrische Grösse, die vom Koordinatensystem 
unabhängig definiert werden kann.
Die Formel~\eqref{skript:koord:metrik} darf also ihre Form nicht verändern,
wenn man die Kurvenlänge in einem anderen Koordinatensystem berechnen will.

Die allgemeine Kovarianz fordert, dass es $\bar g_{\alpha\beta}(y^1,\dots,y^n)$
gibt derart, dass die Kurvenlänge mit der Formel
\begin{equation}
l = \int_a^b
\sqrt{\bar g_{\alpha\beta}(y^\sigma(s))\,\frac{dy^\alpha}{ds}\,\frac{dy^\beta}{ds}}
\,ds
\label{skript:koord:metrikbar}
\end{equation}
berechnet werden kann.

Damit die Längenmessung mit der Formel
\eqref{skript:koord:metrikbar} auf das gleiche Resultat wie 
\eqref{skript:koord:metrik} führt, müssen die Integranden übereinstimmen.
Setzt man \eqref{skript:trafo:kontra} in \eqref{skript:koord:metrikbar}
ein, erhält man unter der Wurzel
\begin{equation*}
\bar g_{\alpha\beta}
\frac{dy^\alpha}{ds}\frac{dy^\beta}{ds}
=
\bar g_{\alpha\beta}
\frac{\partial y^\alpha}{\partial x^\mu}
\frac{\partial y^\beta}{\partial x^\nu}
\frac{dx^\mu}{ds}\frac{dx^\nu}{ds}
=
g_{\mu\nu}
\frac{dx^\mu}{ds}\frac{dx^\nu}{ds}.
\end{equation*}
Gleichheit ist nur möglich, wenn
\begin{equation}
\bar g_{\alpha\beta}
\frac{\partial y^\alpha}{\partial x^\mu}
\frac{\partial y^\beta}{\partial x^\nu}
=
g_{\mu\nu}
\end{equation}
gilt.
Die metrischen Koeffizienten $g_{\mu\nu}$ sind folglich in beiden
Indizes kovariant.

\subsubsection{Der inverse metrische Tensor $g^{\mu\nu}$}
Die Grössen $g^{\mu\nu}$ sind so definiert, dass in jedem Koordinatensystem
gilt
\[
\delta^\mu_\nu = g^{\mu\sigma}g_{\sigma\nu}.
\]
Wir müssen die Transformationsregeln für $g^{\mu\nu}$ herausfinden.
Dazu rechnen wir
\begin{align*}
\delta^\mu_\nu
&=
\bar g^{\mu\sigma}\bar g_{\sigma\nu}
=
\bar g^{\mu\sigma}
\biggl(
g_{\alpha\beta}
\frac{\partial x^\alpha}{\partial y^\mu}
\frac{\partial x^\beta}{\partial y^\sigma}
\biggr)
=
\biggl(
\bar g^{\mu\sigma}
\frac{\partial x^\alpha}{\partial y^\mu}
\frac{\partial x^\beta}{\partial y^\sigma}
\biggr)
g_{\alpha\beta}
\quad
\Rightarrow
\quad
g^{\alpha\beta}
=
\bar g^{\mu\sigma}
\frac{\partial x^\alpha}{\partial y^\mu}
\frac{\partial x^\beta}{\partial y^\sigma}.
\end{align*}
Die Koeffizienten sind daher, wie die Notation suggeriert, tatsächlich
kontravariant.

\subsection{Tensoren}
In Erweiterung der eben diskutierten Beispiele nennen wir die Grössen
$
A^{\mu\nu}\mathstrut_{\alpha\beta\gamma}
$
einen zweifach kontravarianten und dreifach kovarianten Tensor, wenn
das Transformationsgesetz
\[
\bar 
A^{\mu\nu}\mathstrut_{\varrho\sigma\tau}
\frac{\partial y^\delta}{\partial x^\mu}
\frac{\partial y^\epsilon}{\partial x^\nu}
=
A^{\delta\epsilon}\mathstrut_{\alpha\beta\gamma}
\frac{\partial y^\alpha}{\partial x^\varrho}
\frac{\partial y^\beta}{\partial x^\sigma}
\frac{\partial y^\gamma}{\partial x^\tau}
\]
gilt.

Tensoren sollen dazu dienen, vom Koordinatensystem unabhängige
geometrische Aussagen oder Naturgesetze zu formulieren.
Dazu sind Rechenregeln nötig, die aus Tensoren neue Tensoren
machen.
In den folgenden Abschnitten stellen wir eine Reihe von Operationen
mit dieser Eigenschaft zusammen.

\subsubsection{Index hochziehen}
Zum {\em Hochziehen eines Index} wird ein Tensor mit einem kovarianten
Index $\mu$ mit dem kontravarianten metrischen Tensor $g^{\mu\nu}$
multipliziert.
\index{Index!hochziehen}%
\index{Hochziehen eines Index}%
Die anderen Indizes des Tensors spielen dabei offenbar keine Rolle,
ihr Transformationsgesetz wird durch die Operation nicht tangiert.
Wir schreiben daher nur diesen einen Index, betrachten also den Tensor
$A_\mu$.
Die Operation des Hochziehens ist also
\begin{align*}
\bar A^\mu
=
\bar g^{\mu\nu}\bar A_\nu
&=
g^{\alpha\beta}
\frac{\partial y^\mu}{\partial x^\alpha}
\frac{\partial y^\nu}{\partial x^\beta}
A_\gamma
\frac{\partial x^\gamma}{\partial y^\nu}
=
g^{\alpha\beta}
\frac{\partial y\mu}{\partial x^\alpha}
A_\gamma
\underbrace{
\frac{\partial x^\gamma}{\partial y^\nu}
\frac{\partial y^\nu}{\partial x^\beta}
}_{\displaystyle \mathstrut= \delta^\gamma_\beta}
=
g^{\alpha\beta}
A_\beta
\frac{\partial y^\mu}{\partial x^\alpha}.
\end{align*}
Dies ist genau die Transformationsformel für einen kontravarianten Index.

Auf die gleiche Weise kann mit dem metrischen Tensor $g_{\mu\nu}$ ein
kontravarianter Index heruntergezogen werden zu einem kovarianten Index.
\index{Herunterziehen eines Index}%
\index{Index!herunterziehen}%

\subsubsection{Verjüngung}
Setzt man in einem Tensor wie $A^{\mu\nu}\mathstrut_{\alpha\beta\gamma}$
einen kovarianten und einen kovarianten gleich und summiert,
erhält man die sogenannte {\em Verjüngung} über die beiden Indizes.
\index{Verjüngung}%
Für die Indizes $\mu$ und $\alpha$ bedeutet dies
\[
A^{\sigma\nu}\mathstrut_{\sigma\beta\gamma}
=
B^\nu\mathstrut_{\beta\gamma}.
\]
Man kann die Verjüngung auch etwas komplizierter als
\begin{equation}
A^{\sigma\nu}\mathstrut_{\sigma\beta\gamma}
=
A^{\mu\nu}\mathstrut_{\alpha\beta\gamma}\delta^\alpha_\mu
\label{skript:trafo:verjung}
\end{equation}
schreiben, darin wird sowohl über $\mu$ als auch über $\alpha$
summiert.
Wir behaupten, dass die Verjüngung wieder ein Tensor ist.

Da die Transformationsformeln für die übrigen Indizes von dieser
Operation offenbar nicht betroffen werden, schreiben wir der Einfachheit
halber nur die beiden von der Verjüngung betroffenen Indizes.
Wir müssen also nur nachprüfen, dass $A^\mu_\mu$ sich wie ein Skalar
verhält, dass also gilt $A^\mu_\mu=\bar A^\alpha_\alpha$ gilt.

Die Tensoreigenschaft der Verjüngung ergibt sich jetzt aus der Schreibweise
\eqref{skript:trafo:verjung}
und der Eigenschaft 
\label{label:trafo:t}
der Transformationsmatrix mittels der Rechnung
\begin{align*}
A^\mu_\mu
&=
A^\mu_\sigma
\delta^\sigma_\mu
=
A^\mu_\sigma
\frac{\partial x^\sigma}{\partial y^\alpha}
\frac{\partial y^\alpha}{\partial x^\mu}
=
\bar A^\alpha_\alpha.
\end{align*}
Die Verjüngung führt also tatsächlich wieder zu einem Tensor.

