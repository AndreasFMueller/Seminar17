%
% orbit.tex -- Orbit in einem Schwarzschildpotential
%
% (c) 2017 Prof Dr Andreas Müller, Hochschule Rapperswil
%
\documentclass[tikz]{standalone}
\usepackage{times}
\usepackage{txfonts}
\usepackage{pgfplots}
\usepackage{csvsimple}
\usetikzlibrary{arrows,intersections}
\begin{document}
\begin{tikzpicture}[thick,scale=0.7]
\coordinate (O) at (0,0);

\csvreader[head to column names,%
	late after head=\xdef\xold{\x}\xdef\yold{\y},%
	after line=\xdef\xold{\x}\xdef\yold{\y}]%
	{orbit.csv}{}{%
		\draw[red,line width=1.5] (0.1 * \xold,0.1 * \yold)--(0.1 * \x,0.1 * \y) node {};
	};

\draw[->] (0,0)--(11,0) coordinate[label = {above:$r$}];

\draw (0,0)--({11 * cos(6.6604 * 180 / 3.1415)},{11 * sin(6.6604 * 180 / 3.1415}) {};

\draw[fill=white] (0,0) circle[radius=0.1] {};
\draw[blue] (0,0) circle[radius=0.1] {};

\node[label={right:$\alpha$}] at (1,0.3) {};
\node[label={above:$r_g$}] at (0,0) {};

\end{tikzpicture}
\end{document}


