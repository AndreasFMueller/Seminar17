%
% lichtablenkung.tex -- 
%
% (c) 2017 Prof Dr Andreas Müller, Hochschule Rapperswil
%
\documentclass[tikz]{standalone}
\usepackage{times}
\usepackage{txfonts}
\usepackage{pgfplots}
\usepackage{csvsimple}
\usetikzlibrary{arrows,intersections}
\begin{document}
\begin{tikzpicture}[thick,scale=5]
\coordinate (O) at (0,0);
\begin{scope}
\clip (0,0) rectangle (1.6,2);
\csvreader[head to column names,%
	late after head=\xdef\phiold{\phi}\xdef\rold{\r},%
	after line=\xdef\phiold{\phi}\xdef\rold{\r}]%
	{lichtablenkung.csv}{}{%
		\draw[red,line width=1.5] (\phiold,0.1 * \rold)--(\phi,0.1 * \r) node {};
};
\draw[line width=0.5] (3.14159/2,0)--(3.14159/2,100);

\end{scope}

\draw (-0.02,0.1)--(0.02,0.1);
\node[label={left:$1$}] at (0.0,0.1) {};

\draw[->] (-0.1, 0  )--(1.7,  0   ) coordinate[label = {above:$\varphi$}];
\draw[->] ( 0  ,-0.1)--(0  , 2   ) coordinate[label = {left:$r$}];

\draw (3.14159/2, -0.02)--(3.14159/2, 0.02);

\node[label={below:$\pi$}] at (3.14159/2,0) {};

\end{tikzpicture}
\end{document}


