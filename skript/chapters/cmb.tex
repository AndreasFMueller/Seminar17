%
% cmb.tex
%
% (c) 2017 Prof Dr Andreas Müller, Hochschule Rapperswil
%
\chapter{Der kosmische Mikrowellenhintergrund%
\label{chapter:cmb}}
\lhead{Der kosmische Mikrowellenhintergrund}
\rhead{}

\section{Schwarzkörperstrahlung}
\rhead{Schwarzkörperstrahlung}
Ein ungelöstes Problem der Physik im ausgehenden neunzehnten Jahrhundert
war das Strahnungsspektrum eines schwarzen Körpers zu berechnen.
Ein schwarzer Körper ist eine idealsierte thermische Strahlungsquelle,
die auftreffende elektromagnetische Strahlung jeder Wellenlänge absorbiert
und deren Strahlungsspektrum nicht von der Oberflächebeschaffenheit
abhängt, sondern nur von dessen Temperatur.
Eine mögliche Realisierung ist ein mit Strahlung gefüllter Hohlraum, dessen
Wände auf konstanter Temperatur gehalten werden.

Die klassische Thermodynamik, die im
Abschnitt~\ref{skript:cmb:section:klassisch} zusammengestellt wird,
stellt die Werkzeuge bereit, mit der wir
dem Strahungsgesetz auf den Grund gehen können.
In Abschnitt~\ref{skript:cmb:section:planck} leiten wir daraus
das Plancksche Strahlungsgesetz her.
Insbesondere interessiert uns dabei, dass eine Vergrösserung aller
Wellenlängen um den gleichen Streckungsfaktor die Form des
Strahlungsgesetzes nicht ändert, so dass die Expansion des Universums
die Eigenschaften des Schwarzkörperspektrums erhält.
Daraus werden wir auch ableiten, wie sich die zugehörige Temperatur
mit der Expansion verändert.
Die kosmische Mikrowellenhintergrundstrahlung wurde in dem Moment
ausgestrahlt, als das ionisierte Gas frühen Universums neutral wurde,
man nennt dies die Rekombination.
Die Dynamik dieses Ereignisses wird in
Abschnitt~\ref{skript:cmb:section:rekombination} untersucht.

\subsection{Klassische Theorie%
\label{skript:cmb:section:klassisch}}
Ein thermodynamisches System besteht aus einer grossen Zahl von 
Teilchen, die ständig interagieren und dabei Energie austauschen.
Der mikroskopische Zustand des Systems ändert sich ständig.
\index{mikroskopischer Zustand}
\index{Zustand, mikroskopisch}
Es ist daher gar nicht möglich, den aktuellen mikroskopischen
Zustand zu beschreiben.
Jede makroskopische Messung liefert daher immer nur einen mittleren
Wert für die gemessene Grösse.

Die fundamentale Zustandsgrösse der klassischen Thermodynamik ist
die Entropie, eine Funktion $S(U,V)$ von inneren Energie $U$ und 
Volumen $V$ des Systems. 
\index{Entropie}
Alle mikroskopischen Zustände zur gleichen Energie und im gleichen
Volumen sind gleichermassen zulässig.
Offenbar gibt es auch keinen Grund, warum einer dieser Zustände
bevorzugt werden soll.
Die statistische Wärmelehre
{\em postuliert} daher, dass alle mikroskopischen Zustände gleich
wahrscheinlich sind.
Dies bedeutet auch, dass das System immer alle mikroskopischen
Zustände ausnützt, die mit der vorgegebenen inneren Energie $U$ und
dem Volumen $V$ vereinbar sind.
Die Zahl $\Omega(U,V)$ der mikroskopischen Zustände, zwischen denen das
System rasche Übergänge vollführt, nimmt zum Beispiel zu, wenn ein
Ventil geöffnet wird, und dem System damit ein grösseres Volumen
zur Verfügung steht.

Andererseits ist aus der klassichen Thermodynamik bekannt, dass ein
System immer denjenigen Zustand annimmt, in dem die Entropie $S(U,V)$
als Funktion von $U$ und $V$ ihr Maximum annimmt.
Man darf daher schliessen, dass die Entropie als Funktion der Anzahl
$\Omega(U,V)$ der mikroskopischen Zustände ausgedrückt werden kann.
Allerdings ist die Entropie additiv, fügt man zwei Systeme zusammen,
so ist die Entropie des Gesamtsystems die Summe der Entropien der
Teilsysteme:
\[
S(U,V) = S_1(U_1,V_1) + S_2(U_2,V_2).
\]
Die Anzahl der Zustände hingegen ist multiplikativ: da es für jeden
der $\Omega_1(U_1,V_1)$ Zustände des ersten Systems $\Omega_2(U_2,V_2)$
Zustände des zweiten Systems gibt, ist die Gesamtzahl der Zustände
deren Produkt
\[
\Omega(U,V)=\Omega_1(U_1,V_1)\cdot \Omega_2(U_2,V_2).
\]
Die Logarithmus-Funtion erfüllt diese Bedingung, und bis auf einen
Faktor ist sie auch die einzige Funktion mit dieser Eigenschaft.
Man kann also immer
\begin{equation}
S(U,V)=k_B\log \Omega(U,V)
\label{skript:cmb:boltzmann}
\end{equation}
setzen, wobei der Faktor $k_B$ im wesentlichen die Masseinheit von
$S$ festlegt.
Durch die Wahl
\[
k_B = 1.3807\cdot 10^{-23} \text{J}/\text{K}
\]
wird Übereinstimmung mit der Kelvin-Temperaturskala erreicht.

\subsubsection{Kanonischer Formalismus}
Dieser Formalismus ist jedoch für die Beschreibung eines Systems 
nicht geeignet, welches sich in Kontakt mit einem Wärmereservoir
befindet, mit welchem es zur Erhaltung der Temperatur Energie
austauschen kann.
Dabei ändert sich die innere Energie des Systems.
Zu dessen Beschreibung ist daher eine Zustandsfunktion nötig, die von
der Temperatur und dem Volumen abhängt, während die Entropie
von innerer Energie und Volumen abhängt.
Man kann dieses Problem lösen, indem man das Wärmereservoir als
Teil eines grösseren Systems modelliert, mit dem das untersuchte
System Energie austauschen kann.
Die Gesamtenergie $E_\text{tot}$ des kombinierten Systems ist dann zwar eine
Zustandsvariable, aber das ursprüngliche Teilsystem kann verschiedene
Zustände annehmen.
Bezeichnen wir die verschiedenen Zustände mit einem Index $j$
und ihre Energie mit $E_j$, dann können wir die Wahrscheinlichkeit
berechnen, dass sich das ursprüngliche System im Zustand $j$ befindet.
Dazu berechnen wir die Anzahl der Zustände, in denen sich das 
Reservoir in einem Zustand mit Energie $E_\text{tot} - E_j$ befindet
und dividieren durch die Gesamtzahl der Zustände mit Energie $E_\text{tot}$.
Wir finden
\[
w_j
=
\frac{\Omega_\text{res}(E_\text{tot}-E_j)}{\Omega_\text{tot}(E_\text{tot})}
\]
für die Wahrscheinlichkeit.
Indem man \eqref{skript:cmb:boltzmann} nach $\Omega$ auflöst, kann $w_j$
durch die Entropie ausgedrückt werden:
\begin{equation}
w_j
=
\frac{\exp\bigl(S_\text{res}(E_\text{tot}-E_j)/k_B\bigr)}{\exp\bigl(S_\text{tot}(E_\text{tot})/k_B\bigr)}
\label{skript:cmb:fj}
\end{equation}
Ist $U$ die mittlere Energie des ursprünglichen Systems im Gleichgewicht,
dann gilt die Additivität der Entropie für das kombinierte System in der Form
\[
S_\text{tot}(E_\text{tot})
=
S(U) + S_\text{res}(E_\text{tot}-U).
\]
Da das Reservoir sehr gross ist, sind die Energien $E_j$ der Zustände des
ursprünglichen Systems im Vergleich zur Gesamtenergie sehr klein,
es ist daher zulässig, die Entropie des Reservoirs linear zu approximieren
\begin{align*}
S_\text{res}(E_\text{tot}-E_j)
&=
S_\text{res}(E_\text{tot}-U+U-E_j)
\\
&=
S_\text{res}(E_\text{tot}-U) + \frac{\partial S_\text{res}}{\partial U}(U-E_j)
\\
&=
S_\text{res}(E_\text{tot}-U) + \frac{U-E_j}{T}
\end{align*}
Setzt man dies in den Ausdruck~\eqref{skript:cmb:fj} für $w_j$ ein, 
erhält man
\[
w_j = e^{(U-TS(U))/k_BT} e^{-E_j/k_BT}.
\]
Man sieht, dass in den Formeln vor allem der Ausdruck $1/k_BT$ vorkommt,
denn man daher mit $\beta=1/k_BT$ abkürzt.
Die Wahrscheinlichkeiten $w_j$ können damit etwas prägnanter als
\begin{equation}
w_j = e^{\beta(U-TS(U))}e^{-\beta E_j}
\label{skript:cmb:fj2}
\end{equation}
geschrieben werden.

\subsubsection{Freie Energie}
Der Ausdruck $F=U-TS(U)$ im Exponenten von~\eqref{skript:cmb:fj2}
heisst die {\em helmholtzsche freie Energie}.
\index{freie Energie}
\index{Energie, freie}
\index{helmholzshe freie Energie}
Die helmholtzsche freie Energie $F$ eines thermodynamischen Systems
gibt an, wieviel Energie bei einem reversiblen, isothermen Prozess zur
Arbeitsleistung zur Verfügung steht. 
Je höher die Temperatur oder die Entropie eines Systems ist, desto geringer
ist die für Arbeitsleistung zur Verfügung stehende freie Energie.
Die freie Energie ist eine Zustandsfunktion der Variablen $T$ und $V$.
Da die Temperatur $T$ als Variable auftritt, eignet sich die freie
Energie als Zustandsfunktion zur Beschreibung eines Systems, welches
sich in Kontakt mit einem Wärmereservoir befindet.

Die Wahrscheinlichkeiten $w_j$ müssen sich zu $1$ summieren, es muss also
gelten
\begin{equation}
1=\sum_{j} w_j = e^{\beta F}\sum_j e^{-\beta E_j}
\qquad
\Rightarrow
\qquad
e^{-\beta F}=\sum_je^{-\beta E_j}=Z.
\label{skript:cmb:Z}
\end{equation}
Die Zahl $Z$ heisst die {\em kanonische Zustandssumme}.
Aus ihr kann man die freie Energie berechnen, indem man
\eqref{skript:cmb:Z} nach $F$ auflöst:
\begin{equation}
-\beta F=\log\sum_j e^{-\beta E_j} = \log Z.
\end{equation}
Die Wahrscheinlichkeiten $w_j$ kann man jetzt auch ganz unabhängig vom
Reservoir als
\begin{equation}
w_j
=
\frac{e^{-\beta E_j}}{\sum_j e^{-\beta E_j}}
\end{equation}
ausdrücken.

\subsubsection{Mittlere Energie}
Schliesslich lässt sich auch die mittlere Energie des ursprünglichen
Systems durch $Z$ ausdrücken.
Dazu verwendet man zunächst die Definition 
\[
U
=
\sum_j E_jw_j
=
\frac{1}{\sum_i e^{-\beta E_j}}
\sum_j E_je^{-\beta E_j}
=
\frac1Z \sum_j E_je^{-\beta E_j}
\]
des Erwartungswertes der Energie.
Die Ableitung von $\log Z$ nach $\beta$ liefert aber fast dasselbe:
\begin{align*}
\frac{d}{d\beta} \log Z
&=
\frac1{Z}\frac{dZ}{d\beta}
=
\frac1{Z}\sum_j\frac{de^{-\beta E_j}}{d\beta}
=
-\frac1Z\sum_j E_je^{-\beta E_j}
=
-U,
\end{align*}
die mittlere Energie ist also bis auf ein Vorzeichen auch die Ableitung
von $\log Z$ nach $\beta$.
Damit lässt sich auch die innere Energie
\[
U=-\frac{d}{d\beta}\log Z
\]
des ursprünglichen Teilsystems im Gleichgewicht mit dem Reservoir der
Temperatur $T$ aus der Zustandssumme berechnen.

\subsection{Plancksches Strahlungsgesetz%
\label{skript:cmb:section:planck}}
Die im vorangegangenen Abschnitt zusammengestellten Grundlagen der
statischen Mechanik erlauben uns, das Plancksche Strahlungsgesetz
zu verstehen.

\subsubsection{Frequenzdichte}
Elektromagnetische Strahlung in einem verspiegelten Hohlraum bildet
stehende Wellen,
deren Wellenlänge ein ganzzahliger Bruchteil der Abmessungen des
Hohlraumes sein muss.
Eine solche stehende Welle kann durch einen Vektor
\[
\vec{k} = \begin{pmatrix}k_x\\k_y\\k_z\end{pmatrix}
\]
von ganze Zahlen $k_x,k_y,k_z\in\mathbb Z$ beschrieben werden, der
die Anzahl der Wellenlängen angibt, die in jeder Richtung in die
Abmessungen des Hohlraums hineinpassen.
Sei $L$ die Seitenlänge des Hohlraumes, das Volumen ist dann $V=L^3$.
Eine Welle kann dann mit Hilfe der Funktionen
\[
\sin \vec{x}\cdot\vec{k}\frac{2\pi}{L}
\qquad\text{und}\qquad
\cos \vec{x}\cdot\vec{k}\frac{2\pi}{L}
\]
beschreiben.
Der Faktor $2\pi/L$ stellt sicher, dass eine Zunahme einer Koordinate
um $L$ das Argument der Winkelfunktion genau um $2\pi$ vergrössert,
so dass die resultierende Funktion in jeder Koordinate $L$-periodisch ist.

Die Ausbreitungsrichtung einer solchen Welle ist $\vec{k}$.
Die Wellenlänge kann man an einem Vektor $\vec{x}$ ablesen, der
$\vec{x}\cdot\vec{k}=L$ erfüllt, er muss daher die Länge
\[
\lambda
=
|\vec{x}|
=
\frac{L}{|\vec{k}|}
\]
haben.
Die zugehörige Frequenz wird daher
\[
\nu(\vec{k})
=
\frac{c}{\lambda}=|\vec{k}|\frac{c}{L}.
\]

Später wird uns die Anzahl der möglichen Vektoren $\vec{k}$ interessieren,
die auf Frequenzen in einem vorgegeben Intervall $[\nu, \nu+d\nu]$
führen.
Dies ist auch die Anzahl der Vektoren mit ganzzahligen Koordinaten derart,
dass
\[
\frac{L}{c}\nu
\le
|\vec{k}|
\le
\frac{L}{c}(\nu+d\nu)
\]
gilt.
Dies ist aber auch ungefähr das Volumen einer Kugelschale zwischen den
Radien 
\[
r=\frac{L}{c}\nu
\qquad\text{und}\qquad
r+dr=\frac{L}{c}(\nu+d\nu)
\]
Die Volumenformel
\[
V(r)
=
\frac{4}{3}\pi r^3
\]
für eine Kugel vom Radius $r$ liefert für das Volumen
zwischen den beiden Radien
\[
V(r+dr)-V(r)
\simeq
\frac{d}{dr} \frac{4}{3}\pi  r^3 \, dr
=
4\pi r^2\,dr.
\]
Daraus liest man ab, dass im Interval $[\nu,\nu+d\nu]$ 
\[
4\pi \frac{L^3}{c^3}\nu^2\,d\nu
=
4\pi \frac{V}{c^3}\nu^2\,d\nu
\]
Frequenzen möglich sind.
Bezogen auf das Volumen $V$ bleibt die Zustandsdichte
\begin{equation}
g(\nu)\,d\nu = \frac{4\pi}{c^3}\nu^2\,d\nu.
\label{skript:cmb:zustandsdichte}
\end{equation}
Allerdings ist zu beachten, dass ein elektromagnetisches Strahlungsfeld
zu jedem Wellenzahlvektor $\vec{k}$ zwei voneinander unabhängige 
Felder mit orthogonalen Polarisierungen ermöglicht, was die Zahl
der Zustände nochmals verdoppelt.

\subsubsection{UV-Katastrophe}
Ein klassischer harmonischer Oszillator kann jede beliebige Energie haben,
da diese nur von der Amplitude abhängt.
Die klassische Thermodynamik leitet aus dem kanonischen Formalismus den
Gleichverteilungssatz ab, der besagt,
dass jeder Freiheitsgrad die gleiche mittlere Energie beiträgt.
Da zu jedem Wellenzahlvektor ein eigener Oszillator und damit ein 
eigener Freiheitsgrad zur Verfügung steht, trägt jeder Wellenzahlvektor
die gleiche Energiemenge bei.
Da es unendlich viele Wellenzahlvektoren gibt, wird die Gesamtenergie
unendlich gross.
Man nennt dies die UV-Katastrophe.
Sie zeigt, dass die klassischen Vorstellungen und insbesondere der
Gleichverteilungssatz nicht anwendbar sind.
Das Problem kann nur gelöst werden indem man dafür sorgt, indem man
annimmt, dass die Oszillatoren mit hoher Frequenz nur mit geringerer
Wahrscheinlichkeit angeregt werden.

\subsubsection{Quantenhypothese}
Planck nahm in seiner Verzweiflung an, dass in einem Oszillator mit
Freqeuenz $\nu$ nur die diskreten Vielfachen des Energiequantums $h\nu$ 
erlaubt sind, wobei $h$, das {\em Plancksche Wirkungsquantum}, derart
klein ist, dass es bei hohen Energien gar nicht erst in Erscheinung
tritt.
Der Oszillator mit Frequenz $\nu$ hat daher die Energiezustände
\[
E_j = jh\nu,
\]
und damit die Zustandssumme
\[
Z_\nu
=
\sum_{j=0}^\infty e^{-j\beta h\nu}
=
\sum_{j=0}^\infty \bigl(e^{-\beta h\nu}\bigr)^j.
\]
Dies ist eine geometrische Reihe mit dem Quotienten
$q=e^{-\beta h\nu}$ mit der Summe
\[
Z_\nu
=
\frac1{1-q}=\frac1{1-e^{-\beta h\nu}}.
\]
Die mittlere Energie, die sich im Gleichgewicht in diesem Oszillator 
befindet, ist
\[
E(\nu,T)
=
-\frac{d}{d\beta}\log Z_\nu
=
\frac{h\nu e^{-\beta h \nu}}{1-e^{-\beta h\nu}}
=
\frac{h\nu}{e^{\beta h\nu}-1}
=
\frac{h\nu}{e^{h\nu/k_BT}-1}.
\]

Nun ist aber auch noch zu berücksichtigen, dass nicht jede beliebige
Frequenz möglich ist.
Die Zustandsdichte für die Frequenzen wurde
in~\eqref{skript:cmb:zustandsdichte} bereits bestimmt.
Somit erhalten wir das Plancksche Strahlungsgesetz.
Die im Frequenzinterval $d\nu$ abgestrahlte Energie eines
schwarzen Körpers ist
\begin{equation}
U(\nu,T)\,d\nu
=
2g(\nu)\,d\nu
\frac{h\nu}{e^{h\nu/k_BT}-1}
=
\frac{8\pi}{c^3}\nu^2\,d\nu\frac{h\nu}{e^{h\nu/k_BT-1}}
=
\frac{8\pi h\nu^3}{c^3}\frac{d\nu}{e^{h\nu/k_BT}-1}
\label{skript:cmb:plancknu}
\end{equation}
Der Faktor $2$ rührt von den zwei Polarisierungen her, die die Strahlung
bei gleichem Wellenzahlvektor $\vec{k}$ und damit gleicher Frequenz haben
kann.

\subsubsection{Wellenlänge}
In~\eqref{skript:cmb:plancknu} haben wir das Plancksche Strahlungsgesetz
durch die Frequenz ausgedrückt.
Wir hätten stattdessen auch die Wellenlänge $\lambda=c/\nu$ verwenden 
können.
Einem Interval $[\lambda,\lambda+d\lambda]$ von Wellenlängen entspricht
das Frequenzinterval
\[
\biggl[
\frac{c}{\lambda}-\frac{c}{\lambda^2}d\lambda
,
\frac{c}{\lambda}
\biggr]
\]
und damit das Strahlungsgesetz
\begin{equation}
U(\lambda,T)\,d\lambda
=
\frac{8\pi h}{\lambda^3}\frac{1}{e^{hc/\lambda k_BT}-1}\frac{c}{\lambda^2}\,d\lambda
\frac{8\pi hc}{\lambda^5}\frac{1}{e^{hc/\lambda k_BT}-1}\,d\lambda
\label{skript:cmb:planck:lambda}
\end{equation}
in Abhängigkeit von der Wellenlänge $\lambda$ und der Temperatur $T$.

\subsubsection{Verschiebungsgesetz}
Ändert man die Temperatur des Hohlraumes um den Faktor $a$ zur $T_1=aT$,
dann wird sich das Strahlungsspektrum $U_1$ zu anderen Wellenlängen
$\lambda_1$ verschieben.
Wir wollen zeigen, dass mit $\lambda_1=\lambda/a$ wieder ein Plancksches
Strahlungsgesetz entsteht, allerdings mit einer geringeren abgestrahlten
Energie.
Setzen wir dies in das Strahlungsgesetz ein, erhalten wir
\begin{align*}
U_1(\lambda_1,T_1)\,d\lambda_1
&=
B\cdot
U(a\lambda_1,T_1/a)a\,d\lambda_1
\\
&=
\frac{B}{a^4}
\cdot
\frac{8\pi hc}{\lambda_1^5}\frac{1}{e^{hc/\lambda_1k_BT_1}-1}\,d\lambda_1
\\
&=
\frac{B}{a^4}\cdot U(\lambda_1,T_1)\,d\lambda_1.
\end{align*}
Das Strahlungsspektrum bei der Temperatur $T_1=aT$ ist also identisch mit dem
Spektrum bei der Temperatur $T$ mit Wellenlängen, die um den Faktor $a^{-1}$
verkürzt wurden.
Der Faktor $B$ muss $B=a^4$ sein, damit dies zutrifft.
Wir schliessen daraus, dass die Gesamtstrahlung eines schwarzen Körpers
proportional zu $T^4$ sein muss (Wiensches Verschiebungsgesetz).

\subsubsection{Expansion}
In der Form~\eqref{skript:cmb:planck:lambda}
des Strahlungsgesetztes können wir untersuchen was
passiert, wenn sich das ganze Universum, betrachtet als ein gigantischer
Hohlraum, sich um den Faktor $a$ ausdehnt.

Wir bezeichnen die Wellenlänge nach der Expansion mit $\lambda_1=a\lambda$
und bestimmen die Energiedichte $U_1(\lambda_1,T_1)$ in Abhängigkeit von
$\lambda_1$.
Im besten Fall ist $U_1(\lambda_1,T_1)$ wieder das Plancksche
Strahlungsgesetz, möglicherweise mit einer anderen Temperatur $T_1$, die
es ebenfalls zu bestimmen gilt.

Aus dem Strahlungsgesetz~\label{skript:cmb:planck:lambda} in Abhängigkeit
von der Wellenlänge wird dann
\begin{align*}
U_1(\lambda_1,T_1)\,d\lambda_1
&=
U(\lambda_1/a,T)\,d(\lambda_1/a)
\\
&=
\frac{8\pi hc}{(\lambda_1/a)^5}
\frac{1}{e^{hc/(\lambda_1/a) k_BT}-1}\,d(\lambda_1/a)
\\
&=
a^4
\frac{8\pi hc}{\lambda_1^5}
\frac{1}{e^{hc/\lambda_1 k_B(T/a)}-1}\,d\lambda_1
\end{align*}
Bis auf den Faktor $a^4$ ist dies wieder ein Plancksches Spektrum
für die Temperatur $T_1=T/a$.
Der Faktor $a^4$ passt aber zu der Tatsache, dass die Gesamtstrahlung
bei einer Temperaturänderung um den Faktor $1/a$ ebenfalls um Faktor
$a^{-4}$ abnehmen muss.
Der Faktor $a^4$ kompensiert also genau diese Abnahme.
Wir kommen daher zum Schluss, dass die Expansion des Universums um den
Faktor $a$ die Temperatur der darin enthaltenen Strahlung um den Faktor
$1/a$ gesenkt wird.
Die Energiedichte der Strahlung nimmt dagegen um den Faktor $a^{-4}$ ab.

\subsubsection{Photonenzahl}
Die Herleitung des Planckschen Strahlungsgesetzes erlaubt auch, die
erwartete Anzahl der angeregten Zustände, also der Photonen im
Hohlraum zu ermitteln.
Wir notieren hier nur das Resultat.
Die Dichte der Photonen in einem Hohlraum der Temperatur $T$ ist
\begin{equation}
n_\gamma
=
16\pi\zeta(3)
\biggl(\frac{k_BT}{hc}\biggr)^3.
\label{skript:cmb:photonenzahl}
\end{equation}
Darin ist $\zeta(3)$ der Wert der Riemannschen Zetafuntion an der
Stelle $3$.
Ein numerischer Wert ist $\zeta(3)=1.20205690\dots$.

\section{Rekombination%
\label{skript:cmb:section:rekombination}}
\rhead{Rekombination}
Im frühen Universum war die Energiedichte der Strahlung viel höher
als die Energiedichte der Materie.
Die Friedmann-Gleichung zeigt auch, wie man den Zeitpunkt berechnen
kann, zu dem die Energiedichte von Strahlung und Materie gleich
waren.
Im frühen Universum gab es daher einerseits sehr viel mehr Strahlung
als Materie, andererseits war die Dichte höher, so dass es sehr
häufig zu Reaktionen zwischen Atomen und Photonen kam, wodurch die
Atome ionisiert wurden.
Erst als das Universum genügend abgekühlt war konnten sich Atomkerne
und Elektronen für längere Zeit zusammenfinden.
Man nennt diese Phase die Rekombination, obwohl die Atomkerne und
Elektronen davor noch nie kombiniert gewesen waren.

In diesem Abschnitt wollen wir daher berechnen, wie der Prozess der
Rekombination abläuft und bei welchem Alter des Universum dies
etwas stattgefunden haben muss.

\subsection{Zusammensetzung des Universums}
Wir nehmen in diesem Abschnitt an, dass das Universum nur aus 
Protonen, Elektronen und Strahlung besteht.
Insbesonderen vernachlässigen wir den beträchtlichen Teil
Helium, der im Rahmen des Big Bang erzeugt worden ist.
Da die Ionisationsenergie von Helium etwa dopopelt so hoch ist wie
diejenige von Wasserstoff, muss Helium im wesentlichen bereits
rekombiniert haben in dem Moment, wo die Rekombination des Wasserstoff
stattfindet.

Die Materie im Universum wird daher beschrieben durch die Anzahldichte
der freien Elektronen $n_e$, die Anzahldichte der freien Protonen $n_p$
und die Anzahldichte $n_\text{H}$ der Wasserstoffatome.
Da das Universum neutral ist, muss $n_e=n_p$ gelten.
Da jedes Wasserstoffatom auch ein Proton als Atomkern enthält, ist die
Gesamtzahl der Baryonen im Universum $n_\text{Baryonen} = n_p+n_\text{H}$.
Der Ionisationsgrad des Unviersums ist daher
\begin{equation}
X=\frac{n_p}{n_\text{Baryonen}}=\frac{n_p}{n_p+n_\text{H}}.
\label{skript:cmb:Xb}
\end{equation}
Ziel dieses Abschnitts ist die Berechnung von $X$ in Abhängigkeit 
von der Zeit.
Löst man die Definition~\eqref{skript:cmb:Xb} nach $n_\text{H}$ auf,
findet man
\begin{align*}
\frac1X
&=
\frac{n_p+n_\text{H}}{n_p}
=
1+
\frac{n_\text{H}}{n_p}
\\
\Rightarrow
\qquad
\frac{n_\text{H}}{n_p}
&=
\frac{1}{X}-1
=
\frac{1-X}{X}
\\
n_\text{H}
&=
\frac{1-X}{X}n_p.
\label{skript:cmb:XnH}
\end{align*}

Ausserdem dürfen wir annehmen, dass die Anzahldichte $n_\gamma$
der Photonen währen der Rekombination wesentlich grösser war als
die Zahl der Baryonen.
Wir bezeichnen mit $\eta$ das Verhältnis
\[
\eta = \frac{n_\text{Baryonen}}{n_\gamma}
\]
von Baryonenzahl zu Photonenzahl.
Mit Hilfe der Definition von $X$ in Gleichung~\eqref{skript:cmb:Xb}
finden wir den Zusammenhang
\begin{equation}
\eta = \frac{n_p}{Xn_\gamma}
\label{skript:cmb:etaX}
\end{equation}
zwischen $\eta$ und $X$.

\subsection{Reaktionsgleichgewicht und Saha-Gleichung}
Die Voraussetzungen über die Zusammensetzung des Universums bedeuten,
dass wir uns nur für die Ionisationsreaktion
\begin{equation}
\text{H} + \gamma 
\rightleftharpoons
p + e^-
\label{skript:cmb:reaktionsgleichung}
\end{equation}
interessieren.
Wir nehmen sogar an, dass die Ionisation eines Wasserstoffatoms 
die konstante Energie $Q=13.6\,\text{eV}$ benötigt.
Damit ignorieren wir angeregt Zustände des Wasserstoffs, die auch
mit geringerem Energieaufwand erreicht werden können.
Bei der Rekombination ist auch ein Übergang in einen angeregten
Zustand möglich, der später durch Emission eines Photons in den Grundzustand
übergehen kann.

Wir möchten berechnen, wie der Ionisationsgrad sich mit der Zeit ändert.
Wenn die Dichte der Photonen sinkt wird sich das Reaktionsgleichgewicht
auf die linke Seite von~\eqref{skript:cmb:reaktionsgleichung} verschieben.
Im statistischen Gleichgewicht kann die Maxwell-Boltzmann-Gleichung
verwendet werden, um die Teilchendichte in Abhängigkeit von der Temperatur
auszudrücken, es gilt für die Komponente $x$
\[
n_x
=
g_x \biggl(\frac{m_xk_BT}{2\pi h^2}\biggr)^{\frac32}
\exp\biggl(-\frac{m_xc^2}{k_BT}\biggr).
\]
Die Zahl $g_x$ ist die Anzahl der Zustände mit gleichem Impuls.
Für Elektronen und Protonen ist $g_e=g_p=2$, da beide genau zwei
Spin-Zustände haben.
Das Wasserstoff-Atom hat dagegen $g_\text{H}=4$ Spinzustände.

Das Verhältnis der Anzahldichten der Reaktionsteilnehmer ist
\begin{equation}
\frac{n_\text{H}}{n_pn_e}
=
\underbrace{\frac{g_\text{H}}{g_pg_e}}_{\displaystyle=1}
\biggl(\frac{m_\text{H}}{m_pm_e}\biggr)^{\frac32}
\biggl(\frac{k_BT}{2\pi h^2}\biggr)^{-\frac32}
\exp\biggl( \frac{(m_p+m_e-m_\text{H})c^2}{k_BT} \biggr)
\label{skript:cmb:saha1}
\end{equation}
Mit unseren vereinfachenden Annahmen ist der Zähler im letzten Term
die Ionisationsenergie
\[
Q=(m_p+m_e-m_\text{H})c^2
\]
des Wasserstoffatoms.
Da der Unterschied zwischen $m_p$ und $m_\text{H}$ sehr gering ist,
setzen wir den Quotienten $=1$.
Mit diesen Vereinfachungen wird die Gleichung~\eqref{skript:cmb:saha1}
zu
\begin{equation}
\frac{n_\text{H}}{n_pn_e}
=
\biggl(\frac{m_ek_BT}{2\pi h^2}\biggr)^{-\frac32}
\exp\biggl( \frac{Q}{k_BT} \biggr),
\label{skript:cmb:saha}
\end{equation}
sie heisst die {\em Saha-Gleichung}.
\index{Saha-Gleichung}

\subsection{Rekombination}
Die Gleichung~\eqref{skript:cmb:XnH} erlaubt uns, in der
Saha-Gleichung~\eqref{skript:cmb:saha} das Verhältnis $m_\text{H}/n_p$
durch $X$ ausdrücken, wir erhalten
\begin{equation}
\frac{1-X}{X}
=
n_p
\biggl(\frac{m_ek_BT}{2\pi h^2}\biggr)^{-\frac32}
\exp\biggl( \frac{Q}{k_BT} \biggr).
\end{equation}
Die Protonendichte $n_p$ lässt sich dank~\eqref{skript:cmb:etaX}
durch $\eta$ und die Photonendichte $n_\gamma$ ausdrücken als
\[
\frac{1-X}{X^2}=\eta n_\gamma
\biggl(\frac{m_ek_BT}{2\pi h^2}\biggr)^{-\frac32}
\exp\biggl( \frac{Q}{k_BT} \biggr).
\]
Für die Hohlraumstrahlung ist die Photonendichte $n_\gamma$ in Abhängigkeit
von der Temperatur aus \eqref{skript:cmb:photonenzahl} bekannt.
\begin{align}
\frac{1-X}{X^2}
&=\eta
16\pi\zeta(3)
\biggl(\frac{k_BT}{hc}\biggr)^3
\biggl(\frac{m_ek_BT}{2\pi h^2}\biggr)^{-\frac32}
\exp\biggl( \frac{Q}{k_BT} \biggr)
\notag
\\
&=
16\pi\zeta(3)(2\pi)^{\frac32}
\,
\eta
\,
\biggl(\frac{k_BT}{m_ec^2}\biggr)^{\frac32}
\exp\biggl( \frac{Q}{k_BT} \biggr)
\label{skript:cmb:xqauadr}
\end{align}
Schreibt man $S(T,\eta)$ für die rechte Seite von \eqref{skript:cmb:xqauadr},
dann wird daraus eine quadratische Gleichung
\[
1-X-X^2 S(T,\eta)=0
\]
für den Ionisationsgrad, die die Lösungen
\[
X = \frac{-1\pm\sqrt{1+4S}}{2S}
\]
hat.
Nur das positive Zeichen liefert eine positive Lösung, so dass wir
\[
X=\frac{-1+\sqrt{1+4S}}{2S}
\]
erhalten.

Für $T\to 0$, also für die Zukunft, wird $S\to \infty$ gehen
und damit
\[
\lim_{S\to\infty}\frac{-1+\sqrt{1+4S}}{2S}
=
\lim_{S\to\infty}\frac{1}{\sqrt{S}}
=
0.
\]
Der Ionisationsgrad fällt also in Zukunft gegen 0.

Für $T\to\infty$, also im ganz frühen Universum, geht $S\to\infty$
und damit auch $X\to 0$
Hilfe der Regel von de l'H\^opital wird
\begin{align*}
\lim_{S\to 0}\frac{-1+\sqrt{1+4S}}{2S}
&=
\lim_{S\to 0}
\frac{\frac12(1+4S)^{-\frac12} 4S}{2}
\\
\lim_{S\to 0}
\frac{\frac12(1+4S)^{-\frac12}\cdot 4}{2}
=
1.
\end{align*}
Im frühen Universum war also das Gas fast vollständig ionisiert.

Definieren wir willkürlich den Zeitpunkt der Rekombination als
denjenigen Zeitpunkt, zu dem $X=\frac12$ ist, dann gilt auf Grund
von \eqref{skript:cmb:xqauadr}
\[
\frac{1-\frac12}{\bigl(\frac12\bigr)^2}
=
2
=
16\pi\zeta(3)(2\pi)^{\frac32}
\,
\eta
\,
\biggl(\frac{k_BT_\text{rec}}{m_ec^2}\biggr)^{\frac32}
\exp\biggl( \frac{Q}{k_BT_\text{rec}} \biggr).
\]
Durch Lösung dieser Gleichung können wir die Temperatur bestimmen,
bei der die Rekombination stattgefunden hat.
Nimmt man $\eta=5.5\cdot 10^{-10}$, kann man für die Rekombinationstemperatur
\[
T_\text{rec} = 3740\text{K}
\]
finden.

\section{Anisotropie des CMB}
\rhead{Anistrotopie des CMB}


