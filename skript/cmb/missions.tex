\subsection{COBE}
Die COBE (Cosmic Background Explorer) Mission, hatte den Zweck, exakte 
Messungen der kosmischen Infrarot-(hier nicht näher behandelt) und 
Mikrowellenhintergrundstrahlung durchzuführen.
Der Satellit wurde am 18. November 1989 ins All geschossen und verfügte über 
drei Messinstrumente \ref{COBE}:
\begin{itemize}
	\item DIRBE (Diffuse Infrared Background Experiment): Um nach der 
	kosmischen Infrarotstrahlung zu suchen.
	Dank der Messung dieser Strahlung konnten unter anderem Modelle der 
	Entstehung von Sternen erstellt werden.
	\item DMR (Differential Microwave Radiometer): Um die Anisotropie der 
	kosmischen Hintergrundstrahlung nachzuweisen.
	\item FIRAS (Far Infrared Absolute Spectrophotometer): Um die Temperatur 
	der kosmischen Hintergrundstrahlung nachzuweisen. 
\end{itemize}

Das Resultat dieser Messungen ist in Abbildung \ref{fig:COBE} zu sehen.
\begin{figure}
	\includegraphics[width=\linewidth]{cmb/images/COBE_CMB.jpg}
	\caption{Das erste Bild der kosmischen Hintergrundstrahlung}
	\label{fig:COBE}
\end{figure}
Die COBE Mission war ein voller Erfolg.
Die letzten Zweifel an der Urknall-Theorie konnten zerstreut werden.
DMR konnte Temperaturschwankungen zwischen verschiedenen Stellen am Himmel 
von $10^{-5}$ nachweisen, womit die Anisotropie der Strahlung bewiesen war.
FIRAS mass den spektralen Verlauf der Strahlung und kam zum Ergebnis, dass ihre 
Temperatur extrem genau der eines Schwarzkörpers entspricht, nämlich 2.725 K 
(siehe Abbildung \ref{fig:CMB_spectrum}).
Damit war bewiesen, dass es sich bei der gemessen Strahlung tatsächlich um die 
kosmische Hintergrundstrahlung handelt, da man weiss, dass eine zur 
Rekombination entstandene Schwarzkörperstrahlung sich von 3 K auf eben diese 
2.725 K abgekühlt haben muss.

John C. Mather (NASA) und George F. Smoot (University of California), erhielten 
für ihre Arbeit an der COBE Mission 2006 den Nobel Preis für Physik.
\begin{figure}
	\includegraphics[width=\linewidth]{cmb/images/CMB_spectrum.png}
	\caption{Das von COBE gemessene Spektrum stimmt perfekt mit dem erwarteten 
	Schwarzkörper-Spektrum überein.}
	\label{fig:CMB_spectrum}
\end{figure}

\subsection{WMAP und Planck}
WMAP (NASA) und Planck (ESA, European Space Agency) folgten auf COBE.
Ihr Ziel war es unter anderem, präzisere Bilder der kosmischen 
Mikrowellenhintergrundstrahlung zu erhalten.

\subsubsection{WMAP (Wilkinson Microwave Anisotrophy Probe)}
WMAP wurde im Juni 2001 ins All geschossen, mit dem Ziel verschiedenste 
kosmologische Messungen durchzuführen.
Die Resultate dieser Messungen sind sehr vielfältig, deswegen werden die 
interessantesten hier kurz aufgeführt:
\begin{itemize}
	\item Messung der kosmischen Hintergrundstrahlung: Durch die WMAP Messungen 
	konnte ein sehr exaktes Bild der Strahlung erzeugt werden (siehe Abbildung 
	\ref{fig:CMB_WMAP}).
	\begin{figure}
		\includegraphics[width=\linewidth]{cmb/images/CMB_WMAP.png}
		\caption{Kosmischer Mikrowellenhintergrund gemessen von WMAP}
		\label{fig:CMB_WMAP}
	\end{figure}
	\item Alter des Universums: Das Alter des Universums konnte auf ein halbes 
	Prozent genau auf 13.77 Milliarden Jahre bestimmt werden
	\item Dunkle Materie und dunkle Energie: Es konnte bestimmt werden, dass 
	das Universum zu 24.0 Prozent aus dunkler Materie und zu 71.4 Prozent aus 
	dunkler Energie besteht.
\end{itemize}
Dies ist nur ein kleiner Ausschnitt aus den vielfältigen Resultaten der 
WMAP Mission. \ref{CMB_WMAP}

\subsubsection{Planck}
Planck wurde 2009 abgeschossen, um die kosmische 
Mikrowellenhintergrundstrahlung noch genauer als bisher studieren zu können.
Das Ziel war es das derzeitige Standardmodell des Universums so zu bestätigen, 
dass es über jeden Zweifel erhaben ist.
Aus den Daten konnte ein noch besser aufgelöstes Bild (siehe Abbildung \ref{fig:CMB_Planck}) generiert werden:
\begin{figure}
	\includegraphics[width=\linewidth]{cmb/images/CMB_Planck.jpg}
	\caption{Kosmischer Mikrowellenhintergrund gemessen von Planck}
	\label{fig:CMB_Planck}
\end{figure}
Das Ziel von Planck wurde erreicht und das zurzeit anerkannte Modell konnte 
bestätigt werden.

\subsection{Vergleich}
Stellt man die Bilder im direkten Vergleich dar, sieht man, wie stark die Auflösung verbessert wurde (siehe Abbildung \ref{fig:COBE_WMAP_PLANCK}).
\begin{figure}
	\includegraphics[width=\linewidth]{cmb/images/COBE_WMAP_Planck.jpg}
	\caption{Direkter Vergleich der aus den verschiedenen Missionen 
	entstandenen Bildern}
	\label{fig:COBE_WMAP_PLANCK}
\end{figure}
Wie man sieht, sind die Verbesserungen beträchtlich.
Das Bild von Planck dient auch als Grundlage für die Berechnungen, die später 
noch aufgezeigt werden.