\subsection{Die Formen des Universums}
Eine der spannendsten Fragen der Kosmologie ist eine ganz simple: Welche Form 
hat unser Universum?

Nach der Entdeckung und Messung der kosmischen Mikrowellenhintergrundstrahlung 
ist es uns möglich, diese Frage zu beantworten.
Die Frage nach der Form eines Körpers ist überraschend einfach durch die 
klassische Geometrie zu klären.
Grundsätzlich sind drei mögliche Formen denkbar:
\begin{itemize}
	\item Das Universum könnte positiv gekrümmt sein, wie eine Kugel.
	Zwei parallele Linien auf der Kugeloberfläche kreuzen sich und die 
	Winkelsumme eines Dreiecks ist grösser als 360 Grad.
	\item Das Universum könnte negativ gekrümmt sein, wie ein Sattel.
	Zwei parallele Linien weichen immer stärker voneinander ab und die 
	Winkelsumme eines Dreiecks ist kleiner als 360 Grad.
	\item Das Universum könnte flach sein, wie ein Blatt Papier.
	Zwei parallele Linien berühren sich nie und die Winkelsumme eines Dreiecks 
	beträgt 360 Grad.
\end{itemize}
Dieses Konzept lässt sich auch auf unser Universum übertragen.
Da es uns möglich ist in der Zeit zurückzublicken, können wir die Bilder aus 
der Vergangenheit analysieren und so auf die Form des Universums schliessen.
Wir wissen welche Temperatur die kosmische Mikrowellenhintergrundstrahlung zum 
Zeitpunkt der Rekombination gehabt haben muss (durch zurückrechnen).
Dann können die aufgenommenen Bilder der Strahlung als Vergleich herebeiziehen:
\begin{itemize}
	\item Ist das Universum positiv gekrümmt, wäre ein Vergrösserungseffekt 
	festzustellen.
	Die Temperatur würde bei uns höher erscheinen? als sie es wirklich ist.
	\item Ist das Universum negativ gekrümmt, wäre ein Verkleinerungseffekt 
	festzustellen.
	Die Temperatur würde bei uns tiefer erscheinen, als sie es wirklich ist.
	\item Ist das Universum flach, würde die Temperatur genau so sein, wie 
	berechnet.
\end{itemize} 