\subsection{RGB Konvertierung}

Für die Analyse des CMBs werden die sphärischen harmonische Koeffizienten 
$c_l^m$ benötigt. Dafür müssen aber erst die Bilddaten, welche im Rot-Grün-Blau 
(RGB) Format vorliegen, in Kelvin Werte umgewandelt werden. Da nicht klar ist, 
wie genau die Ursprüngliche Transformation aussah, wird die Analyse mittels 
einer Annäherung an die wirkliche Transformationsfunktion durchgeführt.

Den dafür benötigten Referenzfarbverlauf finden wir in den Resultaten der 
Planck Mission \cite{cmb:planck_overview}, zu sehen in 
Abbildung~\ref{fig:color-strip-orig}. Ein Blick auf die einzelnen RGB Kanäle 
zeigt, wie eine mögliche Funktion aussehen könnte (siehe 
Abbildung~\ref{fig:color-strip-orig-rgb}). Da es dabei aber um einen Screenshot 
der im PDF Dokument enthaltenen Grafik handelt, ist die Farbtreue nicht gegeben 
und somit ist auch dies nicht die echte Transformationsfunktion. Als mögliche 
Annäherung wird daher der Farbverlauf in Abbildung~\ref{fig:color-strip} mit 
dem RGB Profil in Abbildung~\ref{fig:color-strip-rgb} verwendet. 
Diese hat den Vorteil, dass man, wenn man die einzelnen Kanäle aufsummiert, 
zwei einfache Lineare Funktionen
\begin{equation*}
	y =
	\begin{dcases}
		\frac{765}{500}x + 765 & \text{für} x \leq 0,\\
		-\frac{765}{500}x + 765 & \text{für} x \geq 0,\\
	\end{dcases}
\end{equation*}
erhält. Da wir ja die $y$-Werte kennen, müssen wir die Funktion umkehren, 
was uns
\begin{align*}
	x_1 &= 500\frac{y - 765}{765}\\
	x_2 &= -500\frac{y - 765}{765}
\end{align*}
liefert. Um die beiden Fälle zu unterscheiden, genügt es, lediglich den Anteil 
des blauen und roten Kanals miteinander zu vergleichen. Enthält ein Pixel mehr 
Rot als Blau gilt die Formel für $x_2$ andernfalls diejenige für $x_1$.

\begin{figure}
	\centering
	\begin{subfigure}
		\centering
		\includegraphics[width=0.9\linewidth]{cmb/images/color-strip-full.png}
		\caption{Farbverlauf der für die Codierung des CMB Bildes der ESA 
		Planck 
			Mission verwendet wurde. Mit einer Skala von $-500\mu K$ bis 
			$500\mu 
			K$.}
		\label{fig:color-strip-orig}
	\end{subfigure}
	\hfill
	\begin{subfigure}
		\centering
		\includegraphics[width=\linewidth]{cmb/converter/rgb-graph.pdf}
		\caption{RGB Profil der Abbildung~\ref{fig:color-strip-orig}.}
		\label{fig:color-strip-orig-rgb}
	\end{subfigure}
\end{figure}

\begin{figure}
	\centering
	\begin{subfigure}
		\centering
		\includegraphics[width=0.9\linewidth]{cmb/converter/converter-function-strip.png}
		\caption{Der für die Analyse verwendete Farbverlauf.}
		\label{fig:color-strip}
	\end{subfigure}
	\hfill
	\begin{subfigure}
		\centering
		\includegraphics[width=\linewidth]{cmb/converter/converter-function.pdf}
		\caption{RGB Profil der Abbildung~\ref{fig:color-strip}.}
		\label{fig:color-strip-rgb}
	\end{subfigure}
\end{figure}
