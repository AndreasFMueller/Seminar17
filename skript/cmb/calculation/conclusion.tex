\section{Fazit}

Wir konnten erfolgreich das Leistungsspektrums des kosmischen 
Mikrowellenhintergrunds aus den Bildern der ESA Planck Mission berechnen. Es 
stellte sich heraus, dass man bereits alleinig aus den Bilddaten, die ähnliche 
Resultate erhält, wie die drei Weltraummissionen (COBE, WMAP und Planck) 
bereits gezeigt hatten. Natürlich leidet die Genauigkeit darunter, da die Werte 
einmal in RGB-Werte und wieder zurück gerechnet werden müssen. Schwierigkeiten 
bereitet dabei zudem, dass die für die erste Transformation verwendete Funktion 
nicht direkt bekannt ist. Das hat zur Folge, dass nur eine Näherungsfunktion 
verwendet werden kann, welche wiederum zu einem noch grösseren Fehler führt.

Die Forschung im Zusammenhang mit dem CMB ist noch lange nicht abgeschlossen 
und man entdeckt immer noch weitere neue Dinge, die man sich noch nicht 
erklären kann. So ist es beispielsweise noch immer nicht ganz klar, woher der 
grosse kalte Fleck stammt. Es gibt dazu allerlei Spekulationen, die aber 
noch nicht bestätigt sind. Somit bleibt es also spannend, was noch alles mit 
dem ersten Abdruck unseres Universums erklärt werden kann.
