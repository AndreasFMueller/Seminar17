\subsection{Das Leistungsspektrum\label{subsec:cmb:cl}}

Für die Analyse des CMBs ist, wie in Abbildung~\ref{fig:planck_spectrum} zu 
sehen, vor allem das Leistungsspektrum relevant. Für dessen Berechnung werden 
die verschiedenen
\begin{equation}
	C_l = \overline{|c_l^m|^2} = \langle |c_l^m|^2 \rangle =  \frac{1}{2l + 
	1}\sum_{m=-l}^{l}|c_l^m|^2 = \langle c_l^m 
	c_l^{m*} \rangle = 
	\frac{1}{2l + 
	1}\sum_{m=-l}^{l}c_l^{m*}c_l^m
	\label{eq:cmb:cl}
\end{equation}
benötigt. Teilweise wird statt der Summengleichung~\ref{eq:cmb:cl} auch
\begin{equation*}
	C_l = \langle c_{l}^{m}c_{l'}^{m'*} \rangle = 
	c_{l}^{m}c_{l'}^{m'*}\delta_{l'l}\delta_{m'm}
	\label{eq:cmb:cl-single}
\end{equation*}
verwendet. Die für die Berechnung benötigten sphärischen harmonischen 
Koeffizienten $c_l^m$ werden im Kapitel~\ref{skript:chapter:kugelfunktionen} 
näher beschrieben. Die $\delta$-Funktionen in Gleichung~\ref{eq:cmb:cl-single} 
nennt man Kronecker-Delta. Es gilt dabei
\begin{equation}
\delta_{ij} =
\begin{cases}
0 & \text{für} j \neq j, \\
1 & \text{für} i = j.
\end{cases}
\end{equation}

Bei der Gleichung~\ref{eq:cmb:cl-single} wird ein einzelnes 
$c_l^m$ für jedes $l$ ausgewählt. Dies mit der Begründung, dass bei einer 
unendlichen Anzahl Universen die Durchschnitte der jeweiligen $c_l^m$ für 
jeweils ein $l$ genau gleich wären. \cite{cmb_klauber}

Da wir nur ein einzelnes Universum kennen und messen können, bleibt uns 
nichts anderes übrig als auf die Gleichung~\ref{eq:cmb:cl} zurückzugreifen. Die 
verschiedenen $C_l$ sind dabei die Varianz der sphärischen Harmonischen 
Amplituden $c_l^m$. Weiter ist $C_l$ auch eine Korrelations-Funktion, da bei 
grossen $C_l$ für ein $l$ die Maxima des CMBs mit derjenigen der verschiedenen 
Kugelflächenfunktionen $Y_l^m$ korrelieren. Sollte das Universum also wie im 
Abschnitt~\ref{subsec:cmb:data-meaning} angedeutet flach sein, dann muss das 
globale Maximum der $C_l$-Werte bei rund 180 liegen. Dies weil der Winkel 
zwischen zweier Maxima der $Y_l^m$-Funktion dann $1\degree$ beträgt. Die 
Auswertung der ESA Planck Mission bestätigt dies bereits wie in 
Abbildung~\ref{fig:planck_spectrum} zu sehen. Unsere Resultate nach der CMB 
Analyse sind im Abschnitt~\ref{subsec:cmb:results} zu finden.

Um aus den berechneten $C_l$-Werten schlussendlich das Leistungsspektrum zu 
erhalten, müssen diese jeweils noch zusätzlich mit dem Faktor 
$\dfrac{l(l+1)}{2\pi}$ multipliziert werden.